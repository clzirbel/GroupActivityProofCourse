\yourname

\activitytitle{Construction of the real numbers}{Construction of the real numbers using Dedekind cuts}

\overview{
Many things can be defined and written about that don't actually exist; unicorns, little green men from Mars, and others may come to mind.
To this point in your mathematical career, you have worked with real numbers and used many of their properties, but how do we know that they really exist?
The mathematical answer is that we {\em construct} them from simpler numbers and show that they have the right properties.
 }

\note{{\bf Natural numbers.}  At some point in your life you learned the counting numbers 1, 2, 3, \ldots.
Then you learned to add and multiply, and these operations have the familiar algebraic properties like commutativity, associativity, distributivity.
Their biggest claim to fame:  No matter how high you count, there is always a next number.
Note, however, that we are not {\em defining} the natural numbers.}

\definition{Zero}{Subtraction problems like $9-5$ have answers that are whole numbers, but subtraction problems like $3-3$ and $12-12$ call for a new number.
You can define $0$ as $3-3$ or as $12-12$; there are many ways to write this new number.}

\definition{Negative numbers}{Subtraction problems like $5-9$ and $8-12$ need another set of new numbers to be defined.
You can define $-4$ to be $5-9$ but also $8-12$.
Anytime you want to work with $-4$, you can substitute in $5-9$ instead.
Or $8-12$.}

\exercise{With negative numbers, we say that $a-b = c-d$ if $a+d = c+b$.  Check that this is the case for $5-9$ and $8-12$.
It's important that to check, and you only need to work with natural numbers.}{0in}

\definition{Integers}{The natural numbers, zero, and the negative numbers make up the integers.
Each integer can be written as $a-b$ where $a$ and $b$ are natural numbers.
The integers are closed under addition and multiplication, and these operations have the usual algebraic properties like commutativity and associativity, additive inverses, additive identity, and multiplicative identity.}

\definition{Rational numbers}{\label{rationaldefinition}
Division problems like $15 \div 5$ have answers that are integers, but problems like $5 \div 15$ need yet more new numbers to be defined.
For some reason people decided to write the new numbers as $\frac{5}{15}$ but we could have chosen some other notation like $(5,15)$ or $5 \# 15$.
At any rate, these new ``rational'' numbers are made up of two integers, the second of which needs to be non-zero.
The rules for rational numbers are worth noting in some detail:
\balist{0.0in}
\item Rational numbers $\frac{a}{b}$ and $\frac{c}{d}$ are {\em equal} if $ad = bc$, which is a matter of integer multiplication.
\item We say that $\frac{a}{b} < \frac{c}{d}$ if $ad < bc$, which again comes down to integers
\item The sum of $\frac{a}{b}$ and $\frac{c}{d}$ is the rational number $\frac{ad+bc}{bd}$.  Note that you only need to use integer arithmetic to find the sum of two rational numbers.
\item The product of $\frac{a}{b}$ and $\frac{c}{d}$ is the rational number $\frac{ac}{bd}$.
\elist
It's important to note that everything about these new rational numbers is defined in terms of integers.
There are multiple ways to define each rational number; $\frac{3}{12} = \frac{1}{4}$ for example.
The rational numbers can be shown to have the usual algebraic properties, always because the integers have the property.
Bonus:  non-zero rational numbers have multiplicative inverses.
The set of all rational numbers is denoted by $\Q$.
}

\exercise{Use Definition \ref{rationaldefinition} for each part.
\balist{0.1in}
\item Check that $\frac{1}{3} = \frac{5}{15}$ using the definition.
\item Check that $\frac{1}{4} < \frac{2}{7}$ using the definition.
\item Add $\frac{1}{3} + \frac{1}{4}$ using the definition.
\item Show that $\frac{a}{b} + \frac{c}{d} = \frac{c}{d} + \frac{a}{b}$ using a rewrite proof.
Proofs of other algebraic properties are similar.
\elist
}{0.1in}

\definition{Complex numbers}{Once the real numbers have been defined, we can define the complex numbers by letting $i = \sqrt{-1}$ and then thinking about numbers of the form $a + ib$ where $a$ and $b$ are real numbers.}

\remark{To construct the integers, the rational numbers, and the complex numbers, you put together two numbers of a simpler sort.  As it happens, constructing the real numbers is harder.
The basic idea is this:  to refer to a real number like $\pi$, think of the rational numbers $3,$ $3.1,$ $3.14,$ $3.141$, $3.1415$, and all other rational numbers less than $\pi$.
Then $\pi$ is the ``top'' of this set of rational numbers.
This is how we can use rational numbers to get our hands on real numbers like $\pi$ that are not rational.
In fact, we will literally {\em define} real numbers to be sets of rational numbers like this.}

\remark{\label{rationalsets}
Here are a few examples of the sets we'll be using.
After each set, describe it in words.
\balist{0.1in}
\item $A = \{ q \in \Q : q < 0 \}$
\item $B = \{ q \in \Q : q \leq 7 \}$
\item $C = \{ q \in \Q: q^3 < 5 \}$
\item $D = \{ q \in \Q: q< 0$ or $q^2 < 2 \}$
\elist
It's OK to use words like ``cube root of 5'' in your description, even though that is not a rational number, and so has not been constructed yet.
}

\definition{Closed below}{A set $A$ of rational numbers is said to be {\em cl4osed below} if for all $q \in A$, for all $p \in \Q$ with $p < q$, we have $p \in A$.  In words, if $A$ contains the rational number $q$, then it contains every rational number less than $q$ as well.}

\exercise{Is the set $\{\frac{1}{2}, \frac{1}{3}, \frac{1}{4}, \frac{1}{5}, \ldots \}$ closed below?  Why or why not?}{0in}

\remark{\label{closedbelowremark}
Note that ``closed below'' is unusual in that it has two ``for all'' quantifiers.
To prove that a set $A$ is closed below, follow this format:  ``Let $q \in A$.  Let $p \in \Q$ such that $p < q$.  Show that $p \in A$.}

\exercise{Show that each of the following sets is closed below, following Remark \ref{closedbelowremark}.
\balist{0.5in}
\item $A = \{ q \in \Q : q < 0 \}$.  \Hint Use transitivity.
\item $B = \{ q \in \Q : q \leq 7 \}$
\item $C = \{ q \in \Q: q^3 < 5 \}$.  Do not write $\sqrt[3]{5}$, just use rational numbers and integers.
\item $D = \{ q \in \Q: q< 0$ or $q^2 < 2 \}$.  \Hint Consider two cases, $p < 0$ and $p \geq 0$. Do not write $\sqrt{2}$.
\elist
}{0.6in}

\remark{Sets $C$ and $D$ illustrate how we can use sets of rational numbers to point to a number that we know is irrational, in this case $\sqrt[3]{5}$ and $\sqrt{2}$.  The irrational number we have in mind is at the ``top'' of the set, just like 0 is at the ``top'' of the set $A$.}

\definition{Dedekind cut}{\label{Dedekind}
A set $A \subseteq \Q$ is called a {\em Dedekind cut} if all of the following happen:
\bilist{0in}
\item $A$ is closed below
\item $A$ has no greatest element, meaning that for all $p \in A$, there is a number $q \in A$ with $p < q$.
Keep in mind that $q$ has to be a rational number.
\item There exists an integer $m$ with $m \in A$
\item There exists an integer $n$ with $n \notin A$
\elist
}

\exercise{Check each of the sets below to see if it is a Dedekind cut.  
You have already shown that they are closed below.
If the set is a Dedekind cut, show that.  If not, explain why not.
\balist{0.0in}
\item $A = \{ q \in \Q : q < 0 \}$.  Let $m =$ \blank{1in}.  Let $n = $ \blank{1in}.  To show (ii), let $p \in A$.  Let $q = $ \blank{1in}.
\item $B = \{ q \in \Q : q \leq 7 \}$

\vspace*{0.5in}

\item $C = \{ q \in \Q: q^3 < 5 \}$.  Let $m =$ \blank{1in}.  Let $n = $ \blank{1in}.  To show (ii), given $p \in A$, I suggest you consider two cases.  When $p < 1$, let $q = 1$.  When $p \geq 1,$ let $q = p + c$ where $c =\frac{5-p^3}{100}$ when $p \geq 1$.  Calculate $(p+c)^3$ and use the fac that $c < 1$, so $c^2 < c$ and $c^3 < c$.  Also keep in mind that $p < 2$.

\vspace*{1in}

\item $D = \{ q \in \Q: q< 0$ or $q^2 < 2 \}$.
\elist
}{0.7in}

\pagebreak
\exercise{\label{sumisDedekind}
Let $A$ and $B$ be Dedekind cuts.
Define a new set $C$ by $C = \{ z : $ there exist $a \in A$ and $b \in B$ such that $z = a + b \}$.
Show that $C$ is a Dedekind cut by checking all four requirements in \ref{Dedekind}.
\blist{0.0in}
\item $C$ is closed below.  Let $d \in C$, and let $c \in \Q$ with $c < d$.  We need to show that $c \in C$, so we need to write it as the sum of an element of $A$ and an element of $B$.
Because \blank{0.75in}, there exist $a \in A$ and $b \in B$ such that $d = a + b$.
Let $p = a - (d-c)/2$ and $q = b - (d-c)/2$.
It's clear that $p$ and $q$ are \blank{1in} numbers and that $p < a$ and that \blank{1in}.
Since $p < a$ and $A$ is closed below, we know that $p \in A$.
Since $q < b$ and \blank{1.5in}, we know that \blank{1in}.
Finally, $p+q = \blank{3.5in} = c$, and so \blank{1in}.

\item $C$ has no greatest element.  Let $p \in C$.  Show that there is a number $q \in C$ with $p < q$.
\vspace*{0.5in}

\item There exists an integer $m$ with $m \in C$.  \Hint Use $m_A \in A$ and $m_B \in B$ from \ref{Dedekind}(c).
\vspace*{0.5in}

\item There exists an integer $n$ with $n \notin C$.  \Hint Use $n_A \in A$ and $n_B \in B$.  Then $a < n_A$ for all $a \in A$ and $b < n_B$ for all $b \in B$.  Then $a + b < n_A + n_B$ for all $a \in A$ and all $b \in B$.
Now ... why does that mean that $n_A + n_B$ is not in $C$?
\elist

}{0.5in}

\definition{Real numbers}{We will refer to each Dedekind cut as a ``real number.''  The set of real numbers will be written as $\R$.}

\remark{Yes, you read that right.  A real number is being defined as nothing more, and nothing less, than a Dedekind cut, which is a set of rational numbers.  This is simply one way to use rational numbers to describe and work with real numbers.  It was easier to define 0 as $3-3$ or to define fractions as things you get from pairs of integers.  You'll get used to it.}

\definition{Equality of real numbers}{If $A$ and $B$ are real numbers, we say that $A$ and $B$ are equal if $A \subseteq B$ and $B \subseteq A$.  We write $A =  B$.}

\definition{Less than for real numbers}{If $A$ and $B$ are real numbers we say that $A$ is less than $B$ if $A \subset B$.  We write $A < B$.}

\definition{The real number zero}{Let $\zero = \{ q \in \Q : q < 0\}$.}
\definition{The real number one}{Let $\one = \{ q \in \Q : q < 1\}$.}

\definition{Addition of real numbers}{Let $A$ and $B$ be real numbers.
The sum of $A$ and $B$ is the real number $C = \{ z : $ there exist $a \in A$ and $b \in B$ such that $z = a + b \}$.
This set is a Dedekind cut as explained in \ref{sumisDedekind}.
We write $A \oplus B$ for the sum.
}

\prove{Show that $\zero \oplus \one = \one$ by showing set inclusion both ways.
\balist{0.5in}
\item Show $\zero \oplus \one \subseteq \one$.  Let $c \in \zero \oplus \one$.  
Then $c = a + b$ for some $a \in \zero$ and $b \in \one$.
Thus $a < 0$ and $b< 1$.
Thus, \blank{1in} and so $c \in \one$.
\item Show $ \one \subseteq \zero \oplus \one$.  Let $c \in \one$.  Then $\blank{1in}$.
Write $c = \frac{c-1}{2} + \frac{c+1}{2}$ and check that this means that $c \in \zero \oplus \one$.
\elist
}{0.5in}

\prove{{\bf Commutativity}.  Let $A$ and $B$ be real numbers.  Show that $A \oplus B = B \oplus A$.
Instead of showing set inclusion both ways, do this as a rewrite proof:
\beqnarray{-0.1in}
A \oplus B &=& \{ a + b : a \in A, b \in B \} \\
                 &=& \{ b + a :  \\
                 &=& 
\eeqnarray{-0.5in}
}{0in}

\prove{{\bf Associativity}.  Let $A$, $B$, and $C$ be real numbers.  Show that $A \oplus (B \oplus C) = (A \oplus B) \oplus C$.
Do this as a rewrite proof, abbreviating the definition of the sum.
\beqnarray{-0.1in}
A \oplus (B \oplus C) &=& A \oplus \{ b + c : b \in B, c \in C\} \\
                                  &=& \{ a + (b+c) : \\
                                  &=& 
\eeqnarray{0in}
}{0.0in}

\prove{{\bf Additive identity}.  Let $A$ be a real number.  Show that $A \oplus \zero = A$.
\balist{0.5in}
\item Let $p \in A \oplus \zero$.  Then $p = a + b$ where $a \in$ \blank{1in} and $b $ \blank{1in}.
\Hint You will use the fact that $A$ is closed below.

\item Let $p \in A$.  We need to write $p$ as the sum of an element $a$ of $A$ and a rational number $b$ less than zero.
That means that $a$ will be greater than $p$.  Fortunately, such a number exists because $A$ has no greatest element.

\noindent Let $a \in A$ such that $p < a$.  Let $b = p-a$.  Then ...
\elist
}{0.3in}

\definition{Additive inverse.}{  Let $A$ be a real number.  Define a new set $-A$ by
\[
    -A = \{ b-c : b, c \in Q, b < 0, c \notin A \}
\]
Think of $b$ as being very close to 0, so the elements of $(-A)$ are like the negatives of the rational numbers bigger than the elements of $A$.
It may be hard to understand this set intuitively, and so it may be easier just to work with the definition and not try to get the intuition straight.
}

\guidedproof{
Show that $-A$ is a Dedekind cut.
\bilist{0in}
\item $A$ is closed below.  Let $q \in -A$.  Let $p \in \Q$ with $p < q$.
Since $q \in -A$, we can write $q = b-c$ where $b < 0$ and $c \notin A$.
Check that $p = b+(p-q) - c$ by substituting in for $q$.
Thus, we have written $p = b' - c'$ where $b' < 0$ and $c' \notin A$, by setting $b'=$\blank{1in} and $c'=$\blank{1in}.
This tells us that $p \in -A$, so $-A$ is closed below.

\item $A$ has no greatest element.  Let $p \in -A$.  Then we can write $p = b-c$ where $b < 0$ and $c \notin A$.
Let $q = (b/2) -c $.  Check that $q \in -A$ and that $p < q$.

\vspace*{0.3in}

\item There exists an integer $m$ with $m \in -A$.
Check that $-1-n_A$ where $n_A \notin A$ from \ref{Dedekind}(d) will work.

\vspace*{0.3in}

\item {\bf Challenge:} There exists an integer $n$ with $n \notin -A$.
Probably $-m_A+1$ from \ref{Dedekind}(c) will work.

\vspace*{0.3in}

\elist
}

\prove{{\bf Additive inverse property.} Let $A$ be a real number.  Show that $A \oplus (-A) = \zero$.
\balist{0.1in}
\item Let $z \in A \oplus (-A)$.  Then $z = p + q$ where $p \in A$ and $q = b-c$ where $b < 0$ and $c \notin A$.
That is, $z = p + b - c$.
Since $c \notin A$, we know that $c > p$, because \blank{2in}.
Then $z < 0$ because \blank{2in}.
Thus $z \in \zero$, and so $A \oplus (-A) \subseteq \zero$.

\item Let $z \in \zero$.  Then $z$ is a rational number and $z < 0$.
We need to write $z = p + q$ where $p \in A$ and $q \in (-A)$, and $q$ needs to be written as $b-c$ where $b < 0$ and $c \notin A$.
If $z$ is close to 0, then $p$ and $c$ will need to be close to the ``top'' of the set $A$.
The next result will show that there exists $p \in A$ and $c \notin A$ with $p-c = z/2$.
Also let $b = z/2$.
Then $z = z/2 + z/2 = p - c + b$ as required.
So $z \in A \oplus (-A)$ and thus $\zero \subseteq A \oplus (-A)$.
\elist
}{0in}

\guidedproof{Let $A$ be a real number.
Let $c > 0$ be a rational number.
Then there exists $a \in A$ and $b \notin A$ such that $b-a = c$.

\noindent
By the definition of a Dedekind cut, there are integers $m \in A$ and $n \notin A$.
Consider the numbers $m + kc$ for $k = 0, 1, 2, \ldots$.
Draw a picture of these on a number line below.
When $k = 0$, $m+kc \in A$.
When $k > (n-m)/c$, $m+kc > n$ and so $m+kc \notin A$.
Thus, for some value of $k$, $m+kc \in A$ but $m+(k+1)c \notin A$. 
Let $a = m+kc$ and $b = m+(k+1)c$.
}

\vspace*{0.2in}

\challenge{$\zero$ is unique.
That is, if there is another real number $Z$ for which $A \oplus Z = A$ for all real numbers $A$, then $Z = \zero$.}{0.5in}

\challenge{Given a real number $A$, the additive inverse $-A$ is unique.}{0.5in}

\definition{Multiplication of positive real numbers}{Let $A$ and $B$ be real numbers with $\zero < A$ and $\zero < B$.
The product of $A$ and $B$ is the real number $\{ z : z \in \Q $ and $ z \leq 0$ or there exists $a \in A$ with $a > 0$ and $b \in B$ with $b > 0$ such that $z = ab \}$.
We write $A \otimes B$ to denote the product.}

\remark{It is easiest to define multiplication of positive real numbers.  The definition is very much like the definition of the sum, only a bit more complicated because of the need to include the negative rational numbers.}

\challenge{Show that $A \otimes B$ is a Dedekind cut.}{1in}

\prove{{\bf Multiplicative identity for positive real numbers.} Let $A$ be a real number with $0 < A$.  Then $A \otimes \one = A$.}{1.5in}

\prove{{\bf Distributivity for positive real numbers.} Let $A$, $B$, and $C$ be real numbers with $0<A$, $0<B$, and $0<C$.
Show that $A \otimes (B \oplus C) = (A \otimes B) \oplus (A \otimes C)$.}{1.5in}

\definition{Multiplicative inverse of positive real numbers}{Let $A$ be a real number with $0 < A$.
Define $A^{-1}$ to be $\{ z : z \in \Q$ and $z \leq 0$ or $1/z \in A \}$.}

\prove{Show that $A^{-1}$ is a Dedekind cut.}{1.5in}

\prove{Show that $A \otimes A^{-1} = \one$}{1.5in}

\prove{Let $D = \{ q \in \Q: q< 0$ or $q^2 < 2 \}$ and let $E = \{ q \in \Q: q< 2 \}$.  Show that $D \otimes D = E.$
This confirms that $D$ corresponds to the square root of 2.}{2in}

\prove{Let $A$ and $B$ be real numbers.  Show that $A \otimes B = B \otimes A$.}{0.1in}

\prove{Let $A$, $B$, and $C$ be real numbers.  Show that $A \otimes (B \otimes C) = (A \otimes B) \otimes C$.}{0.1in}

\prove{Prove that the multiplicative identity $\one$ is unique}{0.1in}

\prove{Prove that the multiplicative inverse $A^{-1}$ is unique}{0.1in}

\definition{Multiplication of non-positive real numbers.}{Let $A$ and $B$ be real numbers.
\balist{0in}
\item If $0 < A$ and $B < 0$, then $A \times B = A \times (-B)$
\item If $A < 0$ and $0 < B$, then $A \times B = (-A) \times B$
\item If $A < 0$ and $B < 0$, then $A \times B = (-A) \times (-B)$
\elist
}
\prove{Show that the properties of multiplication extend to multiplication of non-positive numbers.}{0.1in}

\definition{Least upper bound}{Let $S$ be a collection of real numbers.  A number $B$ is the least upper bound of $S$ if $A \leq B$ for all $A \in S$, and if for all other upper bounds $C$ of $S$, we have $B \leq C$.}

\prove{Given a set $S$ of real numbers, the number $B = \cup_{A \in S} A$ is the least upper bound of $S$.
Thus, every collection of real numbers has a least upper bound.}{0in}

\vfill          % pad the rest of the page with white space



