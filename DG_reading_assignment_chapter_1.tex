\activitytitle{Reading assignment \#1}{Due on the second day of class.}

The idea is to read Chapter 1 of the textbook by Daepp and Gorkin\DGreference.
The assignment is to read it in a particular way.
It may take 3 hours to get it done, but you will learn something in those three hours, and you will start to develop a very important skill.

Get a copy of Chapter 1, ``The How, When, and Why of Mathematics.''
Get out your notebook or some paper.
Go somewhere quiet, where you won't be interrupted for a while.
Turn off your phone so you aren't disturbed.
Don't listen to music that will distract you, and make sure there is no TV or youtube on where you can see it or hear it.

Put the notebook or paper right in front of you.
Put the textbook itself a bit farther away.
Make note of the time that you start reading in your notebook, maybe in the left margin.
Read the first paragraph of the chapter, then write one or more sentences in your notes which capture the main idea(s) of the paragraph.

Read the second paragraph, about Geogre P\'{o}lya's list of guidelines.
Look up the list in the Appendix.
Consider writing them in your notebook, or abbreviated versions of them.

Continue to write a sentence summarizing each paragraph.
I believe that if you are not writing, you are probably not thinking as hard as you need to.
Read slowly.
If you run into a word you don't know, google it or look it up in a dictionary.  If you really don't know it, write the definition in your notebook.
It is OK to spend 15 minutes on each page of the book.  Really.
It is not a goal of the course to learn how to read faster.
The goal is to learn how to get more out of the time you spend reading.
If you stop to take a break, note the time that you stopped and the time you start again.

Read Exercise 1.1 and the text that walks you through P\'{o}lya's guidelines.
Use your notebook to try to solve the puzzle yourself.
I've printed the alphabet twice at the bottom of this page.  If you cut off the bottom version, you can slide it along the top one and easily keep track of how the letters correspond.  That should save you a little time.

A second example starts on page 3 of the textbook.
As you read it, draw diagrams in your notebook.
Yes, there are diagrams printed in the textbook, but you will think harder about the diagram and understand more if you draw your own.

Example 1.2 asks a question.  Read the question and see if you can answer it on your own, without reading further in the book.

On page 7, you will see that solutions of the exercises are provided.
Resist the urge to turn your brain off and just read the solutions.  That is not what they are there for!

Work through some of the problems that begin on page 7 in the book.
You do not need to do all of them, but you should at least understand most of them and attempt a few of them.

You may be able to do problem 1.9 because we have done similar things in class.
Give it a try.
Problem 1.10 doesn't interest me.  Does it interest you?

Problem 1.12 is good.  Will it help to make a graph?
Problem 1.13 seems silly.  Do you like it anyway?

Read the Tips on Doing Homework.
At the end, tally up how much time you have spent on reading this chapter.
Write this number in your notebook and remember the number when you come to class.

\vfill          % pad the rest of the page with white space

{\tt ABCDEFGHIJKLMNOPQRSTUVWXYZABCDEFGHIJKLMNOPQRSTUVWXYZ}

\vspace{0.1in}

{\tt ABCDEFGHIJKLMNOPQRSTUVWXYZABCDEFGHIJKLMNOPQRSTUVWXYZ}

