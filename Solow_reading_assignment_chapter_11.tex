\activitytitle{Reading assignment Chapter 11}{Due on Tuesday, November 21.  20 points}

Read Chapter 11 in the book by Daniel Solow.
It is about showing uniqueness.
We have seen an example already in the Division Algorithm, when you showed that given an integer $k > 0$ and an integer $n$, there are unique integers $q$ and $r$ with $0 \leq r < k$ so that $n = kq + r$.
You used the Direct Uniqueness Method:  you supposed that you could also write $n = kq_2 + r_2$ with $0 \leq r_2 < k$ and showed that $q=q_2$ and $r = r_2$.
There is also an Indirect Uniqueness Method, where you would pretend for a minute that $q \neq q_2$ or that $r \neq r_2$ and argue to a false statement, so that you know this is fantasyland.

\vspace{0.1in}
\noindent
{\bf Specific requirements}
\vspace*{-0.15in}

\begin{itemize}
\item Do exercise 11.2.  The notation may be confusing.  In part (a), your goal is to show that $x^*$ equals $y^*$.  How could you do that?  It might be easier to call the numbers $x_1^*$ and $x_2^*$.  In part (b), the function $f$ is given and fixed.  You might want to think of the problem as showing that $G_1 = G_2$.  In part (c), $p$ and $q$ play the role of $a$.

\item Do exercise 11.5.  The answer to part (c) is ``specialization'' which was covered in Chapter 6, which we did not read.  Please explain what specialization means in the context of this question.  It's not a hard concept.

\item Do exercise 11.6.  
Draw a relevant picture.
Rewrite the proof so that each step is on a different line, and give a justification for each step.
Explain which uniqueness method is being used.

\item Do exercise 11.7.
Draw a relevant picture.
Rewrite the proof so that each step is on a different line, and give a justification for each step.
Explain which uniqueness method is being used.

\item Do exercise 11.9.

\item Do exercise 11.11.

\item At the end, tally up how much time you have spent on this chapter.
Write this number in your notebook.
Bring your notebook to class and turn it in for grading.
\end{itemize}

\noindent
{\bf General comments}

Set yourself up in a place where you won't be disturbed.
Read slowly, and write notes in your own words that reflect your understanding of the material.
