\yourname

\activitytitle{Review exercises}{Due on Tuesday, December 5.  20 points.}

\problem{Let $E = \{ m \in \Z : $ there exists $j \in \Z$ such that $m = 2j \}$.
Let $O = \{ m \in \Z : $ there exists $k \in \Z$ such that $m = 2k + 1 \}$.
Show that $E \cap O = \emptyset$ by letting $m \in E \cap O$ and showing that this leads to a contradiction.
}{1in}

\problem{Continuing the previous problem, show that $E \cup O = \Z$ by showing set inclusion both ways.
}{1in}

\problem{Using standard interval notation, show that $[2,6) \cap [3,8) = [3,6)$ by showing set inclusion both ways. As above, write compound inequalities, then individual inequalities, then compound inequalities again.  Use a number line to illustrate.}{2in}

\problem{Show that $[2,6) \cup [3,8) = [2,8)$ by showing set inclusion both ways.}{0in}

\prove{Show that for all $x \in \R$, there exists an integer $n\geq 1$ such that $x \in (-n,n)$.
Use good form for proofs with nested quantifiers, and be sure to cover both positive and negative values of $x$.
If you have trouble getting started, do scratchwork with $x = 4.2, x = -13.1, x = 0$.}{1.5in}

\prove{Show that $[2,5] \cup (4,7) = [2,7)$.}{2.5in}

\problem{Let ${\displaystyle A = \bigcup_{n=1}^{\infty}} (\frac{1}{n}, 1)$ and let $B = (0,1)$.
Show that $A = B$ by showing containment both ways.
}{2in}

\problem{Suppose that $x \leq 5 + \frac{1}{n}$ for all $n = 1, 2, 3, \ldots$.
Show that $x \leq 5$.
\Hint Consider different types of proof including direct, contrapositive, contradiction, etc.}{0in}

\pagebreak

\problem{Let ${\displaystyle A = \bigcap_{n = 1}^{\infty}} [0, 1+\frac{1}{n}]$ and $B = [0,1]$.
Prove that $A = B$ by showing containment both ways.
}{2in}

\example{Write the following numbers as multiples of 5 plus a remainder from 0 to 4.

38 = 5 $\cdot$ \blank{0.5in} + \blank{0.5in}
\qqqq
40 = 5 $\cdot$ \blank{0.5in} + \blank{0.5in}

39 = 5 $\cdot$ \blank{0.5in} + \blank{0.5in}
\qqqq
41 = 5 $\cdot$ \blank{0.5in} + \blank{0.5in}
}{-0.2in}

\prove{Let $k > 0$ be an integer.
For each integer $n$, let $P(n)$ be the statement: ``There exist integers $q$ and $r$ with $0 \leq r < k$ such that $n = kq + r$.''
Use mathematical induction to show that $P(n)$ is true for all integers $n$.

\balist{0.3in}
\item Show that $P(0)$ is true.

\item Let $n$ be an integer.
Suppose that $P(n)$ is true, so that $n = kq+r$ for some integers $q$ and $r$, with $0 \leq r < k$.
Show that there exist integers $q'$ and $r'$ with $0 \leq r' < k$ so that $n+1 = kq' + r'$, and thus conclude that $P(n+1)$ is true.
Note that you will define $q'$ and $r'$ in terms of $q$ and $r$.
It is helpful to do this with two cases, depending on the value of $r$:

\noindent
Case 1.  Suppose $0 \leq r < k-1$.

\vspace*{0.5in}

\noindent
Case 2.  Suppose $r = k-1$.

\vspace*{0.2in}

\item Suppose that $P(n)$ is true and show that $P(n-1)$ is true.
It is helpful to do this with two cases.
\elist

\vfill
\noindent
Use steps b and c and the idea of mathematical induction to conclude the proof that $P(n)$ is true for all $n$.

}{0in}

