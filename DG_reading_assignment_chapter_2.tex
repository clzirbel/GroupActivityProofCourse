\activitytitle{Reading assignment, Chapter 2}{Due on Wednesday, September 2.}

Read Chapter 2 of the book by Daepp and Gorkin.
As with Chapter 1,
\blist{0.0in}
\item Read somewhere quiet, minimizing distractions from phones and friends
\item Note the times that you start and stop reading, and add up the minutes
\item Read with a pencil in your hand and your notebook open in front of you
\item Write a sentence to summarize each paragraph, re-draw diagrams, work out examples and exercises on your own
\item Look up words you don't know, and write down ones you really don't know
\item Read slowly.  You are not reading a comic book or a newspaper.  It is not a goal of this class for you to learn how to read faster.  The goal is to learn how to get more out of the time you spend reading, and to learn to concentrate for longer periods of time.
\item At the end, tally up how much time you have spent on reading this chapter.
Write this number in your notebook and remember the number when you come to class.
\elist

You will read about ``statements.''
Focus on the ones about mathematical things, and don't worry too much about interpreting the ones that are non-mathematical.

{\bf Note that on page 14, there is a statement about the color of the cover of the book.  Books from Springer used to be plain yellow, but the authors must not have realized that someone would put a big blue bar on the cover of this edition of the book.  Just imagine that the book cover is all yellow.}

Fill out every truth table that is suggested in the chapter.
Truth tables are an excellent way to get great clarity about complicated combinations of statements.
The idea is to consider every possible combination of True and False for the basic statements.
For example, if there are two statements, $P$ and $Q$, there will be four rows in the table, running through the four possible combinations of True and False for $P$ and $Q$.
On page 21, there is a truth table for three statements, $P$, $Q$, and $R$.
It has eight rows.

The most important use of truth tables is to tell when two complicated combinations of logical expressions are, in fact, the same.

For me, the hardest thing about truth tables is making columns for implications like $P \to Q$.
Here is the best way I know to think about them.
Each row of the truth table for $P$ and $Q$ covers one combination of truth values for $P$ and $Q$.
Some of these combinations are consistent with the implication that $P$ implies $Q$.
For example, when $P$ is True and $Q$ is True, this is consistent with $P \to Q$, so we put $T$ in the $P \to Q$ column.
The row in which $P$ is True and $Q$ is False, however, is inconsistent with the implication $P \to Q$, so we put $F$ in that row.
The cases in which $P$ is False are a bit different, but they are also consistent with $P \to Q$, since $P \to Q$ only has anything to say about $P$ and $Q$ when $P$ is True.
So we put $T$ in those rows too.

{\bf Problems 1 to 7 are good, so please do those.  
Rather than working on problems 9-21, I would much prefer that you spend your time making the truth tables I describe below.
\blist{0.0in}
\item Make a truth table for $\neg (P \vee Q)$ and $\neg P \wedge \neg Q$.
\item Make a big truth table for $P, Q, R,$ $P \wedge (Q \vee R)$, $P \vee (Q \wedge R)$, $(P \wedge Q) \vee (P \wedge R)$, and $(P \vee Q) \wedge (P \vee R)$.  Which of these are equal?  How can you rememeber that?
\elist
}
\vfill          % pad the rest of the page with white space
