\activitytitle{Possible questions for the quiz over nested quantifiers and negation of quantifiers}{Quiz on Thursday, November 16}

This will be a 40--point quiz.
The problems may be chosen from the ones below or from new problems related to that activity or from Chapter 8.
I will be happy to look at your practice solutions in office hours or just before or after class.

\blist{0.1in}
\item I saw this quote in the news today: ``There is no one here that doesn't know that I'm not an angel.''  Please rewrite this with quantifiers and with as few ``nots'' as possible.

\item Suppose you want to prove the statement $P:$ ``for every integer $n > 0$, there are prime numbers $p$ and $q$ with $q = p + 2$.  (The numbers $p$ and $q$ are called twin primes, like 5 and 7 or 11 and 13.)
No one in the world right now knows whether the statement is true or false, but people are trying!

a. Suppose you want to show that $P$ is true.
What kind of proof would you need?  Use the words construct and choose.
Write an outline of the proof, starting where it needs to start, ending where it would need to end. 

b. Suppose you want to show that $P$ is false.  Negate $P$.  Probably best to end with $q \neq p + 2$.
Explain what it would take to prove $\lnot P$, using the words construct and choose.
Write an outline of the proof, starting where it needs to start, ending where it would need to end.

\item For each of the following statements, if the statement is true, prove it with good form.
If it is not true, negate it and prove that the negation is true.
It would be a good idea to be clear about ``choose'' and ``construct''.
If you need to work backwards, or do some scratchwork, fine.
If you need to do a construction, be clear when that happens.

a. For all real numbers $b$, there exists an integer $n$ such that $n \leq b < n+1$.

b. For all real numbers $b > 0$, there exists an integer $n$ such that $n^2 \leq b < (n+1)^2$.

c. For all real numbers $b$, there exists an integer $n$ such that $n^2 \leq b$.

d. For all real numbers $a > 0$, there exists an integer $n$ such that $\frac{1}{\sqrt{n}} < a$.

e. For all real numbers $a > 0$, there exists an integer $n$ such that for all $m > n$, $\frac{1}{\sqrt{m}} < a$.
Approach this carefully using the words ``construct'' and ``choose'' to make sure you do what you need to do.

\item Define $\ln x = \int_{0}^{x} \frac{1}{t} dt$ for all real numbers $x > 0$.  
For all real numbers $x$ and $y$ with $0 < x < y$, show that $\ln x < \ln y$.

\item Show that for all real numbers $y > 0$, there is a real number $x$ such that $x^3 + x + 1 > y$.

\item Show that for all real numbers $y > 0$, there exists a real number $x$ such that $x^3 + x + 1 = y$.
Rather than use a construction, use a graphical argument.

\item Show that for all integers $m < n$, $2^m < 2^n$.

\item Consider the expression ``not all who wander are lost''.  Rewrite it by pushing the ``not'' past the quantifier.
How does the result differ from the statement ``all who wander are not lost''?

\elist
\vfill          % pad the rest of the page with white space
