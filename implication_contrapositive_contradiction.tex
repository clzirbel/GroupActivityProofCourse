\yourname

\activitytitle{Implication, contrapositive, contradiction}{}
\vspace*{-0.2in}

\overview{
A central part of mathematics is dealing with logical statements and showing which statements imply other statements.
There are a few different techniques for that, and we'll need to use them all.
Note: you may have seen this same material presented using truth tables, but this particular activity specifically avoids truth tables.
}

\definition{Logical statement}{A {\em logical statement} is a sentence that is either true or false.  
Sometimes logical statements have an unknows such as $n$, but for each value of $n$, the statement is either true or false.
We often label logical statements with capital letters.}

\example{For each of the logical statements below, label it with its truth value T or F.
If the sentence is not a logical statement, explain why not.

\balist{0.0in}
\item $P$: 18 is even
\item $Q$: 19 is even
\item $R$: 19 is a large number
\item $S$: 13 is prime
\item $T$: $2^5-1$ is prime
\item $U$: $\sqrt{2}$ is rational
\elist
}{0.0in}

\example{For each of the logical statements below, give five values of the integer $n$ for which the statement is true, if possible, and five values of the integer $n$ for which the statement is false, if possible.

\balist{0.0in}
\item $2^n-1$ is prime
\item $n$ is a perfect square
\item $n^2$ is a prime number
\item $n^2 + 3n + 1$ is odd
\elist
}{0.0in}

\definition{Conjunction, logical and}{The {\em conjunction} of two logical statements $P$ and $Q$ is a new logical statement denoted $P \wedge Q$ which is true when both $P$ and $Q$ are true, and false otherwise.
It is usually read as ``and''.}

\definition{Disjunction, logical or}{The {\em disjunction} of two logical statements $P$ and $Q$ is a new logical statement denoted $P \vee Q$ which is true when $P$ is true, when $Q$ is true, or when both are true, but false when both are false.}

\example{Give the truth value T or F of each of the new statements below, using the statements from above.

\balist{0.0in}
\item $P \vee Q$
\item $Q \wedge S$
\item $P \wedge Q \wedge T$
\item $P \vee (Q \wedge U)$
\elist
}{0.0in}

\definition{Negation}{The {\em negation} of a logical statement $P$ is a new statement denoted $\lnot P$ which is true when $P$ is false and false when $P$ is true.
$\lnot P$ is read as ``not $P$''.}

\example{Give the truth value T or F of each of the new statements below.

\balist{0.0in}
\item $\lnot Q$
\item $P \wedge \lnot Q$
\item $Q \vee \lnot S$
\elist
}{0.0in}


\definition{Implication}{For logical statements $P$ and $Q$, we say that $P$ implies $Q$ and write $P \to Q$ if $P$ being true guarantees that $Q$ is true.  
}

\example{For each statement below, identify the statement corresponding to $P$ and the statement corresponding to $Q$ in the implication $P \to Q$.
In every case, $n$ is an integer.

\balist{0.2in}
\item If $n$ is even, this implies that $n^2$ is even.
\item If $n^2$ is odd, then $n$ is odd.
\item If $n$ is odd, then $n^3-n$ is a multiple of 24.
\elist
}{0.0in}

\definition{Direct proof}{A {\em direct proof} of an implication is where we start with the statement $P$ and use the information in it together with a series of valid logical steps to show that $Q$ is true.  This establishes that $P \to Q$.
We have seen a number of direct proofs, including proofs by rewriting.}

\example{Let $n$ be an integer and suppose that $n$ is even.
Show that $n^2+8n+7$ is odd.}{0.6in}

\example{\label{trydirectoddeven}
Let $n$ be an integer and suppose that $n^2+8n+7$ is odd.
Try to write a direct proof that $n$ is even.
If you don't see a way to do it, you can stop trying.
}{0.6in}

\definition{Contrapositive}{When showing $P \to Q$, another way to think about it is that you need to show that you can be sure to avoid the situation where $P$ is true but $Q$ is false.
You can do this by showing that whenever $Q$ is false, $P$ is also false.
In other words, show that $\lnot Q \to \lnot P$.
}

\example{For each implication below, identify $P$ and $Q$ and write out the implication $\lnot Q \to \lnot P$.
In every case, $n$ is an integer.

\balist{0.4in}
\item If $n^2 + 8n + 7$ is odd, then $n$ is even.
\item If $n^2$ is a multiple of 3, then $n$ is a multiple of 3.
\elist
}{0.0in}

\show{Use the contrapositive to show that if $n^2$ is even, then $n$ is even.}{1in}

\show{Use the contrapositive to show that if $n^2 + 8n + 7$ is odd, then $n$ is even.
Once you are done, compare to what you did in \ref{trydirectoddeven}}{1in}

\show{Use the contrapositive to show that if $n^2$ is a multiple of 3, then $n$ is a multiple of 3.}{1in}

\note{{\bf Showing that a statement is false.} Sometimes we want to show that a statement $P$ is false.
Here is a method to do that.
Suppose that $P$ is true, and use rules of algebra, previously-proven results, theorems, etc. to make a series of logical implications $P \to Q$, $Q \to R$, $R \to S$, $S \to T$ until you arrive at a statement that you know to be false.
Suppose, for example, that $T$ is known to be false.
Then you can be certain that $P$ is false.
Be careful when doing this, because you have to be certain that all of the local deductions you make are valid.}

\definition{Rational\label{rational}}{A real number is said to be {\em rational} if it can be written as the quotient of two integers.}

\definition{Irrational\label{irrational}}{A real number is said to be {\em irrational} if it cannot be written as the quotient of two integers.}

\guidedproof{Let $R$ be the statement that $\sqrt{2}$ is rational.
The goal is to show that $R$ is false,  by making a series of deductions that lead to a conclusion that is known to be false.

Suppose that $R$ is true, that is, suppose that $\sqrt{2}$ is rational.

Then there exist integers $p$ and $q$ for which $\sqrt{2} = \frac{p}{q}$, and we can arrange it so that $p$ and $q$ are not both even.  
(If they were both even numbers, we can factor out 2 from each until they are not both even.)

Using algebra, $2q^2 = p^2$.  Thus, $p^2$ is \blank{2in}.
Thus, $p$ is \blank{2in} and so can be written as $p = \blank{1in}$ for some \blank{1in}.

Using algebra, $q^2 = $ \blank{2in}, and so $q^2$ is \blank{1in}.  Thus $q$ is \blank{1in}.

But we know that this is false, because $\blank{3in}$.
Thus, the statement $R$ is false, and thus $\lnot R$ is true, and so $\sqrt{2}$ is irrational.
}

\note{When proving that statement $P$ is false, we will start by writing ``Pretend for a minute that $P$ is true.''
This language is similar to what people use when writing proofs by contradiction, but is intentionally different.
The goal is to have you repeatedly recognize that we are using a chain of known implications $P \to Q$, $Q \to R$, until we arrive at a false statement, which tells us that $P$ is false.
We do not really believe that $P$ is true, we are just following the chain of implications.
}

\prove{Let $L$ be the statement ``There is a largest integer.''
Prove that $L$ is false.

\noindent
Pretend for a minute that $L$ is true.
Write $n$ for the largest integer.
Consider $n+1$.

\vspace{0.7in}

\noindent
Thus, $L$ is false.
}{0in}

\prove{Let $n$ be an integer.
Prove that for $P$: $n$ is even and $Q$: $n$ is odd, the logical statement $P \wedge Q$ is false.

\noindent
Pretend for a minute that $P \wedge Q$ is true.  Then ...

\vspace{0.7in}

\noindent
Thus, $P \wedge Q$ must be false.
}{0in}

\note{We cannot conclude specifically that $P$ is false, nor can we conclude that $Q$ is false, only that the conjunction $P \wedge Q$ is false.
But read on.}

\prove{Let $n$ be an integer and suppose that $n$ is even.
Prove that the statement $Q$: $n$ is odd is false.

\noindent
Pretend for a minute that $Q$ is true.  The argument above leads us to a false statement.
Thus, $Q$ must be false.

\noindent
What is different here is that before we encounter the statement $Q$, we have supposed that $n$ is an integer and $n$ is even, both of which can be true.
In that context, we can see that $Q$ is false.
}{0in}

\note{Consider the statements $P:$ $n$ is an integer and $Q:$ $n^2$ is 6.
It's clear enough that $P \wedge Q$ is false.
Once again, it's hard to say which statement, $P$ or $Q$ is false.
If $n$ is an integer, then $Q: n^2 = 6$ is false.
If $n^2 = 6$, then $n$ is not an integer.
So we have the flexibility to decide ahead of time which of the two we want to suppose to be true, and then show that the other one is false.}

\prove{Let $S$ be the statement that there are finitely many prime numbers.
Show that $S$ is false by filling in the blanks.

\noindent
Pretend for a minute that $S$ is true, so there are only finitely many \blank{2in}.
Let $k$ be how many prime numbers there are, and call the prime numbers $p_1, p_2, p_3, \ldots, p_k$.
Consider the number $n = p_1 p_2 p_3 \cdots p_k + 1$.
Then $n$ is larger than all prime numbers and so $n$ is not \blank{1in}, so it must be \blank{1.5in}.

By considering factors of $n$, at least one factor must be a \blank{1in} number.
But by the \blank{2in} part of the \blank{2in}, $n$ is not a multiple of $p_1$, $n$ is not a multiple of $p_2$, etc.
Thus, $n$ is not a multiple of a prime number.
We have arrived at the false statement that $n$ is composite and yet has no prime factors.
Thus, statement $S$ must be false.
}{0in}

\prove{
Suppose there are 20 children playing musical chairs, with 19 chairs.
Prove that, when the music stops, at least one child will not have a chair to sit on.
Do this by contradiction, showing that the statement $M:$ ``all children have a chair to sit on by themselves'' is false.

\noindent
Pretend for a moment that $M$ is true.
}{0.7in}

\definition{Process of elimination}{Consider logical statements $P, Q$, and $R$ and suppose we know that $P \vee Q \vee R$ is true.
Suppose now that we show that $Q$ is false and $R$ is false.
We can conclude that $P$ is true.
Hopefully that is obviously true.
If not, one can use truth tables to make it extra clear, which is one place where truth tables really help.}

\prove{\label{nsquaredmultipleof5}
Let $n$ be an integer.
Suppose that $n^2$ is a multiple of 5.
Use the Division Algorithm to produce statements $P,$  $Q,$ $R,$ $S,$ and $T$ of the form $n = 5k + r$ for different values of $r$, so that $P \vee Q \vee R \vee S \vee T$ is true.
Then show that $Q$, $R$, $S$, and $T$ are false, and conclude that $P$ is true, so that $n$ is a multiple of 5.
Do a really good job on these cases, because you'll use them a few more times in the next questions.
}{2.2in}

\prove{Let $n$ be an integer.  Consider $A$: $n^2$ is a multiple of 5 and $B$: $n$ is a multiple of 5.
Clearly state $\lnot B$ and $\lnot A$ and then show that $\lnot B$ implies $\lnot A$, to provide a contrapositive proof that $A \to B$.
Use the cases from \ref{nsquaredmultipleof5}.}{0.9in}

\prove{Let $n$ be an integer and suppose that $n^2$ is a multiple of 5.
Consider $B$: $n$ is not a multiple of 5.
Show that $B$ is false by pretending for a minute that $B$ is true.
Use the cases from \ref{nsquaredmultipleof5}.}{0.9in}

\prove{Show that $\sqrt{5}$ is irrational, following the proof that $\sqrt{2}$ is irrational.}{0in}


\vfill          % pad the rest of the page with white space
