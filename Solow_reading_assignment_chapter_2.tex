\readingtitle{Read Chapter \#2, The Forward-Backward Method}

Read Chapter 2 in the book by Daniel Solow, about the forward-backward method for finding a proof that statement $A$ implies statement $B$.

As you read the chapter, write short summaries of each paragraph, so that your notes provide a short version of the ideas in the chapter.
Specific things to do are listed below to help you check them off.

\begin{enumerate}
\item Keep track of the time it takes you to read the chapter.

\item Read Proposition 1 and make sure you understand what it is saying.
Apparently the area of a right triangle is not always equal to the hypotenuse squared divided by 4.

\item When reading Section 2.1, recall our work on vector sums and dot product, where we were able to write down where we need to start, leave a lot of space, and then write down where we need to end.
At first, Solow is working up from the bottom, from statement B back to B1 back to B2

For working backward, he introduces the idea of a ``key question,'' which is a way to put your finger on what is needed to know that $B$ is true.
The key question needs to be specific enough to be helpful to the problem at hand, but general enough that it make sense to someone who is not immersed in the details of the problem.
It may help to think of formulating the key question as an internet search, since most people have experience with writing searches that are not too specific and not too general at the same time.

\item When starting Section 2.2, use some space in your notes to write out A and A1 at the top, leave 10 lines of space, and write out B2, B1, and B at the bottom.
Fill in steps as you read the section.  

\item When reading Section 2.3, use the numbering of Table 2.1 to list out the statements that are made in each of the four proofs of Proposition 1, in the order that they are made.
If a statement is not actually made, don't write the corresponding label.
This will help to illustrate the order in which the proofs are written and what steps are left out.

\item At the end of the chapter there is an illustration of a maze.
Work through the maze from A to B and count how many dead ends there are on the way to B.
Work through the maze from B to A and count how many dead ends there are.
Note that some of the dead ends are different, depending which way you are going.

\item Do exercise 2.5.

\item Do exercise 2.7.

\item Do exercise 2.11.

\item Do exercise 2.14a and 2.15b.

\item Do exercise 2.19.

\item Do exercise 2.24.

\item Do exercise 2.30.

\item Do exercise 2.37, by writing out the steps in order from A to B.

\item At the end, tally up how much time you have spent on this chapter.
Write this number in the upper left corner of your notes.
\end{enumerate}

