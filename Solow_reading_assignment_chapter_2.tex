\activitytitle{Reading assignment \#2}{Due on Tuesday, September 5.  20 points}

Read Chapter 2 in the book by Daniel Solow.
The ``key question'' in this chapter needs to be specific enough to be helpful to the problem at hand, but general enough that it make sense to someone who is not immersed in the details of the problem.
It may help to think of formulating the key question as an internet search, since most people have experience with writing searches that are not too specific and not too general at the same time.

\noindent
{\bf Specific requirements}

\begin{itemize}
\item When reading the discussion of Proposition 1 in Section 1.1, recall our work on vector sums and dot product, where we were able to write down where we need to start, leave a lot of space, and then write down where we need to end.
Write out the statements A, B, B1, B2 in this format, so that A is at the top, then some space, then B2, B1, and B are last.
At first, Solow is working up from the bottom.

\item When reading Section 1.2, write out A, A1, A2, at the top, leave some space, and write out B2, B1, and B at the bottom, to keep track of progress as you read the section.  Stop reading sometimes and really look to see if you can find additional ``A'' statements and ``B'' statements to connect A and B.

\item When seading Section 1.3, use the numbering of Table 2.1 to list out the statements that are made in each of the four proofs of Proposition 1, in the order that they are made.
If a statement is not actually made, don't write the corresponding label.
This will help to illustrate the order in which the proofs are written and what steps are left out.

\item At the end of the chapter there is an illustration of a maze.
Work through the maze from A to B and count how many dead ends there are on the way to B.
Work through the maze from B to A and count how many dead ends there are.
Note that some of the dead ends are different, depending which way you are going.

\item Do exercise 2.5.

\item Do exercise 2.7.

\item Do exercise 2.11.

\item Do exercise 2.14a and 2.15b.

\item Do exercise 2.19.

\item Do exercise 2.24.

\item Do exercise 2.30.

\item Do exercise 2.38.

\item At the end, tally up how much time you have spent on this chapter.
Write this number in your notebook and remember the number when you come to class.
Bring your notebook to class and turn it in for grading.
\end{itemize}

\noindent
{\bf General comments}

The idea is to read Chapter 2 of the textbook by Daniel Solow.
The assignment is to read it in a particular way.
It may take 3 hours to get it done, but you will learn something in those three hours, and you will start to develop a very important skill.

Get a copy of Chapter 2, ``The Forward-Backward Method.''
Get out your notebook or some paper.
Go somewhere quiet, where you won't be interrupted for a while.
Silence your phone so you aren't disturbed.
Don't listen to music that will distract you, and make sure there is no video going where you can see it or hear it.

Put the notebook or paper right in front of you.
Put the textbook itself a bit farther away.
Make note of the time that you start reading in your notebook, maybe in the left margin.
Read the first paragraph of the chapter, then write one or two sentences in your notes which capture the main idea(s) of the paragraph.

Continue reading and writing a sentence summarizing each paragraph.
I believe that if you are not writing, you are probably not thinking as hard as you need to.
Read slowly.
If you run into a word you don't know, look it up.  
If you really don't know it, write the definition in your notebook.
It is OK to spend 15 minutes on each page of the book.  Really.
It is not a goal of the course to learn how to read faster.
The goal is to learn how to get more out of the time you spend reading.
If you stop to take a break, note the time that you stopped and the time you start again so you can calculate the total time.

Note that solutions of the exercises marked with W are available online at 
\url{http://higheredbcs.wiley.com/legacy/college/solow/1118164024/sm/sm.pdf}
Resist the urge to turn your brain off and just read the solutions.  That is not what they are there for!

