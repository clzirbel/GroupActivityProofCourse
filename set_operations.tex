\yourname

\activitytitle{Operations on sets}{This activity works with set identities and relates them to logic.}

\overview{Sets are absolutely fundamental to mathematics.
This chapter focuses on building up set identities, relationships between sets that are always true.}

\problem{Let $A$ and $B$ be sets.  Show that $(A \cup B)^{c} = A^{c} \cap B^{c}$ by showing set inclusion both ways.  The first part is done for you.  This is one of de Morgan's laws.  Draw a really nice Venn diagram to illustrate.\\
\vspace*{-0.3in}
\begin{itemize}
\item Let $x \in (A \cup B)^{c}$.  
Then $x \notin A \cup B$.  
So $x \notin A$ and $x \notin B$.  
That means that $x \in A^{c}$ and $x \in B^{c}$, and so $x \in A^{c} \cap B^{c}$.  
Since $x$ was arbitrary, $(A \cup B)^{c} \subseteq A^{c} \cap B^{c}$.

\item Let $x \in A^{c} \cap B^{c}$.

\end{itemize}}{1in}

\problem{Let $A$ and $B$ be sets.  Show that $(A \cap B)^{c} = A^{c} \cup B^{c}$ by showing set inclusion both ways.  This is the other one of de Morgan's laws.  Draw a really nice Venn diagram to illustrate.}{2in}

\problem{\label{setsaslogic}Let $A$ and $B$ be sets.  Let $P$ be the logical statement $x \in A$, and let $Q$ be the logical statement $x \in B$.
Use $P$ and $Q$ and logic symbols ($\wedge$ for {\em and}, $\vee$ for {\em or}, $\lnot$ for {\em not}) to translate statements about sets into logic statements:
\blist{0.1in}
\item $x \in A \cup B$ is \blank{1.5in}
\item $x \in A^c$ is \blank{1.5in}
\item $x \in B^c$ is \blank{1.5in}
\item $x \in A^c \cap B^c$ is \blank{1.5in}
\item $x \in (A \cup B)^{c}$ is \blank{1.5in}
\elist
\pagebreak

\noindent
In the white space above and to the right, make a truth table for $P$, $Q$, and each of the other logical statements in the previous problem to establish that $x \in (A \cup B)^c$ is logically equivalent to $x \in A^c \cap B^c$.
Compare the truth values in the columns corresponding to $x \in (A \cup B)^c$ to the Venn diagram you made above.  Explain how they agree.
}{0in}

\problem{Let $A$ and $B$ be sets.  Follow the previous exercise to use a truth table to show that $x \in (A \cap B)^{c}$ is logically equivalent to $x \in A^{c} \cup B^{c}$.  Compare the truth table to the Venn diagram again.}{2in}

\problem{Let $D, E,$ and $F$ be sets.
Use one of de Morgan's laws that you showed above to establish that $(D \cup E \cup F)^c = D^c \cap E^c \cap F^c$.
This proof works by rewriting, not by showing inclusion both ways.
{\bf Hint:} Let $A = D \cup E$ and $B = F$.}{1.5in}

\problem{Let $D, E,$ and $F$ be sets.
Use one of de Morgan's laws to show that $(D \cap E \cap F)^c = D^c \cup E^c \cup F^c$.}{1.5in}

\problem{Let $A, B,$ and $C$ be sets.
Use logical statements $P, Q,$ and $R$ and a truth table to show that $x \in A \cup (B \cap C)$ is logically equivalent to $x \in (A \cup B) \cap (A \cup C)$.  Be sure to define $P$, $Q$, and $R$ at the beginning.}{2in}

\problem{Let $A, B,$ and $C$ be sets.
Show that $A \cup (B \cap C) = (A \cup B) \cap (A \cup C)$ by showing inclusion both ways.
When you encounter a union, use a proof by cases.
For example, if you know that $x \in A \cup B$, one case is that $x \in A$, the other is that $x$ is not in $A$, but $x \in B$.
Organize your writing carefully to make the steps of this argument really clear.}{3in}

\definition{Set difference}{Let $A$ and $B$ be sets.  The {\em set difference} $A \backslash B$ is the set $A \cap B^{c}$, which is all points that are in $A$ but not in $B$.  Draw a Venn diagram to illustrate this definition.}

\definition{Symmetric difference}{Let $A$ and $B$ be sets.  The {\em symmetric difference} of $A$ and $B$ is the set $A \bigtriangleup B = (A \backslash B) \cup (B \backslash A)$.  Draw a Venn diagram to illustrate this definition.}

\problem{Consider again the logical statements from \ref{setsaslogic}.  Write a logical statement that is equivalent to $x \in A \bigtriangleup B$.
Make a truth table with 4 rows, labeled 1, 2, 3, 4, and three columns, one for $x \in A$, one for $x \in B$, and the third for $x \in A \bigtriangleup B$.
Draw a Venn diagram and label the regions in it 1, 2, 3, 4 so that they correspond to the truth table.}{1in}

\problem{Let $A$ and $B$ be sets.  Show that $A \bigtriangleup B = B \bigtriangleup A$ by showing set inclusion both ways.  Draw a nice Venn diagram to illustrate.}{2in}

\problem{Let $A, B,$ and $C$ be sets.  Show that $(A \bigtriangleup B) \bigtriangleup C = A \bigtriangleup (B \bigtriangleup C)$ in three ways.
\blist{0.1in}
\item Draw separate Venn diagrams for the two sets.
\item Show set inclusion both ways.
\item Convert inclusion in $A \bigtriangleup B$, $B \bigtriangleup C$, and other sets to logical statements and use a truth table to show the equality.
\elist
}{0in}

\vfill          % pad the rest of the page with white space
