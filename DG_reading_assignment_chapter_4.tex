\activitytitle{Reading assignment, Chapter 4}{Due in the third week of classes.}

Read and understand Chapter 4 of the textbook by Daepp and Gorkin.
As with previous chapters,
{\small
\blist{0.0in}
\item Read somewhere quiet, minimizing distractions from phones and friends
\item Note the time that you start and stop reading, and add up the minutes
\item Read with a pencil in your hand and your notebook open in front of you
\item Write a sentence to summarize each paragraph, re--draw diagrams, work out examples and exercises on your own
\item Look up words you don't know, and write down ones you really don't know
\item Read slowly.  You are not reading a comic book or a newspaper.  It is not a goal of this class for you to learn how to read faster.  The goal is to learn how to get more out of the time you spend reading, and to learn to concentrate for longer periods of time.
\item At the end, tally up how much time you have spent on reading this chapter.
Write this number in your notebook and remember the number when you come to class.
\elist
}

This is a very important chapter, and one with real substance.  Hopefully you will feel that way when you read it, and will enjoy it more as a result.

This chapter has a large number of very dense expressions involving quantifiers, implications, and logical operators.  Slow way down when you run into one of them.  Pick them apart in your mind and then write them down so they are crystal clear.  Every symbol is important.  It's a bit like when you're reading someone your credit card number or you're giving your phone number to someone you really want to call you.  Every symbol is important.

Exercises 4.1, 4.2, 4.3, and 4.6 are all useful to do.
The discussion that begins at the bottom of page 36 is very important, negating statements with quantifiers.

There are 20 problems.  The more of them you do, the better, of course, but you may not be able to work through all of them.  {\bf Please at least do problems \# 1--7, and 20.}  Read \# 11.  Does this joke work on your friends?

People have asked about grading, or about a rubric that Ying-Ju Chen is using when she reads the notebook.
She'll be assigning numeric values between 0 and 5 for each chapter.
The bulk of the points go toward the notes on the chapter itself.
This is to emphasize that reading and taking notes is the primary concern.
Less than half of the points go toward attempting the exercises, with more emphasis on attempting than on getting them all the way right.
Ying-Ju writes as many helpful comments as she can on each notebook, but there is only so much time, and sometimes things that are incorrect do not get marked as incorrect.
Even so, I think that having you take notes and having Ying-Ju read them every class is working very well.

Pay attention to the phrase ``only if.'' It is often used in a way that can be confusing.  Compare these two statements for example, in which $R$ means Race and $P$ means prize:
\blist{0.0in}
\item I will race if there is a prize offered.  $P \to R$.  This is the most common way that people use the word ``if.''  The prize will make me race.
\item I will race only if there is a prize offered.  $R \to P$.  People say this sort of thing pretty often too, but it's a bit less clear unless you think about it carefully.  Part of the problem is the time order in which things happen, because the racing comes {\em after} the prize is offered.  ``If you see me racing, you can be sure that there was a prize offered. (But offering a prize is no guarantee that I will race.)''
\elist



\vfill          % pad the rest of the page with white space
