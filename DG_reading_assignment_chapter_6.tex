\activitytitle{Reading assignment, Chapter 6}{Due in the fifth week of class.}

Read and understand Chapter 6 of the textbook by Daepp and Gorkin.
As with previous chapters,
{\small
\blist{0.0in}
\item Read somewhere quiet, minimizing distractions from phones and friends
\item Note the time that you start and stop reading, and add up the minutes
\item Read with a pencil in your hand and your notebook open in front of you
\item Write a sentence to summarize each paragraph, re--draw diagrams, work out examples and exercises on your own
\item Look up words you don't know, and write down ones you really don't know
\item Read slowly.  You are not reading a comic book or a newspaper.  It is not a goal of this class for you to learn how to read faster.  The goal is to learn how to get more out of the time you spend reading, and to learn to concentrate for longer periods of time.
\item At the end, tally up how much time you have spent on reading this chapter.
Write this number in your notebook and remember the number when you come to class.
\elist
}

This chapter introduces sets, subsets, equality of sets, and how to tell what the members of a set are.
As you read, take time to write out several members of each set that is introduced.
Note that $A$ being a subset of $B$ is the same as the logical implication $x \in A$ implies $x \in B$.
There is a tight connection between statements in set theory and logical statements.
Here is another:  Set $A$ being equal to set $B$ is the same as the logical implications $x \in A$ if and only if $x \in B$.

There are many examples in this chapter.
Work through them by rewriting them and adding useful steps in your notes.

On page 64, intersections, unions, and complements of sets are introduced.
As you read about them, explain in your notes how these relate logical statements such as $x \in A$ and $x \in B$ to $x \in A \cap B$.

You may enjoy reading about the paradoxes on page 67.
Give them a try.
Even if they are not your cup of tea, try to see what the issue is.

Problems 1 -- 9 are essential.  Do them.

Problem 10 is a good thought problem.  Think about it.

Starting with Problem 11, there are things for you to prove.
I would be happy to see you do some of these by yourself.
We will do these problems in class, but I'd like us to move through them fairly quickly, so have a look at them before class.

\vfill          % pad the rest of the page with white space
