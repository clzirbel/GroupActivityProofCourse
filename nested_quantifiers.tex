\yourname

\activitytitle{Nested quantifiers}{}

\overview{Quantification is a key part of mathematics.
The phrases ``for all'' and ``there exists'' can be combined in different ways to express important relationships between concepts.
Note that the order of the two is very important.}

\definition{For all}{The phrase {\em for all} means that what follows is supposed to be true for many cases, and will probably require a generic proof to cover all cases, or split all cases into a few sub-cases.
Alternative words are ``for each.''
Sometimes people write ``for any'' but please avoid that because it can be ambiguous, as we will see below.
The notation $\forall$ is often used to represent ``for all''.}

\definition{There exists}{The phrase {\em there exists} claims that what follows can be shown to exist.
Often, you prove this by doing a construction of the object that is needed, but occasionally the proof works differently.
The notation $\exists$ is often used to represent ``there exists.''}

\definition{Let}{The word ``let'' has two distinct mathematical meanings which will become apparent in the proofs you write below.
First, when you want to introduce some generic new objects which have some particular property but you want to make no further assumptions about them, you can say something like ``Let $r$ be a rational number.''
This calls into existence a new object $r$ and all you know about it is that it is a rational number.
Second, when you want to construct a new object with a specific value, you also use the word ``let.''
For example, ``Let $x=\sqrt{2}$.''
}

\prove{Show that for all odd integers $m$ and $n$, there exists an integer $p$ such that $m+n = 2p$.
Start your proof by writing ``Let $m$ and $n$ be odd integers.'' After that, be sure to construct the integer $p$.
It's OK to define $p$ after you know what value it needs to have.
At the end, generalize by noting that you made no further assumptions about $m$ and $n$.}{1in}

\prove{Show that for every rational number $r$, there exists a rational number $s$ such that $rs = 1$.
Remember to generalize at the end.
Is there more than one possible value for $s$?}{1.2in}

\prove{Show that for every real number $y$ there is a value of $x$ for which $x^3 = y$.
Is there more than one possible value for $x$?}{1.2in}

\prove{(Calculus required.)  Show that for every continuous function $f:\R \to \R$, there exists a function $F$ with $F' = f$ and $F(0) = 0$.
Is there more than one possible choice for $F$?}{1.2in}

\prove{(Linear algebra required.)  Suppose that the 3 by 3 matrix $A$ is invertible.
Show that for all 3-dimensional vectors $b$, the equation $Ax = b$ has a solution $x$.
Is there more than one possible value for $x$?}{1.2in}


\exercise{~
\balist{0.5in}
\item Show that for every integer $p$, there is an integer $n$ with $2^n > p$.
\item Using the integer $n$ that you constructed, show that for all $m > n$, we have $2^m > p$.
\elist
}{0.3in}

\exercise{~
\balist{0.5in}
\item Show that for every real number $a > 0$, there is an integer $n$ with $\frac{1}{n^2 + 10} < a$.
\item Using the integer $n$ that you constructed, show that for all $m > n$, we have $\frac{1}{m^2+10}$.
\elist
}{0.3in}

\exercise{~
\balist{0.5in}
\item Show that for every real number $a > 0$, there is an integer $n$ with $\frac{1}{n^2 - 100} < a$.
\item Using the integer $n$ that you constructed, show that for all $m > n$, we have $\frac{1}{m^2 - 100}$.
\elist
}{0.3in}

\exercise{Write the following statements symbolically.
Introduce new notation as you need it.
The first one is done for you.
\balist{0.5in}
\item Between every two locations in the US, there is a shortest driving route.  Use the variables $L_1, L_2,$ and $r$.\\
$\forall$ locations $L_1$ and $L_2$, $\exists$ route $r$ such that $r$ starts at $L_1$ and ends at $L_2$.\\
\vspace*{-0.5in}
\item Every rose has a thorn.  Use the variables $r$ and $t$.
\item Every broken (analog) clock is right twice a day.  Use the variables $c, t_1, t_2$.
\item Every married couple with a child gets a tax deduction.  Use variables $m, c$, and $d$.
\elist
}{0.1in}

\exercise{
Write the following statements symbolically:
\balist{0.4in}
\item For every $a$, there is a $b$ for which $b^2 = a$
\item For every $b$, there is an $a$ for which $b^2 = a$
\item For every $a$ and every $b$, it is the case that $b^2 = a$
\item There exists an $a$ and there exists a $b$ such that $b^2 = a$
\elist
}{0.1in}

\exercise{Which of the statements in the previous problem are true if the universe for both $a$ and $b$ is the set of non--negative integers?
If not true, explain why not.
\balist{0.2in}
\item
\item
\item
\item
\elist
}{0.1in}

\definition{Suppose}{The word ``suppose'' is  used to restrict a generic object to have one more property.
For example, you might want to consider a generic integer, but you might want your proof to break into two cases.
So you might write, ``Let $n$ be an integer.  First, suppose that $n \geq 0$.''  Then you write a proof that covers this first case.  Later, you can write ``Second, suppose that $n < 0$.''  This is useful when the proof is different for the case $n \geq 0$ than it is for $n < 0$.}

\exercise{Explain how existence in the Division Algorithm is covered by the previous idea of using ``let'' and ``suppose.''}{0.1in}


\exercise{
Show that there exists a number $a$ such that for all $x \in \R$, $5\sin(x) + 7\cos(3x) < a$.
Note that in this problem, you first construct $a$ and then you show a ``for all'' statement.
}{0.1in}

\exercise{
Show that for the integers, there exists a number $a$ for which, for all integers $b$, $ab = b$.
}{0.1in}



\vfill          % pad the rest of the page with white space
