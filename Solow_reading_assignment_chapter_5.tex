\activitytitle{Reading assignment \#5}{Due on Tuesday, October 3.  20 points}

Read Chapter 5 in the book by Daniel Solow.
It is about showing that something happens ``for all'' objects with a certain property.
We have already done a number of proofs of this general form.

Pay particular attention to the beginning of the chapter, about set theory.
We will soon start doing set theory activities in class.

\vspace{0.1in}
\noindent
{\bf Specific requirements}
\vspace*{-0.15in}

\begin{itemize}
\item Read Section 5.1 slowly and make sure to think through every sentence.
There is a lot of new content in just a few pages.
Set theory is super important, and this is a very nice introduction to certain aspects of it that we will spend a lot of time with.
Make sure that your notes reflect the time you spend and your understanding of the material.

\item Read Section 5.2.  It has an extended discussion of using the forward-backward method to do a proof.
After you have read it, write out the statements in order and using the labels {\bf A, A1, A2, \ldots, B6, B5, \ldots B2} so that it is clear that you understand exactly how the proof works.
I think this will help make it clearer to you, also.
Because only half of the proof is being done here, start with the definitions of sets $S$ and $T$, which is part {\bf A} of the proof, write {\bf A1:} as ``Let $x$ be an element of $S$.'' and end with {\bf B2:} $S$ is a subset of $T$.
In {\bf A1:}, the word ``Let'' means that a specific object is being brought into existence, with a specific property, for you to work with.

\item Do exercise 5.2.

\item Do exercise 5.6.

\item Do exercise 5.7.

\item Do exercise 5.14.  Following the chapter, first identify the objects, the certain property, and the something that happens in the for-all statements.  Then do a nice job explaining what is right or wrong about a, b, c, d, and e.

\item At the end, tally up how much time you have spent on this chapter.
Write this number in your notebook.
Bring your notebook to class and turn it in for grading.
\end{itemize}

\noindent
{\bf General comments}

Set yourself up in a place where you won't be disturbed.
Read slowly, and write notes in your own words that reflect your understanding of the material.
