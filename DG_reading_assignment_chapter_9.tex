\activitytitle{Reading assignment, Chapter 9}{Due in the tenth week of class.}

Read and understand Chapter 9 of the textbook by Daepp and Gorkin, called ``The Power Set and the Cartesian Product.''
This is the last chapter on plain set theory.
It should stretch your mind in a few new directions.
Prepare to move slowly and think carefully.

When $A$ is a set, the power set of $A$ is the collection of all subsets of $A$.  Read Example 9.1 and do Exercise 9.3 and then {\bf do Problem 9.1.}
Work through Exercise 9.2 and then {\bf do Problem 9.2.}
Problem 9.2 is hard, but excellent for you.  Take it very slowly.
Work through Exercise 9.4 and then {\bf do Problem 9.5.}
{\bf Do Problem 9.8.}

{\bf Do Problem 9.11.}  For 9.11, you have already seen the power set of a set containing 2 elements and 3 elements.  {\bf Hint:}  When you are making a subset of a set $A$, for each element of $A$, you have to decide whether it goes in or out of the subset.  There are two choices (in or out) each time.  If the hint doesn't help you, write out the power set of $\{1, 2, 3, 4\}$, then read the hint again.  Hopefully you don't have to write out the power set of $\{1, 2, 3, 4, 5\}!$

You are already very familiar with one Cartesian product:  making ordered pairs $(x,y)$ of real numbers is the Cartesian product $\R \times \R$, which you know better as the $xy$ plane.
Every problem involving Cartesian products of sets containing real numbers can be depicted as points in the $xy$ plane.
Make a graph in every case.
This will help your intuition.
When there are only finitely many points, like with $\{0,1\} \times \{ 2,3\}$, also list out all of the $(x,y)$ pairs.

{\bf Answer these questions:  Who is the Cartesian product named after?  Why, exactly?}

Work through Exercise 9.5 a, b, e.

{\bf For Theorem 9.7, draw $A$ and $C$ as intervals on the $x$ axis and draw $B$ and $D$ as intervals on the $y$ axis, then draw out the sets in the statement of the theorem on two separate sets of axes.}
Make sure you are crystal clear about what these sets are, and you will be close to mastering Cartesian products.

{\bf Do Problem 9.12.}  It connects Cartesian products to things you learned in geometry.

{\bf Do Problem 9.17a.}  Notice that this is an ``if and only if'' proof, and it has three set equalities to show.
Suppose that $A \times B = C \times D$ and show that $A = C$ and $B = D$ by showing containment each way.
Here is one part of the argument:  Let $x \in A$.  Also let $y \in B$.  Then $(x,y) \in A \times B = C \times D$, and so $x \in C$.  Thus $A \subseteq C$.
After that part is done, suppose that $A=C$ and $B=D$ and argue that $A \times B = C \times D$.

{\bf Think about Problem 9.19.}

\noindent As with the previous chapters,
\blist{0.0in}
\item Read somewhere quiet, minimizing distractions from phones and friends
\item Note the time that you start and stop reading, and add up the minutes
\item Read with a pencil in your hand and your notebook open in front of you
\item Write a sentence to summarize each paragraph, re-draw diagrams, work out examples and exercises on your own
\item Look up words you don't know, and write down ones you really don't know
\item Read slowly. 
\item At the end, tally up how much time you have spent on reading this chapter.
Write this number in your notebook and remember the number when you come to class.
\elist

\vfill          % pad the rest of the page with white space
