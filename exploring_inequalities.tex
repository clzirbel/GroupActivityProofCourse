\yourname

\activitytitle{Exploring inequalities}{}
\vspace*{-0.2in}

\overview{In this activity you will explore properties of inequalities, but without proving the inequalities.  The point here is to use examples and counterexamples to sharpen your intuition about inequalities and their properties.
Be adventurous when you look for counterexamples.
If you find a counterexample, put a box around it.
If the conclusion about inequalities seems to be correct, put a big check mark next to it.}

\question{Is the statement $7 \leq 7$ true?  Explain.}{0.0in}

\question{Is the statement $7 < 7$ true?  Explain.}{0.0in}

\question{Is the statement $6 \leq 7$ true?  Explain.}{0.0in}

\question{Suppose $a < 7$.  Can you conclude that $a \leq 7$?  This is counterintuitive for many people.
Writing $a \leq 7$ does not mean, ``I certify that $a$ really could equal 7.''  Instead, it means more like ``I certify that the value of $a$ can be at most 7'' or that ``$a > 7$ is false.''  Again, is it also true that $a \leq 7$?}{0.0in}

\question{Suppose $a \leq 7$.  Can you be certain that $a < 7$?  A good technique is to write down five numbers satisfying $a \leq 7$ and see if they also satisfy $a < 7$.  Try to find a counterexample.  If you find one, put a box around it, otherwise put a check mark.}{0.0in}

\question{Suppose $a < b$ and $b \leq c$.  Is it guaranteed that $a < c$?  Work with examples if it helps, and look for a counterexample.}{0.0in}

\question{Suppose $a < b$ and $b \leq c$.  Is it guaranteed that $a \leq c$?  Work with examples if it helps, and look for a counterexample.}{0.0in}

\question{Suppose $a \leq b$ and $b \leq c$.  Is it guaranteed that $a < c$?  Work with examples if it helps, and look for a counterexample.}{0.0in}

\question{Suppose $a > 12$.  Consider the inequality $-a > -12$.  Write down five numbers satisfying $a > 12$ and check whether or not they satisfy $-a > -12$.  Look for a counterexample.  If you find a counterexample, put a box around it.  If the result is OK, put a check mark.}{0.0in}

\question{Suppose $a > 12$.  Consider the inequality $-a < -12$.  Write down five numbers satisfying $a > 12$ and check whether or not they satisfy $-a < -12$.  If you find a counterexample, put a box around it.}{0.0in}

\question{Suppose $c < 5$.  Use examples to check whether $c^2 < 25$.  If you find a counterexample, put a box around it and consider whether an additional condition on $c$ would guarantee $c^2 < 25$.}{0.0in}

\pagebreak

\question{Suppose $c < 7$ and $d \leq 8$.
Use examples to check whether $c + d < 15$.
If you find a counterexample, put a box around it and consider whether an additional condition on $c$ and $d$ would guarantee $c + d < 15$.}{0.1in}

\question{Suppose $c < 3$ and $d \leq 4$.
Use examples to check whether $cd < 12$.
If you find a counterexample, put a box around it and consider whether an additional condition on $c$ and $d$ would guarantee $cd < 12$.}{0.1in}

\question{Suppose $a \leq b$ and $c \geq 0$.
Use examples to check whether $ac \leq bc$, as above.
}{0.1in}


\question{Suppose $a \leq b$ and $c \leq d$.
Use examples to check whether $a+c < b+d$, as above.
If you find a counterexample, put a box around it and consider whether an additional condition would guarantee $a+c < b+d$.}{0.1in}

\question{Suppose $a < b$ and $c \leq d$.
Use examples to check whether $ac < bd$.
If you find a counterexample, put a box around it and consider whether an additional condition would guarantee $ac < bd$.}{0.2in}

\question{Suppose $a \leq b$.
Use examples to check whether $a^2 < b^2$.
If you find a counterexample, put a box around it and consider whether an additional condition would guarantee $a^2 < b^2$.}{0in}

\show{If $1 < p$, cite a general result from above to conclude that $5 < 5p$.  }{0.2in}

\show{Suppose $p$ is an integer with $-5 < 5p < 5$.
Without dividing by 5, check whether or not it is possible that $p=1$, $p>1$, $p=-1$, $p<-1$.
Conclude that $p = 0$.
If you use properties of inequalities, cite the number from above that you are using.
}{0.5in}

\question{Find an integer $n>0$ for which $\frac{1}{n} < 0.1$.}{0in}

\question{Find the smallest integer $n>0$ for which $\frac{1}{n} < 0.03$.}{0in}

\question{Find the smallest integer $n>0$ for which $\frac{1}{n} < 0.0002$.}{0in}

\question{Let $\varepsilon > 0$.  Describe a procedure for finding the smallest integer $n>0$ for which $\frac{1}{n} < \varepsilon$.}{0in}

\question{Suppose $a < b$.
Use examples to check whether $\frac{1}{a} < \frac{1}{b}$.
If you find a counterexample, put a box around it and consider whether an additional condition would guarantee $\frac{1}{a} < \frac{1}{b}$.}{0in}

\question{Suppose $a \leq b$.
Use examples to check whether $\frac{1}{a} \geq \frac{1}{b}$.
If you find a counterexample, put a box around it and consider whether an additional condition would guarantee $\frac{1}{a} \geq \frac{1}{b}$.}{0in}


\vfill          % pad the rest of the page with white space
