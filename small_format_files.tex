\documentclass[a6paper]{article}
\pagestyle{empty}
\usepackage[margin=2mm]{geometry}
\usepackage{tgheros}
\usepackage[T1]{fontenc}
\renewcommand*\familydefault{\sfdefault}

\usepackage{amsmath}    % allows AMS math things like the cases environment
\usepackage{amssymb}    % allows the use of AMS symbols like blackboard bold
\usepackage{amsthm}     % allows AMS definitions for theorems, proofs, etc.
\theoremstyle{definition}  % use roman font in theorems, definitions, etc.

\newtheorem{theorem}{\bf Theorem}                  % Allow numbered theorems
\newtheorem{proposition}[theorem]{\bf Proposition} % Number props with theorems
\newtheorem{remark}[theorem]{\bf Remark}           % Number remarks with theorems
\newtheorem{definitionX}[theorem]{{\bf Definition}}
\newtheorem{notationX}[theorem]{\bf Notation}
\newtheorem{exampleX}[theorem]{\bf Example}
\newtheorem{problemX}[theorem]{\bf Problem}
\newtheorem{showX}[theorem]{\bf Show}
\newtheorem{recallX}[theorem]{\bf Recall}
\newtheorem{noteX}[theorem]{\bf Note}
\newtheorem{guidedproofX}[theorem]{\bf Guided proof}
\newtheorem{proveX}[theorem]{\bf Prove}
\newtheorem{groupworkX}[theorem]{\bf Group work}
\newtheorem{questionX}[theorem]{\bf Question}
\newtheorem{challengeX}[theorem]{\bf Challenge}

%\newtheorem{theorem}[equation]{\bf Theorem} % this would number theorems
                                             % with equations, as some people like
%\numberwithin{equation}{section}  % equation 3 in section 2 is (2.3)

% ------------------------------------------ new commands

\newcommand{\coursenumber}{Math (insert course number here)}
\newcommand{\coursename}{Mathematical Foundations and Techniques}

\newcommand{\yourname}{\hfill {\bf Your name: \underline{\hspace{2.5in}}}
%\vspace*{-0.1in}
}

\newcommand{\anonymous}{\hfill {\bf Anonymous!}}

\newcommand{\blank}[1]{\underline{\hspace{#1}}}

\newcommand{\activitytitle}[2]{\noindent {\bf \LARGE #1}\\\noindent #2\vspace*{0.1in}}

\newcommand{\overview}[1]{\noindent{\bf \Large Overview}\\\framebox{\parbox{7in}{#1}}}

\newcommand{\definition}[2]{\begin{definitionX}{\bf #1.} #2\end{definitionX}}
\newcommand{\notation}[1]{\begin{notationX}#1\end{notationX}}
\newcommand{\example}[2]{\begin{exampleX}#1\end{exampleX}\vspace*{#2}}
\newcommand{\problem}[2]{\begin{problemX}#1\end{problemX}\vspace*{#2}}
\renewcommand{\show}[2]{\begin{showX}#1\end{showX}\vspace*{#2}}
\newcommand{\recall}[2]{\begin{recallX}#1\end{recallX}\vspace*{#2}}
\newcommand{\note}[1]{\begin{noteX}#1\end{noteX}}
\renewcommand{\stop}[1]{\noindent{\bf \Large \underline{Stop.}} #1}
\newcommand{\Hint}{{\bf Hint}}
\newcommand{\guidedproof}[1]{\begin{guidedproofX}#1\end{guidedproofX}}
\newcommand{\prove}[2]{\begin{proveX}#1\end{proveX}\vspace*{#2}}
\newcommand{\groupwork}[2]{\begin{groupworkX}#1\end{groupworkX}\vspace*{#2}}
\newcommand{\question}[2]{\begin{questionX}#1\end{questionX}\vspace*{#2}}
\newcommand{\challenge}[2]{\begin{challengeX}#1\end{challengeX}\vspace*{#2}}

% ----------------------------- Facilitate repeating things with no number (NN)

\newcommand{\definitionNN}[2]{\noindent{\bf Definition. #1.} #2}
\newcommand{\notationNN}[1]{\begin{notationX}#1\end{notationX}}
\newcommand{\exampleNN}[2]{\noindent{\bf Example.} #1 \\ \vspace*{#2}} \newcommand{\showNN}[2]{\noindent{\bf Show.} #1 \\ \vspace*{#2}}
\newcommand{\recallNN}[2]{\noindent{\bf Recall.} #1 \\ \vspace*{#2}}
\newcommand{\noteNN}[1]{\begin{noteX}#1\end{noteX}}
\newcommand{\HintNN}{{\bf Hint}}
\newcommand{\guidedproofNN}[1]{\begin{guidedproofX}#1\end{guidedproofX}}
\newcommand{\proveNN}[2]{\begin{proveX}#1\end{proveX}\vspace*{#2}}
\newcommand{\groupworkNN}[2]{\begin{groupworkX}#1\end{groupworkX}\vspace*{#2}}
\newcommand{\questionNN}[2]{\begin{questionX}#1\end{questionX}\vspace*{#2}}

\newcommand{\blist}[1]{\begin{list}{{\bf \arabic{enumi}.}}{\usecounter{enumi}\setlength{\itemsep}{#1}}} 
                                     % begin a numbered list.  The optional
                                     % argument is the spacing between items
\newcommand{\elist}{\end{list}}      % end the list

\newcommand{\balist}[1]{\begin{list}{{\bf \alph{enumii}.}}{\usecounter{enumii}\setlength{\itemsep}{#1}}} 
                                     % begin a numbered list.  The optional
                                     % argument is the spacing between items
\newcommand{\ealist}{\end{list}}      % end the list

\newcommand{\vect}[1]{\langle #1 \rangle}   % angle brackets for a vector
\newcommand{\tvec}[1]{\vect{#1_1, #1_2, #1_3}}
\newcommand{\qq}{\quad\quad}
\newcommand{\qqq}{\quad\quad\quad}
\newcommand{\qqqq}{\quad\quad\quad\quad}

\newcommand{\R}{\mathbb{R}}
\newcommand{\Z}{\mathbb{Z}}
\newcommand{\Q}{\mathbb{Q}}
\newcommand{\N}{\mathbb{N}}
\newcommand{\Rp}{\R^{+}}

\newcommand{\DGreference}{\footnote{Reading, Writing, and Proving: A Closer Look at Mathematics, 2011, by Ulrich Daepp and Pamela Gorkin}}

% --------------------------------------------- what to process

\includeonly{DG_reading_assignment_chapter_18_2015}

% --------------------------------------------------- begin document
\begin{document}        % What came before is the 'preamble'

\yourname

\activitytitle{Syllabus}{This syllabus is an assignment for you to read and respond to.
Please read it carefully, fill in answers, put your name on it, and turn it in on the second day of class.}

\noindent {\bf Course description.}
There are two main goals for the course:
\blist{0.1in}
\item Improving your ability to work with definitions, examples, counterexamples, claims, and proofs.
\item Improving your ability to read a mathematics textbook on your own.
\elist
My hope is that your new abilities in these two areas will make you unstoppable in your mathematics classes.
We will spend most of class time on \#1.  
I will design activities for us to do together in class for this purpose.
Most of your time outside of class will be spent on \#2.
Being able to read mathematics on your own is a fantastic skill.
Be sure to set aside quiet time to read the textbook.

\noindent {\bf Q: How comfortable are you already with the ``definition, example, theorem, proof'' sequence in mathematics classes?}

\vspace*{1in}

\noindent {\bf Q: What kinds of experiences have you had in the past with proofs?}

\vspace*{1in}

\noindent {\bf Professor and contact information.}  

\noindent {\bf Q: Do you check your university email regularly?}

\vspace*{0.5in}

\noindent {\bf Schedule.}

\vspace*{0.5in}

\noindent {\bf Office hours.}
You are welcome to visit me in my office, which is room XXX in the mathematics building.
The best way to arrange a time to meet is to send an email listing a few times that would work for you.
I will reply with one that works for me as well.

\noindent {\bf (Sample question to get students to find my office) Q: If you take the elevator to the fourth floor, do you turn right or left to get to my office?}

\vspace*{0.5in}

\noindent {\bf (Sample question to get students to find my office) Q: What are the two flyers on my door about?}

\vspace*{0.5in}

\noindent {\bf Textbook.}  (A suggested textbook; reading assignments are given in these materials for this book.)  
The textbook for the course is {\em Reading, writing and proving, A closer look at mathematics}, second edition, by Ulrich Daepp and Pamela Gorkin.  2011.  The textbook is very good, but not perfect.  You can learn from it, especially if you take time to read it.

\noindent {\bf Q: Have you ever had success reading a mathematics textbook and really learning from it?  If so, please tell what book, what course, and what made it work.  If not, please tell me what you think prevented you from being able to read the book.}

\vspace*{1in}

\noindent {\bf Q: Do you have a hard copy of the textbook that you can read?  Have you been able to get a PDF file of the first chapter from the library?}

\vspace*{0.5in}

\noindent {\bf Graduate assistant.} 

\noindent {\bf Q: Do you have any interest in going to graduate school?  Please explain.}

\vspace*{0.5in}

\noindent {\bf Coursework.} Here are the main things that you will be doing:
\blist{0.1in}
\item Written work on in--class activities.
\item Taking notes on each chapter in the textbook in your notebook.  Bring your notebook to class so that the graduate assistant can read through it in class and give it back.
\item Occasional quizzes (very much like the work you'll already be doing in class) instead of one or two big exams
\item A final exam (which should be very similar to what we have been doing all semester long)
\elist

\noindent {\bf Q: Do you have any questions or concerns about the coursework?}

\vspace*{1in}

\noindent {\bf Grading.}  My general plan is this.  Many things you do during the semester will have a point value attached to them.  The number of points will indicate their relative importance to your grade.  In--class work and homework will count for a larger share than in most courses, while quizzes and exams will count for a lower share.  I will announce the relative percentages at least two weeks before the first exam.

Reading assignments will be assigned numeric values between 0 and 10 for each chapter.
The bulk of the points go toward the notes on the chapter itself.
This is to emphasize that reading and taking notes is the primary concern.
Less than half of the points go toward attempting the exercises, with more emphasis on attempting than on getting them all the way right.
I will write as many helpful comments as we can on each notebook, but there is only so much time, and sometimes things that are incorrect do not get marked as incorrect.

\noindent {\bf Q: Do you have any questions or concerns about the grading?}

\vspace*{1in}

\noindent {\bf Attendance.} Attendance and class participation will be vitally important.
Class time is the best time to make attempts and get immediate feedback.
If you cannot attend a class, notify me as soon as possible by email or phone, before class if possible.
Don't even imagine that you can miss a class without letting me know.
I don't particularly need to know {\bf why}, but I do need to know.

\noindent {\bf Q: What is the most likely reason that you will miss class?  I'm just curious.}

\noindent {\bf Background questions.}
\blist{0.8in}
\item What mathematics courses are you taking this semester?
It's OK to just list the numbers, like Math 3410.

\item What mathematics courses have you already taken here?

\item Including this one, how many semesters until you graduate?

\item Please let me know anything you think I should know about you.  I'll read it all.  Sometimes people like to tell about their hobbies, movies they like, where they're from, etc.

\elist

\vfill          % pad the rest of the page with white space

\activitytitle{Homework problems, week 12}{Due on (put date here).}

Write up solutions of each of the problems below.
They are designed to be straightforward problems.
The goal is to come as close to perfection in your solutions as you can.
\begin{itemize} \itemsep 1pt
\item Do not take shortcuts.
\item If you need to show that something is true for all $n$, or for all $x,y$, start the proof with ``Let \ldots''
\item If you need cases, explain what the cases are and why they cover all the possibilities.
\item If you are doing a proof by contradiction, start that part by saying ``Assume \ldots''
\item If you are doing a proof by contrapositive, tell what $P$ and $Q$ are, and that you will be showing that $\lnot Q$ implies $\lnot P$.
\item Take small steps in each proof, and explain each step.
\item Follow good form.
\item If your proof started with ``Let \ldots'' it will probably end by saying ``We made no further assumption \ldots''
\end{itemize}
Here are the problems to do.  You can write them in your notebook or on separate paper.
\blist{0.1in}
\item Show that if $n$ is an integer and $7n$ is odd, then $n$ is odd.
{\bf Hint:} Be clear what facts you are using about even and odd numbers.

\item Without consulting your book or your notes, prove that $\sqrt{2}$ is irrational.
I mean it.  
Do this from memory.
You should be able to write a very nice proof, with no missing steps.

\item Let $x$ and $y$ be real numbers, and suppose that the product $xy$ is irrational.
Show that either $x$ or $y$ (or both) must be irrational.
{\bf Hint:} You can do this.  Be patient, think about it.

\item Let $A = \{2k+1 : k \in \Z\}$ and let $B = \{ 2m-11 : m \in \Z \}$.
Show that $A = B$ by showing containment both ways.
{\bf Hint:}  Use good form!

\item Let $A = \{ (x,y) \in \R^2 : y = 5x/7 - 2/7 \}$ and $B = \{ (x,y) \in \R^2 : 5x - 7y = 2 \}$.
Show that $A = B$ by showing containment both ways.

\item Let $A = \{ m \in \Z : m = 15k$ for some $k \in \Z \}$, let $B = \{ m \in \Z : m = 35j$ for some $j \in \Z \}$, and let $C = \{ m \in \Z : m = 105n$ for some $n \in \Z \}$.
Show that $A \cap B = C$ by showing containment both ways.
One direction is easier than the other.
Label one of them ``the easy direction'' and the other ``the hard direction''.
{\bf Hint:} Yes, we worked on a problem just like this in class.
Don't go back and find it, work through this one on your own.
{\bf Another hint:}  In the hard direction, you should come to something like $3k = 7j$ where $j$ and $k$ are integers.
You will need to conclude that $j$ is a multiple of 3.
If you are up for the challenge, show this using the division algorithm.
Don't use any ideas about prime factorization.
\elist

\vfill          % pad the rest of the page with white space

\activitytitle{Reading assignment \#1}{Due on the second day of class.  10 points}

The idea is to read Chapter 1 of the textbook by Daepp and Gorkin\DGreference.
The assignment is to read it in a particular way.
It may take 3 hours to get it done, but you will learn something in those three hours, and you will start to develop a very important skill.

Get a copy of Chapter 1, ``The How, When, and Why of Mathematics.''
Get out your notebook or some paper.
Go somewhere quiet, where you won't be interrupted for a while.
Turn off your phone so you aren't disturbed.
Don't listen to music that will distract you, and make sure there is no TV or youtube on where you can see it or hear it.

Put the notebook or paper right in front of you.
Put the textbook itself a bit farther away.
Make note of the time that you start reading in your notebook, maybe in the left margin.
Read the first paragraph of the chapter, then write one or more sentences in your notes which capture the main idea(s) of the paragraph.

Read the second paragraph, about Geogre P\'{o}lya's list of guidelines.
Look up the list in the Appendix.
Consider writing them in your notebook, or abbreviated versions of them.

Continue to write a sentence summarizing each paragraph.
I believe that if you are not writing, you are probably not thinking as hard as you need to.
Read slowly.
If you run into a word you don't know, google it or look it up in a dictionary.  If you really don't know it, write the definition in your notebook.
It is OK to spend 15 minutes on each page of the book.  Really.
It is not a goal of the course to learn how to read faster.
The goal is to learn how to get more out of the time you spend reading.
If you stop to take a break, note the time that you stopped and the time you start again.

Read Exercise 1.1 and the text that walks you through P\'{o}lya's guidelines.
Use your notebook to try to solve the puzzle yourself.
I've printed the alphabet twice for you.  That should save you a little time.

A second example starts on page 3 of the textbook.
As you read it, draw diagrams in your notebook.
Yes, there are diagrams printed in the textbook, but you will think harder about the diagram and understand more if you draw your own.

Example 1.2 asks a question.  Read the question and see if you can answer it on your own, without reading further in the book.

On page 7, you will see that solutions of the exercises are provided.
Resist the urge to turn your brain off and just read the solutions.  That is not what they are there for!

Read through each of the problems that begin on page 8 in the book.
Figure out what each problem is asking for and write that in your notes.
If you can solve the problem, do that.
If not, that's OK.

Problems 1.1 to 1.8 look nice.

You might start Problem 1.9 by trying some possible values for $n$.

Problem 1.10 doesn't interest me.  Does it interest you?

Can you draw the region described in Problem 1.11?

Problem 1.12 is good.  Would it help to make a graph?

Problem 1.13 seems silly.  Do you like it anyway?

Read the Tips on Doing Homework.
At the end, tally up how much time you have spent on reading this chapter.
Write this number in your notebook and remember the number when you come to class.

\activitytitle{Reading assignment, Chapter 2}{Due in the second week of class.}

Read Chapter 2 of the book by Daepp and Gorkin.
As with Chapter 1,
\blist{0.0in}
\item Read somewhere quiet, minimizing distractions from phones and friends
\item Note the time that you start and stop reading, and add up the minutes
\item Read with a pencil in your hand and your notebook open in front of you
\item Write a sentence to summarize each paragraph, re-draw diagrams, work out examples and exercises on your own
\item Look up words you don't know, and write down ones you really don't know
\item Read slowly.  You are not reading a comic book or a newspaper.  It is not a goal of this class for you to learn how to read faster.  The goal is to learn how to get more out of the time you spend reading, and to learn to concentrate for longer periods of time.
\item At the end, tally up how much time you have spent on reading this chapter.
Write this number in your notebook and remember the number when you come to class.
\elist

You will read about ``statements.''
Focus on the ones about mathematical things, and don't worry too much about interpreting the ones that are non-mathematical.

{\bf Note that on page 14, there is a statement about the color of the cover of the book.  Books from Springer are always yellow, but the authors must not have realized that someone would put a big blue bar on the cover of this edition of the book.  Just imagine that the book cover is all yellow.}

Fill out every truth table that is suggested in the chapter.
Truth tables are an excellent way to get great clarity about complicated combinations of statements.
The idea is to consider every possible combination of True and False for the basic statements.
For example, if there are two statements, $P$ and $Q$, there will be four rows in the table, running through the four possible combinations of True and False for $P$ and $Q$.
On page 21, there is a truth table for three statements, $P$, $Q$, and $R$.
It has eight rows.

The most important use of truth tables is to tell when two complicated combinations of logical expressions are, in fact, the same.

For me, the hardest thing about truth tables is making columns for implications like $P \to Q$.
Here is the best way I know to think about them.
Each row of the truth table for $P$ and $Q$ covers one combination of truth values for $P$ and $Q$.
Some of these combinations are consistent with the implication $P$ implies $Q$.
For example, when $P$ is True and $Q$ is True, this is consistent with $P \to Q$, so we put $T$ in the $P \to Q$ column.
The row in which $P$ is True and $Q$ is False, however, is inconsistent with the implication $P \to Q$, so we put $F$ in that row.
The cases in which $P$ is False are a bit different, but they are also consistent with $P \to Q$, since $P \to Q$ only has anything to say about $P$ and $Q$ when $P$ is True.
So we put $T$ in those rows too.

{\bf Problems 1 to 8 are good.  Rather than working on problems 9-21, I would much prefer that you spend your time making some truth tables.
Think of this as a specific assignment.
\blist{0.0in}
\item Do Problem 3 and also make a truth table for $\neg (P \vee Q)$ and $\neg P \wedge \neg Q$.
\item Make a big truth table for $P, Q, R,$ $P \wedge (Q \vee R)$, $P \vee (Q \wedge R)$, $(P \wedge Q) \vee (P \wedge R)$, and $(P \vee Q) \wedge (P \vee R)$.  Which of these are equal?  How can you rememeber that?
\elist
}
\vfill          % pad the rest of the page with white space

\activitytitle{Reading assignment, Chapter 3}{}

Read Chapter 3 of the textbook by Daepp and Gorkin.
As with Chapters 1 and 2,
\blist{0.0in}
\item Read somewhere quiet, minimizing distractions from phones and friends
\item Note the time that you start and stop reading, and add up the minutes
\item Read with a pencil in your hand and your notebook open in front of you
\item Write a sentence to summarize each paragraph, re--draw diagrams, work out examples and exercises on your own
\item Look up words you don't know, and write down ones you really don't know
\item Read slowly.  You are not reading a comic book or a newspaper.  It is not a goal of this class for you to learn how to read faster.  The goal is to learn how to get more out of the time you spend reading, and to learn to concentrate for longer periods of time.
\item At the end, tally up how much time you have spent on reading this chapter.
Write this number in your notebook and remember the number when you come to class.
\elist

Theorem 3.1 lists three properties of logical statements.
Please make truth tables for each of them to check that they are tautologies.
Also add de Morgan's laws from Theorem 2.9.
Then you'll have the whole set.
Having de Morgan's laws handy should make Exercise 3.2 easier.

How can you remember the distributive property?

The contrapositive is really important.
See if you can explain it just by thinking about $P \to Q$ and $\neg Q \to \neg P$, without using truth tables.

Theorem 3.3 is proven using the contrapositive.
This is a very useful method of proof.
Please note that it differs from proof by contradiction.

Read about the converse, and make sure never to confuse an implication with its converse.

Problems 2, 3, 4, 9, 14, 16, 18, 5, 6, 8, 19, 15 are good to work on, in that order.
Work through at least half of these problems.


Chapters 1 to 5 are mostly there to help develop proof techniques.  After Chapter 5, we will spend more of our time on definitions, examples, theorems, and proofs.
Use your time now to develop basic logic and proof techniques that will help you for the rest of the semester and beyond!


\vfill          % pad the rest of the page with white space

\activitytitle{Reading assignment, Chapter 4}{Due in the third week of classes.}

Read and understand Chapter 4 of the textbook by Daepp and Gorkin.
As with previous chapters,
{\small
\blist{0.0in}
\item Read somewhere quiet, minimizing distractions from phones and friends
\item Note the time that you start and stop reading, and add up the minutes
\item Read with a pencil in your hand and your notebook open in front of you
\item Write a sentence to summarize each paragraph, re--draw diagrams, work out examples and exercises on your own
\item Look up words you don't know, and write down ones you really don't know
\item Read slowly.  You are not reading a comic book or a newspaper.  It is not a goal of this class for you to learn how to read faster.  The goal is to learn how to get more out of the time you spend reading, and to learn to concentrate for longer periods of time.
\item At the end, tally up how much time you have spent on reading this chapter.
Write this number in your notebook and remember the number when you come to class.
\elist
}

This is a very important chapter, and one with real substance.  Hopefully you will feel that way when you read it, and will enjoy it more as a result.

This chapter has a large number of very dense expressions involving quantifiers, implications, and logical operators.  Slow way down when you run into one of them.  Pick them apart in your mind and then write them down so they are crystal clear.  Every symbol is important.  It's a bit like when you're reading someone your credit card number or you're giving your phone number to someone you really want to call you.  Every symbol is important.

Exercises 4.1, 4.2, 4.3, and 4.6 are all useful to do.
The discussion that begins at the bottom of page 36 is very important, negating statements with quantifiers.

There are 20 problems.  The more of them you do, the better, of course, but you may not be able to work through all of them.  {\bf Please at least do problems \# 1--7, and 20.}  Read \# 11.  Does this joke work on your friends?

People have asked about grading, or about a rubric that Ying-Ju Chen is using when she reads the notebook.
She'll be assigning numeric values between 0 and 5 for each chapter.
The bulk of the points go toward the notes on the chapter itself.
This is to emphasize that reading and taking notes is the primary concern.
Less than half of the points go toward attempting the exercises, with more emphasis on attempting than on getting them all the way right.
Ying-Ju writes as many helpful comments as she can on each notebook, but there is only so much time, and sometimes things that are incorrect do not get marked as incorrect.
Even so, I think that having you take notes and having Ying-Ju read them every class is working very well.

Pay attention to the phrase ``only if.'' It is often used in a way that can be confusing.  Compare these two statements for example, in which $R$ means Race and $P$ means prize:
\blist{0.0in}
\item I will race if there is a prize offered.  $P \to R$.  This is the most common way that people use the word ``if.''  The prize will make me race.
\item I will race only if there is a prize offered.  $R \to P$.  People say this sort of thing pretty often too, but it's a bit less clear unless you think about it carefully.  Part of the problem is the time order in which things happen, because the racing comes {\em after} the prize is offered.  ``If you see me racing, you can be sure that there was a prize offered. (But offering a prize is no guarantee that I will race.)''
\elist



\vfill          % pad the rest of the page with white space

\activitytitle{Reading assignment, Chapter 5}{Due in the fourth week of classes.}

Read and understand Chapter 5 of the textbook by Daepp and Gorkin.
As with previous chapters,
{\small
\blist{0.0in}
\item Read somewhere quiet, minimizing distractions from phones and friends
\item Note the time that you start and stop reading, and add up the minutes
\item Read with a pencil in your hand and your notebook open in front of you
\item Write a sentence to summarize each paragraph, re--draw diagrams, work out examples and exercises on your own
\item Look up words you don't know, and write down ones you really don't know
\item Read slowly.  You are not reading a comic book or a newspaper.  It is not a goal of this class for you to learn how to read faster.  The goal is to learn how to get more out of the time you spend reading, and to learn to concentrate for longer periods of time.
\item At the end, tally up how much time you have spent on reading this chapter.
Write this number in your notebook and remember the number when you come to class.
\elist
}

This chapter walks you through a number of types of proofs and gives examples of each.  {\bf Rewrite these proofs in your notes, in your own words as much as possible, so that you make them yours.}  By the end of reading the chapter, you should {\bf know} the proof that the square root of 2 is irrational and you should know the other proofs as well.

It might help, in your notes, to make a list of proof techniques from the chapter and from previous chapters.
What chapter talked about proof by contrapositive?  Is that in Chapter 5?
What about truth tables?  You can prove things with those.  What kinds of things?

Read and understand Problem 1.  It is important.

Read the other problems, find the ones that are easy, and do them.
This may seem like a strange assignment, but I really mean it.  Think about each problem (if you can get through all of them), and make sure that if a problem is easy, that you recognize that and write out the solution.
Don't worry if a problem looks hard but turns out to be easy.
That happens all the time.
But hopefully you will spot a number of them that really are easy, and do them.
We will go over these problems in class next week.

\vfill          % pad the rest of the page with white space

\yourname

\activitytitle{Quiz on some problems from Chapter 5 of Daepp and Gorkin}{15 points}

\blist{5in}
\item Let $x$ and $y$ be real numbers.  Use the triangle inequality to show that $||x| - |y|| \leq |x - y|$.

\item Prove or refute the following conjecture:  There are no positive integers $x$ and $y$ such that $x^2 - y^2 = 10$.  You can use the back of the sheet if you like.

\elist
\vfill          % pad the rest of the page with white space

\activitytitle{Reading assignment, Chapter 6}{Due in the fifth week of class.}

Read and understand Chapter 6 of the textbook by Daepp and Gorkin.
As with previous chapters,
{\small
\blist{0.0in}
\item Read somewhere quiet, minimizing distractions from phones and friends
\item Note the time that you start and stop reading, and add up the minutes
\item Read with a pencil in your hand and your notebook open in front of you
\item Write a sentence to summarize each paragraph, re--draw diagrams, work out examples and exercises on your own
\item Look up words you don't know, and write down ones you really don't know
\item Read slowly.  You are not reading a comic book or a newspaper.  It is not a goal of this class for you to learn how to read faster.  The goal is to learn how to get more out of the time you spend reading, and to learn to concentrate for longer periods of time.
\item At the end, tally up how much time you have spent on reading this chapter.
Write this number in your notebook and remember the number when you come to class.
\elist
}

This chapter introduces sets, subsets, equality of sets, and how to tell what the members of a set are.
As you read, take time to write out several members of each set that is introduced.
Note that $A$ being a subset of $B$ is the same as the logical implication $x \in A$ implies $x \in B$.
There is a tight connection between statements in set theory and logical statements.
Here is another:  Set $A$ being equal to set $B$ is the same as the logical implications $x \in A$ if and only if $x \in B$.

There are many examples in this chapter.
Work through them by rewriting them and adding useful steps in your notes.

On page 64, intersections, unions, and complements of sets are introduced.
As you read about them, explain in your notes how these relate logical statements such as $x \in A$ and $x \in B$ to $x \in A \cap B$.

You may enjoy reading about the paradoxes on page 67.
Give them a try.
Even if they are not your cup of tea, try to see what the issue is.

Problems 1 -- 9 are essential.  Do them.

Problem 10 is a good thought problem.  Think about it.

Starting with Problem 11, there are things for you to prove.
I would be happy to see you do some of these by yourself.
We will do these problems in class, but I'd like us to move through them fairly quickly, so have a look at them before class.

\vfill          % pad the rest of the page with white space

\activitytitle{Reading assignment, Chapter 7}{Due Monday, November 2.}

Read and understand Chapter 7 of the textbook by Daepp and Gorkin called ``Operations on sets.''

This is a short chapter, all about working with sets.
You can approach these problems in a number of ways.
Often it helps to draw a nice Venn diagram and get the right intuitive idea for what is being claimed, but don't stop there.
You can also just focus on letting $x \in A$ or whatever and working with that, without thinking about Venn diagrams.

Most of the chapter is devoted to one example, showing that, if $A,$ $B,$ and $C$ are sets, then $A \cup (B \cap C) = (A \cup B) \cap (A \cup C)$.
The book suggests working forward from one side, and backward from the other, just as people sometimes build a bridge by starting at each bank of a river and meeting in the middle.

It also suggests breaking into cases at some point.  For example, if $x \in A \cup (B \cap C)$, you can consider the case $x \in A$, which is great because then it's pretty clear that $x \in (A \cup B) \cap (A \cup C)$.  But you also need to consider the case $x \notin A$, so that $x \in B \cap C$.  But that's helpful, because then $x \in B$ and $x \in C$, and pretty soon it is clear that 
$x \in (A \cup B) \cap (A \cup C)$.

{\bf Do problem 7.1, parts a, c, d, e, f.}  Take your time and use really good form so that the proof is crystal clear.  Notice that part (c) (statement 18 in the theorem) is an ``if and only if'' statement, so it has two parts.  It's going to look something like this:
\blist{0in}
\item Suppose that $A \subseteq B$.  We want to show that $(X \backslash B) \subseteq (X \backslash A).$  Let $x \in X \backslash B$.  Then $x \notin B$.  (More steps here.) Thus, $x \in X \backslash A$, and so $(X \backslash B) \subseteq (X \backslash A).$
\item Suppose that $(X \backslash B) \subseteq (X \backslash A).$  We want to show that $A \subseteq B$.  Let $x \in A$. (More steps here.)  Thus, $x \in B$.
\elist

{\bf Do problem 7.4.}

{\bf Do problem 7.6.}

I guess that these problems are a bit dull, but it really is helpful to be good at proving things about sets.

\noindent As with the previous chapters,
\blist{0.0in}
\item Read somewhere quiet, minimizing distractions from phones and friends
\item Note the time that you start and stop reading, and add up the minutes
\item Read with a pencil in your hand and your notebook open in front of you
\item Write a sentence to summarize each paragraph, re-draw diagrams, work out examples and exercises on your own
\item Look up words you don't know, and write down ones you really don't know
\item Read slowly.  You are not reading a comic book or a newspaper.  It is not a goal of this class for you to learn how to read faster.  The goal is to learn how to get more out of the time you spend reading, and to learn to concentrate for longer periods of time.
\item At the end, tally up how much time you have spent on reading this chapter.
Write this number in your notebook and remember the number when you come to class.
\elist


\vfill          % pad the rest of the page with white space

\activitytitle{Reading assignment, Chapter 8}{Make a good effort by Monday, November 9; due on Friday, November 13.}

Read and understand Chapter 8 of the textbook by Daepp and Gorkin, called ``More on operations on sets.''

This chapter is a challenge.
You will really need to use all the reading skills you have been practicing when you read this chapter.
The ideas are harder, and some are really hard, but not impossible.
Just slow yourself down and write things out in lots of detail.

Example 8.2(a) would be a great one to write out concrete fractions with different values of $p$ and $q$ to understand the sets $A_q$ and then the union of these sets.  For Example 8.2(b), do the same to understand what the sets $B_i$ are, and then what their intersection is.  No shortcuts!  Write out elements for each set.

Exercise 8.3 is also good.

In the middle of page 82 the phrase ``collection of subsets of $X$'' appears.
This is a very new, very difficult concept; do not underestimate how tricky it can be, but patiently think about it and keep coming back to it.
For example, $\cal{A}$ might be all intervals of the form $[k,k+1]$ and you might want to take the union of all such intervals, or the intersection.

Exercise 8.4 is excellent.  Draw pictures until everything is crystal clear.
Exercise 8.5 is also excellent.

Rewrite the proofs of Examples 8.6 and 8.7 to make them your own.  Really.

Exercises 8.9 and 8.10 are also excellent.  Do them on your own, then compare to the solutions in the book.

Do problems 1, 2, and 3.

Here is a challenge problem.  Let $a < b$.  Show that $\bigcup_{n=1}^{\infty} [a, b-\frac{1}{n}] = [a,b)$.  Draw pictures, then show set inclusion both ways.

Here is another challenge problem.  Let $a < b$.  Show that $\bigcap_{n=1}^{\infty} [a,b+\frac{1}{n}) = [a,b]$.  Draw pictures, then show set inclusion both ways.

\noindent As with the previous chapters,
\blist{0.0in}
\item Read somewhere quiet, minimizing distractions from phones and friends
\item Note the time that you start and stop reading, and add up the minutes
\item Read with a pencil in your hand and your notebook open in front of you
\item Write a sentence to summarize each paragraph, re-draw diagrams, work out examples and exercises on your own
\item Look up words you don't know, and write down ones you really don't know
\item Read slowly.  You are not reading a comic book or a newspaper.  It is not a goal of this class for you to learn how to read faster.  The goal is to learn how to get more out of the time you spend reading, and to learn to concentrate for longer periods of time.
\item At the end, tally up how much time you have spent on reading this chapter.
Write this number in your notebook and remember the number when you come to class.
\elist


\vfill          % pad the rest of the page with white space

\activitytitle{Reading assignment, Chapter 9}{Due in the tenth week of class.}

Read and understand Chapter 9 of the textbook by Daepp and Gorkin, called ``The Power Set and the Cartesian Product.''
This is the last chapter on plain set theory.
It should stretch your mind in a few new directions.
Prepare to move slowly and think carefully.

When $A$ is a set, the power set of $A$ is the collection of all subsets of $A$.  Read Example 9.1 and do Exercise 9.3 and then {\bf do Problem 9.1.}
Work through Exercise 9.2 and then {\bf do Problem 9.2.}
Problem 9.2 is hard, but excellent for you.  Take it very slowly.
Work through Exercise 9.4 and then {\bf do Problem 9.5.}
{\bf Do Problem 9.8.}

{\bf Do Problem 9.11.}  For 9.11, you have already seen the power set of a set containing 2 elements and 3 elements.  {\bf Hint:}  When you are making a subset of a set $A$, for each element of $A$, you have to decide whether it goes in or out of the subset.  There are two choices (in or out) each time.  If the hint doesn't help you, write out the power set of $\{1, 2, 3, 4\}$, then read the hint again.  Hopefully you don't have to write out the power set of $\{1, 2, 3, 4, 5\}!$

You are already very familiar with one Cartesian product:  making ordered pairs $(x,y)$ of real numbers is the Cartesian product $\R \times \R$, which you know better as the $xy$ plane.
Every problem involving Cartesian products of sets containing real numbers can be depicted as points in the $xy$ plane.
Make a graph in every case.
This will help your intuition.
When there are only finitely many points, like with $\{0,1\} \times \{ 2,3\}$, also list out all of the $(x,y)$ pairs.

{\bf Answer these questions:  Who is the Cartesian product named after?  Why, exactly?}

Work through Exercise 9.5 a, b, e.

{\bf For Theorem 9.7, draw $A$ and $C$ as intervals on the $x$ axis and draw $B$ and $D$ as intervals on the $y$ axis, then draw out the sets in the statement of the theorem on two separate sets of axes.}
Make sure you are crystal clear about what these sets are, and you will be close to mastering Cartesian products.

{\bf Do Problem 9.12.}  It connects Cartesian products to things you learned in geometry.

{\bf Do Problem 9.17a.}  Notice that this is an ``if and only if'' proof, and it has three set equalities to show.
Suppose that $A \times B = C \times D$ and show that $A = C$ and $B = D$ by showing containment each way.
Here is one part of the argument:  Let $x \in A$.  Also let $y \in B$.  Then $(x,y) \in A \times B = C \times D$, and so $x \in C$.  Thus $A \subseteq C$.
After that part is done, suppose that $A=C$ and $B=D$ and argue that $A \times B = C \times D$.

{\bf Think about Problem 9.19.}

\noindent As with the previous chapters,
\blist{0.0in}
\item Read somewhere quiet, minimizing distractions from phones and friends
\item Note the time that you start and stop reading, and add up the minutes
\item Read with a pencil in your hand and your notebook open in front of you
\item Write a sentence to summarize each paragraph, re-draw diagrams, work out examples and exercises on your own
\item Look up words you don't know, and write down ones you really don't know
\item Read slowly. 
\item At the end, tally up how much time you have spent on reading this chapter.
Write this number in your notebook and remember the number when you come to class.
\elist

\vfill          % pad the rest of the page with white space

\activitytitle{Reading assignment, Chapter 10}{Due in the eleventh week of class.}

Read and understand Chapter 10 of the textbook by Daepp and Gorkin, called ``Relations.''

The main definition for Chapter 10 appears at the end of Chapter 9, on page 93.  Here is the deal.  A {\em relation} $S$ from a set $X$ to a set $Y$ is a subset of $X \times Y$.  If $Y = X$, we say the relation is a relation on $X$.
At the beginning of Chapter 10, we see that we are going to be only working with relations on a set $X$.

Suppose that $S$ is a relation on a set $X$.
That is, suppose that $S$ is a subset of $X \times X$, which means that $S$ is a set of points of the form $(x,y)$, where $x \in X$ and $y \in X$.
Rather than write $(x,y) \in S$, we usually write $x \sim y$.
How to read this out loud?  There is no perfect solution.  I would suggest that you read it as ``$x$ tilde $y$'' (because $\sim$ is the tilde that appears above the n in some Spanish words).

Suppose that $X = \R$ and let $S = \{ (x,y) : x \leq y\}$.
Then $x \sim y$ means that $(x,y) \in S$, which means that $x \leq y$.
In this way, we see that $\leq$ is a relation on $\R$.
{\bf Write out the set $S$ corresponding to the relations $<$, $\leq$, =, $\geq$, and $>$.  Then also sketch these as regions in the $xy$ plane.}

Note that relations are between two elements.
Thus, ``divisible by 4'' is not a relation.
However, if $X = \Z^+$, you could say that $x \sim y$ if $y$ is divisible by $x$, and then you would have a relation.
People often write $x | y$ for this relation and say that $x$ divides $y$.
Call this relation $S$.
{\bf Write out at least ten of the ordered pairs in $S$, using at least five different values of $x$.}

Read Exercises 10.1 and 10.2.  

Read the definitions of reflexive, symmetric, and transitive.
A relation that satisfies all three is called an equivalence relation.
This is where most of the action is with relations.
{\bf Do Problem 10.2.}
{\bf \Large You should start every part of the problem by writing down examples.}
For example, for (a), the example $3 < 3$ will tell you whether the relation is reflexive, $3 < 5$ and $5 < 3$ will tell you about symmetry, and $3 < 5, 5 < 7$, and $3 < 7$ will get you started on transitivity.

Read Example 10.3, then {\bf do Problem 10.3.}
Use examples to check reflexivity, symmetry, and transitivity.

Equivalence relations are very important, as are equivalence classes.
An equivalence relation is like the equality relation (=), but applied to other contexts.
Here is an example that is useful.
Think of the integers, $\Z$.
Say that $x \sim y$ if $x$ and $y$ have the same remainder when you divide by 2.
Then $6 \sim 22$ and $31 \sim 7$.
This relation is reflexive, because $x \sim x$.
It is symmetric because if $x \sim y$ then $y \sim x$.
And it is transitive because if $x \sim y$ and $y \sim z$, then $x \sim z$.
Now we can say that 6 is equivalent to 22, and 31 is equivalent to 7, according to this definition of equivalence.
The equivalence class that contains 6 and 22 is all even numbers, and the equivalence class containing 31 and 7 is all odd numbers.
Let this sink into your mind, and you will start to see that it makes for a useful way to organize things, when an equivalence relation is available.

{\bf Do Problem 10.1}  Start by writing out examples for the pairs $(x,y)$ and $(w,z)$.  Think about lines and circles in the plane.

{\tiny 
\blist{0.0in}
\item Read somewhere quiet, minimizing distractions from phones and friends
\item Note the time that you start and stop reading, and add up the minutes
\item Read with a pencil in your hand and your notebook open in front of you
\item Write a sentence to summarize each paragraph, re-draw diagrams, work out examples and exercises on your own
\item Look up words you don't know, and write down ones you really don't know
\item Read slowly. 
\item Tally up how much time you have spent on reading this chapter.
\elist
}
\vfill          % pad the rest of the page with white space

\activitytitle{Reading assignment, Chapter 18}{Reading due on November 30, notes and exercises due on December 2.}

Read and understand Chapter 18 of the textbook by Daepp and Gorkin, called ``Mathematical Induction,'' up to the statement, but not the proof, of Theorem 18.6.

Mathematical induction and recursion play an important role especially in discrete mathematics. Prepare to move slowly and think carefully. To understand the proof of Theorem 18.1, you will need {\bf Well-ordering principle of the natural numbers}: Every nonempty subset of the natural numbers contains a minimum.

Read Theorem 18.1 and then {\bf do Problem 18.1} and {\bf Problem 18.3}. Follow the steps in Theorem 18.1, defining the assertion $P(n)$ for the problem first. You will need the condition ``$P(n)$ is true'' to show the induction step. Work through Exercises 18.3 to 18.5 and then {\bf do Problem 18.9} without going back to Exercise 18.5. You can do it!

Recursion is a very useful tool to define functions, sequences and sets. Before you move to Theorem 18.6, read the definition of $n$ factorial for $n\in \mathbb{N}$. Write out $3!, 4!$ and $5!$.  As an exercise, simplify $\frac{6!}{2!4!}$. More generally, simplify $\frac{n!}{m!(n-m)!}$ where $n$ and $m$ are two positive integers with $n\geq m$. These fractions are called {\em binomial coefficients} and are useful in probability.

Here is another example of using recursion:  
Let $n\in \mathbb{Z^+}$. Consider the function $S(n) = S(n-1) + n$ with $S(0) = 0$. Write out $S(1), S(2), S(3)$ and $S(4)$. Can you figure out what this function does for us?  Together with Problem 18.1, you should be able to see the connection between induction and recursion.

Theorem 18.6 shows the existence and uniqueness of a recursive function $g: N \rightarrow X$ given a function $f: X \rightarrow X$ and $a\in X$, where $X$ is a nonempty set. The function $g$ satisfies
\begin{itemize}
\item[(i)] The base step: $g(0) = a$, and
\item[(ii)] The recursive step: $g(n+1)=f(g(n))$ for all $n \in \mathbb{N}$.
\end{itemize}

\noindent The proof of Theorem 18.6 is too long for us to read this semester.  You can come back it later.

\noindent As with the previous chapters,
\blist{0.0in}
\item Read somewhere quiet, minimizing distractions from phones and friends
\item Note the time that you start and stop reading, and add up the minutes
\item Read with a pencil in your hand and your notebook open in front of you
\item Write a sentence to summarize each paragraph, re-draw diagrams, work out examples and exercises on your own
\item Look up words you don't know, and write down ones you really don't know
\item Read slowly.
\item At the end, tally up how much time you have spent on reading this chapter.
Write this number in your notebook and remember the number when you come to class.
\elist

\vfill          % pad the rest of the page with white space


%-----------------------------------------------------------------------------

\end{document}  % end of the document

