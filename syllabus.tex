\yourname

\activitytitle{Syllabus}{This syllabus is an assignment for you to read and respond to.
Please read it carefully, fill in answers, put your name on it, and turn it in on the second day of class.}

\noindent {\bf Course description.}
There are two main goals for the course:
\blist{0.1in}
\item Improving your ability to work with definitions, examples, counterexamples, claims, and proofs.
\item Improving your ability to read a mathematics textbook on your own.
\elist
My hope is that your new abilities in these two areas will make you unstoppable in your mathematics classes.
We will spend most of class time on \#1.  
I will design activities for us to do together in class for this purpose.
Most of your time outside of class will be spent on \#2.
Being able to read mathematics on your own is a fantastic skill.
Be sure to set aside quiet time to read the textbook.

\noindent {\bf Q: How comfortable are you already with the ``definition, example, theorem, proof'' sequence in mathematics classes?}

\vspace*{1in}

\noindent {\bf Q: What kinds of experiences have you had in the past with proofs?}

\vspace*{1in}

\noindent {\bf Professor and contact information.}  

\noindent {\bf Q: Do you check your university email regularly?}

\vspace*{0.5in}

\noindent {\bf Schedule.}

\vspace*{0.5in}

\noindent {\bf Office hours.}
You are welcome to visit me in my office, which is room XXX in the mathematics building.
The best way to arrange a time to meet is to send an email listing a few times that would work for you.
I will reply with one that works for me as well.

\noindent {\bf (Sample question to get students to find my office) Q: If you take the elevator to the fourth floor, do you turn right or left to get to my office?}

\vspace*{0.5in}

\noindent {\bf (Sample question to get students to find my office) Q: What are the two flyers on my door about?}

\vspace*{0.5in}

\noindent {\bf Textbook.}  (A suggested textbook; reading assignments are given in these materials for this book.)  
The textbook for the course is {\em Reading, writing and proving, A closer look at mathematics}, second edition, by Ulrich Daepp and Pamela Gorkin.  2011.  The textbook is very good, but not perfect.  You can learn from it, especially if you take time to read it.

\noindent {\bf Q: Have you ever had success reading a mathematics textbook and really learning from it?  If so, please tell what book, what course, and what made it work.  If not, please tell me what you think prevented you from being able to read the book.}

\vspace*{1in}

\noindent {\bf Q: Do you have a hard copy of the textbook that you can read?  Have you been able to get a PDF file of the first chapter from the library?}

\vspace*{0.5in}

\noindent {\bf Graduate assistant.} 

\noindent {\bf Q: Do you have any interest in going to graduate school?  Please explain.}

\vspace*{0.5in}

\noindent {\bf Coursework.} Here are the main things that you will be doing:
\blist{0.1in}
\item Written work on in--class activities.
\item Taking notes on each chapter in the textbook in your notebook.  Bring your notebook to class so that the graduate assistant can read through it in class and give it back.
\item Occasional quizzes (very much like the work you'll already be doing in class) instead of one or two big exams
\item A final exam (which should be very similar to what we have been doing all semester long)
\elist

\noindent {\bf Q: Do you have any questions or concerns about the coursework?}

\vspace*{1in}

\noindent {\bf Grading.}  My general plan is this.  Many things you do during the semester will have a point value attached to them.  The number of points will indicate their relative importance to your grade.  In--class work and homework will count for a larger share than in most courses, while quizzes and exams will count for a lower share.  I will announce the relative percentages at least two weeks before the first exam.

Reading assignments will be assigned numeric values between 0 and 10 for each chapter.
The bulk of the points go toward the notes on the chapter itself.
This is to emphasize that reading and taking notes is the primary concern.
Less than half of the points go toward attempting the exercises, with more emphasis on attempting than on getting them all the way right.
I will write as many helpful comments as we can on each notebook, but there is only so much time, and sometimes things that are incorrect do not get marked as incorrect.

\noindent {\bf Q: Do you have any questions or concerns about the grading?}

\vspace*{1in}

\noindent {\bf Attendance.} Attendance and class participation will be vitally important.
Class time is the best time to make attempts and get immediate feedback.
If you cannot attend a class, notify me as soon as possible by email or phone, before class if possible.
Don't even imagine that you can miss a class without letting me know.
I don't particularly need to know {\bf why}, but I do need to know.

\noindent {\bf Q: What is the most likely reason that you will miss class?  I'm just curious.}

\noindent {\bf Background questions.}
\blist{0.8in}
\item What mathematics courses are you taking this semester?
It's OK to just list the numbers, like Math 3410.

\item What mathematics courses have you already taken here?

\item Including this one, how many semesters until you graduate?

\item Please let me know anything you think I should know about you.  I'll read it all.  Sometimes people like to tell about their hobbies, movies they like, where they're from, etc.

\elist

\vfill          % pad the rest of the page with white space
