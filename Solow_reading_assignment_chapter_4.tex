\activitytitle{Reading assignment \#4}{Due on Tuesday, September 26.  20 points}

Read Chapter 4 in the book by Daniel Solow.
It is about showing that there is an ``object'' with a ``certain property'' such that ``something happens.''
We have already done a number of proofs of this general form.

From the class survey, I am reminded that people like to take notes in different ways.
Do what works for you, but make sure that your notes show that you read each section and that you found and understood the main messages there.

Also, from the class survey, people would like to work on something related to the reading, so we'll start with something from this reading on Tuesday.
Good idea!

\vspace{0.1in}
\noindent
{\bf Specific requirements}
\vspace*{-0.15in}

\begin{itemize}
\item Read  Section 4.1 and take notes.
Then, look back through the Even and Odd activity and the Vector Sum and Dot Product activity and list by number all of the exercises that are of the form ``Show that there is an object with a certain property such that something happens.''
I've posted previous activities on the Syllabus section on Canvas.
Note:  Showing that $n$ is even means showing that there exists an integer $k$ for which $n=2k$.

\item The existence part of the Division Algorithm is of the form described in this chapter.
Write it out following the general pattern that there is an ``object' with a ``certain property'' such that ``something that happens,'' in that order.
Hint:  The last thing to write is $n = qk + r$.

\item In Section 4.3, Proposition 5 assumes that $m$ is even.
Suppose instead that $m$ is odd, and show that $m^2+n^2-1$ is a multiple of 4.

\item Do exercise 4.2.

\item Do exercise 4.9.  In each case, explain how you found the object.

\item Do exercise 4.11.  There are two objects getting constructed here, $k$ and $x$.
Where do their values come from?
Under what condition could you produce an additional rational root?

\item Do exercise 4.13.  This is an excellent project with several parts.  Work hard on it.

\item Do exercise 4.16.

\item Do exercise 4.22.

\item At the end, tally up how much time you have spent on this chapter.
Write this number in your notebook.
Bring your notebook to class and turn it in for grading.
\end{itemize}

\noindent
{\bf General comments}

Set yourself up in a place where you won't be disturbed.
Read slowly, and write notes in your own words that reflect your understanding of the material.
