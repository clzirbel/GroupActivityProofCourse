\documentclass[11pt]{article}  % 11 point font

\textheight  9.4in      % height of the text before a page break
\topmargin  -0.8in      % extra distance from top of page to header, if any
\textwidth   7in      % the default text width is much narrower
\oddsidemargin -0.25in   % extra left margin on odd-numbered pages.
\renewcommand{\baselinestretch}{1.15} % line spacing; use any decimal
\pagestyle{empty}

\usepackage{amsmath}    % allows AMS math things like the cases environment
\usepackage{amssymb}    % allows the use of AMS symbols like blackboard bold
\usepackage{amsthm}     % allows AMS definitions for theorems, proofs, etc.
\theoremstyle{definition}  % use roman font in theorems, definitions, etc.

\newtheorem{theorem}{\bf Theorem}                  % Allow numbered theorems
\newtheorem{proposition}[theorem]{\bf Proposition} % Number props with theorems
\newtheorem{remark}[theorem]{\bf Remark}           % Number remarks with theorems
\newtheorem{definitionX}[theorem]{{\bf Definition}}
\newtheorem{notationX}[theorem]{\bf Notation}
\newtheorem{exampleX}[theorem]{\bf Example}
\newtheorem{problemX}[theorem]{\bf Problem}
\newtheorem{showX}[theorem]{\bf Show}
\newtheorem{recallX}[theorem]{\bf Recall}
\newtheorem{noteX}[theorem]{\bf Note}
\newtheorem{guidedproofX}[theorem]{\bf Guided proof}
\newtheorem{proveX}[theorem]{\bf Prove}
\newtheorem{groupworkX}[theorem]{\bf Group work}
\newtheorem{questionX}[theorem]{\bf Question}
\newtheorem{challengeX}[theorem]{\bf Challenge}

%\newtheorem{theorem}[equation]{\bf Theorem} % this would number theorems
                                             % with equations, as some people like
%\numberwithin{equation}{section}  % equation 3 in section 2 is (2.3)

% ------------------------------------------ new commands

\newcommand{\coursenumber}{Math (insert course number here)}
\newcommand{\coursename}{Mathematical Foundations and Techniques}

\newcommand{\yourname}{\hfill {\bf Your name: \underline{\hspace{2.5in}}}
%\vspace*{-0.1in}
}

\newcommand{\anonymous}{\hfill {\bf Anonymous!}}

\newcommand{\blank}[1]{\underline{\hspace{#1}}}

\newcommand{\activitytitle}[2]{\noindent {\bf \LARGE #1}\\\noindent #2\vspace*{0.1in}}

\newcommand{\overview}[1]{\noindent{\bf \Large Overview}\\\framebox{\parbox{7in}{#1}}}

\newcommand{\definition}[2]{\begin{definitionX}{\bf #1.} #2\end{definitionX}}
\newcommand{\notation}[1]{\begin{notationX}#1\end{notationX}}
\newcommand{\example}[2]{\begin{exampleX}#1\end{exampleX}\vspace*{#2}}
\newcommand{\problem}[2]{\begin{problemX}#1\end{problemX}\vspace*{#2}}
\renewcommand{\show}[2]{\begin{showX}#1\end{showX}\vspace*{#2}}
\newcommand{\recall}[2]{\begin{recallX}#1\end{recallX}\vspace*{#2}}
\newcommand{\note}[1]{\begin{noteX}#1\end{noteX}}
\renewcommand{\stop}[1]{\noindent{\bf \Large \underline{Stop.}} #1}
\newcommand{\Hint}{{\bf Hint}}
\newcommand{\guidedproof}[1]{\begin{guidedproofX}#1\end{guidedproofX}}
\newcommand{\prove}[2]{\begin{proveX}#1\end{proveX}\vspace*{#2}}
\newcommand{\groupwork}[2]{\begin{groupworkX}#1\end{groupworkX}\vspace*{#2}}
\newcommand{\question}[2]{\begin{questionX}#1\end{questionX}\vspace*{#2}}
\newcommand{\challenge}[2]{\begin{challengeX}#1\end{challengeX}\vspace*{#2}}

% ----------------------------- Facilitate repeating things with no number (NN)

\newcommand{\definitionNN}[2]{\noindent{\bf Definition #1.} #2}
\newcommand{\notationNN}[1]{\begin{notationX}#1\end{notationX}}
\newcommand{\exampleNN}[2]{\noindent{\bf Example.} #1 \\ \vspace*{#2}} \newcommand{\showNN}[2]{\noindent{\bf Show.} #1 \\ \vspace*{#2}}
\newcommand{\recallNN}[2]{\noindent{\bf Recall.} #1 \\ \vspace*{#2}}
\newcommand{\noteNN}[1]{\begin{noteX}#1\end{noteX}}
\newcommand{\HintNN}{{\bf Hint}}
\newcommand{\guidedproofNN}[1]{\begin{guidedproofX}#1\end{guidedproofX}}
\newcommand{\proveNN}[2]{\begin{proveX}#1\end{proveX}\vspace*{#2}}
\newcommand{\groupworkNN}[2]{\begin{groupworkX}#1\end{groupworkX}\vspace*{#2}}
\newcommand{\questionNN}[2]{\begin{questionX}#1\end{questionX}\vspace*{#2}}

\newcommand{\blist}[1]{\begin{list}{{\bf \arabic{enumi}.}}{\usecounter{enumi}\setlength{\itemsep}{#1}}} 
                                     % begin a numbered list.  The optional
                                     % argument is the spacing between items
\newcommand{\elist}{\end{list}}      % end the list

\newcommand{\balist}[1]{\begin{list}{{\bf \alph{enumii}.}}{\usecounter{enumii}\setlength{\itemsep}{#1}}} 
                                     % begin a numbered list.  The optional
                                     % argument is the spacing between items
\newcommand{\ealist}{\end{list}}      % end the list

\newcommand{\vect}[1]{\langle #1 \rangle}   % angle brackets for a vector
\newcommand{\tvec}[1]{\vect{#1_1, #1_2, #1_3}}
\newcommand{\qq}{\quad\quad}
\newcommand{\qqq}{\quad\quad\quad}
\newcommand{\qqqq}{\quad\quad\quad\quad}

\newcommand{\R}{\mathbb{R}}
\newcommand{\Z}{\mathbb{Z}}
\newcommand{\Q}{\mathbb{Q}}
\newcommand{\N}{\mathbb{N}}
\newcommand{\Rp}{\R^{+}}

\newcommand{\DGreference}{\footnote{Reading, Writing, and Proving: A Closer Look at Mathematics, 2011, by Ulrich Daepp and Pamela Gorkin}}

% --------------------------------------------- what to process

%\includeonly{even_and_odd}
%\includeonly{vector_sum_dot_product}
%\includeonly{division_algorithm}
%\includeonly{pigeonhole_principle}
%\includeonly{square_roots_are_irrational}
%\includeonly{DG_reading_assignment_chapter_3}
%\includeonly{DG_reading_assignment_chapter_4}
%\includeonly{even_and_odd_and_vector_quiz}
%\includeonly{DG_reading_assignment_chapter_5}
%\includeonly{DG_reading_assignment_chapter_6}
%\includeonly{set_examples_and_relations}
%\includeonly{new_vector_operation_duplicate_quiz}
%\includeonly{DG_reading_assignment_chapter_7}
%\includeonly{set_operations}
%\includeonly{infinite_set_operations_prequel}
%\includeonly{DG_reading_assignment_chapter_8}
\includeonly{infinite_set_operations}


% --------------------------------------------------- begin document
\begin{document}        % What came before is the 'preamble'

\noindent {\bf What you might say on the first day of class}

This class is designed to help you develop key skills in math that will help you succeed in higher level, more abstract courses.
My goal is that by working hard in this course, you will become unstoppable in your future math courses.

There are two main components to the course:  learning how to read a math book on your own, and practicing the basic steps in mathematics, where we start with examples and non--examples, make a definition, examine more more examples, make a conjecture, see if we can find a counterexample to the conjecture, make another conjecture, and if we can find a proof, call the result a theorem.

The simplest way to describe this course is that it is the ``proof course.''
A better way to describe it is that it focuses on the core ideas of mathematics, over and over again, in different settings.
The core idea of mathematics is to make definitions of mathematical objects, consider examples and non--examples of those definitions to make sure we understand them well, make conjectures about what we think might be true, and then try to prove those things.
It is all about understanding mathematical ideas and how they fit together.
Proofs are a part of that, but they are only a part.
This course will not have the same feel as algebra and calculus courses, which are heavier on calculations and don't have quite as many different ideas or different types of examples.

You will work on activities in class, then hand them in at the end of class.
I will read over them and make comments, then give them back to you at the beginning of the next class.
There are often multiple correct ways to do a problem.
It's not question of ``what I want'' as the teacher, but what works?

There is no need to rush through the activities.
Take your time, think about what you're doing.
Work with your group to make sure you all understand everything along the way.
You do not need to finish the activity; I always try to add extra material at the end so that no group runs out of things to do.

Most people learn by doing.
Me talking at the board is not the same as you learning.
Most of your time in the course will be spent working on activities as a group while I go from group to group, seeing how you are doing, answering questions, and making suggestions.

Outside of class, you will be reading the textbook, taking notes on what you read, and solving some exercises.
You will bring your notes to class and they will be read over quickly to make sure that you are doing the reading and thinking.
Each chapter will also have a short reading quiz.
I will not lecture over the material in the book; that is one of the keys to you learning how to read a book on your own.

There will be a final exam, but rather than mid-term exams, there will be quizzes over the material in activities, once you have had a chance to get good at it.

\yourname

\activitytitle{Even and odd}{Our first example of definitions, examples, theorems, and proofs}

\overview{Definitions are important to read and understand by looking at examples.  Many proofs are little more than working with the definitions and rewriting things.  With a bit of practice, these become very routine.  This activity has you work through two definitions, a few examples, and then some proofs.  Everything relies on the definitions, so keep coming back to them.  We will use a similar model many times during the semester.}

\definition{Even\label{evendef}}{An integer $n$ is {\em even} if there exists an integer $k$ for which $n = 2k$.}

\definition{Odd\label{odddef}}{An integer $n$ is {\em odd} if there exists an integer $k$ for which $n = 2k+1$.}

\note{19 meets the definition to be odd because 19 is an integer and 19 = 2(9)+1.}

\example{Check that 12 meets the definition to be even by writing $12 = 2k$ for an appropriate value of $k$ and make sure $k$ is an integer.}{0.2in}

\example{Does $-9$ meet the definition to be odd?  Write $-9 = 2k+1$.}{0.2in}

\example{Does 0 meet the definition to be even?  Write $0 = 2k$ for an integer $k$.}{0.2in}

\example{Does 1.73 meet the definition to be odd?  Explain.}{0.4in}

\note{Suppose that $m$ is an integer.  Then $2m+1$ is an integer and it is odd because it meets the definition to be odd.  Also, $2m+2$ is even because it is an integer and can be rewritten as $2(m+1)$, which is of the form $2k$ where $k=m+1$, which is an integer.}

\show{Suppose that $m$ is an integer.  Show that $2m+6$ is even by rewriting it until it meets the definition to be even.  Connect your statements with $=$ signs.}{0.5in}

\show{Suppose that $m$ is an integer.  Show that $4m + 9$ is odd by rewriting it until it meets the definition to be odd.  Connect your statements with $=$ signs.}{0.5in}

\stop{Compare your answers to the questions above with the other people in your group before you move on.  Resolve any differences in your answers.}

\show{Suppose that $m$ is even.  Then $m = 2k$ for some integer $k$.
Show that $m+8$ is even by rewriting it as $2k+8$ and continuing until it is 2 times an integer.  Connect your statements with $=$ signs.}{1in}

\note{You already know that the sum of two even numbers is even.  The next item guides you through a proof of this fact, using the definitions of even and odd above.}

\guidedproof{Suppose that $m$ and $n$ are even.  Fill in the blanks to show that $m+n$ is even.\label{sumofevens}
\blist{0.1in}
\item There exist integers $j$ and $k$ such that $m = $\blank{0.75in} and $n = $\blank{0.75in}.
\item Thus, $m + n = $ \blank{2in} = $2(j+k)$.
\item This number meets the definition to be even because it is an \blank{0.75in} and because $(j+k)$ is an \blank{0.75in}.
\item We saw that if $m$ and $n$ are even, then $m+n$ is even.  
We made no further assumption about $m$ and $n$.
Thus, the sum of any two even numbers is even. 
\elist
}

\guidedproof{Suppose that $m$ is even and $n$ is odd.
Fill in the blanks to show that $mn$ is even.  
Use the previous exercise as a model.
\blist{0.1in}
\item There exist \blank{2in} such that $m = $\blank{0.75in} and $n = $\blank{0.75in}.
\item Thus, $mn = $\blank{3in} = $2($\blank{0.75in}$)$.
\item This number satisfies the definition to be even because \blank{1.25in} and \blank{1.25in}.
\item We saw that \blank{4in}.  We made no \\\blank{3in}.
Thus, \blank{2.5in}.
\elist
}

\prove{Let $m$ and $n$ be odd.  Follow the examples above as a model to show that $mn$ is odd by rewriting $mn$ until it meets the definition to be odd.  Good form is critically important in proofs.
\blist{0.1in}
\item
\item
\item
\item
\elist
}{0.1in}
\pagebreak
\prove{Let $m$ and $n$ be odd.  Follow the examples above as a model to show that $m + n$ is even by rewriting it.  Good form is critically important in proofs.
\blist{0.1in}
\item
\item
\item
\item
\elist
}{0.1in}

\groupwork{Step 1 in each of your proofs uses the definition of even or odd to write $m$ and $n$ in a more useful, more informative way.
How does this help your proof along?
Discuss with members of your group.}{0.75in}

\groupwork{Step 3 in each of your proofs also uses the definition.
How does this use of the definition differ from what happens in Step 1?
Discuss with members of your group.}{0.75in}

\prove{Let $m$ be odd.  Follow the models above to prove that $m^2$ is odd.  Use good form.}{1.5in}

\prove{Let $m$ be even.  Follow the models above to prove that $m^2$ is even.  Use good form.}{1.5in}

\question{What does it mean that an integer is a multiple of 4?  Give your own definition analogous to the definitions of even and odd.}{0.25in}

\prove{Let $m$ be even.  Show that $m^2 +2m + 4$ is a multiple of 4.}{1in}

\question{Let $m$ be even.  Can $m^2 +2m + 4$ be a multiple of 8?  Explain.}{1in}

\noindent
{\bf Challenges.}  Here are some statements that are harder to prove, because they require a bit more than simply restating the definitions.
See if you can make a good argument for them.

\question{Pretend for a minute that there is an integer $m$ which is both even and odd.
Work with the definitions to see that this really must be fantasyland.}{1in}

\prove{If $m$ is an integer, then $m$ is even or $m$ is odd; it has to be one of the two, there is no third possibility.\\
Here is one suggestion.  0 is even.  If $n$ is even, then $n+1$ is odd.  If $n$ is odd, then $n+1$ is even.  This should cover all positive integers.  Also, if $n$ is even, then $-n$ is even, which tells us about negative integers.}{1.5in}

\prove{If $m$ is an integer and $m^2$ is odd, then $m$ is odd.
\Hint:  There are two cases to check, the case in which $m$ is even and the case in which $m$ is odd.}{0.0in}

\vfill          % pad the rest of the page with white space

\yourname

\activitytitle{Background  and syllabus questions -- Math 3280}{Do your best with these questions and turn this sheet in on Thursday.
Some of them reference information from the syllabus. This assignment is worth 10 points}

\blist{0.8in}

\item (2 points) What mathematics courses have you already taken in college?
It's OK to just list the numbers, like Math 3410.

\item (2 points) Please list all the courses you are taking this semester, aside from this one.

\item (2 points) Including this semester, how many semesters have you been at BGSU?

\item How comfortable are you with the ``definition, example, theorem, proof'' progression in mathematics classes?

\item What kinds of experiences have you had in the past with proofs?

\item Have you ever had success reading a mathematics textbook and really learning from it?  If so, please tell what course and what made it work.  If not, please tell me what you think prevented you from being able to read the book.

\item (2 points) If you take the elevator to the fourth floor, do you turn right or left to get to my office?

\item (2 points) What is the most interesting thing to you on the door of my office?

\item Do you have a hard copy of the textbook that you can read?  Electronic copy?

\item Do you have any interest in going to graduate school?  Please explain.

\item Do you have any questions or concerns about the coursework?

\item Do you have any questions or concerns about the grading?

\item What is the most likely reason that you will miss class?  I'm just curious.

\item Please let me know anything you think I should know about you.  I'll read it all.  Sometimes people like to tell about their hobbies, movies they like, other academic interests, clubs they're in, where they're from, etc.

\elist

\vfill          % pad the rest of the page with white space

\include{syllabus2015}
\yourname

\activitytitle{Sum and dot product of 3--dimensional vectors}{\vspace*{-0.2in}}

\overview{In Calculus III and Linear Algebra, we define vectors and work with them.  
They have a geometric interpretation, but here we will simply give an algebraic definition of 3--dimensional vectors and some operations on them and work with their algebraic properties.
This activity illustrates proofs in which all that is needed is the definition and a ``rewrite'' proof, where you can work forward and backward to show a series of equalities.
Notice how we often use the same definition twice in one proof, once to ``unpack'' and the second time to ``re--pack.''}

\definition{3--dimensional vector\label{3dvectordef}}{A three--dimensional vector is an ordered triple $\vect{ a_1, a_2, a_3 }$, where $a_1, a_2,$ and $a_3$ are real numbers.
The numbers  $a_1, a_2,$ and $a_3$ are called {\em components} of the vector.
}

\notation{A 3--dimensional vector $\vect{ a_1, a_2, a_3 }$ is often denoted by a single letter with an arrow over the top, like this $\vec{a}$.  When it is written like $\vect{ a_1, a_2, a_3 }$ it is said to be in {\em open form.}
The commas and brackets are part of the definition and are important.}

\definition{Equality of 3--dimensional vectors\label{3dvectorequalitydef}}{3--dimensional vectors $\vect{ a_1, a_2, a_3 }$ and $\vect{ b_1, b_2, b_3 }$ are equal if $a_1=b_1, a_2=b_2,$ and $a_3 = b_3$.  Note: The order of the numbers is important.}

\definition{Sum of 3--dimensional vectors\label{3dvectorsumdef}}{The sum of 3--dimensional vectors $\vect{ a_1, a_2, a_3 }$ and $\vect{ b_1, b_2, b_3 }$ is the 3--dimensional vector $\vect{ a_1+b_1, a_2+b_2, a_3+b_3 }$.  We write $\vec{a} \oplus \vec{b}$ for the sum of $\vec{a}$ and $\vec{b}$, using a new symbol so we don't confuse addition of vectors with addition of real numbers.}

\example{Is $\vect{ 3,9,12 }$ a 3--dimensional vector? Explain.  

\vspace*{0.1in}
\noindent
Is it equal to $\vect{12,3,9}$?  Explain.}{0.1in}

\example{Is $\vect{ \sqrt{3},\sqrt[3]{9},\sqrt{-12} }$ a 3--dimensional vector? Explain.}{0.3in}

\example{Is $\vect{ 8,13.35321,\pi,-7 }$ a 3--dimensional vector? Explain.}{0.3in}

\example{Is $\vect{ 3 + 9 + 12 }$ a 3--dimensional vector? Explain.}{0.3in}

\example{Is $\vect{ \left[ \begin{array}{cc} 6 & 0 \\ 2 & 5 \end{array} \right], -4, 7 }$ a 3--dimensional vector? Explain.}{0.4in}

\example{Let $x$ be a real number.  Is $\vect{ \frac{14}{3},2-7x,\sqrt{16} }$ a 3--dimensional vector? Explain.}{0.3in}

\stop{Compare your answers to the questions above with the members of your group.  Make sure you agree on everything.}
\pagebreak

\example{Calculate the sum of $\vec{c} = \vect{ 12,-5,3 }$ and $\vec{d} = \vect{ 6,4,-11 }$.  Start by writing $\vec{c} \oplus \vec{d} = \ldots$ and write the vectors in open form next.


\vspace*{0.2in}
\noindent
Calculate $\vec{d} \oplus \vec{c}$ in the same way in a separate calculation.
}{0.1in}

\show{You are going to show that addition of 3--dimensional vectors is commutative.  
Fill in the blanks.
This is a ``rewrite'' proof.
You can work forward from the top, backward from the bottom, or a bit of both.\\
Let $\vec{a}$ and $\vec{b}$ be 3--dimensional vectors.  Then,
\begingroup
\addtolength{\jot}{0.7em}
\begin{eqnarray*}
    \vec{a} \oplus \vec{b}
    &=& \vect{ \qqq,\qqq,\qqq } \oplus \vect{\qqq,\qqq,\qqq} \qqqq \qqqq \qqqq \qqqq \\
    &=& \vect{ \qqq\qqq,\qqq\qqq,\qqq\qqq } \\
    &=& \vect{ \qqq\qqq,\qqq\qqq,\qqq\qqq } \\
    &=& \vect{ \qqq,\qqq,\qqq } \oplus \vect{\qqq,\qqq,\qqq} \\
    &=& \vec{b} \oplus \vec{a}
\end{eqnarray*}
\endgroup
We have seen that $\vec{a} \oplus \vec{b} = \vec{b} \oplus \vec{a}$.
We made no further assumption about $\vec{a}$ and $\vec{b}$.
Thus, for all 3--dimensional vectors $\vec{a}$ and $\vec{b}$, we know that  $\vec{a} \oplus \vec{b} = \vec{b} \oplus \vec{a}$.
Thus, addition of 3--dimensional vectors is commutative.
}{0in}

\show{Go back to each line of the proof above and give exactly one reason for the equality on that line at the very right side of the line.  
The first one is ``Write in open form.''  
Two of them are Definition \ref{3dvectorsumdef}.
In the middle you will use the fact that addition of real numbers is commutative.  
Thus, at the heart of it, commutativity of vector addition comes from commutativity of addition of real numbers.}{0in}

\show{Show that addition of 3--dimensional vectors is associative.
Start with arbitrary 3--dimensional vectors $\vec{a}, \vec{b},$ and $\vec{c}$.
Write $(\vec{a} \oplus \vec{b}) \oplus \vec{c}$ and rewrite it until it becomes $\vec{a} \oplus (\vec{b} \oplus \vec{c})$.
Take small steps and write exactly one reason for each equality.
Since you know what equality you need to show, you can work forward from the top, backward from the bottom, or both.\\
Let $\vec{a}, \vec{b},$ and $\vec{c}$ be \blank{3in}.
\begingroup
\addtolength{\jot}{0.7em}
\begin{eqnarray*}
    (\vec{a} \oplus \vec{b}) \oplus \vec{c}
    &=&  \qqqq \qqqq \qqqq \qqqq \qqqq\qqqq\qqqq\qqqq \qqqq \\
    &=&  \\
    &=&  \vect{ ( \qqq + \qqq ) + \qqq, ( \qqq + \qqq ) + \qqq, ( \qqq + \qqq ) + \qqq} \\
    &=&  \vect{ \qqq + ( \qqq + \qqq), \qqq + ( \qqq + \qqq), \qqq + ( \qqq + \qqq)} \\
    &=&  \\
    &=&  \\
    &=& \vec{a} \oplus (\vec{b} \oplus \vec{c})
\end{eqnarray*}
\endgroup
We have seen that ... 
}{0in}

\stop{Compare your argument to the rest of the members of your group.  Make sure that you agree on absolutely every step and every justification.}
\pagebreak

\definition{Scalar product for 3--dimensional vectors}{Let $c$ be a real number and let $\vec{a} = \tvec{a}$ be a 3--dimensional vector.  The {\em scalar product} of $c$ and $\vec{a}$ is a 3--dimensional vector defined as:
\[
    c\vec{a} = \vect{ca_1,ca_2,ca_3}.
\]}

\example{Let $c = 3$ and $\vec{a} = \vect{7,-4,\sqrt{2}}$.  Calculate $c\vec{a}$, starting by writing $c\vec{a} = 3 \vect{7, -4,\sqrt{2}} = \ldots$.}{0.3in}

\example{Calculate $\pi \vect{9,4,1}$ = }{0.2in}

\example{Calculate $(2+\sqrt{3}) \vect{5,b,c}$ = }{0.2in}

\show{You are going to show that the scalar product is distributive over vector addition.  First use the word ``Let'' to settle on one real number $c$ and two 3--dimensional vectors, $\vec{a}$ and $\vec{b}$.  
Then start with the expression $c(\vec{a} \oplus \vec{b})$ and rewrite it three times.
Then, move to the last expression and work backwards, until you meet in the middle.  
Provide one reason for each equality, on the right, on the same line as the equality.
At the end, follow the model to conclude that you have shown distributivity in general.

\vspace{0.2in}
\noindent
Let ...
\begingroup
\addtolength{\jot}{0.7em}
\begin{eqnarray*}
    c(\vec{a} \oplus \vec{b})
    &=&  \qqqq \qqqq \qqqq \qqqq \qqqq\qqqq\qqqq\qqqq \qqqq \\
    &&  \\
    &&  \\
    &&  \\
    &&  \\
    &&  \\
    &&  \\
    &&  \\
    &&  \\
    &=& c\vec{a} \oplus c\vec{b}
\end{eqnarray*}
\endgroup
We have seen that ...
}{0.5in}

\vfill

\stop{Check over what everyone in your group has done, and make sure that you completely agree.}

\newpage

\show{Show that the scalar product is distributive over real number addition.  
Start with ``Let.''  
Write $(c+d)\vec{a}$ and rewrite it until it equals $c\vec{a} \oplus d\vec{a}$.
Work forward from the top and backward from the bottom.
Provide one reason for each equality.
At the end, follow the model to conclude that this shows distributivity in general.
Explain why some addition signs are $+$ and others are $\oplus$.}{4in}

\definition{Zero vector}{The vector $\vect{0,0,0}$ is a special 3--dimensional vector, called the {\em zero vector}.
We denote it by $\vec{0}$.}

\definition{Additive inverse}{Let $\vec{a}$ be a 3--dimensional vector, with open form $\tvec{a}$.
Define a new vector by $-\vec{a} = \vect{-a_1,-a_2,-a_3}.$
It is called the {\em additive inverse} of $\vec{a}$.}

\show{Let $\vec{a}$ be a 3--dimensional vector.  
Use a rewrite proof to show that $\vec{a} \oplus \vec{0} = \vec{a}$.  
This is called the {\em additive identity} property.
It's not very exciting.  Make a general conclusion.}{1.2in}

\show{Let $\vec{a}$ be a 3--dimensional vector, and let $-\vec{a}$ be its additive inverse.  
Use good form to show that $\vec{a} \oplus (-\vec{a}) = \vec{0}$.  
This is called the {\em additive inverse} property.
This is also not very exciting.  Make a general conclusion.}{0in}

\pagebreak
\definition{Dot product of 3--dimensional vectors}{The dot product of 3--dimensional vectors $\vect{ a_1, a_2, a_3 }$ and $\vect{ b_1, b_2, b_3 }$ is the real number $a_1 b_1 + a_2 b_2 + a_3 b_3$.}

\notation{The dot product of 3--dimensional vectors $\vec{a}$ and $\vec{b}$ is denoted $\vec{a} \bullet \vec{b}$.}

\example{Calculate the dot product of $\vec{a} = \vect{ 12,-5,3 }$ and $\vec{b} = \vect{ 6,4,-11 }$  Do this by writing
\begin{eqnarray*}
    \vec{a} \bullet \vec{b} &=& \tvec{a} \bullet \tvec{b}\\
    &=& a_1 b_1 + a_2 b_2 + a_3 b_3
\end{eqnarray*}
and then substituting in the numbers.
This makes the calculation just a matter of rewriting, so it is a good way to do calculations like this.}{1.3in}

\show{Show that the dot product is commutative, just as multiplication of real numbers is commutative.
Start with ``Let''.
Write one expression at the top of the space below, and write your goal expression at the bottom, and then work
forward and backward until you have a rewrite proof.
Follow the models from previous examples, and be sure to make a general conclusion.}{3.5in}

\example{Calculate $\vec{a} \bullet \vec{0}$.  Is this a general result?  If so, make your calculation into a general result.}{0in}

\newpage

\show{Show that the dot product is distributive over vector addition.
That is, show that $(\vec{a} \oplus \vec{b}) \bullet \vec{c} = \vec{a} \bullet \vec{c} + \vec{b} \bullet \vec{c}.$  
Start with ``Let''.  
Write the first expression at the top, the last expression at the bottom, and then work forward and backward.
Also explain why one addition symbol is $\oplus$ and the other is $+$.}{3.5in}

\show{Let $\vec{a}$ and $\vec{b}$ be 3--dimensional vectors and let $c$ be a real number.  
Show in general that $c(\vec{a} \bullet \vec{b}) = (c\vec{a}) \bullet \vec{b} = \vec{a} \bullet (c\vec{b})$.  
Use parentheses {\em every} time three things are multplied together, to be clear about order of operations.
Since there are two equalities to show, think about how you will organize the equalities.}{0in}

\vfill          % pad the rest of the page with white space

\yourname

\activitytitle{Quantifier assessment}{This is not part of your grade in the course, but it will be checked for correctness.}

\noindent {\bf A.} Write the following statements symbolically:
\blist{0.4in}
\item For every $a$, there is a $b$ for which $b^2 = a$
\item For every $b$, there is an $a$ for which $b^2 = a$
\item For every $a$ and every $b$, it is the case that $b^2 = a$
\item There exists an $a$ and there exists a $b$ such that $b^2 = a$
\elist
\noindent 

\noindent {\bf B.} Which of the statements in the previous problem are true if the universe for both $a$ and $b$ is the set of non--negative integers?
If not true, explain why not.
\blist{0.2in}
\item
\item
\item
\item
\elist

\noindent {\bf C.} Negate the statements from problem A.
\blist{0.4in}
\item
\item
\item
\item
\elist

\noindent {\bf D.} Write the following statements symbolically:
\blist{0.5in}
\item Every rose has a thorn.
\item Every married couple with a child gets a tax deduction.
\elist

\vfill          % pad the rest of the page with white space

\yourname

\activitytitle{The Division Algorithm}{Dividing integers with remainders will form the basis for several things we want to prove.}

\overview{We would like to distribute $n$ objects evenly among $k$ people and find out how many are left over.  We will investigate a procedure for doing this, which is called division, even though there will be no fractions in this activity.  Procedures that are guaranteed to work are called {\em algorithms} after the 9th century Persion mathematician al-Khwarizmi, who worked on procedures for arithmetic.  The division algorithm itself dates to Euclid's {\em Elements} from around 300 BC.}

\example{\label{37among5} You are the dealer in a card game that has 37 cards.  (It's not a standard deck of cards.)  There are 5 people playing, and everyone needs to end up with the same number of cards.  Dealing one card to each player leaves 32 cards in your hands.  Write down the numbers 37, 32, and continue until you cannot deal out any more cards evenly.  Let $r$ denote the number of cards left at the end, and let $q$ denote the number of times you subtracted 5, which is also the number of cards that each person got.  You see that $37 - 5q = r$, which you can rewrite as $37 = 5q + r$.  Fill in $q$ and $r$ and write out these two equations.}{1.2in}

\example{Now you're playing a card game that you have not played before, and you haven't taken the time to count how many cards are in the deck.  You are the dealer again, and there are 5 people who need cards.  Let $n$ denote the number of cards in the deck.  Imagine that you repeat the procedure from the previous example until you can no longer deal out cards evenly.  Again, let $r$ denote the number of cards you have left at the end and $q$ denote the number of cards that each person got.  What do we know for sure about the possible values of $r$?  What do we know for sure about the possible values of $q$?  Write the relationship between $n$, 5, $q$, and $r$ analogous to $37 - 5q = r$ and the expression analogous to $37 = 5q + r$.  The last expression accounts for where all of the $n$ cards have gone; some are dealt out, some are left in your hands.  Write out a sentence that explains this.}{1.5in}

\example{Continuing to divide by 5, complete this sentence:  Given an integer $n \geq 0$, there exist integers $q$ and $r$ where (list properties of $q$ here) \blank{2in} and (list properties of $r$ here) \blank{2in}, such that (write the relationship between $n$, $q$, and $r$ analogous to $37 = 5q + r$) \blank{2in}.}{0in}

\example{\label{positivepositivedivision} 
Once again, we have $n$ cards, but now there are $k$ people playing, where $k > 0$ is an integer.
Your friend is dealing.
Write instructions telling your friend how to deal the cards out, when to stop dealing, and say what you can about how many cards she will have left over and how many cards each person will get.
Using $q$ to denote the number of cards each person gets and $r$ to denote the number of cards left over, write out the relationship between $n, k, q,$ and $r$, and write inequalities concerning $q$ and $r$.}{2in}

\stop Compare your work to the others in your group.  Work to get the best possible phrasing of \ref{positivepositivedivision}.

\question{What happens when $k=0$?  Can you satisfy $n = qk + r$?  What goes wrong?}{1in}

\note{As above, suppose that $n$ and $k$ are integers that are greater than 0.
Suppose you find integers $q$ and $r$ for which $n = qk + r$ and $0 \leq r < k$.
Suppose your friend tries to do the same thing and finds integers $q_2$ and $r_2$ for which $n = q_2 k + r_2$ and $0 \leq r_2 < k$.
Must it be the case that $q = q_2$ and $r = r_2$?
That is, are the values of $q$ and $r$ {\em unique}?
If you think about dealing $n$ cards to $k$ people, it's pretty clear that you and your friend will get the same values of $q$ and $r$, but how could we see this without thinking about card dealing?
It will take a few steps.
}

\show{\label{remainderinequality} Suppose that $r$ and $b$ are integers for which $0 \leq r < k$ and $0 \leq b < k$.
Carefully combine these inequalities to show that $-k < r-b < k$.
Your goal is to provide a crystal clear argument with no extra steps.
You can add inequalities that run the same direction; for example, if $a < b$ and $c < d$, then $a+c < b+d$.
}{2in}

\show{\label{DivisionAlgorithmUniqueness}Suppose that $n$ and $k$ are integers, that $n \geq 0$ and $k > 0$, and that $q$, $r$, $q_2$, and $r_2$ are integers such that $n = qk + r$ and $n = q_2 k + r_2$ and that $0 \leq r < k$ and $0 \leq r_2 < k$.
Set $qk + r$ equal to $q_2 k + r_2$ and use \ref{remainderinequality} argue that $q = q_2$ and $r = r_2$.
It will probably not be obvious how to get started; this proof requires a spark of genius.
Use scratch paper, write down everything you know, including \ref{remainderinequality}, and then write a final argument here.
}{2.6in}

\theorem{\label{divisionalgorithm}{\bf The Division Algorithm.} Let $n$ and $k$ be integers greater than 0.
There are two parts to the theorem.
\blist{0in}
\item {\bf Existence.} There exist integers $q$ and $r$ for which $n = qk + r$ and for which $0 \leq r < k$.
\item {\bf Uniqueness.}  The numbers $q$ and $r$ are unique; $n = qk + r$ is the only way to write $n$ as a multiple of $k$ plus a remainder from $0, 1, 2, \ldots, k-1$.
\elist
The number $q$ is called the {\em quotient} and $r$ is called the {\em remainder}.
You have proven this theorem above in two problems.
The existence part was proven in \blank{0.8in}.  The uniqueness part was proven in \blank{0.8in}.
}

\example{Rewrite Theorem \ref{divisionalgorithm} for the case where $k=2$.
Be specific about the possible values of $r$.}{0.6in}

\prove{Let $n$ be an integer greater than 0.
Recall the definitions of even and odd.
Use the existence part of the division algorithm with $k=2$ to show that $n$ must satisfy at least one of these definitions.}{0.8in}

\prove{Let $n$ be an integer greater than 0.
Use the uniqueness part of the division algorithm with $k=2$ to conclude that if $n$ is even, then it cannot be odd.  Also, if $n$ is odd, it cannot be even.  Thus, each integer is even or odd, not both.
}{0in}

\note{Now we will divide negative numbers by positive numbers, with remainder.}

\example{Start with $-37,$ add 5 to get $-32$, and add 5 repeatedly, writing down the numbers you come to, until you reach a number between 0 and 4.
Count the number of 5's that you added to write $-37 + 5k = r$, where you fill in $k$ and $r$.
Rewrite this as $-37 = 5q + r$ and note the sign of $q$.
This represents division of a negative number with remainder.
How does it differ from division of a positive number with remainder?}{2in}

\prove{Let $n$ be an integer less than 0.
Let $k$ be an integer greater than 0.
Add $k$ to $n$ repeatedly until you reach a number between 0 and $k-1$.
Can you be sure that you will ever get all the way to non-negative numbers?
Can you be sure that you don't jump over the numbers $0, 1, \ldots, k-1$ and keep adding $k$ forever?
Use the result to argue that you can write $n + pk = r$ for some integer $p$, and rearrange it to read $n = qk + r$.
What do you know about the values of $q$ and $r$?
}{2in}

\prove{Let $n$ be an integer less than 0 and let $k$ be an integer greater than 0.
Scrutinize your proof of \ref{DivisionAlgorithmUniqueness}.
The problem assumed that $n \geq 0$, but where did the proof use $n \geq 0$?
If it did not, then the proof was more general than we needed at the time.
What can we conclude about uniqueness when writing $n = qk + r$ for negative values of $n$?
}{0in}
\pagebreak

\show{Suppose that $n$ is an integer and $n = 3m + 1$ where $m$ is an integer.
Use the division algorithm to argue clearly that $n$ cannot be a multiple of 3.}{1.5in}

\show{Let $n$ be an integer and suppose that $n^2$ is a multiple of 3.
We would like to conclude that $n$ is a multiple of 3.
Let's use the contrapositive:  We'll show that if $n$ is not a multiple of 3, then $n^2$ is not a multiple of 3.
Use the division algorithm to explain that there are three ways to write $n$ as $3q+r$, and two of these make $n$ not be a multiple of 3.
In each of these two cases, compute $n^2$ and check that $n^2$ is not a multiple of 3, again using the division algorithm.
}{2in}

\show{Suppose $n$ is an integer.
Can $n^2$ be of the form $3m + 2$ where $m$ is an integer?
Examine the cases in the previous question carefully.}{0in}
\pagebreak

\show{Let $n$ be an integer and consider the numbers $n$, $n+1$, and $n+2$.
Show that exactly one of these is a multiple of 3.
Use three cases, $n = 3k$, $n = 3k + 1$, and $n = 3k + 2$, and follow the guide below.

\vspace*{0.2in}
\noindent
Case 1:  $n = 3k$.  Then $n+1 = $ \blank{1.5in} and $n+2 = $ \blank{1.5in}.\\
Exactly one of these is a multiple of three (circle it).
Use the division algorithm to argue that the other two are not multiples of 3.

\vspace*{0.5in}
\noindent
Case 2:  $n = 3k+1$.  Then $n+1 = $ \blank{1.5in} and $n+2 = $ \blank{1.5in}.\\
Exactly one of these is a multiple of three (circle it).
Use the division algorithm to argue that the other two are not multiples of 3.

\vspace*{0.5in}
\noindent
Case 3:  $n = 3k+2$.  Then $n+1 = $ \blank{1.5in} and $n+2 = $ \blank{1.5in}.
}{0.5in}

\show{Let $n$ be even and consider the numbers $n$ and $n+2$.
Use two cases to show that exactly one of these is a multiple of 4.}{2in}

\show{Let $n$ be an odd integer.  Show that $n^3-n$ is a multiple of 24.  Here again, you will need a spark of genius.  Use scratch paper to brainstorm different approaches that you could try, then try the one that looks the most promising.}{0in}

\vfill          % pad the rest of the page with white space

\include{pigeonhole_principle}
\anonymous

\activitytitle{Class survey}{I would like your feedback to improve the course.  Many thanks in advance!}

\blist{0.5in}
\item In class, we work through activities without much ``lecture.''
How does this work for you?

\item What is going well in the class, so that we should not change it?

\item Is there anything we should change about the class to help you learn better?

\item If there are specific things you are able to do in other classes because of taking this class, please list them.

\item What else could we do to make you unstoppable in your other math courses?

\item Do you look forward to coming to class?  Why or why not?

\item You have been asked to read the textbook, and we have checked your notes to make sure this is happening.
Is this working well for you?
Why or why not?

\vspace{0.5in}
Would you recommend that other faculty do the same in their courses?
{\bf yes \quad no}
\\
Please explain.
\vspace*{0.5in}

\item In what way(s) have you changed how you work with the textbook in other courses that you are taking?  Do you read them more?  Differently?
Please explain.

\item Make any other comments you like here or on the back of the sheet.

\elist
\vfill          % pad the rest of the page with white space

\yourname

\activitytitle{Square roots of prime numbers are irrational}{This is a classic example of proof by contradiction.}

\overview{Most students are familiar with the fact that $\sqrt{2}$ is irrational, but few can prove it.  Having read a proof of this fact in your textbook or online, the starting point is to re-create the proof from memory, then to move on to showing that $\sqrt{3}$ is irrational.  The proof is similar and yet different.}

\definition{Irrational\label{irrational}}{A real number is said to be {\em irrational} if it cannot be written as the quotient of two integers.}

\prove{Prove that $\sqrt{2}$ is irrational by contradiction.  The proof begins with ``Assume that $\sqrt{2}$ can be written as $\frac{p}{q}$ where $p$ and $q$ are integers.''
Argue to a contradiction.}{3in}

\note{If you are new to proof by contradiction, you might prefer to start the previous proof by writing ``Let's pretend for a minute that $\sqrt{2}$ can be written as $\frac{p}{q}$ where $p$ and $q$ are integers.''  This makes it extra clear that you don't really believe that $\sqrt{2}$ is rational, you are just exploring what would happen if that were true.  When you arrive at a contradiction, you realize it's time to stop pretending; $\sqrt{2}$ must be irrational.}

\show{A key step in the proof is that if $n$ is an integer and $n^2$ is even, then $n$ is even.  You may have already shown this, using a proof by contradiction (which begins, ``Assume that $n$ is odd.'') or a proof by contrapositive (which begins, ``Let us show the contrapositive, that if $n$ is not odd, then $n^2$ is not odd.'') or a proof by cases (which begins, ``There are two possibilities for $n$, that $n$ is even or that $n$ is odd.)  Whichever one you used, choose another and write the proof here.}{2in}

\show{Mimic the proof that $\sqrt{2}$ is irrational to show that $\sqrt{3}$ is irrational.
Work with the members of your group to figure out how to do this.}{3in}

\show{A key step in the proof that $\sqrt{3}$ is irrational is the fact that if $n$ is an integer and $n^2$ is a multiple of 3, then $n$ is also a multiple of 3.  This is not as straightforward as with multiples of 2, but it can still be done by contradiction, by contrapositive, or by cases.
Think about these possibilities and choose the one that seems to you to be the best approach.}{2in}





\vfill          % pad the rest of the page with white space

\yourname

\activitytitle{Examples of sets and relations between them}{This activity introduces sets, ways to write them, and the relations between them.}

\overview{Many familiar ideas can be expressed using sets.
We begin with examples of sets and the relations between them.}

\problem{Let $A  = \{x : x$ solves $ax=b$ where $a$ and $b$ are integers and $a \ne 0 \}$. 
Let $\Q  = \{x : x$ is a rational number $\}$.
Show that $A = \Q$ by showing set inclusion in both directions.
The first part is done for you; read that carefully.\\
\vspace*{-0.3in}
\begin{itemize}
\item Let $x \in A$.  
Then there exist $a$ and $b$ such that $ax = b$ and $a \ne 0$.
Dividing through by $a$, $x = \frac{b}{a}$ where $a$ and $b$ are integers and $a$ is not zero.
Thus, $x \in \Q$.
Since $x$ was arbitrary, $A \subseteq \Q$.

\item Let $x \in \Q$.

\end{itemize}}{0.8in}

\problem{Let $A = \{ f : f $ is a continuous function from $\R$ to $\R \}$.
Let $B = \{ f : f $ is a differentiable function from $\R$ to $\R \}$.
Determine whether $A \subset B$, $B \subset A$, $A \subseteq B$, or $B \subseteq A$ and then write a clean argument that it is so.
Remember that to show $\subset$, you need an example of an element that is in one set but not in the other.
}{1.2in}

\problem{Let $A  = \{x \in \R : x$ solves $x^2=a$ where $a$ is an integer and $a \geq 0\}$. 
Show that $A \not\subset \Q$.
Make your logic crystal clear.
}{1in}

\problem{Continuing the previous problem, show that $\Q \not\subset A$.
Make your logic crystal clear.
}{1in}

\problem{Let $E = \{ m \in \Z : $ there exists $j \in \Z$ such that $m = 2j \}$.
Let $O = \{ m \in \Z : $ there exists $k \in \Z$ such that $m = 2k + 1 \}$.
Show that $E \cap O = \emptyset$ by letting $m \in E \cap O$ and showing that this leads to a contradiction.
}{1in}

\problem{Continuing the previous problem, show that $E \cup O = \Z$ by showing set inclusion both ways.
}{1in}

\problem{Let $2\Z = \{ m \in \Z :$ there exists $j \in \Z$ such that $m = 2j\}$.
Let $3\Z = \{ m \in \Z :$ there exists $j \in \Z$ such that $m = 3j\}$, and similarly with other sets like $5\Z$ and $15Z$.
Show that $2\Z \cap 3\Z = 6\Z$ by showing set inclusion both ways.
}{1in}

\problem{Write out all elements in $6\Z \cap 8\Z \cap \{1, 2, 3, \ldots, 100\}$.
}{1in}

\note{Inequalities between real numbers have the {\em transitivity} property:  If $a \leq b$ and $b \leq c$, then we can conclude that $a \leq c$.  Similar inequalities are true with $\geq, <,$ and $>$.}

\problem{Suppose that $x > 4$.  Argue that $x \geq 2$.}{1in}

\problem{Using standard interval notation, show that $(4,9] \subset [2,9]$.
Begin with ``Let $x \in (4,9].''$ then rewrite this as a compound inequality, then rewrite as two separate inequalities.  Use transitivity along the way. Make sure to }{1.5in}

\problem{Using standard interval notation, show that $[2,6) \cap [3,8) = [3,6)$ by showing set inclusion both ways. As above, write compound inequalities, then individual inequalities, then compound inequalities again.  Use a number line to illustrate.}{2in}

\problem{Show that $[2,6) \cup [3,8) = [2,8)$ by showing set inclusion both ways.}{0in}


\vfill          % pad the rest of the page with white space

\yourname

\activitytitle{Operations on sets}{This activity works with set identities and relates them to logic.}

\overview{Sets are absolutely fundamental to mathematics.
This chapter focuses on building up set identities, relationships between sets that are always true.}

\problem{Let $A$ and $B$ be sets.  Show that $(A \cup B)^{c} = A^{c} \cap B^{c}$ by showing set inclusion both ways.  This is one of de Morgan's laws.  Draw a really nice Venn diagram to illustrate.}{2in}

\problem{Let $A$ and $B$ be sets.  Show that $(A \cap B)^{c} = A^{c} \cup B^{c}$ by showing set inclusion both ways.  This is the other one of de Morgan's laws.  Draw a really nice Venn diagram to illustrate.}{2in}

\problem{\label{setsaslogic}Let $A$ and $B$ be sets.  Let $P$ be the logical statement $x \in A$, and let $Q$ be the logical statement $x \in B$.
Use $P$ and $Q$ and logic symbols to fill in the simplest expressions:
\blist{0.1in}
\item $x \in A \cup B$ is \blank{1in}
\item $x \in (A \cup B)^{c}$ is $\lnot (P \wedge Q)$
\item $x \in A^c$ is \blank{1in}
\item $x \in B^c$ is \blank{1in}
\item $x \in A^c \cap B^c$ is \blank{1in}
\elist
Make a truth table for $P$, $Q$, and each of the other logical statements here to establish that $x \in (A \cup B)^c$ is logically equivalent to $x \in A^c \cap B^c$.
Compare the truth values in the columns corresponding to $x \in (A \cup B)^c$ to the Venn diagram you made above.  Explain how they agree.
}{0in}

\problem{Let $A$ and $B$ be sets.  Use the approach of the previous exercise to show that $x \in (A \cap B)^{c}$ is logically equivalent to $x \in A^{c} \cup B^{c}$.  Compare the truth table to the Venn diagram again.}{2in}

\problem{Let $D, E,$ and $F$ be sets.
Use one of de Morgan's laws that you showed above to establish that $(D \cup E \cup F)^c = D^c \cap E^c \cap F^c$.
This proof works by rewriting, not by showing inclusion both ways.
{\bf Hint:} Let $A = D \cup E$ and $B = F$.}{1.5in}

\problem{Let $D, E,$ and $F$ be sets.
Use one of de Morgan's laws to show that $(D \cap E \cap F)^c = D^c \cup E^c \cup F^c$.}{1.5in}

\problem{Let $A, B,$ and $C$ be sets.
Use a proof by cases to show that $A \cup (B \cap C) = (A \cup B) \cap (A \cup C)$.  Remember to show inclusion both ways.
Organize your writing carefully to make the steps of this argument really clear.}{3in}

\problem{Let $A, B,$ and $C$ be sets.
Use logical statements $P, Q,$ and $R$ and a truth table to show that $x \in A \cup (B \cap C)$ is logically equivalent to $x \in (A \cup B) \cap (A \cup C)$.  Be sure to define $P$, $Q$, and $R$ at the beginning.}{2.5in}

\definition{Symmetric difference}{Let $A$ and $B$ be sets.  The {\em symmetric difference} of $A$ and $B$ is the set $A \bigtriangleup B = (A \backslash B) \cup (B \backslash A)$.}

\problem{Let $A$ and $B$ be sets.  Show that $A \bigtriangleup B = B \bigtriangleup A$ by showing set inclusion both ways.  Draw a nice Venn diagram to illustrate.}{2in}

\problem{Consider again the logical statements from \ref{setsaslogic}.  Write a logical statement that is equivalent to $x \in A \bigtriangleup B$.
Make a truth table with 4 rows, labeled 1, 2, 3, 4, and three columns, one for $x \in A$, one for $x \in B$, and the third for $x \in A \bigtriangleup B$.
Draw a Venn diagram and label the regions in it 1, 2, 3, 4 so that they correspond to the truth table.}{1in}

\problem{Let $A, B,$ and $C$ be sets.  Show that $(A \bigtriangleup B) \bigtriangleup C = A \bigtriangleup (B \bigtriangleup C)$ in these ways.
\blist{0.1in}
\item Draw separate Venn diagrams for the two sets.
\item Show set inclusion both ways.
\item Convert inclusion in $A \bigtriangleup B$, $B \bigtriangleup C$, and other sets to logical statements and use a truth table to show the equality.
\elist
}{0in}

\vfill          % pad the rest of the page with white space

\yourname

\activitytitle{Infinite unions, intersections, and a few other things}{This activity is a prequel to working with infinite unions and intersections.}

\overview{Infinite unions and intersections take a bit of getting used to.  Fortunately, we can understand them with quantifiers.}

\definition{Union}{Let $A_1, A_2, \ldots$ be sets, with universe $X$.
The {\em union} of $A_1, A_2, \ldots$, which is denoted $\bigcup_{n=1}^{\infty} A_n$, is all elements of $X$ which are in $A_n$ for some $n = 1, 2, 3, \ldots.$}

\definition{Intersection}{Let $A_1, A_2, \ldots$ be sets, with universe $X$.
The {\em intersection} of $A_1, A_2, \ldots$, which is denoted $\bigcap_{n=1}^{\infty} A_n$, is all elements of $X$ which are in $A_n$ for all $n = 1, 2, 3, \ldots.$}

\problem{Use quantifiers to express what it means that $x \in \bigcup_{n=1}^{\infty} A_n.$\\
{\bf Solution:} $\exists n, x \in A_n$.
In words, there is at least one $n$ for which $x$ is in $A_n$; that is what it takes to be in the union.}{0.2in}

\problem{Work with quantifiers to express what it means that $x \notin \bigcup_{n=1}^{\infty} A_n.$
Negate the previous expression and use rules of quantifiers to rewrite it, one small step at a time, until it is as simple as possible.}{1.5in}

\problem{Use quantifiers to express what it means that $x \in \bigcap_{n=1}^{\infty} A_n.$}{1.0in}

\problem{Work with quantifiers to express what it means that $x \notin \bigcap_{n=1}^{\infty} A_n.$
Negate the previous expression, then rewrite again using complements.}{1.5in}

\stop{Go back to each of the four preceding problems and write a sentence explaining the logic of the last expression that you wrote down and how it relates to the expression you started with.}

\problem{de Morgan's law.  Show that $\left( \bigcup_{i \in I} A_i \right)^c = \bigcap_{i \in I} A_i^c$ by writing logical expressions for $x$ being in the set on the left side and for the right side.
Note that here the union is over sets $A_i$ where the index $i$ comes from an index set $I$, but the logic is the same as in the previous problems.
Start by writing a logical expression that means the same thing as $x \in \left( \bigcup_{i \in I} A_i \right)^c$ and work with it until it is a logical expression for $x \in \bigcap_{i \in I} A_i^c$.
When you write the proof this way, you do not need to show containment both ways to show that the two sets are equal.

$x \in \left( \bigcup_{i \in I} A_i \right)^c$ means $\lnot (\exists i \in I, x \in A_i)$, which means \ldots
}{1.0in}

\problem{Show that $(3,\infty) \subset [3,\infty)$; these are both intervals on the real number line.
Remember that when you show $\subset$ there are two things to show:  containment, and that there is an element of one set that is not an element of the other.  Solve this problem by letting $x \in (3,\infty)$ and writing that information as the logical statement ``$x > 3$ is true''.}{0.7in}

\problem{Show that $[2,5) \cap (3,7) \subseteq (3,5)$ using inequalities.
Start by letting $x \in [2,5) \cap (3,7)$.}{0.8in}

\problem{Let $x > 0$.  Show that there exists an integer $n$ such that $0 < \frac{1}{n} < x$.
{\bf Hint:} Look at $\frac{1}{x}$ and round up.
{\bf Another hint:} Suppose that $x = 0.31$. What value of $n$ works?}{1.3in}

\problem{Let $n$ be an integer greater than 0.  Show that $[\frac{1}{n},1] \subseteq (0,1] \subseteq [0,1]$ by working with inequalities.  Then show that $[\frac{1}{n},1] \subset (0,1] \subset [0,1]$ by looking at individual points.}{0in}
\vfill          % pad the rest of the page with white space

\yourname

\activitytitle{Infinite operations on sets}{Unions, intersections, and complements of infinitely many sets}

\overview{We are working with sets of real numbers.  These exercises will give you practice with sets and teach you things about the real numbers as well.}

\problem{Let $A = \bigcup_{n=1}^{\infty} [\frac{1}{n}, 1]$.  
List out the first five sets in this union.  
Draw a picture of them above a number line.  
Make a conjecture about what interval $A$ is equal to, call the new set $B$, then show that $A = B$ by showing containment both ways.
You will need to use this property of real numbers:  if $x > 0$, then there exists a positive integer $n$ with $0 < \frac{1}{n} < x$.}{3in}

\problem{Let $B = \bigcup_{n=0}^{\infty} [n, n^2]$.
List out the first five or more sets in this union.
Draw them on a number line if it helps.
Make a conjecture about how you can write $B$ in a simpler way, call the new set $C$, then prove that $B = C$ by showing containment in both directions.}{3in}

\pagebreak

\problem{For $n = 2, 3, 4, \ldots,$ let $C_n = \{ 2n, 3n, 4n, \ldots \}$.
\blist{1in}
\item Write out the first five of the $C_n$.
\item Let $D = \bigcup_{n=2}^{\infty} C_n$.
Describe the set $D$ in simpler terms, perhaps by writing out the smallest 10 elements of $D$.
\item What is $\N \backslash D$?  Remember that $\N = \{0, 1, 2, 3, \ldots\}$.
\elist
}{0in}

\problem{Let $I$ be a set, and for each $i$ in $I$, let $A_i$ be a set, all subsets of the same universe $X$.
Show de Morgan's law:  $\left( \bigcup_{i \in I} A_i \right)^c = \bigcap_{i \in I} A_i^c$ by showing set containment in both directions.
I hope you will find that it is actually easier to do this for a collection of sets than for two sets.
}{3in}

\problem{Let $E = \bigcap_{n = 1}^{\infty} [0, 1+\frac{1}{n}]$.
List out the first five sets in this union.  
Draw a picture of them above a number line.  
Make a conjecture about what interval $E$ is equal to, call the new set $F$, then prove that $E = F$ by showing containment both ways.

{\bf Hint:} You may want to show that $E \subseteq F$ by showing the logically equivalent statement that $F^c \subseteq E^c$.
This is the same as the contrapositive:  suppose that $y \notin F$, then show that $y \notin E$.
You may find it useful to keep in mind that if $x > 0$, then there exists an integer $n$ for which $0 < \frac{1}{n} < x$.
}{3in}

\problem{Let $F = \bigcup_{r \in \Q} (r-\frac{1}{10}, r+\frac{1}{10})$.
Here, $\Q$ is the set of all rational numbers.
Make a conjecture about a simpler way to describe the set $F$, then prove your conjecture by showing set containment both ways.}{3in}

\problem{Let $G = \bigcup_{k \in \Z} (k, k+1)$.
\blist{1in} 
\item Draw out some of the intervals here.
\item Make a conjecture about what set $G$ is.
\item Use one of de Morgan's laws to re--express $G^c$ as an intersection.
Does that help?
\item What is easier to describe, or to think of, $G$ or $G^c$?
Give the simplest description.
\elist}{0in}

\problem{Let $a < b$.  Show that $\bigcup_{n=1}^{\infty} [a, b-\frac{1}{n}] = [a,b)$.  Draw pictures, then show set inclusion both ways.  Is there a problem if $b - \frac{1}{n} < a$?}{2in}

\problem{Let $a < b$.  Show that $\bigcap_{n=1}^{\infty} [a,b+\frac{1}{n}) = [a,b]$.  Draw pictures, then show set inclusion both ways.}{2in}


\vfill          % pad the rest of the page with white space

\yourname

\activitytitle{The power set and the Cartesian product}{Useful constructions with sets.}

\overview{The power set is our first example of thinking hard about collections of sets.  The Cartesian product is used often when you want ordered pairs or ordered triples of numbers or other objects.}

\problem{Write out the members of the following power sets.  It may be helpful to do \#3, \#4, then \#2, \#1, and finally \#5.
\blist{0.1in}
\item $S = \emptyset$.  ${\cal P}(S) = $
\item $S = \{ 1 \}$.  ${\cal P}(S) = $
\item $S = \{ 1,2 \}$.  ${\cal P}(S) = $
\item $S = \{ 1,2,3 \}$.  ${\cal P}(S) = $
\item $S = \{ 1,2,3,4 \}$.  ${\cal P}(S) = $
\item $S = \{ 1,2,3,4,5 \}$.  ${\cal P}(S) = $
\elist
}{0in}

\question{If $S$ has $n$ elements, how many members will ${\cal P}(S)$ have?  Explain as well as you can.}{1in}

\problem{Write the appropriate symbol between the entities, or mark the statement as true or false.  Give an explanation for anything that is not obvious enough.
\blist{0.0in}
\item $ 1 \quad\quad {\cal P}(\{ 1, 2, 3 \})$
\item $[3, 10] \quad\quad \Z$
\item $[3, 10] \quad\quad \R$
\item $\Q \quad\quad \R$
\item $\Q \quad\quad {\cal P}(\R)$
\item $[3, 10] \quad\quad {\cal P}(\R)$
\item $\N \quad\quad \R$
\item $\emptyset \quad\quad \R$
\item $\emptyset \quad\quad {\cal P}(\R)$
\item $\{ \emptyset \} \subseteq A$?
\item $\emptyset \subset {\cal P}(A)$?
\elist
}{0.0in}

\question{Suppose that $S$ is a set.  Then ${\cal P}(S)$ is also a set, but if we let $A \in {\cal P}(S)$, then $A$ is also a set.
Explain how this can be.
What is the relationship between $A$ and $S$?}{1.5in}

\problem{Let $I$ be a set, and for each $i$ in $I$, let $B_i$ be a set.
Show that ${\cal P}\left( \bigcap_{i \in I} B_i \right) = \bigcap_{i \in I} {\cal P}(B_i)$.\\
Let $A$ be an element of the set on the left-hand side.  Notice that $A$ is a set.  Argue that it is an element of the set on the right-hand side.
\vspace{1in}
Let $A$ be an element of the set on the right-hand side \ldots}{1in}

\problem{
\blist{0.75in}
\item Sketch the Cartesian product $A = [1,3] \times [2,5]$.
\item Sketch the Cartesian product $B = [2,4] \times [1,3]$.
\item Sketch the intersection $A \cap B$.
\item It seems that $A \cap B$ is also a Cartesian product.  Identify the sets whose product is $A \cap B$.
\item What is $([1,3] \cap [2,4]) \times ([2,5]\cap[1,3])$?
\elist
}{0.0in}

\vfill          % pad the rest of the page with white space

\yourname

\activitytitle{Relations}{Often treated as the little brother to functions, relations have unsuspected depth.}

\overview{You are already familiar with a number of relations, including $<, \leq, =, \geq,$ and $>$ for real numbers, plus $\subset$, $\subseteq$, and $=$ for sets.  Many other relations can be defined.  The most useful ones are called equivalence relations; they are analogous to equality for numbers and for sets.  They partition the space into equivalence classes, which are very useful in a number of ways.}

\definition{Relation}{A {\em relation} on a set $X$ is a subset $S$ of $X \times X$.}

\notation{Suppose that $S$ is a relation on a set $X$.
That is, suppose that $S$ is a subset of $X \times X$, which means that $S$ is a set of points of the form $(x,y)$, where $x \in X$ and $y \in X$.
Rather than write $(x,y) \in S$, we usually write $x \sim y$.
How to read this out loud?  There is no perfect solution.  I would suggest that you read it as ``$x$ tilde $y$'' because $\sim$ is the tilde that appears above the n in some Spanish words.}

\problem{
You are going to write out the subsets of $\{1, 2, 3, 4\}$ and then draw arrows between them to indicate the proper subset relation.  You might want to lay the sets out in a nice order to make the arrows easy to draw and to read.
What is the set $X$ on which this relation is defined?
}{2in}

\definition{Reflexive}{A relation $\sim$ is {\em reflexive} if $x \sim x$ for all $x$ in $X$.}

\definition{Symmetric}{A relation $\sim$ is {\em symmetric} if $x \sim y$ implies $y \sim x$.}

\definition{Transitive}{A relation $\sim$ is {\em transitive} if $x \sim y$ and $y \sim z$ implies $x \sim z$.}

\note{Sometimes people have a hard time remembering the words reflexivity, symmetry, and transitivity.  Notice that they are in alphabetical order, and that they involve 1, 2, or 3 objects at a time, respectively.}

\definition{Equivalence relation}{A relation $\sim$ is called an {\em equivalence relation} if it is reflexive, symmetric, and transitive.  Note that equality is an equivalence relation on the set of real numbers.}

\definition{Equivalence class}{Suppose that $\sim$ is an equivalence relation.  Fix $x$ in $X$.  The set of all elements $y$ for which $x \sim y$ is called the {\em equivalence class containing $x$}.}


\problem{Let $X = \Z^+$ and say that $x \sim y$ if $y$ is divisible by $x$.
People often write $x | y$ for this relation and say that $x$ divides $y$.
\blist{0.8in}
\item Check whether this relation is reflexive.  If so, prove that it is, starting with ``Let $x \in \Z^+$.''  If not, give a counterexample.
\item Check whether this relation is symmetric.  If so, prove that it is, starting with ``Let $x, y \in \Z^+$ and suppose that $x \sim y$.''  If not, give a counterexample.
\item Check whether this relation is transitive. If so, prove that it is, starting with  ``Let $x, y, z \in \Z^+$ and suppose that $x \sim y$ and $y \sim z$.''  If not, give a counterexample.
\item Thinking of the relation as a set of ordered pairs, write out ten different ordered pairs satisfying the relation, and graph them on the $xy$ plane.
\elist
}{0.5in}

\problem{Consider all cities in the US that have population over 30,000.
For each of the following relations, determine whether they are reflexive, symmetric, and/or transitive.  Provide a counterexample for any property that fails to hold.  If all three hold, the relation is an equivalence relation.  In that case, identify the equivalence classes and tell how many such classes there are.
\blist{1.2in}
\item Say that $x \sim y$ if the names of cities $x$ and $y$ start with the same letter.
\item Say that $x \sim y$ if $x$ and $y$ are in the same state.
\item Say that $x \sim y$ if cities $x$ and $y$ are within 50 miles of each other.
\elist
}{1.2in}

\problem{Let $X$ be the set of all English words.
Say that $x \sim y$ if the letters in $x$ and $y$ appear on the same number keys on a cell phone, in the same order.
For example, BAR $\sim$ CAP.
\blist{0.5in}
\item Check whether this relation is reflexive.
\item Check whether this relation is symmetric.
\item Check whether this relation is transitive.
\item If all three properties hold, describe the equivalence classes, and tell what the equivalence class of BAR is.
\elist}{0.5in}


\problem{
Let $X = \Z$.
Say that $x \sim y$ if $x$ and $y$ has the same remainder as $y$ when they are divided by 3.
Then, for example, $13 \sim 19$ and $9 \sim 27$.
\blist{0.5in}
\item Show that this relation is reflexive, symmetric, and transitive.
Do this in general, starting with ``Let.''
\vspace*{0.8in}
\item Describe all elements of the equivalence class containing 0.
\item Describe the other equivalence classes.  How many are there?
\elist
}{0.3in}

\problem{Consider the set $X$ of all non--zero 3--dimensional vectors.
For $\vec{a}$ and $\vec{b}$ in $X$, say that $\vec{a} \sim \vec{b}$ if there exists a constant $c$ for which $\vec{a} = c \vec{b}$.
\blist{0.8in}
\item Show that this relation is reflexive, starting with ``Let.''  Tell what $c$ is.
\item Show that this relation is symmetric.  You will need two values of $c$.
\item Show that this relation is transitive.  Here there will be three values of $c$.
\item This is an equivalence relation.  Describe the equivalence classes.  The collection of all equivalence classes is called {\em projective space}.
\item Could you use the angle between lines to define a distance between equivalence classes?  What would the maximum distance be?
\elist}{0in}

\problem{Consider the set of all English words.
Say that $x \sim y$ if one can be obtained from the other by changing exactly one letter.
For example, BAT $\sim$ CAT but BAT $\nsim$ CAR.
Check whether this relation is reflexive, symmetric, and/or transitive.
Provide counterexamples if necessary.}{1in}

\problem{Consider the set of all functions on the real line.
That is, consider the set of all $f : \R \to \R$.
Say that $f \sim g$ if $f$ and $g$ are equal except at a finite number of points.
For example, if $f(x) = x^2$ and $g(x) = \left\{ \begin{array}{cl} x^2, & x \ne 0 \\ 5, & x = 0 \end{array} \right.,$ then $f \sim g$.
Show that this is an equivalence relation.
How can you describe the equivalence classes?}{1.5in}

\note{Problems 10.1 and 10.3 from Daepp and Gorkin\DGreference~are particularly good at this stage in the course.}
\vfill          % pad the rest of the page with white space

\activitytitle{Homework problems, week 12}{Due on (put date here).}

Write up solutions of each of the problems below.
They are designed to be straightforward problems.
The goal is to come as close to perfection in your solutions as you can.
\begin{itemize} \itemsep 1pt
\item Do not take shortcuts.
\item If you need to show that something is true for all $n$, or for all $x,y$, start the proof with ``Let \ldots''
\item If you need cases, explain what the cases are and why they cover all the possibilities.
\item If you are doing a proof by contradiction, start that part by saying ``Assume \ldots''
\item If you are doing a proof by contrapositive, tell what $P$ and $Q$ are, and that you will be showing that $\lnot Q$ implies $\lnot P$.
\item Take small steps in each proof, and explain each step.
\item Follow good form.
\item If your proof started with ``Let \ldots'' it will probably end by saying ``We made no further assumption \ldots''
\end{itemize}
Here are the problems to do.  You can write them in your notebook or on separate paper.
\blist{0.1in}
\item Show that if $n$ is an integer and $7n$ is odd, then $n$ is odd.
{\bf Hint:} Be clear what facts you are using about even and odd numbers.

\item Without consulting your book or your notes, prove that $\sqrt{2}$ is irrational.
I mean it.  
Do this from memory.
You should be able to write a very nice proof, with no missing steps.

\item Let $x$ and $y$ be real numbers, and suppose that the product $xy$ is irrational.
Show that either $x$ or $y$ (or both) must be irrational.
{\bf Hint:} You can do this.  Be patient, think about it.

\item Let $A = \{2k+1 : k \in \Z\}$ and let $B = \{ 2m-11 : m \in \Z \}$.
Show that $A = B$ by showing containment both ways.
{\bf Hint:}  Use good form!

\item Let $A = \{ (x,y) \in \R^2 : y = 5x/7 - 2/7 \}$ and $B = \{ (x,y) \in \R^2 : 5x - 7y = 2 \}$.
Show that $A = B$ by showing containment both ways.

\item Let $A = \{ m \in \Z : m = 15k$ for some $k \in \Z \}$, let $B = \{ m \in \Z : m = 35j$ for some $j \in \Z \}$, and let $C = \{ m \in \Z : m = 105n$ for some $n \in \Z \}$.
Show that $A \cap B = C$ by showing containment both ways.
One direction is easier than the other.
Label one of them ``the easy direction'' and the other ``the hard direction''.
{\bf Hint:} Yes, we worked on a problem just like this in class.
Don't go back and find it, work through this one on your own.
{\bf Another hint:}  In the hard direction, you should come to something like $3k = 7j$ where $j$ and $k$ are integers.
You will need to conclude that $j$ is a multiple of 3.
If you are up for the challenge, show this using the division algorithm.
Don't use any ideas about prime factorization.
\elist

\vfill          % pad the rest of the page with white space

\yourname

\activitytitle{Inequalities}{We can define the $<$ relation for real numbers and establish its properties.}

\overview{Most students at your level take the real numbers as things that simply exist and have a number of properties such as commutativity.  In fact, the real numbers can be {\em constructed} from the rational numbers, and the rational numbers from the integers, and the integers from the positive integers.  In this activity, we back up to the point that the real numbers have been constructed, but before inequalities have been defined.  We define the $<$ relation and prove a number of useful properties that it satisfies.  Since the $>$ relation is so similar, we will not define it or show its properties.}

\note{\label{realnumberproperties}Let $\R$ denote the set of real numbers, and denote addition and multiplication of real numbers in the usual ways.
{\bf Addition} has these properties:  commutativity ($a+b = b+a$), associativity ($a + (b+c) = (a+b) +c$), additive identity (there exists a unique real number called 0 for which $a + 0 = a$ for all $a \in \R$), and additive inverse (for each number $a$ in $\R$, there exists a unique real number $-a$ for which $a + (-a) = 0$).
{\bf Multiplication} has these properties:  commutativity ($ab = ba$), associativity ($a(bc) = (ab)c$), multiplicative identity (there exists a unique real number called 1, with $1 \ne 0$, such that $a\cdot 1 = a$ for all $a$ in $\R$), multiplicative inverse (for each $a$ in $\R$ with $a \ne 0$, there exists a unique number called $a^{-1}$ for which $a \cdot a^{-1} = 1$.
{\bf Addition and multiplication} are related by the distributive property: ($(a+b)c = ac + bc$).}

\definition{Subtraction}{Let $a$ and $b$ be real numbers.  The difference of $a$ and $b$, denoted $a - b$, is the real number $a + (-b)$, where $(-b)$ denotes the additive inverse of $b$.}

\note{\label{negativeproperties}At the end of this activity, you will see how to establish the following useful properties regarding additive inverses and subtraction:
\balist{0in}
\item $a\cdot 0 = 0$ for all real numbers $a$.
\item The additive inverse $(-a)$ is equal to $(-1)\cdot a$, where $(-1)$ is the additive inverse of 1
\item $(-1)(-1) = 1$.
\item The additive inverse of $a+b$ is $(-a)+(-b)$.  Using subtraction notation, $-(a+b) = -a-b$.
\item $-(-a) = a$.
\ealist
You can use subtraction as usual in this activity, but if you would like to avoid subtraction notation and just use additive inverses, that is worth attempting.}

\definition{Positive real numbers\label{positivereals}}{By construction, the real numbers have a subset $\Rp$, called the {\em positive real numbers,} for which:
\balist{0.3in}
\item If $a,b \in \Rp$, then $a + b \in \Rp$.  ($\Rp$ is closed under addition.)
\item If $a,b \in \Rp$, then $a\cdot b \in \Rp$.  ($\Rp$ is closed under multiplication.)
\item For every real number $a$, either $a \in \Rp$ or $(-a) \in \Rp$ or $a = 0$.  Exactly one of the three happens.
\ealist
\vspace{0.2in}

Under each property above, write a sentence that states it in plain English.
Think of $\Rp$ as being the positive, non--zero numbers.
We can't use interval notation to write what $\Rp$ is, because intervals are defined in terms of inequalities, and we have not defined inequalities yet!}

\show{Let $a \in \R$ and suppose that $a \ne 0$.
Show that $a\cdot a \in \Rp$.
When you use properties from \ref{negativeproperties} or \ref{positivereals}, cite them by number.
{\bf Hint:} There are two cases left in \ref{positivereals}c.}{0.9in}

\show{Show that $1 \in \Rp$.  Be careful to cite any previous properties that you use.}{0.5in}

\show{Show that $(-1) \notin \Rp$.
{\bf Hint:} Assume that $(-1) \in \Rp$ and use \ref{positivereals}a.}{0.5in}

\definition{Less than}{\label{lessthan}Let $a$ and $b$ be real numbers.  We write that $a < b$ if $b - a \in \Rp$.}

\note{All of the following problems rely on  Definition \ref{lessthan}, so you will use it over and over.
Note that $>$ has not been defined yet, so be careful not to use it.}

\show{Show that $-1 < 0$. {\bf Hint:} use \ref{negativeproperties}d.}{0.5in}

\show{Show that $1 < 1$ is not true.
Thus, the $<$ relation is not reflexive.
When you use properties from \ref{negativeproperties} or \ref{positivereals}, cite them by number.
}{0.9in}

\show{Show that $0 < 1$ but that $1 < 0$ is not true.
Thus, the $<$ relation is not symmetric.}{0.9in}

\show{Show that the $<$ relation on $\R$ is transitive.
Follow good form by first letting $a, b, c$ be real numbers and supposing that $a < b$ and $b < c$.
When you use properties from \ref{negativeproperties} or \ref{positivereals}, cite them by number.
You may enjoy ticking off the properties of the real numbers that you use.
For example, in this proof, you are likely to use the fact that $(-b) + b = 0$, which is the additive inverse property.}{0.9in}

\show{Let $a,b \in \R$ and suppose that $a < b$.
Show that $-b < -a$.
When you use properties from \ref{negativeproperties} or \ref{positivereals}, cite them by number.
}{0.9in}

\show{Let $a, b, c \in \R$.  Suppose that $a < b$.  Show that $a+c < b+c$.
When you use properties from \ref{negativeproperties} or \ref{positivereals}, cite them by number.
}{0.9in}

\show{Let $a, b, c, d \in \R$.  Suppose that $a < b$ and $c < d$.  Show that $a+c < b+d$.
When you use properties from \ref{negativeproperties} or \ref{positivereals}, cite them by number.
}{0.9in}

\show{Let $a, b, c$ be real numbers.  Suppose that $a < b$ and $0 < c$.
Show that $ac < bc$.
}{0.9in}

\show{Let $a, b, c$ be real numbers.  Suppose that $a < b$ and $c < 0$.
Show that $bc < ac$.
}{0.9in}

\show{Let $a,b \in \R$ and suppose that $0 < a$ and $b < 0$.
Use a previous result to show that $ab < 0$.}{0.9in}

\show{Let $a \in \R$ and suppose that $0 < a$.
Show that $0 < a^{-1}$.
Here $a^{-1}$ is the multiplicative inverse of $a$.
{\bf Hint:}  This one take a bit more effort than the previous ones.
Note that division has not been defined yet, so just use addition, subtraction, and multiplication.}{0.9in}

\show{Let $a,b \in \R$ and suppose that $0 < a$ and $a < b$.
Show that $b^{-1} < a^{-1}$.}{0.9in}

\note{Below, you are asked to prove basic properties of additive inverses and subtraction.}

\show{Let $a$ be a real number.  Show that $a \cdot 0 = 0$.}{0.4in}

\show{People sometimes ask if the additive inverse $(-a)$ is the same as $(-1)\cdot a$, where $(-1)$ is the additive inverse of 1.
It's true, and here is how you show it; you should fill in steps and write the justifications at the right side of each line.
\begin{eqnarray*}
  a + (-1)\cdot a &=& 1\cdot a + (-1) \cdot a \\
                  &=& (1 + (-1)) \cdot a \\
                  &=&  \\
                  &=& 0,
\end{eqnarray*}
This shows that $(-1) \cdot a$ is the additive inverse of $a$.}{0in}

\show{You might think that it is obvious that $(-1)(-1) = 1$, where $(-1)$ is the additive inverse of 1, but this takes a few steps beyond the properties of the real numbers in \ref{realnumberproperties}.  Write justifications and complete the following steps to show it.
\begin{eqnarray*}
(-1) + (-1)(-1) &=& (-1)(1) + (-1)(-1) \\
                &=& (-1)(1 + (-1)) \\
                &=& \\
                &=& 0,
\end{eqnarray*}
which shows that $(-1)(-1)$ is the additive inverse of $-1$, which is 1.}{0in}

\show{The additive inverse of a sum works out nicely.
Let $a$ and $b$ be real numbers and think about the additive inverse of $a+b$.
Write justifications to the right of each statement.
\begin{eqnarray*}
-(a+b) &=& (-1)(a+b) \\
       &=& (-1)(a) + (-1)(b) \\
       &=& (-a) + (-b)
\end{eqnarray*}}{0in}

\show{Let $a \in \R$.  The statement $-(-a) = a$ is just a statement about additive inverses.  Prove that it is true.}{0in}


\vfill          % pad the rest of the page with white space

\yourname

\activitytitle{Mathematical Induction}{Proving that a claim is true for all $n=1, 2, 3, \ldots$.  }

\overview{One important task in mathematics is to find regular patterns and prove that they hold.
The main method we use for this is mathematical induction.
Co-authored by Ying-Ju Chen.}

\theorem{\bf Mathematical induction\label{MItheorem}}\\
{For each integer $n=1, 2, 3, \ldots$, let $P(n)$ denote a true/false statement involving $n$.
\begin{itemize}
\item[(i)] (The basis step) Prove that $P(1)$ is true.
\item[(ii)] (The inductive step) For each $n = 1, 2, 3, \ldots$, suppose that $P(n)$ is true, and use $P(n)$ to prove that $P(n+1)$ is true.
\end{itemize}
From the above two steps, we can conclude that $P(n)$ is true for all $n = 1, 2, 3, \ldots$.}

\note{Proving the inductive step is usually done as a ``rewrite'' proof, where you start with the left hand side of what you want to show and rewrite until you come to the desired right hand side.
Often, some quantity in the statement $P(n+1)$ can be written in terms of a similar quantity in the statement $P(n)$ plus a new part.
You will always use the fact that $P(n)$ is true.}

\question{A student working on an induction problem wrote $P(n+1) = P(n) + \frac{1}{n^2}$.
How can you tell that this must be wrong?}
{0.2in}

\guidedproof{\label{inductionguidedproof}
Use mathematical induction to show that $3^n$ is odd for all $n = 1, 2, 3, \ldots$.
\balist{0.0in}
\item State $P(n)$:  $P(n)$ is that \blank{2in}
\item Basis step:  $P(1)$ is that \blank{1.5in}.  This is true because \blank{1.5in}.
\item State $P(n+1)$:  $P(n+1)$ is that \blank{2in}
\item Inductive step:  Let $n \geq 1$.  suppose that $P(n)$ is true.  Show that $P(n+1)$ is true.
You may use facts you have already proven about odd numbers.

\noindent
Because $P(n)$ is true, \blank{1.5in}.
Now $3^{n+1}=$ \blank{1in}, which is odd because \blank{3.5in}.
Thus $P(n+1)$ is true.
Since $n \geq 1$ was arbitrary, by mathematical induction, $P(n)$ is true for all $n \geq 1$.
\elist
}

\exercise{Fill in the table using your powers of pattern recognition.

\vspace*{0.1in}
\label{examplesum}
\centerline{
\begin{tabular}[c]{c|c|c|c|c|c|c|c|c|c|c}
  $n$ & 1 & 2 & 3 & 4 & 5 & 6 & 7 & $\cdots$ & $n$ & $n+1$ \\
  \hline
  first $n$ odd integers & 1 & 3 & 5 & 7 & & & & $\cdots$ & & \\
  \hline
  sum of first $n$ odd integers & ~1~ & ~4~ & \hspace{0.2in} & \hspace{0.2in} & \hspace{0.2in} & \hspace{0.2in} & \hspace{0.2in} & $\cdots$ & \hspace{0.5in} & \hspace{0.5in} \\
\end{tabular}}

\vspace*{0.1in}
\noindent
Column $n$ of the table contains a conjecture about the sum of the first $n$ odd integers.
In the next problem, you will use mathematical induction to prove it.
}{0.0in}

\pagebreak
\guidedproof{\label{oddsum}
Prove the conjecture in \ref{examplesum} using mathematical induction.
\balist{0.1in}
\item State $P(n)$:  $P(n)$ is that the sum of the first $n$ odd integers equals \blank{1in}
\item Basis step:  $P(1)$ is that \blank{2.25in}.  This is true because \blank{0.75in}.
\item Write out $P(n+1)$:  $P(n+1)$ is that \blank{4in}
\item Inductive step: Let $n \geq 1$.  suppose that $P(n)$ is true, and use that to show that $P(n+1)$ is true.
\begin{eqnarray*}
\lefteqn{\mbox{the sum of the first $n+1$ odd integers}} \\
&=& \mbox{the sum of the first $n$ odd integers plus \blank{1in}} \\
&=& \blank{1.5in} + \blank{1.5in} \mbox{since $P(n)$ is true} \\
&=& \blank{1.5in}
\end{eqnarray*}
Thus $P(n+1)$ is true.
Since \blank{1in} was arbitrary, by \blank{2in} we conclude that the sum of the first $n$ odd integers equals \blank{1in} for all $n = 1, 2, 3, \ldots$.
\elist
}

\notation{Summation notation for $a_1+a_2+a_3+\cdots+a_n$ is $\displaystyle \sum_{k=1}^{n}a_k$.}

\example{Use summation notation to rewrite the result in \ref{oddsum}.}{0.2in}

\exercise{Fill in the blanks.

{\bf a.} $\displaystyle \sum_{k=1}^{n+1} k^2 = \sum_{k=1}^{n} k^2 + \blank{1in}$
\qqqq
{\bf b.} $\displaystyle \sum_{k=1}^{n+1} \frac{1}{k^3} = \sum_{k=1}^{n} \frac{1}{k^3} + \blank{1in}$
}{0in}

\show{Show that $\sum_{k=1}^n 4k-3 = n(2n-1)$ for all $n = 1, 2, 3, \ldots$.
\balist{0.0in}
\item State $P(n)$:  $P(n)$ is that
\item Basis step:  $P(1)$ is that  \hfill Check: \blank{1.5in}
\item State $P(n+1)$:  $P(n+1)$ is that

\vspace*{0.2in}
\noindent
Suggestion:  Use algebra to simplify the right hand side.

\item Inductive step: Let $n \geq 1$.  suppose that $P(n)$ is true, and use that to show that $P(n+1)$ is true.

\vspace*{-0.2in}
\begin{eqnarray*}
\sum_{k=1}^{n+1} 4k-3 &=& \blank{1.5in} + \blank{1.5in} \\
                                         &=& \blank{1.5in} + \blank{1.5in} \qq \mbox{since $P(n)$ is true} \\
                                         &=& \\
                                         &=& 
\end{eqnarray*}
Thus, \blank{1.5in}.  Since ... 
\elist
}{0.0in}

\stop{Compare your proofs with the other people in your group before you move on.}

\show{Use mathematical induction to show that $\sum_{k=1}^n 5^k = \frac{5}{4} (5^n-1)$ for all integers $n \geq 1$.
\balist{0.1in}
\item State $P(n)$:  
\item Basis step:  
\item State $P(n+1)$:  

\vspace*{0.2in}
\noindent
Suggestion:  Multiply out the right hand side.

\item Inductive step: Let $n \geq 1$.  suppose that $P(n)$ is true, and use that to show that $P(n+1)$ is true.
\elist
}{0.9in}

\note{The basis step need not use $n=1$, for example, it can use $n=-3, n= 0,$ or $n = 100$.}

\show{Use mathematical induction to show that $2n+1 < 2^n$ for all integers $n$ with $n \geq 4$.
\balist{0.10in}
\item State $P(n)$:
\item Basis step:  $P(4)$ is that:  \hfill Check: \blank{1.5in}
\item State $P(n+1)$:  
\item Inductive step: Let $n \geq 4$.  suppose that $P(n)$ is true, and use that to show that $P(n+1)$ is true.
\begin{eqnarray*}
2(n+1) + 1 = \blank{1.5in} &=& \blank{1.5in} \\
                                            &<& \blank{1.5in} \qq \mbox{since $P(n)$ is true}\\
                                            &<& \blank{1.5in} \\
                                            &=& 2^{n+1}.
\end{eqnarray*}
Thus, \blank{1.5in}.  Since ... 
\elist
}{0in}

\show{Use mathematical induction to show that $5^n > 2^n + 3^n$ for all integers $n$ with $n \geq 2$.  Use the same format as above.
\balist{0.0in}
\item
\item
\item
\item
\elist
}{1.5in}

\show{Use induction to prove Bernoulli's inequality: For all $x\in \R$, if $1+x >0$, then $(1+x)^n \geq 1+nx$ for all $n = 0, 1, 2, \ldots$.  Use the same format as above.

\noindent
Let $x$ be such that $1+x > 0$.

\vspace*{1.5in}
\noindent
Where did you use the assumption that $1+x > 0$?
}{0in}

\show{Use induction to prove that $\frac{1}{1\cdot 2} + \frac{1}{2\cdot 3} + \frac{1}{3\cdot 4} + \cdots+\frac{1}{n\cdot (n+1)} = \frac{n}{n+1}$ for all positive integers $n$. \label{sumeq}}{1.5in}

\note{It is possible to show that the statement in \ref{sumeq} is true without using mathematical induction, but using a different algebraic technique.  How?}

\show{For each $n\in \Z^+$, let $P(n)$ denote the statement ``$n^2+5n+1$ is an even integer.''
\balist{0.05in}
\item State $P(n+1)$
\item Suppose that $P(n)$ is true, and use that to prove that $P(n+1)$.
\vspace*{0.5in}
\item For which $n$ is $P(n)$ actually true? 
\item What is moral of this exercise?
\elist
}{0in}

\show{Use induction to prove that $n^3-n$ is a multiple of 6 for all integers $n = 0, 1, 2, \ldots$.
Do the induction step as a ``rewrite'' proof.}{1.5in}

\show{Use induction to prove that $11^n-4^n$ is a multiple of 7 for all $n = 0, 1, 2, \ldots$.
Do the induction step as a ``rewrite'' proof.
\Hint Use the equation $11^n-4^n = 7k$ once.}{1.5in}

\show{Prove that $1^2-2^2+3^2-4^2+5^2+\cdots-(2n)^2+(2n+1)^2 = (n+1)(2n+1)$ for all $n = 0, 1, 2, \ldots$. \Hint{ It would be helpful to write down $P(0)$ and $P(1)$ first.}}{1.5in}

\prove{Let $k > 0$ be an integer.
For each integer $n = 0, 1, 2, \ldots$, let $P(n)$ be the statement: ``There exist integers $q$ and $r$ with $0 \leq r < k$ such that $n = kq + r$.''
Use mathematical induction to show that $P(n)$ is true for all integers $n$.

\balist{0.5in}
\item Show that $P(0)$ is true.

\item Suppose that $P(n)$ is true and show that $P(n+1)$ is true.
It is helpful to do this with two cases.

\noindent
Case 1.  Suppose that $n = qk + r$ where $0 \leq r < k-1$.

\vspace*{0.5in}

\noindent
Case 2.  Suppose that $n = qk + r$ where $r = k-1$.

\item Suppose that $P(n)$ is true and show that $P(n-1)$ is true.
It is helpful to do this with two cases.
\elist

\vspace*{1.5in}
\noindent
Use steps b and c and the idea of mathematical to conclude the proof.

}{0in}

\show{
Use mathematical induction to prove that $\displaystyle \sum_{k=1}^{n} k = \frac{1}{2}n(n+1)$ for all integers $n = 1, 2, 3, \ldots$.
}{1.5in}

\show{
Use mathematical induction to prove that $\displaystyle \sum_{k=1}^{n} k^2 = \frac{1}{6}n(n+1)(2n+1)$ for all integers $n = 1, 2, 3, \ldots$.
}{1.5in}

\show{Let $r$ be a real number not equal to 1.
Use induction to prove that $\displaystyle \sum_{k=0}^{n} r^k = \frac{1-r^{n+1}}{1-r}$ for all integers $n = 0, 1, 2, \ldots$.
Note where you use the assumption on $r$.
}{1.5in}

\show{Use mathematical induction to show that $n! > 3^n$ for all $n = 7, 8, 9, \ldots$.}{1.5in}


\vfill          % pad the rest of the page with white space


\yourname

\activitytitle{Quiz on inequalities}{5 points}

\definitionNN{Positive real numbers}{By construction, the real numbers have a subset $\Rp$, called the {\em positive real numbers,} for which:
\balist{0.3in}
\item If $a,b \in \Rp$, then $a + b \in \Rp$.  ($\Rp$ is closed under addition.)
\item If $a,b \in \Rp$, then $a\cdot b \in \Rp$.  ($\Rp$ is closed under multiplication.)
\item For every real number $a$, either $a \in \Rp$ or $(-a) \in \Rp$ or $a = 0$.  Exactly one of the three happens.
\ealist
}

\vspace*{0.2in}

\definitionNN{Less than}{Let $a$ and $b$ be real numbers.  We write that $a < b$ if $b - a \in \Rp$.}

\vspace*{0.5in}

\showNN{Let $a, b, c$ be real numbers.  Suppose that $a < b$ and $0 < c$.
Show that $ac < bc$.
Take very small steps and be careful to cite justifications for every single step.
}{0.9in}

\vfill          % pad the rest of the page with white space

\yourname

\activitytitle{Quiz on induction}{15 points}

\noindent
For each problem below, clearly state $P(1), P(k),$ and $P(k+1)$ as logical statements with double quotes around them.
When proving that $P(k)$ being true implies that $P(k+1)$ is true, do not write down $P(k+1)$ as if it were true, but rather start with one side and work with it until it turns into the other side.

\vspace{0.2in}

\showNN{Use induction to show that for $n > 0$, 8 divides $5^n + 2(3^{n-1}) + 1.$
{\bf Hint:}  As in other proofs of divisibility, add and subtract to be able to use $P(n)$ to simplify $P(n+1)$.}{6in}



\showNN{On the back of this piece of paper, use induction to show that for all $n \geq 1$, we have that $1(1!) + 2(2!) + \cdots + n(n!) = (n+1)! - 1$.}{0in}

\vfill          % pad the rest of the page with white space

\yourname

\activitytitle{Absolute value and related functions}{A careful development of the properties of the absolute value function.}

\overview{The absolute value function is easy to understand for numbers like 9 and $-13$, but it's harder to show its properties because our intuition works so hard to see all variables as having positive values.
In this activity, we will not use the standard notation for the absolute value function and will have to keep our intuition at bay.
We will instead rely completely on the definition.
{\bf When you use a property of inequalities, cite it by number.}}

\definition{Absolute value}{The function $f : \R \to \R$ defined by
\[
    f(x) = \left\{ \begin{array}{cl} x, & \mbox{if $x \geq 0$} \\
                          -x, & \mbox{if $x < 0$} \end{array} \right.
\]
is called the {\em absolute value} function.}

\notation{In this activity, do not use the standard notation for absolute value, not even once.
Every time you work with the absolute value function, use and cite the definition.}

\show{\label{absofproduct}Show that $f(ab) = f(a) f(b)$ for all real numbers $a$ and $b$.
Follow the model.\\

Let $a$ and $b$ be real numbers.
There are four cases.
\blist{0.5in}
\item Suppose that $a \geq 0$ and $b \geq 0$.  Then $ab \geq 0$ so $f(ab) = ab$ and $f(a) = a$ and $f(b) = b$, so $f(ab) = ab = f(a)f(b)$.
\item Suppose that $a \geq 0$ and $b < 0$.
\item Suppose that $a < 0$ and $b \geq 0$.
\item Suppose that $a < 0$ and $b < 0$.
\elist
In each case, we see that \blank{2in}.
We made no further assumptions about \blank{1in}, thus \blank{3in}.}{0in}

\show{Following the model above, show that $f(f(a)) = f(a)$ for all real numbers $a$.}{1in}

\show{Show that $f(-a) = f(a)$ for all real numbers $a$.}{1in}

\show{Show that $f(a-b) = f(b-a)$ for all real numbers $a$ and $b$.}{1in}

\show{Show that $f(a) \geq 0$ for all real numbers $a$.}{1in}

\show{Follow the model in \ref{absofproduct} to show that $f(a+b) \leq f(a) + f(b)$ for all real numbers $a$ and $b$.
When $a$ and $b$ have different signs, consider two cases, $a+b \geq 0$ and $a+b < 0$.
You will probably want to show that if $b < 0$, then $b < -b$.
Make a good, solid argument using transitivity of $<$.}{3in}

\show{Show that for real numbers $a$ and $b$, $f(a) \leq b$ if and only if $-b \leq a \leq b$.
Remember that an ``if and only if'' proof has two directions.
In both directions, you will have to consider two cases, $a \geq 0$ and $a < 0$.
Note that the statement $-b \leq a \leq b$ is equivalent to ($-b \leq a$ and $a \leq b$).}{3in}

\show{Show that for all real numbers $a$ and $b$, $f(a) \geq b$ if and only if ($a \geq b$ or $a \leq -b$).}{2in}

\show{Show that for all real numbers $a$ and $b$, $f(a) \leq f(a-b) + f(b)$. {\bf Hint:} Look at $f((a-b)+b)$.}{1in}

\show{Show that for all real numbers $a$ and $b$, $f(a) - f(b) \leq f(a-b)$ and also $f(b)-f(a) \leq f(b-a)$.}{1in}

\show{Show that for all real numbers $a$ and $b$, $f(a-b) \geq f(f(a)-f(b))$.}{1in}

\definition{Minimum function}{The function $h : [0,\infty) \to \R$ defined by 
\[
    h(x) = \left\{ \begin{array}{cl} x, & \mbox{if $x \leq 1$} \\
                                     1, & \mbox{if $x > 1$} \end{array} \right.
\]
can be called the minimum function.}

\show{Show that for all real numbers $a$, $h(a) = 0$ if and only if $a = 0$.}{1.5in}

\show{Show that if $a \leq b$, then $h(a) \leq h(b)$.}{1.5in}

\show{Show that if $h(a) < h(b)$, then $a < b$.}{1.5in}

\show{Show that for all real numbers $a$ and $b$, $h(a+b) \leq h(a) + h(b)$.
{\bf Hint:}  Use a proof by cases.  But what are the cases?}{0in}

\vfill          % pad the rest of the page with white space

\yourname

\activitytitle{Functions}{One-to-one, onto and bijective functions}

\overview{You are comfortable working with functions already. There are various ways of describing functions. Here you will learn the formal definition of a function.}

\definition{Function}{Let $X$ and $Y$ be sets. A function $f$ from $X$ to $Y$ is a relation from $X$ to $Y$ that satisfies:
\begin{enumerate}
\item for each $x \in X$ there is a $y \in Y$ such that $(x,y)\in f$, and
\item if $(x,y)\in f$ and $(x,z)\in f$, then $y=z$.
\end{enumerate}
The set $X$ is called the domain of $f$ and the set $Y$ is called the codomain of $f$.}

\notation{We write $f:X\rightarrow Y$ to describe a function $f$ from $X$ to $Y$ and we write $f(x)=y$ instead of $(x,y)\in f$.}

\definition{Injective Functions} {Let $X$ and $Y$ be sets and let $f:X \rightarrow Y$ be a function. The function $f$ is said to be injective or one-to-one if whenever $x_1,x_2 \in X$ are such that $x_1\neq x_2$ then $f(x_1) \neq f(x_2)$.}

\definition{Surjective Functions} {Let $X$ and $Y$ be sets and let $f:X \rightarrow Y$ be a function. The function $f$ is said to be surjective or onto if for each $y\in Y$ there exists an $x\in X$ such that $f(x)=y$.}

\definition{Bijective Functions}{Let $X$ and $Y$ be sets and let $f:X \rightarrow Y$ be a function. The function $f$ is said to be bijective if it is both injective and surjective.} 

\example{Let $X=\{Monday, \diamondsuit, \sqrt{\pi}, purple\}$ and $Y=\{\alpha, \heartsuit, fun\}$ be sets and define the relation $f$ from $X$ to $Y$ by $f=\{(Monday, fun), (\diamondsuit, \alpha), (\sqrt{\pi}, fun), (purple, fun)\}$. Draw a diagram to illustrate the relation. Is the relation $f$ a function? Prove your answer by using the definition.} {2in}

\problem{Let $X=\{Cleveland, Chicago, Los Angeles, Miami \}$ be a set of American cities and let $Y=\{Cavaliers, Heat, Lakers, Bulls, Clippers\}$ be a set of NBA teams. 

a) Provide an example of a relation that is a function from $X$ to $Y$ and draw a diagram to illustrate your example.
\vspace{2in}

b) Provide an example of a relation from $X$ to $Y$ that is not a function and draw a diagram to illustrate your example.
\vspace{2in}

c) Provide  an example of an injective function from $X$ to $Y$. Draw a diagram to illustrate your example.
\vspace{2in}

d) Show that there are no surjective functions from $X$ to $Y$.}{2in}

\note{The next problem states an equivalent definition for injectivity. This definition is very useful when proving that a function is injective.}
\problem{Let $X$ and $Y$ be two sets and let $f:X \rightarrow Y$ be a function. Then $f$ is injective if and only if for all $x_1, x_2 \in X$ such that $f(x_1)=f(x_2)$ we have $x_1=x_2$.}{1in}

\problem {Let $f:\N \rightarrow \N$ be a function defined by $f(n)=2n+1$. Show that the function $f$ is injective but not surjective.
 
{\bf Hint:} Use the previous problem to prove injectivity. In order to prove that $f$ is not surjective you need to find an $m\in \N$ which cannot be written as $2n+1$.}{2in}

\problem {Let $f:\N \rightarrow \Z$ be a function defined by 
$$
   f(n) = \left\{
     \begin{array}{lr}
       \frac{n}{2},       & \text{if } n \text{ is even} \\
       \frac{-(n+1)}{2},  & \text{if } n \text{ is odd} 
     \end{array}
   \right.
$$ 
Show that $f$ is a bijective function. 

{\bf Hint:} To show that $f$ is injective let $n_1, n_2 \in \N$ be such that $f(n_1)=f(n_2)$ and look at all the possible cases according to the parity of $n_1$ and $n_2$. 

To prove the surjectivity let $m\in \Z$. Consider the two possible cases; one when $m\geq 0$ and the other one when $m<0$. Then in each case find an $n \in \N$ such that $f(n)=m$.}{3in}

\vfill

\activitytitle{Review questions for the final quiz}{Taken from a variety of sources without attribution.}

\overview{The final quiz will consist of approximately 6 questions and will be worth approximately 30 points.
I will try to make sure that it can be done by a prepared student in 2 hours.
The best way to prepare is to work out problems on the review sheet and on the handouts that we have had in class.}

\vspace{0.2in}

\noindent
{\bf Induction problems}
\blist{0.2in}
\item Show that $1 + 2 + 2^2 + 2^3 + \cdots + 2^n = 2^{n+1} - 1$ for all $n \geq 0$.
\item Show that $\sum_{j=0}^n r^j = \frac{r^{n+1} - 1}{r-1}$ for all $n$ and all real numbers $r \ne 1$.  (There is an easier formula when $r = 1!$)
\item Show that $2^n < n!$ for all $n \geq 4$.
\item Show that $3^n < n!$ for all $n \geq 7$.
\item Show that $n! < n^n$ for all $n > 1$.
\item Show that $1\cdot 2 + 2\cdot 3 +3 \cdot 4 + \cdots + n(n+1) = \frac{n(n+1)(n+2)}{3}$ for all $n \geq 1$.
\item Show that $1 + \frac{1}{4} + \frac{1}{9} + \cdots + \frac{1}{n^2} < 2-\frac{1}{n}$ for all $n \geq 2$.
\item Show that $n^5 - n$ is a multiple of 5 for all $n$.
\item Show that $n^2 - 1$ is a multiple of 8 for all odd $n$.
{\bf Hint:} You could try showing $P(1)$ and then show that $P(n)$ implies $P(n+2)$.
\item Draw $n$ lines in the plane such that no two lines are parallel and no three lines go through a common point.
Show that this divides the plane into $\frac{n^2+n+2}{2}$ regions.
{\bf Hint:} How many regions does the $n+1$st line add?
\item Show that $1 + \frac{1}{2} + \frac{1}{3} + \cdots + \frac{1}{2^n} \geq 1 + \frac{n}{2}$.  This is too hard for the final quiz, but you may enjoy working on it.  Notice that increasing $n$ by 1 will double the number of terms, unlike most of the other problems you have worked on.
This result shows that the harmonic series diverges.
\item Show that $5^n$ is odd for all $n \geq 0$.
\item Use the product rule to show that, for every integer $n \geq 1$, the derivative of $x^n$ is $n x^{n-1}$.
\elist

\pagebreak

\noindent
{\bf Sample problems}
\blist{0.2in}
\item Show that if $n$ is odd, then $n^2 + 2n - 7$ is a multiple of 4.
\item Let $a, b,$ and $c$ be integers.
Suppose that $b$ is a multiple of $a$ or $c$ is a multiple of $a$.
Show that $bc$ is a multiple of $a$.
\item Let $a$ and $b$ be integers.
Suppose that $a$ is a multiple of $b$ and that $b$ is a multiple of $a$.
Show that $a = \pm b$.

\item Suppose that $n$ is an integer and $n^2$ is a multiple of 5.
Show that $n$ is a multiple of 5.
Try to do this without looking back at your notes!

\item Suppose that $n$ is an odd integer.
Show that $n^3 - 25n$ is a multiple of 24.
This is similar to something we did in class.
See if you can do it that way.
Can you do it by induction instead?
Work with $P(n)$ and $P(n+2)$.
Which way is easier?

\item Show that an integer $n$ cannot be both even and odd.
What kind of proof did you use?

\item We used the Division Algorithm, number 73, a lot once we proved it.
It has both an existence part and a uniqueness part.
Please review the statements and arguments for both:

Suppose that $n$ and $k$ are positive integers.
Argue that there {\bf exist} integers $q$ and $r$ with $0 \leq r < k$ such that $n = kq + r$.

Suppose that $n = kq + r$ where $q$ and $r$ and integers and $0 \leq r < k$ and also that $n = ka + b$ where $a$ and $b$ are integers and $0 \leq b < k$.
Show that $q=a$ and $r=b$.

\item Suppose that $m$ and $n$ are integers and that $3m = 7n$.
Show that $n$ is a multiple of 3.
Do this by writing $n$ as $3q+r$ for different possible values of $r$.

\item Give a complete proof that $\sqrt{3}$ is irrational.

\item Show that $A \cap (B \cup C) = (A \cap B) \cup (A \cap C)$ by showing containment in both directions and using cases.
Then show it again using logical statements such as $P = ``x \in A"$ and a truth table.
Try to do this without looking back at Problems 124 and 125.

\item Now that we have practiced inequalities, re--do Problem 130.

\item (Problem 133).  Let $I$ be a set, and for each $i$ in $I$, let $A_i$ be a set, all subsets of the same universe $X$.
Show de Morgan's law:  $\left( \bigcup_{i \in I} A_i \right)^c = \bigcap_{i \in I} A_i^c$ by showing set containment in both directions.

\item Show that $[2,5) \cap (3,7) = (3,5)$ by showing inclusion both ways.
Start by letting $x \in [2,5) \cap (3,7)$, so that $x \in [2,5)$ and $x \in (3,7)$, then write it as $2 \leq x < 5$ and $3 < x < 7$.
There are four inequalities here.  Soon enough you can conclude that $x \in (3,5)$.
Then show containment the other way as well.

\item Let $x > 0$.
Show that there exists an integer $n$ such that $0 < \frac{1}{n} < x$.

\item Show that $\bigcup_{n=1}^{\infty} [\frac{1}{n}, 1] = (0,1]$ by showing inclusion in both directions.
This should be easier now that we have practiced inequalities.

\item Do Problem 137, about unions.

\item Do Problem 138, about intersections.

\item This is a challenge involving two--dimensional sets and the Cartesian product.
Let $T = \{ (x,y) \in \R^2 : x \geq 0, y \geq 0, x+y\leq 1\}.$
Let $U = \bigcup_{0 \leq a \leq 1} [0,a]\times[0,1-a]$.
Show that $T = U$ by showing containment both directions.

\item Do problem 169, about relations.

\item Do problem 170, about relations.

\item Finish problems 225, 226, 227, 229, 230, 231 about absolute value.

\elist

\vfill          % pad the rest of the page with white space

\yourname

\activitytitle{Final quiz}{30 points}

\noindent
Please use one side of a sheet of paper for each problem.
If you are totally stuck, you can ask for a hint, but it may cost you a little something.
Good luck!

\vspace{0.2in}

\blist{0.0in}
\item a) State the definition of odd.

b) Show that the product of two odd numbers is odd.
Do your best to write a picture--perfect proof.

c) Let $P(n)$ be the statement ``The product of $n$ odd numbers is odd.''
You may want to write this as ``$k_1 k_2 \cdots k_n$ is odd.''
Show that $P(n)$ is true for all $n = 1, 2, \ldots$.

\vfill


\item Answer this question on the next page.

a) Let $n$ be an integer.
Explain how you know that you can write $n = 5q + r$ where $q$ and $r$ are integers and $0 \leq r < 5$.

b) Let $m$ be an integer and suppose that $5m = 7n$.
Show that $n$ is a multiple of 5.
Don't use prime factorization.

\pagebreak

\item Let $X$ be the set of all ordered pairs of integers.
Then $X$ contains things like $(1,3), (-4,11),$ and $(9,7)$.
We say that $(a,b) \sim (p,q)$ if $aq = bp$.

\noindent
a) Is $(6,10) \sim (9,15)$?

\vspace*{1in}
\noindent
b) Show that $\sim$ is reflexive.

\vspace*{1in}
\noindent
c) Show that $\sim$ is symmetric.

\vspace*{1in}
\noindent
d) Show that $\sim$ is transitive.

\vspace*{1in}
\noindent
e) Describe all members of the equivalence class containing $(6,10)$.

\vfill

\item Answer this question on the next page.
Show that $\bigcup_{n=1}^{\infty} [2, 5-\frac{1}{n}] = [2,5)$.  Draw pictures, then show set inclusion both ways.

\pagebreak

\definitionNN{176. Positive real numbers}{By construction, the real numbers have a subset $\Rp$, called the {\em positive real numbers,} for which:
\balist{0.1in}
\item If $a,b \in \Rp$, then $a + b \in \Rp$.  ($\Rp$ is closed under addition.)
\item If $a,b \in \Rp$, then $a\cdot b \in \Rp$.  ($\Rp$ is closed under multiplication.)
\item For every real number $a$, either $a \in \Rp$ or $(-a) \in \Rp$ or $a = 0$.  Exactly one of the three happens.
\ealist
}

\item Let $a \in \R$ with $a \ne 0$.
Show that $a\cdot a \in \Rp$.
Be extremely clear about every step that you take.

\vfill

\definitionNN{Absolute value}{The function $f : \R \to \R$ defined by
\[
    f(x) = \left\{ \begin{array}{cl} x, & \mbox{if $x \geq 0$} \\
                          -x, & \mbox{if $x < 0$} \end{array} \right.
\]
is called the {\em absolute value} function.}

\item Answer this question on the next page.
Show that $f(-a) = f(a)$ for all real numbers $a$. 

\elist

\activitytitle{Reading assignment \#1}{Due on the second day of class.}

The idea is to read Chapter 1 of the textbook by Daepp and Gorkin\DGreference.
The assignment is to read it in a particular way.
It may take 3 hours to get it done, but you will learn something in those three hours, and you will start to develop a very important skill.

Get a copy of Chapter 1, ``The How, When, and Why of Mathematics.''
Get out your notebook or some paper.
Go somewhere quiet, where you won't be interrupted for a while.
Turn off your phone so you aren't disturbed.
Don't listen to music that will distract you, and make sure there is no TV or youtube on where you can see it or hear it.

Put the notebook or paper right in front of you.
Put the textbook itself a bit farther away.
Make note of the time that you start reading in your notebook, maybe in the left margin.
Read the first paragraph of the chapter, then write one or more sentences in your notes which capture the main idea(s) of the paragraph.

Read the second paragraph, about Geogre P\'{o}lya's list of guidelines.
Look up the list in the Appendix.
Consider writing them in your notebook, or abbreviated versions of them.

Continue to write a sentence summarizing each paragraph.
I believe that if you are not writing, you are probably not thinking as hard as you need to.
Read slowly.
If you run into a word you don't know, google it or look it up in a dictionary.  If you really don't know it, write the definition in your notebook.
It is OK to spend 15 minutes on each page of the book.  Really.
It is not a goal of the course to learn how to read faster.
The goal is to learn how to get more out of the time you spend reading.
If you stop to take a break, note the time that you stopped and the time you start again.

Read Exercise 1.1 and the text that walks you through P\'{o}lya's guidelines.
Use your notebook to try to solve the puzzle yourself.
I've printed the alphabet twice at the bottom of this page.  If you cut off the bottom version, you can slide it along the top one and easily keep track of how the letters correspond.  That should save you a little time.

A second example starts on page 3 of the textbook.
As you read it, draw diagrams in your notebook.
Yes, there are diagrams printed in the textbook, but you will think harder about the diagram and understand more if you draw your own.

Example 1.2 asks a question.  Read the question and see if you can answer it on your own, without reading further in the book.

On page 7, you will see that solutions of the exercises are provided.
Resist the urge to turn your brain off and just read the solutions.  That is not what they are there for!

Work through some of the problems that begin on page 7 in the book.
You do not need to do all of them, but you should at least understand most of them and attempt a few of them.

You may be able to do problem 1.9 because we have done similar things in class.
Give it a try.
Problem 1.10 doesn't interest me.  Does it interest you?

Problem 1.12 is good.  Will it help to make a graph?
Problem 1.13 seems silly.  Do you like it anyway?

Read the Tips on Doing Homework.
At the end, tally up how much time you have spent on reading this chapter.
Write this number in your notebook and remember the number when you come to class.

\vfill          % pad the rest of the page with white space

{\tt ABCDEFGHIJKLMNOPQRSTUVWXYZABCDEFGHIJKLMNOPQRSTUVWXYZ}

\vspace{0.1in}

{\tt ABCDEFGHIJKLMNOPQRSTUVWXYZABCDEFGHIJKLMNOPQRSTUVWXYZ}


\vfill          % pad the rest of the page with white space

{\tt ABCDEFGHIJKLMNOPQRSTUVWXYZABCDEFGHIJKLMNOPQRSTUVWXYZ}

\vspace{0.1in}

{\tt ABCDEFGHIJKLMNOPQRSTUVWXYZABCDEFGHIJKLMNOPQRSTUVWXYZ}

\vspace{0.1in}

{\tt ABCDEFGHIJKLMNOPQRSTUVWXYZABCDEFGHIJKLMNOPQRSTUVWXYZ}

\vspace{0.1in}

{\tt ABCDEFGHIJKLMNOPQRSTUVWXYZABCDEFGHIJKLMNOPQRSTUVWXYZ}

\vspace{0.1in}

{\tt ABCDEFGHIJKLMNOPQRSTUVWXYZABCDEFGHIJKLMNOPQRSTUVWXYZ}

\vspace{0.1in}

{\tt ABCDEFGHIJKLMNOPQRSTUVWXYZABCDEFGHIJKLMNOPQRSTUVWXYZ}

\vspace{0.1in}

{\tt ABCDEFGHIJKLMNOPQRSTUVWXYZABCDEFGHIJKLMNOPQRSTUVWXYZ}

\vspace{0.1in}

{\tt ABCDEFGHIJKLMNOPQRSTUVWXYZABCDEFGHIJKLMNOPQRSTUVWXYZ}

\vspace{0.1in}

{\tt ABCDEFGHIJKLMNOPQRSTUVWXYZABCDEFGHIJKLMNOPQRSTUVWXYZ}

\vspace{0.1in}

{\tt ABCDEFGHIJKLMNOPQRSTUVWXYZABCDEFGHIJKLMNOPQRSTUVWXYZ}

\vspace{0.1in}

{\tt ABCDEFGHIJKLMNOPQRSTUVWXYZABCDEFGHIJKLMNOPQRSTUVWXYZ}

\vspace{0.1in}

{\tt ABCDEFGHIJKLMNOPQRSTUVWXYZABCDEFGHIJKLMNOPQRSTUVWXYZ}

\vspace{0.1in}

{\tt ABCDEFGHIJKLMNOPQRSTUVWXYZABCDEFGHIJKLMNOPQRSTUVWXYZ}

\vspace{0.1in}

{\tt ABCDEFGHIJKLMNOPQRSTUVWXYZABCDEFGHIJKLMNOPQRSTUVWXYZ}

\vspace{0.1in}

{\tt ABCDEFGHIJKLMNOPQRSTUVWXYZABCDEFGHIJKLMNOPQRSTUVWXYZ}

\vspace{0.1in}

{\tt ABCDEFGHIJKLMNOPQRSTUVWXYZABCDEFGHIJKLMNOPQRSTUVWXYZ}

\vspace{0.1in}

{\tt ABCDEFGHIJKLMNOPQRSTUVWXYZABCDEFGHIJKLMNOPQRSTUVWXYZ}

\vspace{0.1in}

{\tt ABCDEFGHIJKLMNOPQRSTUVWXYZABCDEFGHIJKLMNOPQRSTUVWXYZ}

\vspace{0.1in}

{\tt ABCDEFGHIJKLMNOPQRSTUVWXYZABCDEFGHIJKLMNOPQRSTUVWXYZ}

\vspace{0.1in}

{\tt ABCDEFGHIJKLMNOPQRSTUVWXYZABCDEFGHIJKLMNOPQRSTUVWXYZ}

\vspace{0.1in}

{\tt ABCDEFGHIJKLMNOPQRSTUVWXYZABCDEFGHIJKLMNOPQRSTUVWXYZ}

\vspace{0.1in}

{\tt ABCDEFGHIJKLMNOPQRSTUVWXYZABCDEFGHIJKLMNOPQRSTUVWXYZ}

\vspace{0.1in}

{\tt ABCDEFGHIJKLMNOPQRSTUVWXYZABCDEFGHIJKLMNOPQRSTUVWXYZ}

\vspace{0.1in}

{\tt ABCDEFGHIJKLMNOPQRSTUVWXYZABCDEFGHIJKLMNOPQRSTUVWXYZ}

\vspace{0.1in}

{\tt ABCDEFGHIJKLMNOPQRSTUVWXYZABCDEFGHIJKLMNOPQRSTUVWXYZ}

\vspace{0.1in}

{\tt ABCDEFGHIJKLMNOPQRSTUVWXYZABCDEFGHIJKLMNOPQRSTUVWXYZ}

\vspace{0.1in}

{\tt ABCDEFGHIJKLMNOPQRSTUVWXYZABCDEFGHIJKLMNOPQRSTUVWXYZ}

\vspace{0.1in}

{\tt ABCDEFGHIJKLMNOPQRSTUVWXYZABCDEFGHIJKLMNOPQRSTUVWXYZ}

\vspace{0.1in}

{\tt ABCDEFGHIJKLMNOPQRSTUVWXYZABCDEFGHIJKLMNOPQRSTUVWXYZ}

\vspace{0.1in}

{\tt ABCDEFGHIJKLMNOPQRSTUVWXYZABCDEFGHIJKLMNOPQRSTUVWXYZ}


\activitytitle{Reading assignment, Chapter 2}{Due on Wednesday, September 2.}

Read Chapter 2 of the book by Daepp and Gorkin.
As with Chapter 1,
\blist{0.0in}
\item Read somewhere quiet, minimizing distractions from phones and friends
\item Note the times that you start and stop reading, and add up the minutes
\item Read with a pencil in your hand and your notebook open in front of you
\item Write a sentence to summarize each paragraph, re-draw diagrams, work out examples and exercises on your own
\item Look up words you don't know, and write down ones you really don't know
\item Read slowly.  You are not reading a comic book or a newspaper.  It is not a goal of this class for you to learn how to read faster.  The goal is to learn how to get more out of the time you spend reading, and to learn to concentrate for longer periods of time.
\item At the end, tally up how much time you have spent on reading this chapter.
Write this number in your notebook and remember the number when you come to class.
\elist

You will read about ``statements.''
Focus on the ones about mathematical things, and don't worry too much about interpreting the ones that are non-mathematical.

{\bf Note that on page 14, there is a statement about the color of the cover of the book.  Books from Springer used to be plain yellow, but the authors must not have realized that someone would put a big blue bar on the cover of this edition of the book.  Just imagine that the book cover is all yellow.}

Fill out every truth table that is suggested in the chapter.
Truth tables are an excellent way to get great clarity about complicated combinations of statements.
The idea is to consider every possible combination of True and False for the basic statements.
For example, if there are two statements, $P$ and $Q$, there will be four rows in the table, running through the four possible combinations of True and False for $P$ and $Q$.
On page 21, there is a truth table for three statements, $P$, $Q$, and $R$.
It has eight rows.

The most important use of truth tables is to tell when two complicated combinations of logical expressions are, in fact, the same.

For me, the hardest thing about truth tables is making columns for implications like $P \to Q$.
Here is the best way I know to think about them.
Each row of the truth table for $P$ and $Q$ covers one combination of truth values for $P$ and $Q$.
Some of these combinations are consistent with the implication that $P$ implies $Q$.
For example, when $P$ is True and $Q$ is True, this is consistent with $P \to Q$, so we put $T$ in the $P \to Q$ column.
The row in which $P$ is True and $Q$ is False, however, is inconsistent with the implication $P \to Q$, so we put $F$ in that row.
The cases in which $P$ is False are a bit different, but they are also consistent with $P \to Q$, since $P \to Q$ only has anything to say about $P$ and $Q$ when $P$ is True.
So we put $T$ in those rows too.

{\bf Problems 1 to 7 are good, so please do those.  
Rather than working on problems 9-21, I would much prefer that you spend your time making the truth tables I describe below.
\blist{0.0in}
\item Make a truth table for $\neg (P \vee Q)$ and $\neg P \wedge \neg Q$.
\item Make a big truth table for $P, Q, R,$ $P \wedge (Q \vee R)$, $P \vee (Q \wedge R)$, $(P \wedge Q) \vee (P \wedge R)$, and $(P \vee Q) \wedge (P \vee R)$.  Which of these are equal?  How can you rememeber that?
\elist
}
\vfill          % pad the rest of the page with white space

\activitytitle{Reading assignment, Chapter 3}{}

Read Chapter 3 of the textbook by Daepp and Gorkin.
As with Chapters 1 and 2,
\blist{0.0in}
\item Read somewhere quiet, minimizing distractions from phones and friends
\item Note the time that you start and stop reading, and add up the minutes
\item Read with a pencil in your hand and your notebook open in front of you
\item Write a sentence to summarize each paragraph, re--draw diagrams, work out examples and exercises on your own
\item Look up words you don't know, and write down ones you really don't know
\item Read slowly.  You are not reading a comic book or a newspaper.  It is not a goal of this class for you to learn how to read faster.  The goal is to learn how to get more out of the time you spend reading, and to learn to concentrate for longer periods of time.
\item At the end, tally up how much time you have spent on reading this chapter.
Write this number in your notebook and remember the number when you come to class.
\elist

Theorem 3.1 lists three properties of logical statements.
Please make truth tables for each of them to check that they are tautologies.
Also add de Morgan's laws from Theorem 2.9.
Then you'll have the whole set.
Having de Morgan's laws handy should make Exercise 3.2 easier.

How can you remember the distributive property?

The contrapositive is really important.
See if you can explain it just by thinking about $P \to Q$ and $\neg Q \to \neg P$, without using truth tables.

Theorem 3.3 is proven using the contrapositive.
This is a very useful method of proof.
Please note that it differs from proof by contradiction.

Read about the converse, and make sure never to confuse an implication with its converse.

Problems 2, 3, 4, 9, 14, 16, 18, 5, 6, 8, 19, 15 are good to work on, in that order.
Work through at least half of these problems.


Chapters 1 to 5 are mostly there to help develop proof techniques.  After Chapter 5, we will spend more of our time on definitions, examples, theorems, and proofs.
Use your time now to develop basic logic and proof techniques that will help you for the rest of the semester and beyond!


\vfill          % pad the rest of the page with white space

\activitytitle{Reading assignment, Chapter 4}{Due in the third week of classes.}

Read and understand Chapter 4 of the textbook by Daepp and Gorkin.
As with previous chapters,
{\small
\blist{0.0in}
\item Read somewhere quiet, minimizing distractions from phones and friends
\item Note the time that you start and stop reading, and add up the minutes
\item Read with a pencil in your hand and your notebook open in front of you
\item Write a sentence to summarize each paragraph, re--draw diagrams, work out examples and exercises on your own
\item Look up words you don't know, and write down ones you really don't know
\item Read slowly.  You are not reading a comic book or a newspaper.  It is not a goal of this class for you to learn how to read faster.  The goal is to learn how to get more out of the time you spend reading, and to learn to concentrate for longer periods of time.
\item At the end, tally up how much time you have spent on reading this chapter.
Write this number in your notebook and remember the number when you come to class.
\elist
}

This is a very important chapter, and one with real substance.  Hopefully you will feel that way when you read it, and will enjoy it more as a result.

This chapter has a large number of very dense expressions involving quantifiers, implications, and logical operators.  Slow way down when you run into one of them.  Pick them apart in your mind and then write them down so they are crystal clear.  Every symbol is important.  It's a bit like when you're reading someone your credit card number or you're giving your phone number to someone you really want to call you.  Every symbol is important.

Exercises 4.1, 4.2, 4.3, and 4.6 are all useful to do.
The discussion that begins at the bottom of page 36 is very important, negating statements with quantifiers.

There are 20 problems.  The more of them you do, the better, of course, but you may not be able to work through all of them.  {\bf Please at least do problems \# 1--7, and 20.}  Read \# 11.  Does this joke work on your friends?

People have asked about grading, or about a rubric that Ying-Ju Chen is using when she reads the notebook.
She'll be assigning numeric values between 0 and 5 for each chapter.
The bulk of the points go toward the notes on the chapter itself.
This is to emphasize that reading and taking notes is the primary concern.
Less than half of the points go toward attempting the exercises, with more emphasis on attempting than on getting them all the way right.
Ying-Ju writes as many helpful comments as she can on each notebook, but there is only so much time, and sometimes things that are incorrect do not get marked as incorrect.
Even so, I think that having you take notes and having Ying-Ju read them every class is working very well.

Pay attention to the phrase ``only if.'' It is often used in a way that can be confusing.  Compare these two statements for example, in which $R$ means Race and $P$ means prize:
\blist{0.0in}
\item I will race if there is a prize offered.  $P \to R$.  This is the most common way that people use the word ``if.''  The prize will make me race.
\item I will race only if there is a prize offered.  $R \to P$.  People say this sort of thing pretty often too, but it's a bit less clear unless you think about it carefully.  Part of the problem is the time order in which things happen, because the racing comes {\em after} the prize is offered.  ``If you see me racing, you can be sure that there was a prize offered. (But offering a prize is no guarantee that I will race.)''
\elist



\vfill          % pad the rest of the page with white space

\activitytitle{Reading assignment, Chapter 5}{Due Friday, September 25.}

\noindent
Read and understand Chapter 5 of the textbook by Daepp and Gorkin.
As with previous chapters,
{\small
\blist{0.0in}
\item Read somewhere quiet, minimizing distractions from phones and friends
\item Note the time that you start and stop reading, and add up the minutes
\item Read with a pencil in your hand and your notebook open in front of you
\item Write a sentence to summarize each paragraph, re--draw diagrams, work out examples and exercises on your own
\item Look up words you don't know, and write down ones you really don't know
\item Read slowly.  You are not reading a comic book or a newspaper.  It is not a goal of this class for you to learn how to read faster.  The goal is to learn how to get more out of the time you spend reading, and to learn to concentrate for longer periods of time.
\item At the end, tally up how much time you have spent on reading this chapter.
Write this number in your notebook and remember the number when you come to class.
\elist
}

This chapter walks you through a number of types of proofs and gives examples of each.  {\bf Rewrite these proofs in your notes, in your own words as much as possible, so that you make them yours.}  By the end of reading the chapter, you should {\bf know} the proof that the square root of 2 is irrational and you should know the other proofs as well.

It might help, in your notes, to make a list of proof techniques from the chapter and from previous chapters.
What chapter talked about proof by contrapositive?  Is that in Chapter 5?
What about truth tables?  You can prove things with those.  What kinds of things?

Read and understand Problem 1.  It is important.

Read the other problems, find the ones that are easy, and do them.
This may seem like a strange assignment, but I really mean it.  Think about each problem (if you can get through all of them), and make sure that if a problem is easy, that you recognize that and write out the solution.
Don't worry if a problem looks hard but turns out to be easy.
That happens all the time.
But hopefully you will spot a number of them that really are easy, and do them.
We will go over these problems in class the following week.

\vfill          % pad the rest of the page with white space

\activitytitle{Reading assignment, Chapter 6}{Due in the fifth week of class.}

Read and understand Chapter 6 of the textbook by Daepp and Gorkin.
As with previous chapters,
{\small
\blist{0.0in}
\item Read somewhere quiet, minimizing distractions from phones and friends
\item Note the time that you start and stop reading, and add up the minutes
\item Read with a pencil in your hand and your notebook open in front of you
\item Write a sentence to summarize each paragraph, re--draw diagrams, work out examples and exercises on your own
\item Look up words you don't know, and write down ones you really don't know
\item Read slowly.  You are not reading a comic book or a newspaper.  It is not a goal of this class for you to learn how to read faster.  The goal is to learn how to get more out of the time you spend reading, and to learn to concentrate for longer periods of time.
\item At the end, tally up how much time you have spent on reading this chapter.
Write this number in your notebook and remember the number when you come to class.
\elist
}

This chapter introduces sets, subsets, equality of sets, and how to tell what the members of a set are.
As you read, take time to write out several members of each set that is introduced.
Note that $A$ being a subset of $B$ is the same as the logical implication $x \in A$ implies $x \in B$.
There is a tight connection between statements in set theory and logical statements.
Here is another:  Set $A$ being equal to set $B$ is the same as the logical implications $x \in A$ if and only if $x \in B$.

There are many examples in this chapter.
Work through them by rewriting them and adding useful steps in your notes.

On page 64, intersections, unions, and complements of sets are introduced.
As you read about them, explain in your notes how these relate logical statements such as $x \in A$ and $x \in B$ to $x \in A \cap B$.

You may enjoy reading about the paradoxes on page 67.
Give them a try.
Even if they are not your cup of tea, try to see what the issue is.

Problems 1 -- 9 are essential.  Do them.

Problem 10 is a good thought problem.  Think about it.

Starting with Problem 11, there are things for you to prove.
I would be happy to see you do some of these by yourself.
We will do these problems in class, but I'd like us to move through them fairly quickly, so have a look at them before class.

\vfill          % pad the rest of the page with white space

\activitytitle{Reading assignment, Chapter 7}{Due Monday, November 2.}

Read and understand Chapter 7 of the textbook by Daepp and Gorkin called ``Operations on sets.''

This is a short chapter, all about working with sets.
You can approach these problems in a number of ways.
Often it helps to draw a nice Venn diagram and get the right intuitive idea for what is being claimed, but don't stop there.
You can also just focus on letting $x \in A$ or whatever and working with that, without thinking about Venn diagrams.

Most of the chapter is devoted to one example, showing that, if $A,$ $B,$ and $C$ are sets, then $A \cup (B \cap C) = (A \cup B) \cap (A \cup C)$.
The book suggests working forward from one side, and backward from the other, just as people sometimes build a bridge by starting at each bank of a river and meeting in the middle.

It also suggests breaking into cases at some point.  For example, if $x \in A \cup (B \cap C)$, you can consider the case $x \in A$, which is great because then it's pretty clear that $x \in (A \cup B) \cap (A \cup C)$.  But you also need to consider the case $x \notin A$, so that $x \in B \cap C$.  But that's helpful, because then $x \in B$ and $x \in C$, and pretty soon it is clear that 
$x \in (A \cup B) \cap (A \cup C)$.

{\bf Do problem 7.1, parts a, c, d, e, f.}  Take your time and use really good form so that the proof is crystal clear.  Notice that part (c) (statement 18 in the theorem) is an ``if and only if'' statement, so it has two parts.  It's going to look something like this:
\blist{0in}
\item Suppose that $A \subseteq B$.  We want to show that $(X \backslash B) \subseteq (X \backslash A).$  Let $x \in X \backslash B$.  Then $x \notin B$.  (More steps here.) Thus, $x \in X \backslash A$, and so $(X \backslash B) \subseteq (X \backslash A).$
\item Suppose that $(X \backslash B) \subseteq (X \backslash A).$  We want to show that $A \subseteq B$.  Let $x \in A$. (More steps here.)  Thus, $x \in B$.
\elist

{\bf Do problem 7.4.}

{\bf Do problem 7.6.}

I guess that these problems are a bit dull, but it really is helpful to be good at proving things about sets.

\noindent As with the previous chapters,
\blist{0.0in}
\item Read somewhere quiet, minimizing distractions from phones and friends
\item Note the time that you start and stop reading, and add up the minutes
\item Read with a pencil in your hand and your notebook open in front of you
\item Write a sentence to summarize each paragraph, re-draw diagrams, work out examples and exercises on your own
\item Look up words you don't know, and write down ones you really don't know
\item Read slowly.  You are not reading a comic book or a newspaper.  It is not a goal of this class for you to learn how to read faster.  The goal is to learn how to get more out of the time you spend reading, and to learn to concentrate for longer periods of time.
\item At the end, tally up how much time you have spent on reading this chapter.
Write this number in your notebook and remember the number when you come to class.
\elist


\vfill          % pad the rest of the page with white space

\activitytitle{Reading assignment, Chapter 8}{Due in the eighth week of class.}

Read and understand Chapter 8 of the textbook by Daepp and Gorkin, called ``More on operations on sets.''

This chapter is a challenge.
You will really need to use all the reading skills you have been practicing when you read this chapter.
The ideas are harder, and some are really hard, but not impossible.
Just slow yourself down and write things out in lots of detail.

Example 8.2(a) would be a great one to write out concrete fractions with different values of $p$ and $q$ to understand the sets $A_q$ and then the union of these sets.  For Example 8.2(b), do the same to understand what the sets $B_i$ are, and then what their intersection is.  No shortcuts!  Write out elements for each set.

Exercise 8.3 is also good.

In the middle of page 82 the phrase ``collection of subsets of $X$'' appears.
This is a very new, very difficult concept; do not underestimate how tricky it can be, but patiently think about it and keep coming back to it.
For example, $\cal{A}$ might be all intervals of the form $[k,k+1]$ and you might want to take the union of all such intervals, or the intersection.

Exercise 8.4 is excellent.  Draw pictures until everything is crystal clear.
Exercise 8.5 is also excellent.

Rewrite the proofs of Examples 8.6 and 8.7 to make them your own.  Really.

Exercises 8.9 and 8.10 are also excellent.  Do them on your own, then compare to the solutions in the book.

Do problems 1, 2, and 3.

Here is a challenge problem.  Let $a < b$.  Show that $\bigcup_{n=1}^{\infty} [a, b-\frac{1}{n}] = [a,b)$.  Draw pictures, then show set inclusion both ways.

Here is another challenge problem.  Let $a < b$.  Show that $\bigcap_{n=1}^{\infty} [a,b+\frac{1}{n}) = [a,b]$.  Draw pictures, then show set inclusion both ways.

\noindent As with the previous chapters,
\blist{0.0in}
\item Read somewhere quiet, minimizing distractions from phones and friends
\item Note the time that you start and stop reading, and add up the minutes
\item Read with a pencil in your hand and your notebook open in front of you
\item Write a sentence to summarize each paragraph, re-draw diagrams, work out examples and exercises on your own
\item Look up words you don't know, and write down ones you really don't know
\item Read slowly.  You are not reading a comic book or a newspaper.  It is not a goal of this class for you to learn how to read faster.  The goal is to learn how to get more out of the time you spend reading, and to learn to concentrate for longer periods of time.
\item At the end, tally up how much time you have spent on reading this chapter.
Write this number in your notebook and remember the number when you come to class.
\elist


\vfill          % pad the rest of the page with white space

\activitytitle{Reading assignment, Chapter 9}{Due in the tenth week of class.}

Read and understand Chapter 9 of the textbook by Daepp and Gorkin, called ``The Power Set and the Cartesian Product.''
This is the last chapter on plain set theory.
It should stretch your mind in a few new directions.
Prepare to move slowly and think carefully.

When $A$ is a set, the power set of $A$ is the collection of all subsets of $A$.  Read Example 9.1 and do Exercise 9.3 and then {\bf do Problem 9.1.}
Work through Exercise 9.2 and then {\bf do Problem 9.2.}
Problem 9.2 is hard, but excellent for you.  Take it very slowly.
Work through Exercise 9.4 and then {\bf do Problem 9.5.}
{\bf Do Problem 9.8.}

{\bf Do Problem 9.11.}  For 9.11, you have already seen the power set of a set containing 2 elements and 3 elements.  {\bf Hint:}  When you are making a subset of a set $A$, for each element of $A$, you have to decide whether it goes in or out of the subset.  There are two choices (in or out) each time.  If the hint doesn't help you, write out the power set of $\{1, 2, 3, 4\}$, then read the hint again.  Hopefully you don't have to write out the power set of $\{1, 2, 3, 4, 5\}!$

You are already very familiar with one Cartesian product:  making ordered pairs $(x,y)$ of real numbers is the Cartesian product $\R \times \R$, which you know better as the $xy$ plane.
Every problem involving Cartesian products of sets containing real numbers can be depicted as points in the $xy$ plane.
Make a graph in every case.
This will help your intuition.
When there are only finitely many points, like with $\{0,1\} \times \{ 2,3\}$, also list out all of the $(x,y)$ pairs.

{\bf Answer these questions:  Who is the Cartesian product named after?  Why, exactly?}

Work through Exercise 9.5 a, b, e.

{\bf For Theorem 9.7, draw $A$ and $C$ as intervals on the $x$ axis and draw $B$ and $D$ as intervals on the $y$ axis, then draw out the sets in the statement of the theorem on two separate sets of axes.}
Make sure you are crystal clear about what these sets are, and you will be close to mastering Cartesian products.

{\bf Do Problem 9.12.}  It connects Cartesian products to things you learned in geometry.

{\bf Do Problem 9.17a.}  Notice that this is an ``if and only if'' proof, and it has three set equalities to show.
Suppose that $A \times B = C \times D$ and show that $A = C$ and $B = D$ by showing containment each way.
Here is one part of the argument:  Let $x \in A$.  Also let $y \in B$.  Then $(x,y) \in A \times B = C \times D$, and so $x \in C$.  Thus $A \subseteq C$.
After that part is done, suppose that $A=C$ and $B=D$ and argue that $A \times B = C \times D$.

{\bf Think about Problem 9.19.}

\noindent As with the previous chapters,
\blist{0.0in}
\item Read somewhere quiet, minimizing distractions from phones and friends
\item Note the time that you start and stop reading, and add up the minutes
\item Read with a pencil in your hand and your notebook open in front of you
\item Write a sentence to summarize each paragraph, re-draw diagrams, work out examples and exercises on your own
\item Look up words you don't know, and write down ones you really don't know
\item Read slowly. 
\item At the end, tally up how much time you have spent on reading this chapter.
Write this number in your notebook and remember the number when you come to class.
\elist

\vfill          % pad the rest of the page with white space

\activitytitle{Reading assignment, Chapter 10}{Due in the eleventh week of class.}

Read and understand Chapter 10 of the textbook by Daepp and Gorkin, called ``Relations.''

The main definition for Chapter 10 appears at the end of Chapter 9, on page 93.  Here is the deal.  A {\em relation} $S$ from a set $X$ to a set $Y$ is a subset of $X \times Y$.  If $Y = X$, we say the relation is a relation on $X$.
At the beginning of Chapter 10, we see that we are going to be only working with relations on a set $X$.

Suppose that $S$ is a relation on a set $X$.
That is, suppose that $S$ is a subset of $X \times X$, which means that $S$ is a set of points of the form $(x,y)$, where $x \in X$ and $y \in X$.
Rather than write $(x,y) \in S$, we usually write $x \sim y$.
How to read this out loud?  There is no perfect solution.  I would suggest that you read it as ``$x$ tilde $y$'' (because $\sim$ is the tilde that appears above the n in some Spanish words).

Suppose that $X = \R$ and let $S = \{ (x,y) : x \leq y\}$.
Then $x \sim y$ means that $(x,y) \in S$, which means that $x \leq y$.
In this way, we see that $\leq$ is a relation on $\R$.
{\bf Write out the set $S$ corresponding to the relations $<$, $\leq$, =, $\geq$, and $>$.  Then also sketch these as regions in the $xy$ plane.}

Note that relations are between two elements.
Thus, ``divisible by 4'' is not a relation.
However, if $X = \Z^+$, you could say that $x \sim y$ if $y$ is divisible by $x$, and then you would have a relation.
People often write $x | y$ for this relation and say that $x$ divides $y$.
Call this relation $S$.
{\bf Write out at least ten of the ordered pairs in $S$, using at least five different values of $x$.}

Read Exercises 10.1 and 10.2.  

Read the definitions of reflexive, symmetric, and transitive.
A relation that satisfies all three is called an equivalence relation.
This is where most of the action is with relations.
{\bf Do Problem 10.2.}
{\bf \Large You should start every part of the problem by writing down examples.}
For example, for (a), the example $3 < 3$ will tell you whether the relation is reflexive, $3 < 5$ and $5 < 3$ will tell you about symmetry, and $3 < 5, 5 < 7$, and $3 < 7$ will get you started on transitivity.

Read Example 10.3, then {\bf do Problem 10.3.}
Use examples to check reflexivity, symmetry, and transitivity.

Equivalence relations are very important, as are equivalence classes.
An equivalence relation is like the equality relation (=), but applied to other contexts.
Here is an example that is useful.
Think of the integers, $\Z$.
Say that $x \sim y$ if $x$ and $y$ have the same remainder when you divide by 2.
Then $6 \sim 22$ and $31 \sim 7$.
This relation is reflexive, because $x \sim x$.
It is symmetric because if $x \sim y$ then $y \sim x$.
And it is transitive because if $x \sim y$ and $y \sim z$, then $x \sim z$.
Now we can say that 6 is equivalent to 22, and 31 is equivalent to 7, according to this definition of equivalence.
The equivalence class that contains 6 and 22 is all even numbers, and the equivalence class containing 31 and 7 is all odd numbers.
Let this sink into your mind, and you will start to see that it makes for a useful way to organize things, when an equivalence relation is available.

{\bf Do Problem 10.1}  Start by writing out examples for the pairs $(x,y)$ and $(w,z)$.  Think about lines and circles in the plane.

{\tiny 
\blist{0.0in}
\item Read somewhere quiet, minimizing distractions from phones and friends
\item Note the time that you start and stop reading, and add up the minutes
\item Read with a pencil in your hand and your notebook open in front of you
\item Write a sentence to summarize each paragraph, re-draw diagrams, work out examples and exercises on your own
\item Look up words you don't know, and write down ones you really don't know
\item Read slowly. 
\item Tally up how much time you have spent on reading this chapter.
\elist
}
\vfill          % pad the rest of the page with white space

\activitytitle{Reading assignment, Chapter 18}{Due in the thirteenth week of class.}

Read and understand Chapter 18 of the textbook by Daepp and Gorkin, called ``Mathematical Induction.''

Mathematical induction and recursion play an important role especially in discrete mathematics. Prepare to move slowly and think carefully. To understand the proof of Theorem 18.1, you will need {\bf Well-ordering principle of the natural numbers}: Every nonempty subset of the natural numbers contains a minimum.
Read Theorem 18.1 and then {\bf do Problem 18.1} and {\bf Problem 18.3}. Follow the steps in Theorem 18.1, define the assertion $P(n)$ for the problem first. You will need the condition "$P(n)$ is true" to show the induction step. Work through Exercise 18.3 to Exercise 18.5 and then {\bf do Problem 18.9} without going back Exercise 18.5. You can do it!!

Recursion is a very useful tool to define functions, sequences and sets. Before you move to Theorem 18.6, read the definition of $n$ factorial for $n\in \mathbb{N}$. Write out $3!, 4!$ and $5!$, then try $\frac{6!}{2!4!}$. More general, simplify $\frac{n!}{m!(n-m)!}$ where $n$ and $m$ are two positive integers with $n\geq m$. Here is another simple example:\\
Let $n\in \mathbb{Z^+}$. Consider the function $S(n) = S(n-1) + n$ with $S(0) = 0$. Write out $S(1), S(2)$ and $S(3)$. Can you figure out what this function does for us? With Problem 18.1, you should be able to see the connection between induction and recursion.

Theorem 18.6 shows the existence and uniqueness of a recursive function $g: N \rightarrow X$ given a function $f: X \rightarrow X$ and $a\in X$, where $X$ is a nonempty set. The function $g$ satisfies
\begin{itemize}
\item[(i)] The base step: $g(0) = a$, and
\item[(ii)] The recursive step: $g(n+1)=f(g(n))$ for all $n \in \mathbb{N}$.
\end{itemize}

\noindent The proof of Theorem 18.6 is long and hard. Be patient! You may not get the idea of the proof at beginning, try to write outlines of the proof. You can come back it later.

The keys to a successful recursive solution is to identify the base case and make sure the recursive step is making progress toward the solution. Do Exercise 18.7 and then {\bf do Problem 18.10} and {\bf Problem 18.11}.

\noindent As with the previous chapters,
\blist{0.0in}
\item Read somewhere quiet, minimizing distractions from phones and friends
\item Note the time that you start and stop reading, and add up the minutes
\item Read with a pencil in your hand and your notebook open in front of you
\item Write a sentence to summarize each paragraph, re-draw diagrams, work out examples and exercises on your own
\item Look up words you don't know, and write down ones you really don't know
\item Read slowly.
\item At the end, tally up how much time you have spent on reading this chapter.
Write this number in your notebook and remember the number when you come to class.
\elist

\vfill          % pad the rest of the page with white space


\yourname

\activitytitle{Quiz on irrationality of square roots of odd primes}{20 points}

\showNN{Show that if $n$ is an integer and $n^2$ is a multiple of 47, then $n$ is a multiple of 47.  You choose the type of proof you want to do.  Whatever you do, there are too many cases to check one by one, so organize your thoughts efficiently.}{7in}

\showNN{On the other side of this sheet of paper, show that $\sqrt{47}$ is irrational.  I recommend a proof by contradiction.}{0in}

\vfill          % pad the rest of the page with white space

\yourname

\activitytitle{Quiz on even and odd and three--dimensional vectors}{20 points}

\noindent Work hard to write really nice proofs.

\definitionNN{1. Even}{An integer $n$ is {\em even} if there exists an integer $k$ for which $n = 2k$.}

\definitionNN{2. Odd}{An integer $n$ is {\em odd} if there exists an integer $k$ for which $n = 2k+1$.}

\definitionNN{3. Three--dimensional vector}{A three--dimensional vector is an ordered triple $\vect{ a_1, a_2, a_3 }$, where $a_1, a_2,$ and $a_3$ are real numbers.}

\definitionNN{4. Sum of 3--dimensional vectors}{The sum of 3--dimensional vectors $\vect{ a_1, a_2, a_3 }$ and $\vect{ b_1, b_2, b_3 }$ is the 3--dimensional vector $\vect{ a_1+b_1, a_2+b_2, a_3+b_3 }$.  We write $\vec{a} \oplus \vec{b}$, using a new symbol so we don't confuse addition of vectors with addition of real numbers.}

\definitionNN{5. Scalar product for 3--dimensional vectors}{Let $c$ be a real number and let $\vec{a} = \tvec{a}$ be a 3--dimensional vector.  The {\em scalar product} of $c$ and $\vec{a}$ is defined as:
\[
    c\vec{a} = \vect{ca_1,ca_2,ca_3}.
\]}

\showNN{Show that the product of two odd numbers is odd.}{5in}

\showNN{On the back of this sheet, show that the scalar product is distributive over vector addition. That is, show that $c(\vec{a} \oplus \vec{b}) = c\vec{a} \oplus c\vec{b}$.}{0in}


\vfill          % pad the rest of the page with white space

\yourname

\activitytitle{Quiz on sum and dot product of 3--dimensional vectors}{10 points}

\definitionNN{3--dimensional vector}{A three--dimensional vector is an ordered triple $\vect{ a_1, a_2, a_3 }$, where $a_1, a_2,$ and $a_3$ are real numbers.}

\definitionNN{Equality of 3--dimensional vectors}{3--dimensional vectors $\vect{ a_1, a_2, a_3 }$ and $\vect{ b_1, b_2, b_3 }$ are equal if $a_1=b_1, a_2=b_2,$ and $a_3 = b_3$.  The order of the numbers is important.}

\definitionNN{Sum of 3--dimensional vectors}{The sum of 3--dimensional vectors $\vect{ a_1, a_2, a_3 }$ and $\vect{ b_1, b_2, b_3 }$ is the 3--dimensional vector $\vect{ a_1+b_1, a_2+b_2, a_3+b_3 }$.  We write $\vec{a} \oplus \vec{b}$, using a new symbol so we don't confuse addition of vectors with addition of real numbers.}

\showNN{Show that addition of 3--dimensional vectors is commutative.
Start with ``Let,'' take one step at a time, write the justification for the step, and make a general conclusion.
}{4in}

\definitionNN{Dot product of 3--dimensional vectors}{The dot product of 3--dimensional vectors $\vect{ a_1, a_2, a_3 }$ and $\vect{ b_1, b_2, b_3 }$ is the real number $a_1 b_1 + a_2 b_2 + a_3 b_3$.}

\showNN{On the other side of this sheet of paper, show that the dot product is distributive over vector addition.
That is, show that $(\vec{a} \oplus \vec{b}) \bullet \vec{c} = \vec{a} \bullet \vec{c} + \vec{b} \bullet \vec{c}.$  Start with ``Let \ldots,'' take one step at a time, write the justification for the step, and make a general conclusion.  Please also explain why one addition symbol is $\oplus$ and the other is $+$.}{0in}

\vfill          % pad the rest of the page with white space

\activitytitle{Possible questions for the quiz over the Division Algorithm}{}

This will be a 40--point quiz, with two problems on it.
The problems may be chosen from the ones below or from new problems related to that activity.

I suggest that you write out solutions for each of these before the quiz, and that you try to do them without consulting your notes.
Rediscover the arguments, and you will own them.
Then, some hours later, write them again on a fresh sheet of paper.
This is the best way to learn them.

I will be happy to look at your practice solutions in office hours or just before or after class.

\blist{0.5in}
\item Let $n > 0$ and $k > 0$ be integers.
Argue that there exist integers $q$ and $r$ such that $n = qk + r$ and $0 \leq r < k$.
You can phrase the argument in terms of dealing out $n$ cards to $k$ people, or in terms of starting with $n$ and subtracting $k$ repeatedly.

\item Let $n > 0$ and $k > 0$ be integers.
Suppose that there exist integers $q$ and $r$ for which $n = qk + r$ and $0 \leq r < k$, and at the same time that there exist integers $q_2$ and $r_2$ for which $n = q_2 k + r_2$ and $0 \leq r_2 < k$.
Show that $q = q_2$ and $r = r_2$, including deriving any new inequalities that you need.
This shows that there is at most one way to write $n = qk + r$ with $0 \leq r < k$.

\item Let $n$ be an integer and suppose that $n = 3m+1$ for some integer $m$.
Use the uniqueness part of the Division Algorithm to argue that $n$ cannot be written as $n = 3k$ where $k$ is an integer.
Thus, $n$ is not a multiple of 3.

\item Let $n$ be an integer and suppose that $n^2$ is a multiple of 3.
Use the Division Algorithm to write $n$ as $3m$, $3m+1$, or $3m+2$, and then use the Division Algorithm to rule out the last two cases.
Make clear which part of the Division Algorithm you use in each part.

\item Let $n$ be even, so that $n = 2k$ for some integer $k$.
Use the Division Algorithm to write $k$ as $2j$ or $2j+1$.
For each case, show that exactly one of the numbers $n$ and $n+2$ is a multiple of 4.
This will also require the use of the Division Algorithm.

\elist
\vfill          % pad the rest of the page with white space

\yourname

\activitytitle{Quiz on things related to the Division Algorithm}{10 points}

\blist{5.5in}
\item Let $n > 0$ and $k > 0$ be integers.
Suppose that there exist integers $q$ and $r$ for which $n = qk + r$ and $0 \leq r < k$, and at the same time that there exist integers $a$ and $b$ for which $n = ak + b$ and $0 \leq b < k$.
Show that $q = a$ and $r = b$.
This shows that there is at most one way to write $n = qk + r$ with $0 \leq r < k$.

\item Let $k$ be an integer.
Show that exactly one of the integers $k, k+1, k+2, k+3$ is a multiple of 4.
You can use the back of this sheet of paper.
\elist

\vfill          % pad the rest of the page with white space

\yourname

\activitytitle{Quiz on even and odd}{15 points}

\noindent Work hard to write really nice proofs.

\blist{3.5in}
\item Show that the product of two odd numbers is odd.

\item Show that for all integers $n$, the quantity $n^2 + 10n + 21$ is either odd or is a multiple of 4.

\item The numbers $0, 1, 4, 9, 16, 25, \ldots$ are called perfect squares.  The differences between consecutive perfect squares are $1, 3, 5, 7, 9, \ldots$.
Show that the difference between consecutive perfect squares is always an odd number.


\elist
\vfill          % pad the rest of the page with white space

\yourname

\activitytitle{Quiz on infinite set operations}{5 points}

\blist{5.5in}
\item
Let $B = \bigcup_{n=0}^{\infty} [n, n^2]$.
List out the first five or more sets in this union.
Draw them on a number line if it helps.

Let $C = \{0\} \cup \{1\} \cup [2,\infty)$.
Show that $B = C$ by showing containment in both directions.
You will need to use three cases in each direction to deal with $0, 1,$ and the rest.
\elist

\vfill          % pad the rest of the page with white space

\yourname

\activitytitle{Quiz on a new operation with 3--dimensional vectors}{20 points}

\definitionNN{Sum of 3--dimensional vectors}{The sum of 3--dimensional vectors $\vect{ a_1, a_2, a_3 }$ and $\vect{ b_1, b_2, b_3 }$ is the 3--dimensional vector $\vect{ a_1+b_1, a_2+b_2, a_3+b_3 }$.}

\definitionNN{The {\em twist product} of 3--dimensional vectors $\vect{ a_1, a_2, a_3 }$ and $\vect{ b_1, b_2, b_3 }$ is the 3--dimensional vector $\vect{ a_1 b_3, a_2 b_2, a_3b_1 }$.
It is denoted $\vec{a} * \vec{b}$}

\exampleNN{For example, $\vect{1,3,6} * \vect{2,7,10} = \vect{1 \cdot 10, 3 \cdot 7, 6 \cdot 2} = \vect{10,21,12}$.}{0in}

\showNN{Show that the twist product is distributive over vector addition.
That is, show that $(\vec{a} \oplus \vec{b}) * \vec{c} = \vec{a} * \vec{c} \oplus \vec{b} * \vec{c}.$  Start with ``Let \ldots,'' take one step at a time, write the justification for the step, and make a general conclusion.}{4in}

\showNN{Prove or disprove:  ``The twist product is commutative.''}{2in}


\showNN{On the other side of this piece of paper, show that for all 3--dimensional vectors $\vec{a}$ and $\vec{b}$ and real numbers $c$, $\vec{a} * (c\vec{b}) = (c\vec{a})*\vec{b} = c(\vec{a} * \vec{b})$.
Use parentheses {\em every} time three things are multiplied together.}{0in}

\vfill          % pad the rest of the page with white space

\yourname

\activitytitle{Quiz on a new operation with 3--dimensional vectors}{20 points}

\definitionNN{Sum of 3--dimensional vectors}{The sum of 3--dimensional vectors $\vect{ a_1, a_2, a_3 }$ and $\vect{ b_1, b_2, b_3 }$ is the 3--dimensional vector $\vect{ a_1+b_1, a_2+b_2, a_3+b_3 }$.}

\definitionNN{Duplicate product}{The {\em duplicate product} of 3--dimensional vectors $\vect{ a_1, a_2, a_3 }$ and $\vect{ b_1, b_2, b_3 }$ is the 3--dimensional vector $\vect{ a_1 b_1, a_2 b_3, a_3 b_3 }$.
(That is not a typo, $b_3$ is used twice.  That is why it is called the duplicate product.)
It is denoted $\vec{a} * \vec{b}$.}

\exampleNN{For example, $\vect{1,3,6} * \vect{5,2,4} = \vect{1 \cdot 5, 3 \cdot 4, 6 \cdot 4} = \vect{5,12,24}$.}{0in}

\showNN{Show that the duplicate product is distributive over vector addition.
That is, show that $(\vec{a} \oplus \vec{b}) * \vec{c} = \vec{a} * \vec{c} \oplus \vec{b} * \vec{c}.$  Start with ``Let \ldots,'' take one step at a time, write the justification for the step, and make a general conclusion.}{4in}

\showNN{Prove or disprove:  ``The duplicate product is commutative.''  (You use a proof to prove, a counterexample to disprove.)}{2in}


\showNN{On the other side of this piece of paper, show that for all 3--dimensional vectors $\vec{a}$ and $\vec{b}$ and real numbers $c$, $\vec{a} * (c\vec{b}) = (c\vec{a})*\vec{b} = c(\vec{a} * \vec{b})$.
Use parentheses {\em every} time three things are multiplied together.}{0in}

\vfill          % pad the rest of the page with white space

\yourname

\activitytitle{Quiz on a new operation with 3--dimensional vectors}{20 points}

\definitionNN{Sum of 3--dimensional vectors}{The sum of 3--dimensional vectors $\vect{ a_1, a_2, a_3 }$ and $\vect{ b_1, b_2, b_3 }$ is the 3--dimensional vector $\vect{ a_1+b_1, a_2+b_2, a_3+b_3 }$.}

\definitionNN{The {\em twist product} of 3--dimensional vectors $\vect{ a_1, a_2, a_3 }$ and $\vect{ b_1, b_2, b_3 }$ is the 3--dimensional vector $\vect{ a_1 b_3, a_2 b_2, a_3b_1 }$.
It is denoted $\vec{a} * \vec{b}$}

\exampleNN{For example, $\vect{1,3,6} * \vect{2,7,10} = \vect{1 \cdot 10, 3 \cdot 7, 6 \cdot 2} = \vect{10,21,12}$.}{0in}

\showNN{Show that the twist product is distributive over vector addition.
That is, show that $(\vec{a} \oplus \vec{b}) * \vec{c} = \vec{a} * \vec{c} \oplus \vec{b} * \vec{c}.$  Start with ``Let \ldots,'' take one step at a time, write the justification for the step, and make a general conclusion.}{4in}

\showNN{Prove or disprove:  ``The twist product is commutative.''}{2in}


\showNN{On the other side of this piece of paper, show that for all 3--dimensional vectors $\vec{a}$ and $\vec{b}$ and real numbers $c$, $\vec{a} * (c\vec{b}) = (c\vec{a})*\vec{b} = c(\vec{a} * \vec{b})$.
Use parentheses {\em every} time three things are multiplied together.}{0in}

\vfill          % pad the rest of the page with white space

\yourname

\activitytitle{Quiz on some problems from Chapter 5 of Daepp and Gorkin}{15 points}

\blist{5in}
\item Let $x$ and $y$ be real numbers.  Use the triangle inequality to show that $||x| - |y|| \leq |x - y|$.

\item Prove or refute the following conjecture:  There are no positive integers $x$ and $y$ such that $x^2 - y^2 = 10$.  You can use the back of the sheet if you like.

\elist
\vfill          % pad the rest of the page with white space



%-----------------------------------------------------------------------------

\end{document}  % end of the document

