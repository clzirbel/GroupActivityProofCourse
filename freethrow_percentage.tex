\yourname

\activitytitle{Freethrow percentage}{A few thought problems to work on.}

\overview{The first question is inspired by a problem from the book ``Reading, Writing, and Proving: A Closer Look at Mathematics'' by Daepp and Gorkin.}

\exercise{Frieda Freethrow plays for the Bowling Green State University women's basketball team.
Early in the season, she had made 6 out of 10 freethrow attempts in games, giving her a 60\% freethrow percentage.
That was not good enough for her, so she practiced freethrows and eventually, later in the season, got her freethrow percentage up to 80\% to that point in the season.
The question is, was there a point in the season when her freethrow percentage was exactly 75\%?

Explore this question in two ways:  look at possible sequences of making or missing freethrows and the resulting freethrow percentages to see if you can find a way that she could have avoided hitting exactly 75\%, and look for ways to prove that she must have hit 75\% at some point.
There is no single way to formulate this question, so be creative in finding ways to look at how the freethrow percentage can change over the course of the season so you can make your best argument.
}{3in}

\vfill

\exercise{Freddie Freethrow plays for the BGSU men's basketball team.
His season started out well, he made 8 of his first 10 freethrows, giving him an 80\% freethrow percentage.
Freddie neglected practicing freethrows, and by some point later in the season, his freethrow percentage had dropped to 60\%.
Was there a point in the season when Freddie's freethrow percentage was exactly 75\%?

Again, look at possible sequences of freethrows to see if 75\% is always hit or can be skipped over and make your best argument.
}{0in}


\vfill          % pad the rest of the page with white space
