\yourname

\activitytitle{Integer-valued functions}{}

\overview{It is often helpful to replace a real number with a nearby integer.
There are standard functions from the real numbers to the integers to do this, and this activity helps you learn some of their properties and uses.}

\exercise{Think of a function $f$ with the following properties:  
First, $f : \R \to \Z$, meaning that the input to $f$ is a real number, and the output from $f$ will always be an integer.
Second, $f$ takes on the following values:
\[
  \begin{array}{cccccr}
	f(0.5) & = & 0 &\qqqq f(-3.2) & = & -4 \\
	f(0.9) & = & 0 &\qqqq  f(-10) & = & -10 \\
	f(1) & = & 1 &\qqqq f(-9.5) & = & -10 \\
	f(1.1) & = & 1 &\qqqq f(18.2) & = & 18 \\
  \end{array}
\]

\noindent
Humans have an amazing ability to generalize from examples like this.
Describe what $f$ does to a generic input number $x$:

\vspace*{0.2in}
\noindent
We will need the letter $f$ for other functions.
Common notation in mathematics for $f(x)$ is $\lfloor x \rfloor$.
In programming languages, people write $\floor(x)$.
In what way does the word floor describe what $f$ does?
}{0.3in}

\exercise{Think of a function $g$ with the following properties:  
First, $g : \R \to \Z$.
Second, $g$ takes on the following values:
\[
  \begin{array}{cccccr}
	g(0.5) & = & 1 &\qqqq g(-3.2) & = & -3 \\
	g(0.9) & = & 1 &\qqqq  g(-10) & = & -10 \\
	g(1) & = & 1 &\qqqq g(-9.5) & = & -9 \\
	g(1.1) & = & 2 &\qqqq g(18.2) & = & 19 \\
  \end{array}
\]

\noindent
Describe what $g$ does to a generic input number $x$:

\vspace*{0.2in}
\noindent
Common mathematical notation for $g(x)$ is $\lceil x \rceil$.
In programming languages, people write $\ceil(x)$, where ceil is short for ceiling.
In what way does the word ceiling describe what $g$ does?
}{0.3in}

\exercise{For what values of $x$ do we have ceil($x$) = floor($x$)+1?

\indent
Are there any values of $x$ for which the equality does not hold?
}{0.0in}

\exercise{What exactly is it about the definition of $\floor(x)$ that makes $x - \floor(x) < 1$?
}{0.5in}

\exercise{The inequalities listed below might be true for all $x$, or they might fail for some values of $x$.
If one sometimes fails, give a specific example of $x$ where it fails, called a {\em counterexample}, and calculate the quantities in the inequality to explain the counterexample.
If an inequality appears to be true for all $x$, prove it this way:  ``Let $x$ be a real number.  Consider two cases.  Case 1: Suppose $x$ is an integer.  Case 2: Suppose $x$ is not an integer.''  Show that the inequality is correct in both cases.

\balist{0.6in}

\item $x \leq \ceil(x)$

\item $x < \ceil(x)$

\item $0 \leq \ceil(x) - x < 1$ (check both inequalities)

\ealist
}{0.3in}

\exercise{Let $b$ be a real number.  
Let $n=\ceil(b)+1$.
Show that $n > b$; cite a previous result or give a reason for each step.
\begin{eqnarray*}
    n &=& \ceil(b) + 1 \qq \mbox{given}\\
       &>& \ceil(b) \qqqq \mbox{because \blank{1.5in}} \\
       &\geq& b \qqqq\qq \mbox{because \blank{1.5in}}
\end{eqnarray*}
}{0.0in}

\question{Let $x$ be a real number.  What is special about the integer $\floor(x+1)$?}{0.15in}

\question{Solve the equality $\frac{1}{x} = 0.07$ for $x$.}{0.2in}

\question{Let $x>0$.  Solve the inequality $\frac{1}{x} < 0.07$ for $x$.}{0.2in}

\question{Find the smallest integer $n>0$ for which $\frac{1}{n} < 0.07$.}{0.2in}

\question{Find the smallest integer $n>0$ for which $\frac{1}{n} < 0.004$.}{0.2in}

\question{Find the smallest integer $n>0$ for which $\frac{1}{n} < 0.00009$.}{0.2in}

\question{Let $\varepsilon > 0$.  Describe a procedure for finding the smallest integer $n>0$ for which $\frac{1}{n} < \varepsilon$.}{0in}



\vfill          % pad the rest of the page with white space
