\yourname

\activitytitle{Integer-valued functions}{}

\overview{This activity introduces some functions from the real numbers to the integers, and asks you to establish some of their properties.}

\exercise{Think of a function $f$ with the following properties:  
First, $f : \R \to \Z$, meaning that the input to $f$ is a real number, and the output from $f$ will always be an integer.
Second, $f$ takes on the following values:
\[
  \begin{array}{ccccccc}
	f(0.5) & = & 1 &\qqqq f(-3.2) & = & -3 \\
	f(0.9) & = & 1 &\qqqq  f(-10) & = & -10 \\
	f(1) & = & 1 &\qqqq f(-9.5) & = & -9 \\
	f(1.1) & = & 2 &\qqqq f(18.2) & = & 19 \\
  \end{array}
\]

\noindent
Humans have an amazing ability to generalize from examples like this.
Describe what $f$ does to a generic input number $x$.

\vspace*{0.2in}
\noindent
We will need the letter $f$ for other functions.
Decide among yourselves on new notation for $f$.
You can use special symbols, as with the notation $|x|$ or $n!$, or you can use multiple letters, as with $\sin(x)$ or $\ln(x)$.
}{0.3in}

\exercise{Think of a function $g$ with the following properties:  
First, $g : \R \to \Z$.
Second, $g$ takes on the following values:
\[
  \begin{array}{ccccccc}
	g(0.5) & = & 0 &\qqqq g(-3.2) & = & -4 \\
	g(0.9) & = & 0 &\qqqq  g(-10) & = & -10 \\
	g(1) & = & 1 &\qqqq g(-9.5) & = & -10 \\
	g(1.1) & = & 1 &\qqqq g(18.2) & = & 18 \\
  \end{array}
\]

\noindent
Describe what $g$ does to a generic input number $x$.

\vspace*{0.2in}
\noindent
Decide among yourselves on new notation for $g$.

\vspace*{0.2in}
\noindent
Can you write $g$ in terms of $f$?  If so, how?  If not, why not?
}{0.3in}

\exercise{Think of a function $h$ with the following properties:  
First, $h : \R \to \Z$.
Second, $h$ takes on the following values:
\[
  \begin{array}{ccccccc}
	h(0.5) & = & 1 &\qqqq h(-3.2) & = & -3 \\
	h(0.9) & = & 1 &\qqqq  h(-10) & = & -9 \\
	h(1) & = & 2 &\qqqq h(-9.5) & = & -9 \\
	h(1.1) & = & 2 &\qqqq h(18.2) & = & 19 \\
  \end{array}
\]

\noindent
Describe what $h$ does to a generic input number $x$.

\vspace*{0.2in}
\noindent
Can you write $h$ in terms of $f$ or $g$ or both?  If so, how?  If not, why not?
}{0.0in}

%=================================================================
\pagebreak

\exercise{Using the notation introduced on the previous page, several inequalities are listed below.
They might be true for all $x$, or they might fail for some values of $x$.
If an inequality is true for all $x$, say so.
If it fails for some $x$ values, gives a specific example of $x$, called a {\em counterexample}, and calculate the quantities in the inequality to explain the counterexample.

\balist{0.4in}

\item $x \leq f(x)$

\item $x < f(x)$

\item $f(x) - 1 < x \leq f(x)$

\item $g(x) \leq x$

\item $g(x) < x$

\item $g(x) \leq x < g(x)+1$

\item $0 \leq x - g(x) < 1$

\item $x < h(x)$

\ealist
}{0.0in}




\vfill          % pad the rest of the page with white space
