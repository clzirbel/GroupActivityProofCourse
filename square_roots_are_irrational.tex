\yourname

\activitytitle{Square roots of prime numbers are irrational}{This is a classic example of proof by contradiction.}

\overview{Most students are familiar with the fact that $\sqrt{2}$ is irrational, but few can prove it.  Having read a proof of this fact in your textbook or online, the starting point is to re-create the proof from memory, then to move on to showing that $\sqrt{3}$ is irrational.  The proof is similar and yet different.}

\definition{Irrational\label{irrational}}{A real number is said to be {\em irrational} if it cannot be written as the quotient of two integers.}

\note{If you are new to proof by contradiction, you might prefer to start the next proof by writing ``Let's pretend for a minute that $\sqrt{2}$ can be written as $\frac{p}{q}$ where $p$ and $q$ are integers.''  This makes it extra clear that you don't particularly believe that $\sqrt{2}$ is rational, you are just exploring what would happen if that were true.  When you arrive at a contradiction, you realize it's time to stop pretending; $\sqrt{2}$ must be irrational.}

\prove{Prove that $\sqrt{2}$ is irrational by contradiction.  The proof begins with ``Assume for the sake of contradiction that $\sqrt{2}$ can be written as $\frac{p}{q}$ where $p$ and $q$ are integers.''
Argue to a contradiction.}{2.7in}


\show{A key step in the proof is that if $n$ is an integer and $n^2$ is even, then $n$ is even.  You may have already shown this, using a proof by contradiction (which begins, ``Assume for the sake of contradiction that $n$ is odd.'') or a proof by contrapositive (which begins, ``Let us show the contrapositive, that if $n$ is odd, then $n^2$ is odd.'') or a proof by cases (which begins, ``There are two possibilities for $n$, that $n$ is even or that $n$ is odd.'')  Whichever one you have already seen, choose a different one and write the proof here.}{0in}

\pagebreak

\show{Mimic the proof that $\sqrt{2}$ is irrational to show that $\sqrt{3}$ is irrational.}{2.6in}

\show{A key step in the proof that $\sqrt{3}$ is irrational is the fact for an integer $n$ that: if $n^2$ is a multiple of 3, then $n$ is a multiple of 3.  Prove this by contradiction, starting with ``Assume for the sake of contradiction that $n$ is not a multiple of 3''.}{1.8in}

\show{Now write a proof by contrapositive that if $n^2$ is a multiple of 3, then $n$ is a multiple of 3.  Clearly state what the contrapositive is, then prove it.}{1.8in}

\show{Now write a proof by cases.}{0in}

\pagebreak

\show{Suppose that $n$ is an integer and that $n^2$ is a multiple of 5.
Show that $n$ is a multiple of 5.  Make the logic of your proof crystal clear.}{2in}

\show{Show that $\sqrt{5}$ is irrational.  Now that you are getting good at proofs like this, try to write a picture perfect proof.}{2in}

\show{Suppose that $n$ is an integer and that $n^2$ is a multiple of $p$, where $p$ is a prime number.
Make a good start on showing that $n$ is a multiple of $p$.}{2in}

\show{Suppose that $p$ is a prime number.  Show that $\sqrt{p}$ is irrational, assuming that the previous result is true.}{0in}

\pagebreak

\show{Suppose that $n$ is odd and suppose that $n^3 - n$ is a multiple of 24.  Show that $(n+2)^3 - (n+2)$ is also a multiple of 24.}{2.5in}

\show{Check that when $n=1$, $n^3 - n$ is a multiple of 24.
Use this together with the previous problem to conclude that $n^3-n$ is a multiple of 24 for additional values of $n$.
What values of $n$ can your argument cover?}{2in}

\show{Write an argument that will cover all other odd values of $n$}{0in}



\vfill          % pad the rest of the page with white space
