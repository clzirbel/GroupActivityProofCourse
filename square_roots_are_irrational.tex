\yourname

\activitytitle{Square roots of prime numbers are irrational}{This is a classic example of proof by contradiction.}

\overview{Most students are familiar with the fact that $\sqrt{2}$ is irrational, but few can prove it.  Having read a proof of this fact in your textbook or online, the starting point is to re-create the proof from memory, then to move on to showing that $\sqrt{3}$ is irrational.  The proof is similar and yet different.}

\definition{Irrational\label{irrational}}{A real number is said to be {\em irrational} if it cannot be written as the quotient of two integers.}

\prove{Prove that $\sqrt{2}$ is irrational by contradiction.  The proof begins with ``Assume that $\sqrt{2}$ can be written as $\frac{p}{q}$ where $p$ and $q$ are integers.''
Argue to a contradiction.}{3in}

\note{If you are new to proof by contradiction, you might prefer to start the previous proof by writing ``Let's pretend for a minute that $\sqrt{2}$ can be written as $\frac{p}{q}$ where $p$ and $q$ are integers.''  This makes it extra clear that you don't really believe that $\sqrt{2}$ is rational, you are just exploring what would happen if that were true.  When you arrive at a contradiction, you realize it's time to stop pretending; $\sqrt{2}$ must be irrational.}

\show{A key step in the proof is that if $n$ is an integer and $n^2$ is even, then $n$ is even.  You may have already shown this, using a proof by contradiction (which begins, ``Assume that $n$ is odd.'') or a proof by contrapositive (which begins, ``Let us show the contrapositive, that if $n$ is not odd, then $n^2$ is not odd.'') or a proof by cases (which begins, ``There are two possibilities for $n$, that $n$ is even or that $n$ is odd.)  Whichever one you used, choose another and write the proof here.}{2in}

\show{Mimic the proof that $\sqrt{2}$ is irrational to show that $\sqrt{3}$ is irrational.
Work with the members of your group to figure out how to do this.}{3in}

\show{A key step in the proof that $\sqrt{3}$ is irrational is the fact that if $n$ is an integer and $n^2$ is a multiple of 3, then $n$ is also a multiple of 3.  This is not as straightforward as with multiples of 2, but it can still be done by contradiction, by contrapositive, or by cases.
Think about these possibilities and choose the one that seems to you to be the best approach.}{2in}





\vfill          % pad the rest of the page with white space
