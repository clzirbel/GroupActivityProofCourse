\noindent {\bf After the even and odd activity}

(Pass back the even and odd activities that were collected at the end of the time last time.)

I hope you had a good time reading the chapter from the textbook.
Please open your book or binder to the first page of your notes and write your name in the upper right corner and write the total number of minutes you spent reading it.
Then take it over to Ying-Ju Chen, who will look through, make comments, etc. and give it back by the end of the class.
She'll be recording a number from 0 to 5 for each reading assignment that she looks at.
5 is good, it means that you did all the right things.

I made some comments on what you wrote for the activity from last time about even and odd, but I did not correct everything.
So ... there may be things on your paper that are not perfect but that I did not comment on.
There is plenty of time in the semester, your sensibilities will change and come around to good ways of doing problems.

It's quite OK if you did not finish everything.
We may go back to the ones we missed, but we may not.
Don't worry, we have plenty to do.

From the even and odd activity, I'd like to note that rewriting is a good method to use.  For example,
\[
    2m+6 = 2(m+3) = 2k, k = m+3
\]
Today we will have another activity in which rewriting is the key.

There is more than one correct way to do a problem.
In many cases, I wrote something that would be more direct.
It's not a criticism, just something to compare.
Over time, you will develop a sense for finding more direct arguments.

When we are working on the activity to day, a better question than, ``Is this what you wanted?'' would be, ``Does this work?''

A few people asked last time whether 0 is an integer.
Did you notice that we do not yet have a definition of ``integer''?
This can be a bit of a problem in math; you can't include every definition that you ever need all on one piece of paper.
You just have to jump in somewhere and make your way.
So it's clear, the set of integers is often denoted $\mathbb{Z}$ and consists of the numbers $\{\ldots, -3, -2, -1, 0, 1, 2, 3, \ldots \}$.

Did you notice that something amazing happened in number \ref{sumofevens}?
We established that something is true for {\em all} sums of even numbers.
That was because we started with two arbitrary even numbers $m$ and $n$ and showed, using the definition, that their sum is also even.
We didn't assume anything extra special about $m$ and $n$, like we didn't assume they were even and also assume that they were prime, or that they were multiples of 5, or anything like that.
Being even is all we assumed, so our proof will work again no matter what even numbers you bring to it.
That is how you prove something about all even numbers, even though there are infinitely many even numbers.
