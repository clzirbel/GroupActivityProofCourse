\yourname

\activitytitle{Special words in mathematics}{A short guide to how to use certain words}

\overview{Using the right words in the right situation shows that you understand the logical structure of what you are writing.
It also makes it clear to the reader what you mean.}

\definition{Let}{The word ``Let'' has two main uses in mathematics.

\balist{0.1in}
\item The word ``Let'' is used to introduce a new variable or other object and give it a specific value.  This is often used in proofs where you need to show the existence of some object, but is also used in many other contexts.
\begin{itemize}
\item Let $n = \frac{1}{a} + 1$, rounded up to the nearest integer.
\item Let $x = \frac{-b + \sqrt{b^2-4ac}}{2a}.$
\item Let $f(x) = \sin(x) + \cos(x)$.
\end{itemize}
The word ``set'' can also be used here.
\item The word ``Let'' can also be used to introduce a new variable with a generic value, which is used when writing a generic proof that is supposed to work for many values of the variable.
\begin{itemize}
\item Let $n \geq 1$.
\item Let $x > 0$.
\item Let $r \in \Q$.
\item Let $0 < x < 1$.
\end{itemize}
\elist
A very important point is that whenever you use the word ``Let'', you change the value of the variable.  So for example, if you start a proof by saying ``Let $a > 0$, then $a$ becomes a specific real number, but it could be any real number greater than 0.  
Based on this $a$, you may construct other variables like $n$ which depend on $a$.  For example, $n = 1/a$.
Later in the same proof, if you say, ``Let $a > 1$, then this changes the value of $a$, and any variable that depends on $a$ will lose its connection.
}

\definition{Suppose}{The word ``Suppose'' has two main uses in mathematics.
\balist{0.1in}
\item The word ``Suppose'' can be used to introduce cases, for example, to restrict consideration of an already--introduced variable to a smaller range of variables.  Using ``suppose'' this way does not introduce a new variable.
\begin{itemize}
\item Let $x \in \R$.  Case 1.  Suppose $x \geq 0$. ... Case 2. Suppose $x < 0$.
\item Let $x \in [3,7]$.  Case 1.  Suppose that $x < 4$.  ... Case 2.  Suppose that $x \geq 4$.
\end{itemize}
\item The word ``Suppose'' is also used to introduce a logical statement at the beginning of a theorem or proof.
\begin{itemize}
\item Suppose that the function $f$ is continuous on the interval $[a,b]$.
\item Suppose that $n$ is an odd integer.
\end{itemize}
These are the only ways that ``Suppose'' can be used to introduce a new variable.  Note that ``Let'' could also have been used.
\elist
}

\definition{Assume}{The word ``Assume'' is most often used to introduce a proof by contradiction.
Because it is helpful to know that a proof by contradiction is coming, it is helpful to use familiar wording.
You can say things like:
\balist{0in}
\item Assume that $\sqrt{2}$ is rational.
\item Assume for the sake of contradiction that $\sqrt{2}$ is rational.
\item Pretend for a minute that $\sqrt{2}$ is rational.  (Recommended in this class, but unconventional outside this class.
\elist}

\remark{The word ``Any'' is ambiguous and I recommend that you avoid using it.  Sometimes it means ``for all'' and sometimes it means ``there exists''.
Consider this example:
\balist{0.1in}
\item Suppose that $n > a$ for any non-negative integer $n$.  This would be true {\bf for all} non-negative integers $n$ if $a = -5$.  But it would be true for {\bf some} non-negative integer $n$ if $a=10$.  The meaning is ambiguous.
\elist
}



\vfill          % pad the rest of the page with white space
