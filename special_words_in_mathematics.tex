\yourname

\activitytitle{Special words in mathematics}{A short guide to how to use certain words}

\overview{Using the right words in the right situation shows that you understand the logical structure of what you are writing.
It also makes it clear to the reader what you mean.}

\definition{Let}{The word ``Let'' has two main uses in mathematics, both of them in proofs.

\balist{0.1in}
\item The word ``Let'' is used to introduce a new variable or other object and give it a specific value.  This is often used in proofs where you need to show the existence of some object, but is also used in many other contexts.
\begin{itemize}
\item Let $f(x) = \sin(x) + \cos(x)$.
\item Let $x = \frac{-b + \sqrt{b^2-4ac}}{2a}.$
\item Let $n = \frac{1}{a} + 1$, rounded up to the next integer.
\end{itemize}
The word ``set'' can also be used here in place of ``let.''
\item The word ``Let'' can also be used to introduce a new variable having a particular property:
\begin{itemize}
\item Let $n$ be even.
\item Let $a > 0$.
\item Let $r \in \Q$.
\item Let $0 < x < 1$.
\end{itemize}
These statements cause the variable to take on a specific value.
We don't know the specific value, only that it has the property we make it have.
This is very useful when writing a generic proof that is supposed to work for all values of the variable having the property.
\elist
A very important point is that whenever you use the word ``Let'', you change the value of the variable.  So for example, if you start a proof by saying ``Let $a > 0$,'' then $a$ becomes a specific real number.  
Based on this $a$, you may construct other variables like $b$ which depend on $a$.  For example, $b = 1/a$.
Later in the same proof, if you say, ``Let $a \geq 1$,'' then this changes the value of $a$, and any variable that depends on $a$ will lose its connection.
Instead, you may want to think of $a \geq 1$ as a case to consider and use the word ``suppose.''
}

\definition{Suppose}{The word ``Suppose'' has two main uses in mathematics.
\balist{0.1in}
\item The word ``Suppose'' can be used to introduce cases in a proof, for example, to restrict consideration of an already--introduced variable to a smaller range of variables.  Using ``suppose'' this way does not introduce a new variable or change the value of the variable.
\begin{itemize}
\item Let $a > 0$.  Case 1.  Suppose $a \geq 1$. \ldots  Case 2. Suppose $0 < a < 1$. \ldots
\item Let $x \in [3,7]$.  Case 1.  Suppose that $x < 4$.  ... Case 2.  Suppose that $x \geq 4$.
\end{itemize}
\item The word ``Suppose'' is also used to introduce a logical statement at the beginning of a theorem or proof.
\begin{itemize}
\item Suppose that the function $f$ is continuous on the interval $[a,b]$.
\item Suppose that $n$ is an odd integer.
\end{itemize}
\elist
}

\definition{Assume}{The word ``Assume'' is most often used to introduce a proof by contradiction.
Because it is helpful to know that a proof by contradiction is coming, it is helpful to use familiar wording.
You can say things like:
\balist{0in}
\item Assume that $\sqrt{2}$ is rational.
\item Assume for the sake of contradiction that $\sqrt{2}$ is rational.
\item Pretend for a minute that $\sqrt{2}$ is rational.  (Recommended in this class, but unconventional outside this class.)
\elist}

\remark{The word ``any'' is ambiguous and it is best to avoid using it.  Sometimes it means ``for all'' and sometimes it means ``there exists'' or ``for some'' and sometimes you just can't tell.
Consider these examples:
\balist{0.1in}
\item Let $f(x) = \sin(4x)$ for any real number $x$.  ``any'' means: \blank{1in}
\item Is it true that $\sin(x) = x$ for any value of $x$?  ``any'' means:  \blank{1in}
\item Is it true that $\sin^2(x) + \cos^2(x) = 1$ for any $x$?  ``any'' means:  \blank{1in}
\item Let $a$ be a real number.  Suppose that $n > a$ for any non-negative integer $n$.  This would be true for all / for some (circle one choice) non-negative integers $n$ if $a = -5$.  But it would be true for {\bf some} non-negative integer $n$ if $a=10$.  The meaning is ambiguous.
\elist
}

\example{{\bf A badly told story.}  
Amanda was a sophomore in college.
One day after class, she went to study in the park.
She walked past a family at a picnic table and headed toward a shady tree.
Barney said, ``This next test is going to be really hard!''
Amanda told Barney to relax.

\vspace{0.1in}
\noindent
Who is Barney?!?  We haven't been introduced.  Does he know Amanda?  Were they walking together?  Were they meeting to study?

\vspace{0.1in}
\noindent
Writing a proof is a bit like tellling a story.  It's important to introduce the variables you use.  Don't let a variable barge in without introduction like Barney did.  Make sure to relate a new variable to existing variables the first time it enters the story.}{0in}

\exercise{Rewrite the story about Amanda and Barney.}{1in}

\example{{\bf Another story.}
Rex, Bill, and Ted went for a walk.
Bill said, ``Rex, I see a cat over there.  Please chase it away.''
Rex barked at the cat, the cat ran away, Bill squeaked in delight, and Ted whinnied in amusement.

\vspace{0.1in}
\noindent
In this story, names were introduced, but the names did not suggest that Rex, Bill, and Ted are animals, and we the reader is left to try to figure out what types of animals.
What is Rex? \blank{1in}  Bill? \blank{1in}  Ted? \blank{1in}
When you introduce variables, tell what type of thing each variable represents, whether an integer, rational number, real number, vector, function, set, matrix, etc.
}{0in}

\exercise{Rewrite the story about Rex, Bill, and Ted.}{1in}

\example{Fill in the blanks in this proof with the appropriate words and/or definitions.

Show that for all real numbers $a > 0$, there is an integer $n$ with $\frac{1}{n^2} < a$.

\vspace*{0.1in}
\noindent
Let $a > 0$.

Case 1.  \blank{1in} that $a \geq 1$.  \blank{1.5in}  Then $\frac{1}{n^2} = \frac{1}{4} < 1 \leq a$, and so $\frac{1}{n^2} < a$.

Case 2.  \blank{1in} that $a < 1$.  Let $n$ be the next integer larger than $\frac{1}{\sqrt{a}}$.  Then $n > \frac{1}{\sqrt{a}}$.
Squaring both sides, $n^2 > \frac{1}{a}$.  Taking reciprocals, $\frac{1}{n^2} < a$, as desired.

In each case, we have shown the existence of an integer $n$ with the desired property.
Thus, \blank{1in} $a > 0$, there is an integer $n$ with $\frac{1}{n^2} < a$.
}{0in}




\vfill          % pad the rest of the page with white space
