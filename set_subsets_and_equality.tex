\yourname

\activitytitle{Set subsets and equality}{This activity works on showing that one set is a subset of another, and showing equality between two sets.}

\exercise{Let $T$ be the set of all multiples of 3, and let $S$ be the set of all multiples of 6.

\noindent List at least 5 elements of $T$:

\noindent List at least 5 elements of $S$:

\noindent
Determine which of the following set inclusions is true.
If not true, list at least one element which serves as a counterexample.
\balist{0.0in}
\item $T \subseteq S$
\item $S \subseteq T$
\ealist

\noindent
If one of the following set inclusions is true, list five elements that account for the strict set inclusion.
\balist{0.0in}
\item $T \subset S$
\item $S \subset T$
\ealist
}{0in}

\exercise{Let $C$ denote the set of composite numbers and let $E$ denote the set of even numbers.
Determine which of the following set inclusions is true.
If not true, list up to five elements which serve as counterexamples.
\balist{0.0in}
\item $C \subseteq E$
\item $E \subseteq C$
\ealist
}{0in}

\exercise{Let $\R$ denote the real numbers, $\Z$ denote the integers, and $\Q$ denote the rational numbers.
Express the most informative set inclusions between these sets.
You do not need to prove them.}{0.1in}

\problem{
Let $\Q$ denote the rational numbers.
Let $A  = \{x \in \R : x$ solves $x^2=a$ where $a$ is an integer and $a \geq 0\}$. 
List out at least 5 elements of $A$: \blank{1.5in}

\noindent
Show that $A \not\subseteq \Q$ by giving one concrete example and telling how it relates to $A$ and to $\Q$.
}{0.5in}

\problem{Continuing the previous problem, show that $\Q \not\subseteq A$ by giving one concrete example and telling how it relates to $A$ and $\Q$.
}{0.5in}

\guidedproof{\label{guidedsubsetproof}
When showing that $A \subseteq B$, you need to show that every element of $A$ is also an element of $B$.
Here is how to do it.
Let $x \in A$.
Use this fact and the definitions of $A$ and $B$ to show that $x \in B$.
Since you made no further assumptions about $x$, this shows that every element of $A$ is also an element of $B$.
Note:  Your proof must start with ``Let $x \in A$.'' and must get to ``Thus $x\in B$.'' before generalizing.
Note:  Often $B$ will have a specific membership requirement, and you need to show that $x$ exactly meets the requirement.
}

\guidedproof{\label{guidedpropersubsetproof}
When showing that $A \subset B$, you need to show that $A \subseteq B$ and you need to show that there is an element of $B$ that is not in $A$.
If you can construct such an element, do that, because that should make the proof clearer to the reader.
}

\guidedproof{\label{strictsubsetmodel}
Let $\Z$ denote the integers and $\Q$ denote the rational numbers.
Recall that $\Q = \{ x : x = \frac{p}{q}$ for some integers $p$ and $q$ with $q \ne 0 \}$.
Show that $\Z \subset \Q$ following the model.
\balist{0.1in}
\item Let $m \in \Z$.  $m$ meets the definition to be in $\Q$ because we can write it as \blank{1in} where $p = \blank{0.5in}$ and $q = \blank{0.5in}$.
Thus $m \in \Q$.
We made no further assumption about $m \in \Z$.
Thus for all $m \in Z$, we know that $m \in \Q$.
Thus $\Z \subseteq \Q$.
\item To see that $\Z \subset \Q$, note that \blank{1in} is in $\Q$ but is not in $\Z$.  (One concrete example.)
\ealist
}

\prove{Let $\R$ denote the real numbers and $\C$ denote the complex numbers.
Recall that $\C = \{ a + ib : a, b \in \R\}$ where $i = \sqrt{-1}$, which is not a real number.
Show that $\R \subset \C$ following the model in the previous problem.
\balist{0.3in}
\item
\item
\ealist
}{0.3in}

\prove{Let $2\Z = \{ m \in \Z :$ there exists $j \in \Z$ such that $m = 2j \}$ and $6\Z = \{ m \in \Z :$ there exists $j \in \Z$ such that $m = 6j \}$.
Show that $6\Z \subset 2\Z$, following the model in \ref{strictsubsetmodel}.}{1in}

\definition{Intervals of real numbers.}{Let $a$ and $b$ be real numbers with $a \leq b$.
We define the following four types of intervals:
\balist{0.1in}
\item $(a,b) = \{ x \in \R : a < x < b\}$.  We say that both endpoints are open.
\item $[a,b) = \{ x \in \R : a \leq x < b\}$.  The compound inequality $a \leq x \leq b$ means $a \leq x$ and $x \leq b$.
\item $(a,b] = \{ x \in \R : a < x \leq b\}$
\item $[a,b] = \{ x \in \R : a \leq x \leq b\}$.  We say that both endpoints are closed.
\ealist
}

\exercise{Sketch the following intervals on separate number lines.
To indicate open endpoints, draw an open circle like $\circ$.
To indicate closed endpoints, draw a closed circle like $\bullet$.
\balist{0.1in}
\item $(2,9)$
\item $[2,7)$
\item $(3,5]$
\item $[0,1]$
\elist
}{0in}

\note{Inequalities between real numbers have the {\em transitivity} property:  If $a \leq b$ and $b \leq c$, then we can conclude that $a \leq c$.  Similar inequalities are true with $\geq, <,$ and $>$.}

\exercise{Using standard interval notation, show that $(4,9] \subset [2,9]$.
Begin with ``Let $x \in (4,9].''$ then rewrite this as a compound inequality, then rewrite as two separate inequalities.  
Note when you use transitivity. 
Make sure to write that $2 \leq x \leq 9$.
Make sure to show that $(4,9]$ is a proper subset of $[2,9]$.}{1.5in}

\definition{Set equality}{For sets $A$ and $B$, we say that $A$ equals $B$ when the elements of $A$ are exactly the same as the elements of $B$.
We write $A=B$ when $A$ and $B$ are equal sets.
}

\example{
The set consisting of all colors of a rainbow and the set consisting of colors of white light observed through a prism are equal sets.
}{0in}

\guidedproof{One way to show that $A=B$ is to {\em show inclusion both ways}.
That means to show that $A \subseteq B$ and $B \subseteq A$.
Here is how you do it.
\blist{0.2in}
\item To show $A \subseteq B$:  Let $x \in A$.  Use this fact and the definitions of $A$ and $B$ to show that $x \in B$.
Having made no further assumption about $x \in A$, you can conclude that $A \subseteq B$.
\item To show $B \subseteq A$: \blank{1.5in}.  Use this fact and the definitions of $A$ and $B$ to show that \blank{1in}.
Having made no further assumption about \blank{1in}, you can conclude that \blank{1in}.
\elist
}

\exercise{Let $A  = \{x : x$ solves $ax=b$ where $a$ and $b$ are integers and $a \ne 0 \}$. 
Let $\Q  = \{x : x$ is a rational number$\}$.
Show that $A = \Q$ by showing set inclusion both ways.
\balist{0in}
\item Let $x \in A$.  
Then there exist $a$ and $b$ such that \blank{1in} and $a \ne 0$.
Dividing through by $a$, $x =$ \blank{1in} where $a$ and $b$ are integers and $a$ is not zero.
Thus, $x \in \Q$.
Since $x$ was arbitrary, $A \subseteq \Q$.

\item Let $x \in \Q$.

\elist}{0.6in}

\exercise{
Let $A = \{ x: (x-3)^2 - 1 < 3\}$ and let $B = (1,5)$.
Show that $A = B$ by showing set inclusion both ways.}{1.5in}

\exercise{
Let $A = \{ x: -x^2 + 5x + 14 \geq 0 \}$ and let $B = [-2,7]$.
Show that $A = B$ by showing set inclusion both ways.
}{1.5in}

\problem{Let $2\Z = \{ m \in \Z :$ there exists $j \in \Z$ such that $m = 2j\}$.
Let $3\Z = \{ m \in \Z :$ there exists $j \in \Z$ such that $m = 3j\}$, and similarly with other sets like $5\Z$ and $15\Z$.
Show that $2\Z \cap 3\Z = 6\Z$ by showing set inclusion both ways.
}{1.5in}

\problem{Write out all elements in $6\Z \cap 8\Z \cap \{1, 2, 3, \ldots, 100\}$.
}{0.2in}

\exercise{
Let $\R^3$ denote the set of all 3-dimensional vectors and let 
$S = \{ v : v = t_1 \vect{1,0,0} + t_2 \vect{1, 1, 0} + t_3 \vect{1, 0, 1}$ where $t_1, t_2, t_3$ are real numbers $\}$.
\balist{0.6in}
\item Show that $S \subseteq \R^3$.
\item Show that $\R^3 \subseteq S$.  Now, {\em you} determine what values of $t_1, t_2,$ and $t_3$ will work to make $v$ have the form of an element of $S$.
\elist
}{0.5in}

\vfill          % pad the rest of the page with white space
