\activitytitle{Reading assignment, Chapter 3}{}

Read Chapter 3 of the textbook by Daepp and Gorkin.
As with Chapters 1 and 2,
\blist{0.0in}
\item Read somewhere quiet, minimizing distractions from phones and friends
\item Note the time that you start and stop reading, and add up the minutes
\item Read with a pencil in your hand and your notebook open in front of you
\item Write a sentence to summarize each paragraph, re--draw diagrams, work out examples and exercises on your own
\item Look up words you don't know, and write down ones you really don't know
\item Read slowly.  You are not reading a comic book or a newspaper.  It is not a goal of this class for you to learn how to read faster.  The goal is to learn how to get more out of the time you spend reading, and to learn to concentrate for longer periods of time.
\item At the end, tally up how much time you have spent on reading this chapter.
Write this number in your notebook and remember the number when you come to class.
\elist

Theorem 3.1 lists three properties of logical statements.
Please make truth tables for each of them to check that they are tautologies.
Also add de Morgan's laws from Theorem 2.9.
Then you'll have the whole set.
Having de Morgan's laws handy should make Exercise 3.2 easier.

How can you remember the distributive property?

The contrapositive is really important.
See if you can explain it just by thinking about $P \to Q$ and $\neg Q \to \neg P$, without using truth tables.

Theorem 3.3 is proven using the contrapositive.
This is a very useful method of proof.
Please note that it differs from proof by contradiction.

Read about the converse, and make sure never to confuse an implication with its converse.

Problems 2, 3, 4, 9, 14, 16, 18, 5, 6, 8, 19, 15 are good to work on, in that order.
Work through at least half of these problems.


Chapters 1 to 5 are mostly there to help develop proof techniques.  After Chapter 5, we will spend more of our time on definitions, examples, theorems, and proofs.
Use your time now to develop basic logic and proof techniques that will help you for the rest of the semester and beyond!


\vfill          % pad the rest of the page with white space
