\activitytitle{Possible questions for the quiz over Contrapositive, Elimination, and Contradiction}{}

The problems may be chosen from the ones below or from new problems related to that activity.

I suggest that you write out solutions for each of these before the quiz, and that you try to do them without consulting your notes.
Rediscover the arguments, and you will own them.
Then, some hours later, write them again on a fresh sheet of paper.
This is the best way to learn them.

I will be happy to look at your practice solutions in office hours or just before or after class.

\blist{0.5in}
\item Let $n$ be an integer.  Show that if $n^2$ is a multiple of 7, then $n$ is a multiple of 7 by using the process of elimination.
Carefully and explicitly use the Division Algorithm to generate 7 cases, one of which must be true.
You might want to name these $P_0, P_1, \ldots, P_6$.
Then show that six of the cases are false, so that the remaining case must be true.
To save time, calculate $(7k+r)^2$ just once, and then substitute in different values of $r$.
Make the overall logic of the argument crystal clear.

\item Show that $\sqrt{7}$ is irrational.
Follow the model that $\sqrt{2}$ is irrational, by pretending for a minute that $\sqrt{7}$ is rational and making logical deductions that lead to a statement known to be false.

\item Show that there are infinitely many prime numbers.
The proof in class was a fill-in-the-blank proof.  Here, you will need to know the proof and understand every part.
In particular, explain clearly why $n$ is not a multiple of $p_1$, why $n$ is not a multiple of $p_2$, etc.

\item Let $k$ and $j$ be integers and suppose that $2k = 3j$.

a) Use the process of elimination to show that $j$ is a multiple of 2.

b) Use the process of elimination to show that $k$ is a multiple of 3.

\item Let $n$ be an integer and suppose that $n^2$ is a multiple of 6.
Show that $n$ is a multiple of 6.

\item Show that $\sqrt{6}$ is irrational, using the previous result.

\item Show that $2\sqrt{2}$ is irrational.  Thus, $\sqrt{8}$ is irrational.

\elist
\vfill          % pad the rest of the page with white space
