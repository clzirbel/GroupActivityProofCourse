\activitytitle{Possible questions for the quiz over the Division Algorithm}{}

This will be a 10--point quiz.
I will choose two of the following four problems for the quiz.
I strongly suggest that you write out solutions for each of these before Thursday, and that you try to do them without consulting your notes.
Rediscover the arguments, and you will own them.

Then, some hours later, write them again on a fresh sheet of paper.
This is the best way to learn them.

I will be happy to look at your practice solutions before class on Thursday.
If you have questions, you may ask them by email.

\blist{0.5in}
\item Let $n > 0$ and $k > 0$ be integers.
Argue that there exist integers $q$ and $r$ such that $n = qk + r$ and $0 \leq r < k$.
You can phrase the argument in terms of dealing out $n$ cards to $k$ people, or in terms of starting with $n$ and subtracting $k$ repeatedly.

\item Let $n > 0$ and $k > 0$ be integers.
Suppose that there exist integers $q$ and $r$ for which $n = qk + r$ and $0 \leq r < k$, and at the same time that there exist integers $a$ and $b$ for which $n = ak + b$ and $0 \leq b < k$.
Show that $q = a$ and $r = b$.
This shows that there is at most one way to write $n = qk + r$ with $0 \leq r < k$.

\item Let $k$ be an integer.
Show that exactly one of the integers $k, k+1, k+2, k+3$ is a multiple of 4.
{\bf Note:} I may ask instead for you to show that exactly one of the numbers $k, k+1, k+2$ is a multiple of 3, or that exactly one of the numbers $k, k+1, k+2, k+3, k+4$ is a multiple of 5.  The argument is basically the same in every case.

\item Let $n$ be an odd integer.
Show that $n^3 - n$ is a multiple of 24.

\elist

\vfill          % pad the rest of the page with white space
