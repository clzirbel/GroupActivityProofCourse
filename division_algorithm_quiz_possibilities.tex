\activitytitle{Possible questions for the quiz over the Division Algorithm}{}

This will be a 40--point quiz, with two problems on it.
The problems may be chosen from the ones below or from new problems related to that activity.

I suggest that you write out solutions for each of these before the quiz, and that you try to do them without consulting your notes.
Rediscover the arguments, and you will own them.
Then, some hours later, write them again on a fresh sheet of paper.
This is the best way to learn them.

I will be happy to look at your practice solutions in office hours or just before or after class.

\blist{0.5in}
\item Let $n > 0$ and $k > 0$ be integers.
Argue that there exist integers $q$ and $r$ such that $n = qk + r$ and $0 \leq r < k$.
You can phrase the argument in terms of dealing out $n$ cards to $k$ people, or in terms of starting with $n$ and subtracting $k$ repeatedly.

\item Let $n > 0$ and $k > 0$ be integers.
Suppose that there exist integers $q$ and $r$ for which $n = qk + r$ and $0 \leq r < k$, and at the same time that there exist integers $q_2$ and $r_2$ for which $n = q_2 k + r_2$ and $0 \leq r_2 < k$.
Show that $q = q_2$ and $r = r_2$, including deriving any new inequalities that you need.
This shows that there is at most one way to write $n = qk + r$ with $0 \leq r < k$.

\item Let $n$ be an integer and suppose that $n = 3m+1$ for some integer $m$.
Use the uniqueness part of the Division Algorithm to argue that $n$ cannot be written as $n = 3k$ where $k$ is an integer.
Thus, $n$ is not a multiple of 3.

\item Let $n$ be an integer and suppose that $n^2$ is a multiple of 3.
Use the Division Algorithm to write $n$ as $3m$, $3m+1$, or $3m+2$, and then use the Division Algorithm to rule out the last two cases.
Make clear which part of the Division Algorithm you use in each part.

\item Let $n$ be even, so that $n = 2k$ for some integer $k$.
Use the Division Algorithm to write $k$ as $2j$ or $2j+1$.
For each case, show that exactly one of the numbers $n$ and $n+2$ is a multiple of 4.
This will also require the use of the Division Algorithm.

\elist
\vfill          % pad the rest of the page with white space
