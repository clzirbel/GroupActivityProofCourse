\yourname

\activitytitle{Examples of sets and relations between them}{This activity introduces sets, ways to write them, and the relations between them.}

\overview{Many familiar ideas can be expressed using sets.
We begin with examples of sets and the relations between them.}

\problem{Let $A  = \{x : x$ solves $ax=b$ where $a$ and $b$ are integers and $a \ne 0 \}$. 
Let $\Q  = \{x : x$ is a rational number $\}$.
Show that $A = \Q$ by showing set inclusion in both directions.
The first part is done for you; read that carefully.\\
\vspace*{-0.3in}
\begin{itemize}
\item Let $x \in A$.  
Then there exist $a$ and $b$ such that $ax = b$ and $a \ne 0$.
Dividing through by $a$, $x = \frac{b}{a}$ where $a$ and $b$ are integers and $a$ is not zero.
Thus, $x \in \Q$.
Since $x$ was arbitrary, $A \subseteq \Q$.

\item Let $x \in \Q$.

\end{itemize}}{0.8in}

\problem{Let $A = \{ f : f $ is a continuous function from $\R$ to $\R \}$.
Let $B = \{ f : f $ is a differentiable function from $\R$ to $\R \}$.
Determine whether $A \subset B$, $B \subset A$, $A \subseteq B$, or $B \subseteq A$ and then write a clean argument that it is so.
Remember that to show $\subset$, you need an example of an element that is in one set but not in the other.
}{1.2in}

\problem{Let $A  = \{x \in \R : x$ solves $x^2=a$ where $a$ is an integer and $a \geq 0\}$. 
Show that $A \not\subset \Q$.
Make your logic crystal clear.
}{1in}

\problem{Continuing the previous problem, show that $\Q \not\subset A$.
Make your logic crystal clear.
}{1in}

\problem{Let $E = \{ m \in \Z : $ there exists $j \in \Z$ such that $m = 2j \}$.
Let $O = \{ m \in \Z : $ there exists $k \in \Z$ such that $m = 2k + 1 \}$.
Show that $E \cap O = \emptyset$ by letting $m \in E \cap O$ and showing that this leads to a contradiction.
}{1in}

\problem{Continuing the previous problem, show that $E \cup O = \Z$ by showing set inclusion both ways.
}{1in}

\problem{Let $2\Z = \{ m \in \Z :$ there exists $j \in \Z$ such that $m = 2j\}$.
Let $3\Z = \{ m \in \Z :$ there exists $j \in \Z$ such that $m = 3j\}$, and similarly with other sets like $5\Z$ and $15Z$.
Show that $2\Z \cap 3\Z = 6\Z$ by showing set inclusion both ways.
}{1in}

\problem{Write out all elements in $6\Z \cap 8\Z \cap \{1, 2, 3, \ldots, 100\}$.
}{1in}

\note{Inequalities between real numbers have the {\em transitivity} property:  If $a \leq b$ and $b \leq c$, then we can conclude that $a \leq c$.  Similar inequalities are true with $\geq, <,$ and $>$.}

\problem{Suppose that $x > 4$.  Argue that $x \geq 2$.}{1in}

\problem{Using standard interval notation, show that $(4,9] \subset [2,9]$.
Begin with ``Let $x \in (4,9].''$ then rewrite this as a compound inequality, then rewrite as two separate inequalities.  Use transitivity along the way. Make sure to }{1.5in}

\problem{Using standard interval notation, show that $[2,6) \cap [3,8) = [3,6)$ by showing set inclusion both ways. As above, write compound inequalities, then individual inequalities, then compound inequalities again.  Use a number line to illustrate.}{2in}

\problem{Show that $[2,6) \cup [3,8) = [2,8)$ by showing set inclusion both ways.}{0in}


\vfill          % pad the rest of the page with white space
