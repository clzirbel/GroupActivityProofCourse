\yourname

\activitytitle{Sum and dot product of 3--dimensional vectors}{We define a new mathematical object and do some work with it.}

\overview{In Calculus III and Linear Algebra, one defines vectors and works with them.  They have a geometric interpretation, but here we will simply give an algebraic definition and work with their algebraic properties.
This activity illustrates proofs in which all that is needed is the definition and some rewriting.
Notice how we often use the same definition twice in one proof, once to ``unpack'' and the second time to ``re--pack.''}

\definition{3--dimensional vector\label{3dvectordef}}{A three--dimensional vector is an ordered triple $\vect{ a_1, a_2, a_3 }$, where $a_1, a_2,$ and $a_3$ are real numbers.}

\notation{A 3--dimensional vector $\vect{ a_1, a_2, a_3 }$ is often denoted by a single letter with an arrow over the top, like this $\vec{a}$.  When it is written like $\vect{ a_1, a_2, a_3 }$ it is said to be in {\em open form.}}

\definition{Equality of 3--dimensional vectors\label{3dvectorequalitydef}}{3--dimensional vectors $\vect{ a_1, a_2, a_3 }$ and $\vect{ b_1, b_2, b_3 }$ are equal if $a_1=b_1, a_2=b_2,$ and $a_3 = b_3$.  The order of the numbers is important.}

\definition{Sum of 3--dimensional vectors\label{3dvectorsumdef}}{The sum of 3--dimensional vectors $\vect{ a_1, a_2, a_3 }$ and $\vect{ b_1, b_2, b_3 }$ is the 3--dimensional vector $\vect{ a_1+b_1, a_2+b_2, a_3+b_3 }$.  We write $\vec{a} \oplus \vec{b}$, using a new symbol so we don't confuse addition of vectors with addition of real numbers.}

\example{Is $\vect{ 3,9,12 }$ a 3--dimensional vector? Explain.  Is it equal to $\vect{12,3,9}$?  Explain.}{0.4in}

\example{Is $\vect{ \sqrt{3},\sqrt[3]{9},\sqrt[4]{-12} }$ a 3--dimensional vector? Explain.}{0.3in}

\example{Is $\vect{ 8,13.35321,\pi,-7 }$ a 3--dimensional vector? Explain.}{0.3in}

\example{Is $\vect{ 1,\pm 7,\heartsuit }$ a 3--dimensional vector? Explain.}{0.3in}

\example{Is $\vect{ 3 + 9 + 12 }$ a 3--dimensional vector? Explain.}{0.3in}

\example{Is $\vect{ \left[ \begin{array}{cc} 6 & 0 \\ 2 & 5 \end{array} \right], -4, 7 }$ a 3--dimensional vector?}{0.4in}

\example{Let $x$ be a real number.  Is $\vect{ \frac{14}{3},2-7x,\sqrt{16} }$ a 3--dimensional vector? Explain.}{0.3in}

\stop{Compare your answers to the questions above with the members of your group.  Make sure you agree on everything.}
\pagebreak

\example{Calculate the sum of $\vec{a} = \vect{ 12,-5,3 }$ and $\vec{b} = \vect{ 6,4,-11 }$.  Start by writing $\vec{a} \oplus \vec{b} = \ldots$ and write the vectors in open form next.
}{0.5in}

\show{We are going to show that addition of 3--dimensional vectors is commutative.  We will do this by rewriting.  Follow the model.\\
Let $\vec{a}$ and $\vec{b}$ be 3--dimensional vectors.  Then,
\begin{eqnarray*}
    \vec{a} \oplus \vec{b}
    &=& \vect{ \qqq,\qqq,\qqq } \oplus \vect{\qqq,\qqq,\qqq} \qqqq \qqqq \\
    &=& \vect{ \qqq\qqq,\qqq\qqq,\qqq\qqq } \\
    &=& \vect{ \qqq\qqq,\qqq\qqq,\qqq\qqq } \\
    &=& \vect{ \qqq,\qqq,\qqq } \oplus \vect{\qqq,\qqq,\qqq} \\
    &=& \vec{b} \oplus \vec{a}
\end{eqnarray*}
We have seen that $\vec{a} \oplus \vec{b} = \vec{b} \oplus \vec{a}$.
We made no further assumption about $\vec{a}$ and $\vec{b}$.
Thus, for all 3--dimensional vectors $\vec{a}$ and $\vec{b}$, we know that  $\vec{a} \oplus \vec{b} = \vec{b} \oplus \vec{a}$.
Thus, addition of 3--dimensional vectors is commutative.
}{0in}

\show{Go back to each line of the proof above and give a reason for the equality on that line at the very right side of the line.  The first one is ``Write in open form.''  Somewhere in the middle you will use the fact that addition of real numbers is commutative.  Thus, at the heart of it, commutativity of vector addition comes from commutativity of addition of real numbers.}{0in}

\show{You are going to show that addition of 3--dimensional vectors is associative.
Let $\vec{a}, \vec{b},$ and $\vec{c}$ be 3--dimensional vectors.
Start with $(\vec{a} \oplus \vec{b}) \oplus \vec{c}$ and rewrite it until it becomes $\vec{a} \oplus (\vec{b} \oplus \vec{c})$ following the model of the previous proof, starting with the word ``Let''.
Also write explanations for each step.
Conclude, following the model, that you have shown that vector addition is associative.
This last part is very important.}{2.5in}

\stop{Compare your argument to the rest of the members of your group.  Make sure that you agree on absolutely every step and every justification.}
\pagebreak

\definition{Scalar product for 3--dimensional vectors}{Let $c$ be a real number and let $\vec{a} = \tvec{a}$ be a 3--dimensional vector.  The {\em scalar product} of $c$ and $\vec{a}$ is defined as:
\[
    c\vec{a} = \vect{ca_1,ca_2,ca_3}.
\]}

\example{Let $c = 3$ and $\vec{a} = \vect{7,-4,\sqrt{2}}$.  Calculate $c\vec{a}$, starting by writing $c\vec{a} = \ldots$.}{0.3in}

\example{Calculate $\pi \vect{9,4,1}$.}{0.3in}

\example{Calculate $(2+\sqrt{3}) \vect{5,b,c}$.}{0.3in}

\show{You are going to show that the scalar product is distributive over vector addition.  First use the word ``Let'' to settle on one real number $c$ and two 3--dimensional vectors, $\vec{a}$ and $\vec{b}$.  Then write $c(\vec{a} \oplus \vec{b})$ and rewrite it until it equals $c\vec{a} \oplus c\vec{b}$.  Provide a reason for each step, as in the proofs above.
At the end, follow the model to conclude that you have shown distributivity in general.}{3in}

\show{You are going to show that the scalar product is distributive over real number addition.  Start with ``Let.''  Write $(c+d)\vec{a}$ and rewrite it until it equals $c\vec{a} \oplus d\vec{a}$.
Provide justifications for each step.
At the end, follow the model to conclude that this shows distributivity in general.}{3in}

\stop{Check over what everyone in your group has done, and make sure that you are in complete agreement.}

\definition{Zero vector}{The vector $\vect{0,0,0}$ is a special 3--dimensional vector, called the {\em zero vector}.
We denote it by $\vec{0}$.}

\definition{Additive inverse}{Let $\vec{a}$ be a 3--dimensional vector, with open form $\tvec{a}$.
Define a new vector by $-\vec{a} = \vect{-a_1,-a_2,-a_3}.$
It is called the {\em additive inverse} of $\vec{a}$.}

\show{Let $\vec{a}$ be a 3--dimensional vector.  Show that $\vec{a} \oplus \vec{0} = \vec{a}$.  It's not very exciting.  Make a general conclusion.}{1.5in}

\show{Let $\vec{a}$ be a 3--dimensional vector, and let $-\vec{a}$ be its additive inverse.  Show that $\vec{a} \oplus (-\vec{a}) = \vec{0}$.  This is also not very exciting.  Make a general conclusion.}{1.5in}

\pagebreak
\definition{Dot product of 3--dimensional vectors}{The dot product of 3--dimensional vectors $\vect{ a_1, a_2, a_3 }$ and $\vect{ b_1, b_2, b_3 }$ is the real number $a_1 b_1 + a_2 b_2 + a_3 b_3$.}

\notation{The dot product of 3--dimensional vectors $\vec{a}$ and $\vec{b}$ is denoted $\vec{a} \bullet \vec{b}$.}

\example{Calculate the dot product of $\vec{a} = \vect{ 12,-5,3 }$ and $\vec{b} = \vect{ 6,4,-11 }$  Do this by writing
\begin{eqnarray*}
    \vec{a} \bullet \vec{b} &=& \tvec{a} \bullet \tvec{b}\\
    &=& a_1 b_1 + a_2 b_2 + a_3 b_3
\end{eqnarray*}
and then substituting in the numbers.
This makes the calculation just a matter of rewriting.}{2in}

\show{You will show that the dot product is commutative, just as multiplication of real numbers is commutative.
This time, you write the first line, ``Let $\vec{a}$ and $\vec{b}$ be \ldots.''
Follow the models from previous examples, and be sure to make a general conclusion.}{3in}

\show{You will show that the dot product is distributive over vector addition.
That is, you want to show that $(\vec{a} \oplus \vec{b}) \bullet \vec{c} = \vec{a} \bullet \vec{c} + \vec{b} \bullet \vec{c}.$  Start with ``Let \ldots''.  Please explain why one addition symbol is $\oplus$ and the other is $+$.}{2in}

\example{Calculate $\vec{a} \bullet \vec{0}$.  Is this a general result?}{1in}

\show{Let $\vec{a}$ and $\vec{b}$ be 3--dimensional vectors and let $c$ be a real number.  Show in general that $c(\vec{a} \bullet \vec{b}) = (c\vec{a}) \bullet \vec{b} = \vec{a} \bullet (c\vec{b})$.  Since there are two equalities to show, you might want to think about how you will go about it.}{0in}

\vfill          % pad the rest of the page with white space
