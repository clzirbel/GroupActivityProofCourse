\activitytitle{Syllabus for Math 3280 in Fall 2017}{}
\vspace*{-0.2in}

\noindent {\bf Course description.}
There are two main goals for the course:
\blist{0.0in}
\item Improving your ability to work with definitions, examples, counterexamples, claims, and proofs.
\item Improving your ability to read a mathematics textbook on your own.
\elist
My hope is that your new abilities in these two areas will make you unstoppable in your mathematics classes.
We will spend most of class time on \#1.  
I will design activities for us to do together in class for this purpose.
Most of your time outside of class will be spent on \#2.
Being able to read mathematics on your own is a fantastic skill.
Be sure to set aside quiet time to read the textbook.

\noindent {\bf Topics covered.}
\balist{0.0in}
\item Proofs involving even and odd, the division algorithm, multiples of 3, 5, and irrationality of $\sqrt{2}$ and others
\item Proofs involving vector operations such as sum, scalar product, dot product
\item Proofs involving inequalities, definition of $<$, deriving properties of $<$
\item Proofs by contrapositive, process of elimination, contradiction
\item Proofs requiring a construction, for example, $\forall$ $a > 0, \exists$ integer $n > 0$ such that $\frac{1}{n} < a$.
\item Proofs involving set equality and subset relations, union, intersection
\item Proofs using mathematical induction
\item Proofs about infinite unions and intersections, for example, $\bigcup_{n=1}^{\infty} [\frac{1}{n}, 1] = (0,1]$.
\ealist

\noindent {\bf Professor and contact information.}  
Craig L. Zirbel.
My office is room 438 in the Math building.
Email is the best way to reach me, zirbel@bgsu.edu.
Put ``3280'' in the subject line.
Sending me messages on Canvas does not work well, so please avoid that.
If you want to reach me quickly, try my office phone number, 419-372-7466 and leave a message.

\noindent {\bf Schedule.}
The class meets from 1:00 to 2:15 on Tuesdays and Thursdays in room 228 in the Mathematics building.
There will be no class meeting on Tuesday, October 10 or Thursday, November 23.
The last day of class will be Thursday, December 7.
The final exam is scheduled on Thursday, December 14 at 1:15 PM.

\noindent {\bf Office hours.}
You are welcome to ask me questions in my office, which is room 438 in the mathematics building.
If you are having trouble finishing activities in class, I'll ask you to finish them in my office hours.
I will ask you for your available times on Thursday, then I will schedule office hours.
You can also make an appointment with me.
The best way to arrange a time to meet is to send an email listing a few times that would work for you.
I will reply with one that works for me as well.

\noindent {\bf Textbook.}  
The textbook for the course is {\em How to Read and Do Proofs}, sixth edition, by Daniel Solow, 2014.  
The textbook is very good and you can learn a lot from it.

\noindent {\bf Graduate assistant.} Johanson Berlie will be assisting in the classroom and will help to check the notebooks.
Johanson is a master's student in mathematics.
Ask him about graduate school.

\noindent {\bf Coursework.} Here are the main things that you will be doing:
\blist{0.0in}
\item Written work on in--class activities.
\item Reading, taking notes, and doing exercises from each chapter in the textbook.
\item Half a dozen quizzes, very much like the work you'll already be doing in class
\item A final exam, very similar to what we have been doing all semester long
\elist

\noindent {\bf Grading.}  Many things you do during the semester will have a point value attached to them.  
The number of points will indicate their relative importance to your grade.  
In--class work and reading homework will count for a larger share than in most courses, while quizzes and exams will count for a lower share.
All quizzes will be announced in class at least one week ahead of time.
Grades will be posted on Canvas.

\noindent {\bf Attendance.} Attendance and class participation are vitally important and will contribute directly to your grade.
Class time is the best time to make attempts and get immediate feedback.
If you cannot attend a class, notify me as soon as possible by email or phone, before class if possible.
Don't even imagine that you can miss a class without letting me know.
I don't particularly need to know {\bf why}, but I do need to know.

\noindent {\bf Academic Honesty.} You will work together with members of a group on in-class activities.  You must work on quizzes and exams on your own.  For the reading assignments, you may work together, but you must note who you worked with, and in any case, you must write your own thoughts in your notebook.
\vfill          % pad the rest of the page with white space
