\activitytitle{Reading assignment, Chapter 18}{Due in the thirteenth week of class.}

Read and understand Chapter 18 of the textbook by Daepp and Gorkin, called ``Mathematical Induction.''

Mathematical induction and recursion play an important role especially in discrete mathematics. Prepare to move slowly and think carefully. To understand the proof of Theorem 18.1, you will need {\bf Well-ordering principle of the natural numbers}: Every nonempty subset of the natural numbers contains a minimum.
Read Theorem 18.1 and then {\bf do Problem 18.1} and {\bf Problem 18.3}. Follow the steps in Theorem 18.1, define the assertion $P(n)$ for the problem first. You will need the condition "$P(n)$ is true" to show the induction step. Work through Exercise 18.3 to Exercise 18.5 and then {\bf do Problem 18.9} without going back Exercise 18.5. You can do it!!

Recursion is a very useful tool to define functions, sequences and sets. Before you move to Theorem 18.6, read the definition of $n$ factorial for $n\in \mathbb{N}$. Write out $3!, 4!$ and $5!$, then try $\frac{6!}{2!4!}$. More general, simplify $\frac{n!}{m!(n-m)!}$ where $n$ and $m$ are two positive integers with $n\geq m$. Here is another simple example:\\
Let $n\in \mathbb{Z^+}$. Consider the function $S(n) = S(n-1) + n$ with $S(0) = 0$. Write out $S(1), S(2)$ and $S(3)$. Can you figure out what this function does for us? With Problem 18.1, you should be able to see the connection between induction and recursion.

Theorem 18.6 shows the existence and uniqueness of a recursive function $g: N \rightarrow X$ given a function $f: X \rightarrow X$ and $a\in X$, where $X$ is a nonempty set. The function $g$ satisfies
\begin{itemize}
\item[(i)] The base step: $g(0) = a$, and
\item[(ii)] The recursive step: $g(n+1)=f(g(n))$ for all $n \in \mathbb{N}$.
\end{itemize}

\noindent The proof of Theorem 18.6 is long and hard. Be patient! You may not get the idea of the proof at beginning, try to write outlines of the proof. You can come back it later.

The keys to a successful recursive solution is to identify the base case and make sure the recursive step is making progress toward the solution. Do Exercise 18.7 and then {\bf do Problem 18.10} and {\bf Problem 18.11}.

\noindent As with the previous chapters,
\blist{0.0in}
\item Read somewhere quiet, minimizing distractions from phones and friends
\item Note the time that you start and stop reading, and add up the minutes
\item Read with a pencil in your hand and your notebook open in front of you
\item Write a sentence to summarize each paragraph, re-draw diagrams, work out examples and exercises on your own
\item Look up words you don't know, and write down ones you really don't know
\item Read slowly.
\item At the end, tally up how much time you have spent on reading this chapter.
Write this number in your notebook and remember the number when you come to class.
\elist

\vfill          % pad the rest of the page with white space
