\yourname

\activitytitle{Review exercises}{}

\definition{Cross product}{Let $\vec{a}$ and $\vec{b}$ be 3--dimensional vectors.
The cross product of $\vec{a}$ and $\vec{b}$ is a new vector given by $\vect{a_2 b_3 - a_3 b_2, a_3 b_1 - a_1 b_3, a_1 b_2 - a_2 b_1}$.
The cross product is denoted $\vec{a} \times \vec{b}$.
}

\prove{Show that the cross product is distributive over vector addition, that is, that $\vec{a} \times (\vec{b} \oplus \vec{c}) = \vec{a} \times \vec{b} \oplus \vec{a} \times \vec{c}$.
Take small steps and justify each step.}{2.5in}

\prove{Show that the cross product is anti-symmetric:  for all 3--dimensional vectors $\vec{a}$ and $\vec{b}$, we have $\vec{a} \times \vec{b} = -\vec{b} \times \vec{a}$.}{1.5in}

\prove{Define $\ln x = \int_{1}^{x} \frac{1}{t} dt$ for all real numbers $x > 0$.  
Show that for all real numbers $x$ and $y$ with $0 < x < y$, we have $\ln x < \ln y$.
Working backward, rewrite $\ln x$ and $\ln y$ in terms of integrals, and then write an equality that relates integrals over different intervals.}{2in}

\prove{Let $n$ be an integer and suppose that $n^2$ is a multiple of 7.
Show that $n$ is a multiple of 7.
Use the Division Algorithm to do this by cases, and be crystal clear about the structure of the proof.}{2in}

\prove{Show that $\sqrt{7}$ is irrational.}{1.5in}

\prove{For each $n = 1, 2, 3, \ldots,$ let $a_n = \frac{n^2 + 3}{n^2 + 1}$.  Show that for all $\varepsilon > 0$, there exists an integer $n$ with $a_n - 1 < \varepsilon$.
Be careful to recognize the two quantifiers in the statement and use appropriate proof techniques for each one.
\Hint Rewrite $a_n$ to look like 1 plus something small.
}{1in}

\problem{Suppose that $x \leq 5 + \frac{1}{n}$ for all $n = 1, 2, 3, \ldots$.
Show that $x \leq 5$.
\Hint Consider different types of proof including direct, contrapositive, contradiction, etc.}{1.0in}

\prove{Show that for all $x \in \R$, there exists an integer $n\geq 1$ such that $x \in (-n,n)$.
Use good form for proofs with nested quantifiers, and be sure to cover both positive and negative values of $x$.
If you have trouble getting started, do scratchwork with $x = 4.2, x = -13.1, x = 0$.}{1.5in}


\pagebreak

\problem{Let $E = \{ m \in \Z : $ there exists $j \in \Z$ such that $m = 2j \}$.
Let $O = \{ m \in \Z : $ there exists $k \in \Z$ such that $m = 2k + 1 \}$.
Show that $E \cap O = \emptyset$ by letting $m \in E \cap O$ and showing that this leads to a contradiction.
}{1in}

\problem{Continuing the previous problem, show that $E \cup O = \Z$ by showing set inclusion both ways.
}{1in}

\problem{Using standard interval notation, show that $[2,6) \cap [3,8) = [3,6)$ by showing set inclusion both ways. As above, write compound inequalities, then individual inequalities, then compound inequalities again.  Use a number line to illustrate.}{2in}

\problem{Show that $[2,6) \cup [3,8) = [2,8)$ by showing set inclusion both ways.}{0in}

\prove{Show that $[2,5] \cup (4,7) = [2,7)$.}{2.5in}

\problem{Let ${\displaystyle A = \bigcup_{n=1}^{\infty}} (\frac{1}{n}, 1)$ and let $B = (0,1)$.
Show that $A = B$ by showing containment both ways.
}{2in}

\problem{Let ${\displaystyle A = \bigcap_{n = 1}^{\infty}} [0, 1+\frac{1}{n}]$ and $B = [0,1]$.
Prove that $A = B$ by showing containment both ways.
}{0in}

\vfill          % pad the rest of the page with white space
