\yourname

\activitytitle{Even and odd}{Our first example of definitions, examples, theorems, and proofs}

\overview{Definitions are important to read and understand by looking at examples.  Many proofs are little more than working with the definitions and rewriting things.  With a bit of practice, these become very routine.  This activity has you work through two definitions, a few examples, and then some proofs.  Everything relies on the definitions, so keep coming back to them.  We will use a similar model many times during the semester.}

\definition{Even\label{evendef}}{An integer $n$ is {\em even} if there exists an integer $k$ for which $n = 2k$.}

\definition{Odd\label{odddef}}{An integer $n$ is {\em odd} if there exists an integer $k$ for which $n = 2k+1$.}

\note{19 meets the definition to be odd because 19 is an integer and 19 = 2(9)+1.}

\example{Check that 12 meets the definition to be even by writing $12 = 2k$ for an appropriate value of $k$ and make sure $k$ is an integer.}{0.2in}

\example{Does $-9$ meet the definition to be odd?  Write $-9 = 2k+1$.}{0.2in}

\example{Does 0 meet the definition to be even?  Write $0 = 2k$ for an integer $k$.}{0.2in}

\example{Does 1.73 meet the definition to be odd?  Explain.}{0.4in}

\note{Suppose that $m$ is an integer.  Then $2m+1$ is an integer and it is odd because it meets the definition to be odd.  Also, $2m+2$ is even because it is an integer and can be rewritten as $2(m+1)$, which is of the form $2k$ where $k=m+1$, which is an integer.}

\show{Suppose that $m$ is an integer.  Show that $2m+6$ is even by rewriting it until it meets the definition to be even.  Connect your statements with $=$ signs.}{0.5in}

\show{Suppose that $m$ is an integer.  Show that $4m + 9$ is odd by rewriting it until it meets the definition to be odd.  Connect your statements with $=$ signs.}{0.5in}

\stop{Compare your answers to the questions above with the other people in your group before you move on.  Resolve any differences in your answers.}

\show{Suppose that $m$ is even.  Then $m = 2k$ for some integer $k$.
Show that $m+8$ is even by rewriting it as $2k+8$ and continuing until it is 2 times an integer.  Connect your statements with $=$ signs.}{1in}

\note{You already know that the sum of two even numbers is even.  The next item guides you through a proof of this fact, using the definitions of even and odd above.}

\guidedproof{Suppose that $m$ and $n$ are even.  Fill in the blanks to show that $m+n$ is even.\label{sumofevens}
\blist{0.1in}
\item There exist integers $j$ and $k$ such that $m = $\blank{0.75in} and $n = $\blank{0.75in}.
\item Thus, $m + n = $ \blank{2in} = $2(j+k)$.
\item This number meets the definition to be even because it is an \blank{0.75in} and because $(j+k)$ is an \blank{0.75in}.
\item We saw that if $m$ and $n$ are even, then $m+n$ is even.  
We made no further assumption about $m$ and $n$.
Thus, the sum of any two even numbers is even. 
\elist
}

\guidedproof{Suppose that $m$ is even and $n$ is odd.
Fill in the blanks to show that $mn$ is even.  
Use the previous exercise as a model.
\blist{0.1in}
\item There exist \blank{2in} such that $m = $\blank{0.75in} and $n = $\blank{0.75in}.
\item Thus, $mn = $\blank{3in} = $2($\blank{0.75in}$)$.
\item This number satisfies the definition to be even because \blank{1.25in} and \blank{1.25in}.
\item We saw that \blank{4in}.  We made no \\\blank{3in}.
Thus, \blank{2.5in}.
\elist
}

\prove{Let $m$ and $n$ be odd.  Follow the examples above as a model to show that $mn$ is odd by rewriting $mn$ until it meets the definition to be odd.  Good form is critically important in proofs.
\blist{0.1in}
\item
\item
\item
\item
\elist
}{0.1in}
\pagebreak
\prove{Let $m$ and $n$ be odd.  Follow the examples above as a model to show that $m + n$ is even by rewriting it.  Good form is critically important in proofs.
\blist{0.1in}
\item
\item
\item
\item
\elist
}{0.1in}

\groupwork{Step 1 in each of your proofs uses the definition of even or odd to write $m$ and $n$ in a more useful, more informative way.
How does this help your proof along?
Discuss with members of your group.}{0.75in}

\groupwork{Step 3 in each of your proofs also uses the definition.
How does this use of the definition differ from what happens in Step 1?
Discuss with members of your group.}{0.75in}

\prove{Let $m$ be odd.  Follow the models above to prove that $m^2$ is odd.  Use good form.}{1.5in}

\prove{Let $m$ be even.  Follow the models above to prove that $m^2$ is even.  Use good form.}{1.5in}

\question{What does it mean that an integer is a multiple of 4?  Give your own definition analogous to the definitions of even and odd.}{0.25in}

\prove{Let $m$ be even.  Show that $m^2 +2m + 4$ is a multiple of 4.}{1in}

\question{Let $m$ be even.  Can $m^2 +2m + 4$ be a multiple of 8?  Explain.}{1in}

\noindent
{\bf Challenges.}  Here are some statements that are harder to prove, because they require a bit more than simply restating the definitions.
See if you can make a good argument for them.

\question{Pretend for a minute that there is an integer $m$ which is both even and odd.
Work with the definitions to see that this really must be fantasyland.}{1in}

\prove{If $m$ is an integer, then $m$ is even or $m$ is odd; it has to be one of the two, there is no third possibility.\\
Here is one suggestion.  0 is even.  If $n$ is even, then $n+1$ is odd.  If $n$ is odd, then $n+1$ is even.  This should cover all positive integers.  Also, if $n$ is even, then $-n$ is even, which tells us about negative integers.}{1.5in}

\prove{If $m$ is an integer and $m^2$ is odd, then $m$ is odd.
\Hint:  There are two cases to check, the case in which $m$ is even and the case in which $m$ is odd.}{0.0in}

\vfill          % pad the rest of the page with white space
