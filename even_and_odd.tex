\yourname

\activitytitle{Even and odd}{Our first experience with definitions, examples, theorems, and proofs}

\overview{Definitions are important to read and understand by looking at examples.  Many proofs are little more than working with the definitions and rewriting things.  With a bit of practice, these become very routine.  This activity has you work through two definitions, a few examples, and then some proofs.  Everything relies on the definitions, so keep coming back to them.}

\note{The {\em integers} are positive and negative counting numbers $\ldots, -3, -2, -1, 0, 1, 2, 3, \ldots$}

\definition{Even\label{evendef}}{An integer $n$ is {\em even} if there exists an integer $k$ for which $n = 2k$.}

\definition{Odd\label{odddef}}{An integer $n$ is {\em odd} if there exists an integer $k$ for which $n = 2k+1$.}

\note{19 meets the definition to be odd because 19 is an integer, 19 = 2(9)+1, and 9 is an integer.}

\example{Check that 12 meets the definition to be even by writing $12 = 2k$ for an appropriate value of $k$ and make sure $k$ is an integer.}{0.2in}

\example{Does $-9$ meet the definition to be odd?  Write $-9 = 2k+1$.}{0.2in}

\example{Does 0 meet the definition to be even?  Write $0 = 2k$ and check the value of $k$.}{0.2in}

\example{Does 1.73 meet the definition to be odd?  Explain.}{0.4in}

\note{Suppose that $m$ is an integer.  Then $2m+1$ is an integer and it is odd because it meets the definition to be odd.  Also, $2m+2$ is even because it is an integer and can be rewritten as $2(m+1)$, which is of the form $2k$ where $k=m+1$, which is an integer.}

\show{Suppose that $m$ is an integer.  Show that $2m+6$ is even by rewriting it until it meets the definition to be even.  Connect your expressions with $=$ signs, not implication signs $\Rightarrow$.}{0.5in}

\show{Suppose that $m$ is an integer.  Show that $4m + 9$ is odd by rewriting it until it meets the definition to be odd.  Connect your expressions with $=$ signs.}{0.5in}

\stop{Compare your answers to the questions above with the other people in your group before you move on.  Resolve any differences in your answers.}

\show{Suppose that $m$ is even.  Then $m = 2k$ for some integer $k$.
Show that $m+8$ is even by rewriting it as $2k+8$ and continuing until it is 2 times an integer.  Connect your expressions with $=$ signs.}{1in}

\guidedproof{You already know that the sum of two even numbers is even.  Fill in the blanks to produce a proof of this fact, using Definition \ref{evendef} of even.\label{sumofevens}
\balist{0.1in}
\item Let $m$ and $n$ be even integers.
\item There exist integers $j$ and $k$ such that $m = $\blank{0.75in} and $n = $\blank{0.75in}, by Definition \ref{evendef}.
\item Thus, $m + n = $ \blank{2in} = $2(j+k)$.
\item This number meets the definition to be even because $j+k$ is an \blank{0.75in} and because $m+n$ is \blank{0.75in} times an integer.
\item We saw that if $m$ and $n$ are even, then $m+n$ is even.  
We made no further assumption about $m$ and $n$.
Thus, the sum of any two even numbers is even. 
\elist
}

\guidedproof{
Fill in the blanks to show that the product of two even numbers is even.  
\balist{0.1in}
\item Let $m$ and $n$ \blank{2in}
\item There exist \blank{2in} such that $m = $\blank{0.75in} and $n = $\blank{0.75in}, by \blank{0.75in}.
\item Thus, $mn = $\blank{3in} = $2($\blank{0.75in}$)$.
\item This number satisfies the definition to be even because \blank{1.25in} is an integer and $mn$ is \blank{2in}.
\item We saw that \blank{4in}.  We made no \\\blank{3in}.
Thus, \blank{2.5in}.
\elist
}

\prove{Show that the sum of two odd numbers is even.  
Follow the examples above and remember that good form is critically important in proofs.
\balist{0.1in}
\item
\item
\item
\item
\item (Yes, you need to write this every time!  It's how we make generalizations.)
\elist
}{0.0in}

\pagebreak
\prove{Show that the product of two odd numbers is odd.  
Follow the examples above and remember that good form is critically important in proofs.
\balist{0.1in}
\item
\item
\item
\item
\item
\elist
}{0.1in}

\groupwork{Let $m$ and $n$ be even integers.
Following Definition \ref{evendef}, some students will write $m=2k$ and $n=2k$ where $k$ is an integer.
Does this accurately reflect what we know about $m$ and $n$?
Discuss with members of your group.

\vspace*{0.5in}
\noindent
Find specific values of even $m$ and $n$ for which $m=2k$ and $n=2k$ for some integer $k$ does happen.

\vspace*{0.5in}
\noindent
Find specific values of even numbers $m$ and $n$ for which $m=2k$ and $n=2k$ for some integer $k$ does not happen.

\vspace*{0.5in}
\noindent
How does proof \ref{sumofevens} deal with the collision of notation, where both $m$ and $n$ need to be written as two times an integer?
}{0.5in}

\prove{Show that the square of an even number is even.  That is, if $m$ is even, then $m^2$ is even.  Follow the models above and use good form.\label{evensquared}}{1.5in}

\prove{Show that the square of an odd number is odd.  That is, if $m$ is odd, then $m^2$ is odd.\label{oddsquared}}{0in}

% =======================================
\pagebreak
\question{What does it mean that an integer is a multiple of 4?  Give your own definition analogous to the definitions of even and odd.}{0.25in}

\prove{Let $m$ be even.  Show that $m^2 +2m + 4$ is a multiple of 4.  Use good form.}{1in}

\question{Let $m$ be even.  Can $m^2 +2m + 4$ be a multiple of 8?  Explain.}{1in}

\noindent
{\bf Challenges.}  Here are some statements that are harder to prove because they require a bit more than simply restating the definitions.
See if you can make a good argument for them.

\question{Pretend for a minute that there is an integer $m$ which is both even and odd.
Work with the definitions to see that this really must be fantasyland.}{1in}

\prove{If $m$ is an integer, then $m$ is even or $m$ is odd.  That is, it has to be one of the two, there is no third possibility.\\
Here is one suggestion.  0 is even.  If $n$ is even, then $n+1$ is odd.  If $n$ is odd, then $n+1$ is even.  This should cover all positive integers.  Also, if $n$ is even, then $-n$ is even, which tells us about negative integers.\label{integerevenodd}}{1.5in}

\prove{If $m$ is an integer and $m^2$ is odd, then $m$ is odd.
\Hint  There are two cases to check, the case in which $m$ is even and the case in which $m$ is odd.
You may wish to refer back to \ref{evensquared}, \ref{oddsquared}, and \ref{integerevenodd}.}{0.0in}

\vfill          % pad the rest of the page with white space
