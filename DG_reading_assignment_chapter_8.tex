\activitytitle{Reading assignment, Chapter 8}{Make a good effort by Monday, November 9; due on Friday, November 13.}

Read and understand Chapter 8 of the textbook by Daepp and Gorkin, called ``More on operations on sets.''

This chapter is a challenge.
You will really need to use all the reading skills you have been practicing when you read this chapter.
The ideas are harder, and some are really hard, but not impossible.
Just slow yourself down and write things out in lots of detail.

Example 8.2(a) would be a great one to write out concrete fractions with different values of $p$ and $q$ to understand the sets $A_q$ and then the union of these sets.  For Example 8.2(b), do the same to understand what the sets $B_i$ are, and then what their intersection is.  No shortcuts!  Write out elements for each set.

Exercise 8.3 is also good.

In the middle of page 82 the phrase ``collection of subsets of $X$'' appears.
This is a very new, very difficult concept; do not underestimate how tricky it can be, but patiently think about it and keep coming back to it.
For example, $\cal{A}$ might be all intervals of the form $[k,k+1]$ and you might want to take the union of all such intervals, or the intersection.

Exercise 8.4 is excellent.  Draw pictures until everything is crystal clear.
Exercise 8.5 is also excellent.

Rewrite the proofs of Examples 8.6 and 8.7 to make them your own.  Really.

Exercises 8.9 and 8.10 are also excellent.  Do them on your own, then compare to the solutions in the book.

Do problems 1, 2, and 3.

Here is a challenge problem.  Let $a < b$.  Show that $\bigcup_{n=1}^{\infty} [a, b-\frac{1}{n}] = [a,b)$.  Draw pictures, then show set inclusion both ways.

Here is another challenge problem.  Let $a < b$.  Show that $\bigcap_{n=1}^{\infty} [a,b+\frac{1}{n}) = [a,b]$.  Draw pictures, then show set inclusion both ways.

\noindent As with the previous chapters,
\blist{0.0in}
\item Read somewhere quiet, minimizing distractions from phones and friends
\item Note the time that you start and stop reading, and add up the minutes
\item Read with a pencil in your hand and your notebook open in front of you
\item Write a sentence to summarize each paragraph, re-draw diagrams, work out examples and exercises on your own
\item Look up words you don't know, and write down ones you really don't know
\item Read slowly.  You are not reading a comic book or a newspaper.  It is not a goal of this class for you to learn how to read faster.  The goal is to learn how to get more out of the time you spend reading, and to learn to concentrate for longer periods of time.
\item At the end, tally up how much time you have spent on reading this chapter.
Write this number in your notebook and remember the number when you come to class.
\elist


\vfill          % pad the rest of the page with white space
