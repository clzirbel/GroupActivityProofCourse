\yourname

\activitytitle{Pigeonhole Principle}{It was stated by Johann Peter Gustav Lejeune Dirichlet in 1834 and it is sometimes called Dirichlet's Box Principle.}

\overview{The principle simply states that if we have more pigeons than holes then at least one hole must contain more than one pigeon. The next problem asks you to prove this principle.}

\problem{If $n+1$ pigeons or more are placed in $n$ holes then one of the holes must contain two or more pigeons. 
{\bf Hint:} Prove by induction on $n$.}{4in}

\problem{The human head contains less than $200,000$ hairs. Name three cities in which at least two people have the same number of hairs.}{0.5in}

\problem{A bag contains M\&M's in six different colors: Brown, Yellow, Green, Red, Orange and Blue. How many M\&M's do you need to take out of the bag in order to have at least two of the same color? How many do you need to take out of the bag if you want to have three of the same color?}{1in}

\problem{A classroom floor is painted white and black. Is it always possible to find two points of the same color exactly one inch apart?}{1in}

\problem{Prove that no matter how we choose $51$ natural numbers from $\{1,2,3,\ldots,100\}$, at least two of them must be consecutive.
{\bf Hint:} Consider the pigeonholes $\{1,2\}, \{3,4\}, \ldots , \{99,100\}$.}{2in}

\problem{Prove that given any ten natural numbers we can choose two of them such that their difference is divisible by nine.
{\bf Hint:} Consider the remainders when dividing by nine.}{1.5in}

\problem{Prove that if six integers are selected at random from the set $\{1,2,3,4,5,6,7,8,9,10\}$ then at least two of them add up to eleven.}{1in}