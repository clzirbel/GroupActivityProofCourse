\yourname

\activitytitle{Pigeonhole Principle}{Surprising results can come from simple counting.}

\overview{The Pigeonhole Principle states that if we have more pigeons than pigeonholes to put them in, then at least one pigeonhole must contain more than one pigeon. This idea was stated by Johann Peter Gustav Lejeune Dirichlet in 1834 and so is sometimes called Dirichlet's Box Principle.}

\problem{If $N$ pigeons are placed in $n$ pigeonholes and $N>n$, then one of the pigeonholes must contain two or more pigeons. 
{\bf Hint:} Prove by contradiction.  That is, pretend for a minute that none of the pigeonholes has more than one pigeon.  That leads to a contradiction.}{.8in}

\problem{A bag contains M\&M's in six different colors: Brown, Yellow, Green, Red, Orange and Blue. How many M\&M's do you need to take out of the bag in order to have at least two of the same color? How many do you need to take out of the bag if you want to have three of the same color?}{1in}

\problem{Prove that no matter how we choose $51$ natural numbers from $\{1,2,3,\ldots,100\}$, at least two of them must be consecutive.
{\bf Hint:} Consider the pigeonholes to be the following sets: $\{1,2\}, \{3,4\}, \ldots , \{99,100\}$.}{2in}

\problem{Prove that given ten integers numbers we can choose two of them such that their difference is divisible by nine.
{\bf Hint:} Consider the remainders when dividing by nine.}{0.5in}

\pagebreak

\problem{Prove that if six distinct numbers are selected from the set $\{1,2,3,4,5,6,7,8,9,10\}$ then some set of two of them add up to eleven.}{1in}

\problem{A classroom floor is painted white and black. Is it always possible to find two points of the same color, exactly one foot apart?
{\bf Hint:} Think about an equilateral triangle.}{.9in}

\problem{If you have even more pigeons, you sometimes need more than two pigeons in each pigeonhole.  Suppose we need to place $N$ items into $n$ boxes and $N>n$. Then at least one box must contain at least $\lceil \frac{N}{n} \rceil$ items, which is $\frac{N}{n}$ rounded up to the nearest integer.
{\bf Hint:} Let $N_i$ be the number of items in box $i$, where $1\leq i \leq n$. Then $N=N_1+N_2+ \ldots + N_n$. Suppose by the way of contradiction that $N_i<\lceil \frac{N}{n} \rceil$, for all $i\in \{1, 2, \ldots,n \}$. Prove the fact that $\lceil \frac{N}{n} \rceil< \frac{N}{n} +1$ and use it to reach a contradiction.}{2in}

\problem{The human head contains fewer than 150,000 hairs. Show that there are at least 500 people with the same number of hairs in New York City. Assume that New York City has a population greater than 8,000,000 people.}{1in}

