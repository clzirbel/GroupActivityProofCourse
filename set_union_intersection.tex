\yourname

\activitytitle{Union and intersection of sets}{We introduce the union and intersection of sets and learn how to prove statements about them.}

\noindent
{\bf Note:}  The notation $x \in A$ is usually read ``$x$ is an element of $A$.''  The symbol $\in$ looks like the letter E because it stands for ``element''.  
As far as I can tell, the E is not there to mean ``$x$ exists in $A$''.  Being an element is the point, not existing.

\definition{Union of sets}{
The {\em union} of sets A and B is a new set, consisting of all elements which belong to A or to B or to both. We denote this new set by $A \cup B$.
The logical statement ``$x \in A \cup B$'' is true when ``$x \in A$ or $x \in B$'' is true.
}

\definition{Intersection of sets}{
The {\em intersection} of sets A and B is a new set, consisting of all elements which belong to both A and B. We denote this new set by $A \cap B$.
The logical statement ``$x \in A \cap B$'' is true when ``$x \in A$ and $x \in B$'' is true.
}

\example{
Let $C$ be the set of Computer Science majors and $M$ be the set of Mathematics majors.
Then $C \cap M$ is the set of students double majoring in Computer Science and Mathematics (a powerful combination!) while $C \cup M$ is the set of students majoring in one, the other, or both majors.
}{0in}

\example{Find $[1,5] \cup (3,7)$.  Draw a diagram on a number line to illustrate.}{0.3in}

\example{Find $[1,5] \cap (3,7)$.  Draw a diagram on a number line to illustrate.}{0.3in}

\problem{Let $3\Z = \{ m \in \Z :$ there exists $j \in \Z$ such that $m = 3j\}$.
Let $5\Z = \{ m \in \Z :$ there exists $j \in \Z$ such that $m = 5j\}$, and similarly with other sets like $7\Z$ and $12\Z$.

\balist{0.4in}
\item Find $3\Z \cap 5\Z$ and write it in the most convenient form you can.
\item Find $3\Z \cup 5\Z$ and write it in the most convenient form you can.
\ealist
}{0.4in}

\remark{Remember that to show $S \subseteq T$, you need to show that for all $x$ in $S$, we have $x$ in $T$.
In symbols, $\forall x \in S, x \in T$.  To do a proof like that, you need to start with ``Let $x \in S$.'' and you need to end with something like ``Thus $x \in T$.  Since $x \in S$ was arbitrary, $S \subseteq T$.''}

\prove{Suppose that $A \cup B \subseteq C$.

\balist{0in}
\item Show that $A \subseteq C$ following the model.

Let $x \in A$.  Then \blank{1in} by definition of union.  
Thus \blank{1in} since $A\cup B \in C$.

Since \blank{1in} was arbitrary, \blank{1in}.

\item Show that $B \subseteq C$.  Use good form.
\elist
}{0in}

%============================================================
\pagebreak

\remark{For sets $A, B,$ and $C$, to show that $A \cap B \subseteq C$, you need to start with ``Let $x \in A \cap B$.''  This tells you that $x \in A$ and $x \in B$.
Using those two pieces of information, you need to show that $x \in C$.  You will usually need to use both pieces of information.}

\guidedproof{Show that $[1,5] \cap (3,7) \subseteq (3,5]$.

\noindent
Let $x \in [1,5] \cap (3,7)$.  Then $x \in $ \blank{1in} and $x \in$ \blank{1in} by definition of \blank{1.1in}.

\noindent
Thus, $1 \leq x \leq 5$ and \blank{1.5in}.

\noindent
Thus, \blank{0.5in} $<$ \blank{0.5in} $\leq$ \blank{0.5in}

\noindent
Thus, $x \in (3,5].$  Since $x\in [1,5] \cap (3,7)$ was arbitrary, we conclude that $[1,5] \cap (3,7) \subseteq (3,5]$.
}

\prove{Suppose that $A \subseteq C$ and $B \subseteq C$.
Show that $A \cap B \subseteq C$ following the model above.

\noindent
Let \blank{1in}

\vspace{0.8in}

\noindent
Thus \blank{1in}.
Since \blank{1in} was arbitrary, \blank{3in}.

}{-0.2in}

\prove{Show that $2\Z \cap 7\Z \subseteq 14\Z$.  Use good form.  
You do not need to prove any results about multiples or divisibility, simply use what you know is true from the definition of these sets.

\noindent
Let \blank{1.5in}

\vspace{0.8in}

\noindent
Thus \blank{1in}.
Since \blank{1.5in} was arbitrary, \blank{2.5in}.
}{-0.2in}

\remark{To show that $A \cup B \subseteq C$, you start with ``Let $x \in A \cup B$.'' This tells you that $x \in A$ or $x \in B$.
This is not all that much to work with, since maybe only one of these is true, and you don't know which one.
No matter which one is true, you need to show that $x \in C$.
You should do this by considering two cases.  Case 1, when $x \in A$.  Case 2, when $x \in B$.
In both cases, show that $x \in C$ and then no matter which case applies, you get the result you need.
In the end, you show that $A \subseteq C$ and $B \subseteq C$.
}

\guidedproof{Suppose that $A \subseteq C$ and $B \subseteq C$.
Show that $A \cup B \subseteq C$.

\noindent
Let $x \in A \cup B$.  Then $x \in A$ or $x \in B$ or both, by \blank{2.5in}.

\noindent
Case 1.  Suppose that $x \in A$.  Then $x \in C$ because \blank{1in}.

\noindent
Case 2.  Suppose that $x \in B$.  Then \blank{1in} because \blank{1in}.

\noindent
In both cases, \blank{0.7in}.
Since \blank{1in} was arbitrary, we conclude that \blank{1in}
}

\prove{Show that $[1,5] \cup (3,7) \subseteq [1,7)$.  Use good form.  You will use transitivity for inequalities.}{1in}

\vfill
\pagebreak
% ==========================================================

\prove{Suppose that $B \subseteq A$.
Show that $A \cup B \subseteq A$.
Use good form.}{0.9in}

\prove{Suppose that $A \cup B \subseteq A$.
Show that $B \subseteq A$.
Use good form.}{0.9in}

\remark{To show that $A \subseteq B \cap C$, start with ``Let $x \in A$ and show that $x \in B$ and $x \in C$, so you can conclude that $x \in B \cap C$.
Make sure to write that you have shown that $x \in B \cap C$ before you make a general conclusion.
}

\prove{Show that $14\Z \subseteq 2\Z \cap 7\Z$.  Use transitivity and make a general conclusion at the end.

\noindent
Let \blank{1in}

\vspace{0.8in}

\noindent
Thus \blank{1in}.
Since \blank{0.5in} was arbitrary, \blank{3in}.
}{-0.2in}

\prove{Show that $(3,5] \subseteq [1,5] \cap (3,7)$.  Use good form and make a general conclusion at the end.}{1in}

\remark{To show that $A \subseteq B \cup C$, start with ``Let $x \in A$ and show that $x \in B$ or that $x \in C$.
Often, some $x$ values are in $B$ and others are in $C$, and so you will want to introduce two cases that split the values of $x$ into two groups.  Unfortunately, the cases cannot be ``Case 1:  Suppose $x \in B$'' and ``Case 2:  Suppose $x \in C$'' because you do not yet know that those case cover all possibiliities.  Instead, you may need to use cases like ``Case 1: Suppose $x < 0$'' and ``Case 2:  Suppose $x \geq 0.$''
The details will be different for each problem.
Each case must end with ``Thus $x \in B \cup C$.''}

\prove{Show that $[1,7) \subseteq [1,5] \cup (3,7)$.  Use good form and make a general conclusion at the end.}{1in}

% =======================================================
\pagebreak

\prove{Show that $(3,8) \cup [6,9] = (3,9]$.  There are two steps.
Use inequalities and transitivity, not statements like $(3,8) \subseteq (3,9]$.
\balist{2in}
\item Step 1.  Show that $(3,8) \cup [6,9] \subseteq (3,9]$.
\item Step 2.  Show that $(3,9] \subseteq (3,8) \cup [6,9]$.
\elist
}{2in}

\prove{Show that $(3,8) \cap [6,9] = [6,8)$.  There are two steps.
\balist{2in}
\item Step 1. 
\item Step 2. 
\elist}{-0.2in}


\vfill          % pad the rest of the page with white space
