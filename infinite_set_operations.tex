\yourname

\activitytitle{Infinite operations on sets}{Unions, intersections, and complements of infinitely many sets}

\overview{We are working with sets of real numbers.  These exercises will give you practice with sets and teach you things about the real numbers as well.}

\problem{Let $A = \bigcup_{n=1}^{\infty} [\frac{1}{n}, 1]$.  
List out the first five sets in this union.  
Draw a picture of them above a number line.  
Make a conjecture about what interval $A$ is equal to, call the new set $B$, then show that $A = B$ by showing containment both ways.
You will need to use this property of real numbers:  if $x > 0$, then there exists a positive integer $n$ with $0 < \frac{1}{n} < x$.}{3in}

\problem{Let $B = \bigcup_{n=0}^{\infty} [n, n^2]$.
List out the first five or more sets in this union.
Draw them on a number line if it helps.
Make a conjecture about how you can write $B$ in a simpler way, call the new set $C$, then prove that $B = C$ by showing containment in both directions.}{3in}

\pagebreak

\problem{For $n = 2, 3, 4, \ldots,$ let $C_n = \{ 2n, 3n, 4n, \ldots \}$.
\blist{1in}
\item Write out the first five of the $C_n$.
\item Let $D = \bigcup_{n=2}^{\infty} C_n$.
Describe the set $D$ in simpler terms, perhaps by writing out the smallest 10 elements of $D$.
\item What is $\N \backslash D$?  Remember that $\N = \{0, 1, 2, 3, \ldots\}$.
\elist
}{0in}

\problem{Let $I$ be a set, and for each $i$ in $I$, let $A_i$ be a set, all subsets of the same universe $X$.
Show de Morgan's law:  $\left( \bigcup_{i \in I} A_i \right)^c = \bigcap_{i \in I} A_i^c$ by showing set containment in both directions.
I hope you will find that it is actually easier to do this for a collection of sets than for two sets.
}{3in}

\problem{Let $E = \bigcap_{n = 1}^{\infty} [0, 1+\frac{1}{n}]$.
List out the first five sets in this union.  
Draw a picture of them above a number line.  
Make a conjecture about what interval $E$ is equal to, call the new set $F$, then prove that $E = F$ by showing containment both ways.

{\bf Hint:} You may want to show that $E \subseteq F$ by showing the logically equivalent statement that $F^c \subseteq E^c$.
This is the same as the contrapositive:  suppose that $y \notin F$, then show that $y \notin E$.
You may find it useful to keep in mind that if $x > 0$, then there exists an integer $n$ for which $0 < \frac{1}{n} < x$.
}{3in}

\problem{Let $F = \bigcup_{r \in \Q} (r-\frac{1}{10}, r+\frac{1}{10})$.
Here, $\Q$ is the set of all rational numbers.
Make a conjecture about a simpler way to describe the set $F$, then prove your conjecture by showing set containment both ways.}{3in}

\problem{Let $G = \bigcup_{k \in \Z} (k, k+1)$.
\blist{1in} 
\item Draw out some of the intervals here.
\item Make a conjecture about what set $G$ is.
\item Use one of de Morgan's laws to re--express $G^c$ as an intersection.
Does that help?
\item What is easier to describe, or to think of, $G$ or $G^c$?
Give the simplest description.
\elist}{0in}

\problem{Let $a < b$.  Show that $\bigcup_{n=1}^{\infty} [a, b-\frac{1}{n}] = [a,b)$.  Draw pictures, then show set inclusion both ways.  Is there a problem if $b - \frac{1}{n} < a$?}{2in}

\problem{Let $a < b$.  Show that $\bigcap_{n=1}^{\infty} [a,b+\frac{1}{n}) = [a,b]$.  Draw pictures, then show set inclusion both ways.}{2in}


\vfill          % pad the rest of the page with white space
