\yourname

\activitytitle{Infinite operations on sets}{Unions and intersections of infinitely many sets}

\overview{We are working with sets of real numbers.  These exercises will give you practice with sets and teach you things about the real numbers as well.}

\problem{Let $\displaystyle A = \bigcup_{n=1}^{\infty} [n,n+1)$.  
As indicated below, write $A$ in open form, listing out the first 5 sets in the union.
Figure out what interval $A$ is equal to, call the interval $B$, then show that $A = B$ by showing containment both ways, as indicated below.

\noindent $A = \blank{0.75in} \cup \blank{0.75in} \cup \blank{0.75in} \cup  \blank{0.75in} \cup \blank{0.75in} \cup \cdots$

\noindent $B = \blank{1in}$

\noindent
1) Let $x \in A$.  Then \blank{1.5in} for some $n = 1, 2, 3, \ldots$.

\vspace*{0.9in}

\noindent Thus $x \in B$.  Since $x \in A$ was arbitrary, $A \subseteq B$.  (Always end a proof of inclusion this way!)\\
\noindent
2) Let $x \in B$.  {\bf Note:} You need to show that there exists an $n$ for which $x \in [n,n+1)$.  Tell us how to get the right value of $n$, starting with $x$.
}{0.9in}

\problem{Let $\displaystyle A = \bigcup_{n=1}^{\infty} [-n,n]$.  
As indicated below, write $A$ in open form, listing out the first 5 sets in the union.
Figure out what interval $A$ is equal to, call the interval $B$, then show that $A = B$ by showing containment both ways, as indicated below.

\noindent $A = \blank{0.75in} \cup \blank{0.75in} \cup \blank{0.75in} \cup  \blank{0.75in} \cup \blank{0.75in} \cup \cdots$

\noindent $B = \blank{1in}$

\noindent
1) Let $x \in A$.  Then \blank{1.5in} for some $n = 1, 2, 3, \ldots$.

\vspace*{0.9in}

\noindent
2) Let $x \in B$.  {\bf Note:} You need to show that there exists an $n$ for which $x \in [-n,n]$.  Tell us how to get the right value of $n$, starting with $x$.
}{0.9in}

\problem{Let $\displaystyle A = \bigcap_{n=1}^{\infty} (-n,n)$.  
As indicated below, write $A$ in open form, listing out the first 5 sets in the intersection.
Figure out what interval $A$ is equal to, call the interval $B$, then show that $A = B$ by showing containment both ways, as indicated below.

\noindent $A = \blank{0.75in} \cap \blank{0.75in} \cap \blank{0.75in} \cap  \blank{0.75in} \cap \blank{0.75in} \cap \cdots$

\noindent $B = \blank{1in}$

\noindent
1) Let $x \in A$.  Then \blank{1.5in} for all $n = 1, 2, 3, \ldots$.

\vspace*{0.8in}

\noindent
2) Let $x \in B$.
}{0.8in}

\problem{For $n = 2, 3, 4, \ldots,$ let $A_n = \{ 2n, 3n, 4n, \ldots \}$.
\blist{0.8in}
\item Write out the first five of the $A_n$.
\item Let $\displaystyle B = \bigcup_{n=2}^{\infty} A_n$.
Describe the set $B$ in simpler terms, perhaps by writing out the smallest 10 elements of $B$, then describe $B$ in a sentence.
\item What is $\N \backslash B$?  Remember that $\N = \{0, 1, 2, 3, \ldots\}$.
\elist
}{0in}

\problem{Let $\displaystyle A = \bigcup_{n=0}^{\infty} [n, n^2]$.
List out the first five sets in this union, as you did above.
Draw them on a number line if it helps.
Make a conjecture about how you can write $A$ in a simpler way, call the new set $B$, then prove that $A = B$ by showing containment in both directions.
In each direction, you will need to use three cases.}{2in}

\problem{Let ${\displaystyle A = \bigcup_{n=1}^{\infty}} [\frac{1}{n}, 1]$.  
List out the first five sets in this union, as you did above.  
Draw a picture of them above a number line.  
Make a conjecture about what interval $A$ is equal to, call the new set $B$, then show that $A = B$ by showing containment both ways.
You will need to use this property of real numbers:  if $x > 0$, then there exists a positive integer $n$ with $0 < \frac{1}{n} < x$.}{3in}

\problem{Let ${\displaystyle A = \bigcup_{r \in \Q}} (r-\frac{1}{10}, r+\frac{1}{10})$.
Here, $\Q$ is the set of all rational numbers.
Think of a simpler way to describe the set $A$, then prove your conjecture by showing set containment both ways.}{2in}

\problem{Let ${\displaystyle A = \bigcap_{n = 1}^{\infty}} [0, 1+\frac{1}{n}]$.
List out the first five sets in this intersection, as you did above.  
Draw a picture of them above a number line.  
Make a conjecture about what interval $A$ is equal to, call the new set $B$, then prove that $A = B$ by showing containment both ways.

{\bf Hint:} You may want to show that $A \subseteq B$ by showing the logically equivalent statement that $B^c \subseteq A^c$ (thinking of the universal set as $[0,\infty)$ so you can avoid negative numbers).
This is the same as the contrapositive:  suppose that $x \notin B$, then show that $x \notin A$.
You may find it useful to keep in mind that if $x > 0$, then there exists an integer $n$ for which $0 < \frac{1}{n} < x$.
}{3in}

\problem{Let $\displaystyle A = \bigcup_{k \in \Z} (k, k+1)$.
\blist{0.1in} 
\item Draw out some of the intervals on a number line.
\item Make a conjecture about what set $A$ is.
\elist}{0in}

\problem{Let $a < b$.  Show that ${\displaystyle \bigcup_{n=1}^{\infty}} [a, b-\frac{1}{n}] = [a,b)$.  Draw pictures, then show set inclusion both ways.  If  $b - \frac{1}{n} < a$, the interval is empty.}{2in}

\problem{Let $a < b$.  Show that ${\displaystyle \bigcap_{n=1}^{\infty}} (a-\frac{1}{n},b+\frac{1}{n}) = [a,b]$.  Draw pictures, then show set inclusion both ways.}{0in}


\vfill          % pad the rest of the page with white space
