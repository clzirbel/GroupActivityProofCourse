\yourname

\activitytitle{Infinite unions and intersections}{}

\overview{We are working with infinitely many sets of real numbers.  These exercises will give you practice with sets and teach you things about the real numbers as well.}

\definition{Union}{Let $A_1, A_2, \ldots$ be sets.
The {\em union} of $A_1, A_2, \ldots$ is a new set consisting of all elements that are in $A_n$ for some $n=1, 2, 3, \ldots$.
The union can be written in open form as $A_1 \cup A_2 \cup \cdots$ or in closed form as $\bigcup_{n=1}^{\infty} A_n$.
Note that there is no set $A_{\infty}$.
}

\definition{Intersection}{Let $A_1, A_2, \ldots$ be sets.
The {\em intersection} of $A_1, A_2, \ldots$ is a new set consisting of all elements which are in $A_n$ for all $n = 1, 2, 3, \ldots.$
The intersection can be written in open form as $A_1 \cap A_2 \cap \cdots$ or in  closed form as $\bigcap_{n=1}^{\infty} A_n$.
 }

\problem{Use quantifiers to express what it means that $x \in \bigcup_{n=1}^{\infty} A_n.$\\
{\bf Solution:} $\exists$ $n \geq 1$ such that $x \in A_n$.
In words, there is at least one $n$ for which $x$ is in $A_n$.
It does not take much to be in the union.}{0.0in}

\problem{Use quantifiers to express what it means that $x \in \bigcap_{n=1}^{\infty} A_n.$}{0.2in}

\remark{To show $A \subseteq \bigcap_{n=1}^{\infty} B_n,$ you need to show that for all $x \in A$, we have $x \in \bigcap_{n=1}^{\infty} B_n$.
That is, for all $x \in A$ and all $n = 1, 2, 3, \ldots$, we have $x \in B_n$.
To show this, you need a generic proof that works for all $x$ and all $n$, so start with ``Let $x \in A$ and let $n \geq 1$,'' and then show that $x \in B_n$.
}

\exercise{To show $A \subseteq \bigcup_{n=1}^{\infty} B_n,$ you need to show that $\forall$ $x \in A$, we have $x \in \bigcup_{n=1}^{\infty} B_n$.
That is, $\forall$ $x \in A$, $\exists$ $n \geq 1$ such that $x \in B_n$.
Think ahead to your proof.

\noindent
How will the variable $x$ be introduced?

\vspace*{0.1in}
\noindent
How will the variable $n$ be introduced?

}{0.0in}

\remark{To show $\bigcap_{n=1}^{\infty} A_n \subseteq B:$  Let $x \in \bigcap_{n=1}^{\infty} A_n$.
Then you know that $x \in A_n$ for all $n$, which is a lot of information about $x$.
Use this information to show that $x \in B$.
Exactly how that will work depends on the problem.
}

\remark{To show $\bigcup_{n=1}^{\infty} A_n \subseteq B:$  Let $x \in \bigcup_{n=1}^{\infty} A_n$.
Then you know that $x \in A_n$ for some $n$, but you don't know which $n$, so that is not very informative.
There are infinitely many cases, one for each possible value of $n$.
The best you can do is to start with, ``Let $n \geq 1$. Suppose $x \in A_n$.'' and work forward from there to show that $x \in B$.
In the end, this is the same as showing that $A_n \subseteq B$ for all $n$.
}

% =========================================================================

\problem{Let $\displaystyle A = \bigcup_{n=1}^{\infty} [n,n+1)$.  

\balist{0.0in}
\item Write $A$ in open form, listing out the first five sets in the union:

$A = \blank{0.75in} \cup \blank{0.75in} \cup \blank{0.75in} \cup  \blank{0.75in} \cup \blank{0.75in} \cup \cdots$

\item Let $B = [1,\infty)$.
Explain intuitively why $A = B$.

\vspace*{0.2in}

\item Show that $A \subseteq B$.

\noindent
Let $x \in A$.
Then $x \in [n,n+1)$ for \blank{1.5in}.

\noindent
Thus, $x$ satisfies the following inequalities: \blank{2in}

\vspace{0.7in}
\noindent
Thus, $x \in B$.  Since $x \in A$ was arbitrary, $A \subseteq B$.

\item Show that $B \subseteq A$.

\noindent
Let $x \in B$.  
You need to show that there exists an $n$ for which $x \in [n,n+1)$.  
You need to construct the value of $n$, starting with $x$.

\vspace{0.8in}
\noindent
Thus, $x \in [n,n+1)$, and so $x \in A$.  Since $x \in B$ was arbitrary, $B \subseteq A$.
\elist
}{0.0in}

\problem{Let $\displaystyle A = \bigcup_{n=1}^{\infty} [-n,n]$.  

\balist{0.2in}
\item Write $A$ in open form, listing out the first five sets in the union.

\item Figure out what single interval $A$ is equal to and call the new interval $B$.  $B =$ \blank{1.5in}.
Explain why $A=B$.

\item Show that $A \subseteq B$.  Use good form.

\vspace*{0.3in}

\item Show that $B \subseteq A$.  Use good form.

\elist
}{0.7in}

\problem{Let $\displaystyle A = \bigcap_{n=1}^{\infty} (-n,n)$.  

\balist{0.2in}
\item Write $A$ in open form, listing out the first five sets in the intersection.

\item Figure out what single interval $A$ is equal to and call the interval $B$.  $B = $\blank{1in}.

\item Show that $A \subseteq B$.

\vspace*{0.4in}

\item Show that $B \subseteq A$.

\noindent
Let $x \in B$.  You need to show that $x \in (-n,n)$ for all $n$.  How do you do that?

\elist
}{0.5in}

\problem{  ~

\balist{0.0in}
\item Show that $(5,6] \subseteq [5,6]$.

\item Let $n$ be an integer greater than or equal to 1.  Show that $[5+\frac{1}{n},6] \subseteq (5,6]$

\elist
}{0.0in}

\problem{Let ${\displaystyle A = \bigcup_{n=1}^{\infty}} [\frac{1}{n}, 1]$.  
List out the first five sets in this union, as you did above.  
Draw a picture of them above a number line.  
Make a conjecture about what interval $A$ is equal to, call the new interval $B$, then show that $A = B$ by showing containment both ways.
You will need to use this property of real numbers:  if $x > 0$, then there exists a positive integer $n$ with $0 < \frac{1}{n} < x$.}{3in}

\problem{Suppose that $x \leq 5 + \frac{1}{n}$ for all $n = 1, 2, 3, \ldots$.
Show that $x \leq 5$.
\Hint Consider different types of proof including direct, contrapositive, contradiction, etc.}{1in}

\problem{Let ${\displaystyle A = \bigcap_{n = 1}^{\infty}} [0, 1+\frac{1}{n}]$.
List out the first five sets in this intersection, as you did above.  
Draw a picture of them above a number line.  
Make a conjecture about what interval $A$ is equal to, call the new set $B$, then prove that $A = B$ by showing containment both ways.
}{1.7in}

\problem{Let $a < b$.  Show that ${\displaystyle \bigcup_{n=1}^{\infty}} [a, b-\frac{1}{n}] = [a,b)$.  
(If $b - \frac{1}{n} < a$, the interval is empty.)
Draw pictures, then show set inclusion both ways.  
}{1.7in}

\problem{Let $a < b$.  Show that ${\displaystyle \bigcap_{n=1}^{\infty}} (a-\frac{1}{n},b+\frac{1}{n}) = [a,b]$.  Draw pictures, then show set inclusion both ways.}{1.7in}

\problem{Let ${\displaystyle A = \bigcup_{r \in \Q}} (r-\frac{1}{10}, r+\frac{1}{10})$.
Here, $\Q$ is the set of all rational numbers.
Think of a simpler way to describe the set $A$, then prove your conjecture by showing set containment both ways.}{0.6in}

\pagebreak

\problem{Let $\displaystyle A = \bigcup_{n=0}^{\infty} [n, n^2]$.

\balist{0.2in}
\item List out the first five sets in this union, as you did above.
Draw them on a number line if it helps.

\item Make a conjecture about how you can write $A$ as a union of three simpler sets, and call the new union $B$.

\item Show that $A \subseteq B$.
Let $x \in A$.  Then $x \in [n,n^2]$ for some $n = 0, 1, 2, \ldots.$.
You want to show that $x \in B$.
You can do this with three cases, depending on whether $n=0$, $n=1$, or $n > 1$.
Each case needs to end with $x \in B$.

\noindent
Case 1.  $n=0$

\vspace*{0.2in}
\noindent
Case 2. $n=1$

\vspace*{0.2in}
\noindent
Case 3. 

\vspace*{0.2in}

\item Show that $B \subseteq A$.
Let $x \in B$.
There are three cases, and in each one, you will need to construct $n$ so that $x \in [n, n^2]$.
Each case needs to end with $x \in A$.

\elist
}{1.5in}

\problem{Let $\displaystyle A = \bigcup_{k \in \Z} (k, k+1)$.
\blist{0.2in} 
\item Draw out some of the intervals on a number line.
\item Make a conjecture about what set $A$ is.  You do not need to prove the conjecture.
\elist}{0.5in}

\problem{For $n = 2, 3, 4, \ldots,$ let $A_n = \{ 2n, 3n, 4n, \ldots \}$.
\balist{0.8in}
\item Write out the first five of the $A_n$.
\item Let $\displaystyle B = \bigcup_{n=2}^{\infty} A_n$.
Describe the set $B$ in simpler terms, perhaps by writing out the smallest 10 elements of $B$, then describe $B$ in a sentence.
\item Describe the integers larger than 1 that are not in $B$.
\elist
}{0in}

\problem{
Use quantifiers to write down what it means that $x \in \bigcup_{n=1}^{\infty} A_n$.

\noindent
Use quantifiers to express what it means that $x \notin \bigcup_{n=1}^{\infty} A_n$ by negating quantifiers and rewriting until the expression is as simple as possible.}{1in}

\problem{
Use quantifiers to write down what it means that $x \in \bigcap_{n=1}^{\infty} A_n$.

\noindent
Use quantifiers to express what it means that $x \notin \bigcap_{n=1}^{\infty} A_n$ by negating quantifiers and rewriting until the expression is as simple as possible.}{1in}

\definition{Set complement}{If $A$ is a set, then $A^c$ is all elements under consideration that are not in $A$.  For example, if $A = [0,8)$, then $A^c = (-\infty,0) \cup [8,\infty)$.}

\problem{de Morgan's law.  Show that $\left( \bigcup_{n=1}^{\infty} A_n \right)^c = \bigcap_{n = 1}^{\infty} A_n^c$ by writing logical expressions for $x$ being in the set on the left side and for the right side.
Start by writing a logical expression that means the same thing as $x \in \left( \bigcup_{n=1}^{\infty} A_n \right)^c$ and work with it until it is a logical expression for $x \in \bigcap_{n = 1}^{\infty} A_n^c$.
When you write the proof this way, you do not need to show containment both ways to show that the two sets are equal.

$x \in \left( \bigcup_{n=1}^{\infty} A_n \right)^c$ means $\lnot (\exists $integer $n$ such that $x \in A_n)$, which means \ldots
}{1.0in}



\vfill          % pad the rest of the page with white space
