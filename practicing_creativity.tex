\yourname

\activitytitle{Practicing creativity}{You can get better at generating creative ideas}

\overview{Many proofs require a spark of genius, an idea that might not come to you right away.  It's OK to try the first thing you think of, but when it does not seem to be working out, deliberately look for a large number of possible things to try, and then choose which one looks the most promising.}

\note{These problems ask you to work with your group to generate ideas.  After the groups have had a chance, we'll summarize the ideas from the whole class.  Try to think of good ideas to share with the class!  The point is to generate ideas, not just to solve the problems.}

\groupwork{Problem 1.9 in the book by Daepp and Gorkin\DGreference~asks you to show that if $n$ is odd, then $n^3 - n$ is a multiple of 24.  With your group, take some time to try to think of 5 different things you could possibly do to approach this problem.  Think about $n^3 - n$.  Think about $n$ being odd.  Think about the number 24.  Think about how to show that something is a multiple of 24.  Look at examples.  If you already know how to do this problem, please don't tell the members of your group, but rather think of other things that might be good to try.  Maybe there is more than one way to solve this problem.
\blist{0.5in}
\item
\item
\item
\item
\item
\elist
}{0in}

\groupwork{Let $n$ be an integer.  Could it be that $n$ is both even and odd?  Go back to the definitions.  They don't say that it can't happen.  How can we be sure that it can't happen?  Here again, work with your group to think of a number of ways to possibly convince yourself that a number cannot be both even and odd.
\blist{0.5in}
\item
\item
\item
\item
\item
\elist
}{0in}

\note{At some point you may find yourself wondering just how integers are defined.  This might help:  0 is an integer.  Each integer has a unique ``next'' integer.  For 0, the next integer is 1, and there is no integer between them.  Similarly, each integer has a unique ``previous'' integer.  For 0, the previous integer is $-1$.}

\groupwork{Let $n$ be an integer.  Does $n$ need to be either even or odd?  Could it be neither?  How can we be sure that each integer meets the definition to be odd or meets the definition to be even?  Work with your group to try to think of a number of ways to see that each integer has to be even or odd.
\blist{0.5in}
\item
\item
\item
\item
\item
\elist
}{0in}

\groupwork{Let $n$ be an integer.  Suppose we know that $n^2$ is odd.  In every case that you check, you will find that $n$ is odd.  Is this always the case?  Can we be sure that $n$ will be odd?  Think of a number of possible ways to approach this.
\blist{0.5in}
\item
\item
\item
\elist
}{0in}




\vfill          % pad the rest of the page with white space
