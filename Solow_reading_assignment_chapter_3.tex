\readingtitle{Read Chapter \#3, On Definitions and Mathematical Terminology}

Read Chapter 3 in the book by Daniel Solow.
This chapter is about definitions and how to use them in the forward and backward process.
Fortunately, this does not seem to be a complicated chapter, so work through it and make sure you get the message.
Follow these instructions carefully.

\begin{enumerate}
\item Set yourself up in a place where you won't be disturbed.
Read slowly, and write notes in your own words that reflect your understanding of the material.
You do not need to paraphrase each paragraph, but I would encourage you to write down things that you learn, things that surprise you, or things that you need to remember.

\item When reading Section 3.1, pay close attention to ``if and only if'' because it comes up often in proofs, and you need to know how to show an ``if and only if'' statement.

\item In Section 3.1, there is a page and a half on overlapping notation.
This is a good discussion.

\item In Section 3.2, read the analysis of proof for Proposition 3 and then write down the steps A, A1, B2, B1 in order to make your own proof of the result.
Explain why each step in the proof can be taken.

\item Make yourself a glossary (vocabulary list) for the terms converse, inverse, contrapositive, proposition, theorem, lemma, corollary, axiom, so you learn them well.
In Section 3.3, the author says that a proposition is a true statement that you are trying to prove.
I would emphasize that we don't call it a proposition until we know that it can be proven, and maybe we are trying to understand the proof.
Something we are simply trying to prove should be a called a conjecture; it might not turn out to be true!

\item Do Exercise 3.2.  You do not need to write proofs, only pose the key question, answer it abstractly, and rephrase your answer in terms used in the problem.

\item Do Exercise 3.5 parts c and d.
This will be a bit challenging because it requires you to look up two new definitions and interpret them.
That's an excellent skill, so practice it.

\item Do Exercise 3.12.

\item Do Exercise 3.15.

\item Do Exercise 3.19.
Note that definitions are not the kind of previous knowledge that we are looking for here.
Look for the two previous implications that are used here.

\item Do Exercise 3.21.

\item Do Exercise 3.27.


\item At the end, tally up how much time you have spent on this chapter.
Write this number in your notes.
\end{enumerate}

