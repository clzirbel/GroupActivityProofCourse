\readingtitle{Read Chapter \#12, Mathematical induction}

Read Chapter 12 in the book by Daniel Solow.
This chapter is about induction.
It is well written and will hopefully help you understand induction much better.
Pay special attention to the introduction of strong induction in sections 12.2 and 12.3.

\vspace{0.1in}
\noindent
{\bf Specific requirements}

\begin{itemize}
\item Write useful notes as you read the chapter, and turn those in.

\item Do exercise 12.2, and please do a good job on it.

\item Do exercise 12.6.

\item Do exercise 12.10.

\item Do exercise 12.21.  In addition to doing what the book asks, rewrite the proof and justify each step of the proof.
That is, give a reason that each step of the proof is true, especially with the string of equalities and inequalities.
This is going to take some work.
Roll up your sleeves and get it done.

\item Do exercise 12.22.  This will also take some real work.
In addition to answering the questions in the problem, please answer this question:

d. Could we use $n=1$ as the base case, and save ourselves the trouble of checking the base case for $n=2$?

\item Do exercise 12.23.  
Part (a) is asking about the sentence ``Let $x_0$ be a real number.'' which is called the Choose Method in Chapter 5.

The point $x_*$ is called a {\em fixed point} of the function, and the inequality involving $\alpha$ means that the fixed point is {\em attractive}.  
The point of the result is that if you apply the function $f$ over and over again, the values converge quickly to $x_*$.

Before you do 12.23, you might enjoy playing this little game.
On a calculator, calculate the square root of a number like 20, then take the square root again, and again, and again, and see what happens.
Then start with a number like 0.02 and take the square root again and again and again.
You could also do this with cosine, or with sine, or with exp.
These functions may or may not have a fixed point, and the fixed points may all be different.

If you took Math 3370, Differential Equations, you may have seen the result that Picard iteration has an attractive fixed point, and that is how you show existence of solutions of differential equations.

\leaveout{
\item Do exercise 12.27.
Write out the start of an induction proof by stating $P(n)$, stating and checking $P(1)$, stating $P(n+1)$, and thinking about how to use strong induction to show that $P(n+1)$ is true.
Explain why you will have trouble finishing the induction proof.

This is another case of applying a given function over and over.
It is widely believed that the result stated in this problem is true, but there is no known proof.
The problem has stumped the smartest mathematicians for decades, and has distracted whole departments of mathematics for weeks or months at a time.
If you can solve it, please do!
But do the rest of your homework first.
}

\item At the end, tally up how much time you have spent on this chapter.
Write this number in your notebook.
\end{itemize}

\noindent
{\bf General comments}

Set yourself up in a place where you won't be disturbed.
Read slowly, and write notes in your own words that reflect your understanding of the material.
