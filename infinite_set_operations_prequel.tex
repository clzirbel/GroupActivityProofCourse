\yourname

\activitytitle{Infinite unions, intersections, and a few other things}{This activity is a prequel to working with infinite unions and intersections.}

\overview{Infinite unions and intersections take a bit of getting used to.  Fortunately, we can understand them with quantifiers.}

\definition{Union}{Let $A_1, A_2, \ldots$ be sets, with universe $X$.
The {\em union} of $A_1, A_2, \ldots$, which is denoted $\bigcup_{n=1}^{\infty} A_n$, is all elements of $X$ which are in $A_n$ for some $n = 1, 2, 3, \ldots.$}

\definition{Intersection}{Let $A_1, A_2, \ldots$ be sets, with universe $X$.
The {\em intersection} of $A_1, A_2, \ldots$, which is denoted $\bigcap_{n=1}^{\infty} A_n$, is all elements of $X$ which are in $A_n$ for all $n = 1, 2, 3, \ldots.$}

\problem{Use quantifiers to express what it means that $x \in \bigcup_{n=1}^{\infty} A_n.$\\
{\bf Solution:} $\exists n, x \in A_n$.
In words, there is at least one $n$ for which $x$ is in $A_n$; that is what it takes to be in the union.}{0.2in}

\problem{Work with quantifiers to express what it means that $x \notin \bigcup_{n=1}^{\infty} A_n.$
Negate the previous expression and use rules of quantifiers to rewrite it, one small step at a time, until it is as simple as possible.}{1.5in}

\problem{Use quantifiers to express what it means that $x \in \bigcap_{n=1}^{\infty} A_n.$}{1.0in}

\problem{Work with quantifiers to express what it means that $x \notin \bigcap_{n=1}^{\infty} A_n.$
Negate the previous expression, then rewrite again using complements.}{1.5in}

\stop{Go back to each of the four preceding problems and write a sentence explaining the logic of the last expression that you wrote down and how it relates to the expression you started with.}

\problem{de Morgan's law.  Show that $\left( \bigcup_{i \in I} A_i \right)^c = \bigcap_{i \in I} A_i^c$ by writing logical expressions for $x$ being in the set on the left side and for the right side.
Note that here the union is over sets $A_i$ where the index $i$ comes from an index set $I$, but the logic is the same as in the previous problems.
Start by writing a logical expression that means the same thing as $x \in \left( \bigcup_{i \in I} A_i \right)^c$ and work with it until it is a logical expression for $x \in \bigcap_{i \in I} A_i^c$.
When you write the proof this way, you do not need to show containment both ways to show that the two sets are equal.

$x \in \left( \bigcup_{i \in I} A_i \right)^c$ means $\lnot (\exists i \in I, x \in A_i)$, which means \ldots
}{1.0in}

\problem{Show that $(3,\infty) \subset [3,\infty)$; these are both intervals on the real number line.
Remember that when you show $\subset$ there are two things to show:  containment, and that there is an element of one set that is not an element of the other.  Solve this problem by letting $x \in (3,\infty)$ and writing that information as the logical statement ``$x > 3$ is true''.}{0.7in}

\problem{Show that $[2,5) \cap (3,7) \subseteq (3,5)$ using inequalities.
Start by letting $x \in [2,5) \cap (3,7)$.}{0.8in}

\problem{Let $x > 0$.  Show that there exists an integer $n$ such that $0 < \frac{1}{n} < x$.
{\bf Hint:} Look at $\frac{1}{x}$ and round up.
{\bf Another hint:} Suppose that $x = 0.31$. What value of $n$ works?}{1.3in}

\problem{Let $n$ be an integer greater than 0.  Show that $[\frac{1}{n},1] \subseteq (0,1] \subseteq [0,1]$ by working with inequalities.  Then show that $[\frac{1}{n},1] \subset (0,1] \subset [0,1]$ by looking at individual points.}{0in}
\vfill          % pad the rest of the page with white space
