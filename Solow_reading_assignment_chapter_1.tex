\activitytitle{Reading assignment \#1}{Due on Tuesday, August 29.  20 points}

The idea is to read Chapter 1 of the textbook by Daniel Solow.
The assignment is to read it in a particular way.
It may take 3 hours to get it done, but you will learn something in those three hours, and you will start to develop a very important skill.

Get a copy of Chapter 1, ``The Truth of it All.''
Get out your notebook or some paper.
Go somewhere quiet, where you won't be interrupted for a while.
Silence your phone so you aren't disturbed.
Don't listen to music that will distract you, and make sure there is no video going where you can see it or hear it.

Put the notebook or paper right in front of you.
Put the textbook itself a bit farther away.
Make note of the time that you start reading in your notebook, maybe in the left margin.
Read the first paragraph of the chapter, then write one or two sentences in your notes which capture the main idea(s) of the paragraph.

Continue reading and writing a sentence summarizing each paragraph.
I believe that if you are not writing, you are probably not thinking as hard as you need to.
Read slowly.
If you run into a word you don't know, look it up.  
If you really don't know it, write the definition in your notebook.
It is OK to spend 15 minutes on each page of the book.  Really.
It is not a goal of the course to learn how to read faster.
The goal is to learn how to get more out of the time you spend reading.
If you stop to take a break, note the time that you stopped and the time you start again so you can calculate the total time.

When you read Section 1.1, comment on the goals he lays out.  Do you have the same goals?  Different?  How?

When you read Section 1.2, there are a few vocabulary words in bold, be sure to learn those.
There is a very important example about when you might call your friend a liar; read and reflect on this multiple times, and use the additional examples he gives to understand better.
If you are left with questions, please write them in your notes, perhaps highlight them, and we will try to answer.

Examples 1, 2, 3, and 4 illustrate how to do some of the exercises.

Note that solutions of the exercises marked with W are available online at \url{http://higheredbcs.wiley.com/legacy/college/solow/1118164024/sm/sm.pdf}
Resist the urge to turn your brain off and just read the solutions.  That is not what they are there for!

Please do the following exercises and write solutions in your notebook:

1.2

1.4 

1.7

1.8

1.12 

1.16 Answer the question yourself, then compare your answer with the online solution.

1.18 I suggest that you try various choices for $n$ and $x$ on a calculator and see what you find.  Write out clearly what $A$ and $B$ are, and how you have made $A$ true but $B$ not true.


At the end, tally up how much time you have spent on reading this chapter.
Write this number in your notebook and remember the number when you come to class.
Bring your notebook to class and turn it in for grading.
