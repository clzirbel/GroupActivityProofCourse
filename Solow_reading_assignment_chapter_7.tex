\activitytitle{Reading assignment Chapter 7}{Due on Tuesday, October 17.  20 points}

Read Chapter 7 in the book by Daniel Solow.
This chapter is about nested quantifiers.
We have actually been working with these from day 1 of the course, but now you will write things out more explicitly.

Here is an example.
When you proved that the sum of two even numbers is even, you proved that:
\begin{itemize}
\item For all integers $m$ and $n$, there exists an integer $k$ such that $m+n = 2k$.
\end{itemize}
In order to prove this, you used the choose method (described in Chapter 5) to fix particular values of $m$ and $n$ and then wrote a model proof which worked for all $m$ and $n$.
As part of that model proof, you constructed the new integer $k$, which follows the construction method described in Chapter 4.
There are many theorems following the general pattern ``For all objects $a$, there exists an object $b$ for which something depending on $a$ and $b$ happens.'
Note that in the example, $a$ is the pair $m,n$ and $b$ is the integer $k$.
Also, note that $b$ pretty much always depends on $a$.

There are also theorems following the pattern ``There exists an object $a$ such that for all objects $b$, something depending on $a$ and $b$ happens.
These work differently.
Here you have to do the construction of $a$ in a way that it will work for all $b$ simultaneously, then you fix an object $b$ and write a model proof that will work for all $b$.
The difference here is the order in which the nested quantifiers occur.
Note that here $b$ does not depend on $a$, and $a$ cannot be chosen for any one particular $b$ but needs to work for all $b$.

\vspace{0.1in}
\noindent
{\bf Specific requirements}
\vspace*{-0.15in}

\begin{itemize}
\item As you read Section 7.1, outline how you would use the ``construction'' and the ``choose'' methods to prove S1, S2, S3, and S4, following the model above.

\item Similarly, when reading Section 7.2, outline how you would show that a function is onto.

\item Do exercise 7.2.  

\item Do exercise 7.4.  Instead of doing it exactly as stated, instead create five examples of $(x,y)$ pairs that satisfy the criteria in part a and part b, and then answer part c.
This is why we don't fuss too much about the general pattern ``there exists $a$ such that there exists $b$ for which hsomething depending on $a$ and $b$ happens.''

\item Do exercise 7.7.  Instead of doing it exactly as stated, follow the model at the beginning of this assignment to explain how to use the construction method and the choose method to do a, b, and c.

\item Show that for all $a > 0$, there exists an integer $n > 0$ such that $\frac{1}{n} < a$.
While doing so, mention which part uses the ``construction'' method and which part uses the ``choose'' method.

\item Do exercise 7.18.
I suggest that for every $z$ in $\R$, you show that there exists an $x$ in $\R$ for which $f(g(x)) = z$.
That leaves the letter $y$ available, and you'll want to use it.

\item Do exercise 7.19.  Fortunately, you can construct $x$.

\item At the end, tally up how much time you have spent on this chapter.
Write this number in your notebook.
Bring your notebook to class and turn it in for grading.
\end{itemize}

\noindent
{\bf General comments}

Set yourself up in a place where you won't be disturbed.
Read slowly, and write notes in your own words that reflect your understanding of the material.
