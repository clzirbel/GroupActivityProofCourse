\note{We also can use mathematical induction to show some propositions about sets.}

\show{Prove that if $A_1,A_2,\ldots,A_n$ and $B_1,B_2,\ldots,B_n$ are sets such that $A_j \subseteq B_j$ for $j=1,2,\ldots, n$, then $\bigcap_{j=1}^n A_j \subseteq \bigcap_{j=1}^n B_j$. \Hint{ In the initial step, show that if $A_1 \subseteq B_1$ and $A_2 \subseteq B_2$, then $A_1 \cap A_2 \subseteq B_1 \cap B_2$.}\label{set1}}{2in}

\show{Prove that if $A_1,A_2,\ldots,A_n$ and $B$ are sets, then $(A_1 \cup A_2 \cup \cdots \cup A_n)\cap B = (A_1 \cap B) \cup (A_2 \cap B) \cdots \cup (A_n \cap B)$ for any given positive integer $n$. \HintNN{ In the initial step, we have to show the distributive property of intersection over union of sets: $(A_1 \cup A_2) \cap B = (A_1 \cap B) \cup (A_2 \cap B)$. We have already shown that earlier in the semester.  You will also use this property in the inductive step. \label{set2}}}{2in}

\note{In \ref{set1} and \ref{set2}, instead of showing $P(1)$ is true, we show $P(2)$ is true in the initial step. But why? Explain!}

\notation{The standard notation for the product of $a_1,a_2,\ldots, a_n$ is $\displaystyle a_1\cdot a_2 \cdot a_3 \cdots \cdot a_n = \prod_{k=1}^n a_k$.}
