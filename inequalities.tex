\yourname

\activitytitle{Inequalities}{We can define the $<$ relation for real numbers and establish its properties.}

\overview{Most students at your level take the real numbers as things that simply exist and have a number of properties such as commutativity.  In fact, the real numbers can be {\em constructed} from the rational numbers, and the rational numbers from the integers, and the integers from the positive integers.  In this activity, we back up to the point that the real numbers have been constructed, but before inequalities have been defined.  We define the $<$ relation and prove a number of useful properties that it satisfies.  Since the $>$ relation is so similar, we will not define it or show its properties.}

\note{\label{realnumberproperties}Let $\R$ denote the set of real numbers, and denote addition and multiplication of real numbers in the usual ways.
{\bf Addition} has these properties:  commutativity ($a+b = b+a$), associativity ($a + (b+c) = (a+b) +c$), additive identity (there exists a unique real number called 0 for which $a + 0 = a$ for all $a \in \R$), and additive inverse (for each number $a$ in $\R$, there exists a unique real number $-a$ for which $a + (-a) = 0$).
{\bf Multiplication} has these properties:  commutativity ($ab = ba$), associativity ($a(bc) = (ab)c$), multiplicative identity (there exists a unique real number called 1, with $1 \ne 0$, such that $a\cdot 1 = a$ for all $a$ in $\R$), multiplicative inverse (for each $a$ in $\R$ with $a \ne 0$, there exists a unique number called $a^{-1}$ for which $a \cdot a^{-1} = 1$.
{\bf Addition and multiplication} are related by the distributive property: ($(a+b)c = ac + bc$).}

\definition{Subtraction}{Let $a$ and $b$ be real numbers.  The difference of $a$ and $b$, denoted $a - b$, is the real number $a + (-b)$, where $(-b)$ denotes the additive inverse of $b$.}

\note{\label{negativeproperties}At the end of this activity, you will see how to establish the following useful properties regarding additive inverses and subtraction:
\balist{0in}
\item $a\cdot 0 = 0$ for all real numbers $a$.
\item The additive inverse $(-a)$ is equal to $(-1)\cdot a$, where $(-1)$ is the additive inverse of 1
\item $(-1)(-1) = 1$.
\item The additive inverse of $a+b$ is $(-a)+(-b)$.  Using subtraction notation, $-(a+b) = -a-b$.
\item $-(-a) = a$.
\ealist
You can use subtraction as usual in this activity, but if you would like to avoid subtraction notation and just use additive inverses, that is worth attempting.}

\definition{Positive real numbers\label{positivereals}}{By construction, the real numbers have a subset $\Rp$, called the {\em positive real numbers,} for which:
\balist{0.3in}
\item If $a,b \in \Rp$, then $a + b \in \Rp$.  ($\Rp$ is closed under addition.)
\item If $a,b \in \Rp$, then $a\cdot b \in \Rp$.  ($\Rp$ is closed under multiplication.)
\item For every real number $a$, either $a \in \Rp$ or $(-a) \in \Rp$ or $a = 0$.  Exactly one of the three happens.
\ealist
\vspace{0.2in}

Under each property above, write a sentence that states it in plain English.
Think of $\Rp$ as being the positive, non--zero numbers.
We can't use interval notation to write what $\Rp$ is, because intervals are defined in terms of inequalities, and we have not defined inequalities yet!}

\show{Let $a \in \R$ and suppose that $a \ne 0$.
Show that $a\cdot a \in \Rp$.
When you use properties from \ref{negativeproperties} or \ref{positivereals}, cite them by number.
{\bf Hint:} There are two cases left in \ref{positivereals}c.}{0.9in}

\show{Show that $1 \in \Rp$.  Be careful to cite any previous properties that you use.}{0.5in}

\show{Show that $(-1) \notin \Rp$.
{\bf Hint:} Assume that $(-1) \in \Rp$ and use \ref{positivereals}a.}{0.5in}

\definition{Less than}{\label{lessthan}Let $a$ and $b$ be real numbers.  We write that $a < b$ if $b - a \in \Rp$.}

\note{All of the following problems rely on  Definition \ref{lessthan}, so you will use it over and over.
Note that $>$ has not been defined yet, so be careful not to use it.}

\show{Show that $-1 < 0$. {\bf Hint:} use \ref{negativeproperties}d.}{0.5in}

\show{Show that $1 < 1$ is not true.
Thus, the $<$ relation is not reflexive.
When you use properties from \ref{negativeproperties} or \ref{positivereals}, cite them by number.
}{0.9in}

\show{Show that $0 < 1$ but that $1 < 0$ is not true.
Thus, the $<$ relation is not symmetric.}{0.9in}

\show{Show that the $<$ relation on $\R$ is transitive.
Follow good form by first letting $a, b, c$ be real numbers and supposing that $a < b$ and $b < c$.
When you use properties from \ref{negativeproperties} or \ref{positivereals}, cite them by number.
You may enjoy ticking off the properties of the real numbers that you use.
For example, in this proof, you are likely to use the fact that $(-b) + b = 0$, which is the additive inverse property.}{0.9in}

\show{Let $a,b \in \R$ and suppose that $a < b$.
Show that $-b < -a$.
When you use properties from \ref{negativeproperties} or \ref{positivereals}, cite them by number.
}{0.9in}

\show{Let $a, b, c \in \R$.  Suppose that $a < b$.  Show that $a+c < b+c$.
When you use properties from \ref{negativeproperties} or \ref{positivereals}, cite them by number.
}{0.9in}

\show{Let $a, b, c, d \in \R$.  Suppose that $a < b$ and $c < d$.  Show that $a+c < b+d$.
When you use properties from \ref{negativeproperties} or \ref{positivereals}, cite them by number.
}{0.9in}

\show{Let $a, b, c$ be real numbers.  Suppose that $a < b$ and $0 < c$.
Show that $ac < bc$.
}{0.9in}

\show{Let $a, b, c$ be real numbers.  Suppose that $a < b$ and $c < 0$.
Show that $bc < ac$.
}{0.9in}

\show{Let $a,b \in \R$ and suppose that $0 < a$ and $b < 0$.
Use a previous result to show that $ab < 0$.}{0.9in}

\show{Let $a \in \R$ and suppose that $0 < a$.
Show that $0 < a^{-1}$.
Here $a^{-1}$ is the multiplicative inverse of $a$.
{\bf Hint:}  This one take a bit more effort than the previous ones.
Note that division has not been defined yet, so just use addition, subtraction, and multiplication.}{0.9in}

\show{Let $a,b \in \R$ and suppose that $0 < a$ and $a < b$.
Show that $b^{-1} < a^{-1}$.}{0.9in}

\note{Below, you are asked to prove basic properties of additive inverses and subtraction.}

\show{Let $a$ be a real number.  Show that $a \cdot 0 = 0$.}{0.4in}

\show{People sometimes ask if the additive inverse $(-a)$ is the same as $(-1)\cdot a$, where $(-1)$ is the additive inverse of 1.
It's true, and here is how you show it; you should fill in steps and write the justifications at the right side of each line.
\begin{eqnarray*}
  a + (-1)\cdot a &=& 1\cdot a + (-1) \cdot a \\
                  &=& (1 + (-1)) \cdot a \\
                  &=&  \\
                  &=& 0,
\end{eqnarray*}
This shows that $(-1) \cdot a$ is the additive inverse of $a$.}{0in}

\show{You might think that it is obvious that $(-1)(-1) = 1$, where $(-1)$ is the additive inverse of 1, but this takes a few steps beyond the properties of the real numbers in \ref{realnumberproperties}.  Write justifications and complete the following steps to show it.
\begin{eqnarray*}
(-1) + (-1)(-1) &=& (-1)(1) + (-1)(-1) \\
                &=& (-1)(1 + (-1)) \\
                &=& \\
                &=& 0,
\end{eqnarray*}
which shows that $(-1)(-1)$ is the additive inverse of $-1$, which is 1.}{0in}

\show{The additive inverse of a sum works out nicely.
Let $a$ and $b$ be real numbers and think about the additive inverse of $a+b$.
Write justifications to the right of each statement.
\begin{eqnarray*}
-(a+b) &=& (-1)(a+b) \\
       &=& (-1)(a) + (-1)(b) \\
       &=& (-a) + (-b)
\end{eqnarray*}}{0in}

\show{Let $a \in \R$.  The statement $-(-a) = a$ is just a statement about additive inverses.  Prove that it is true.}{0in}


\vfill          % pad the rest of the page with white space
