\yourname

\activitytitle{Mathematical Induction}{Proving that a claim is true for all $n$}

\overview{One important task in mathematics is to find and distinguish regular patterns or sequences. The main method we use to prove results involving positive integers or about sequences is mathematical induction.}

\theorem{\bf Mathematical induction\label{MItheorem}}\\
{For each integer $n$, let $P(n)$ denote an assertion involving $n$.
\begin{itemize}
\item[(i)] (The basis step) Prove that $P(1)$ is true.
\item[(ii)] (The inductive step) For each $n = 1, 2, 3, \ldots$, assume that $P(n)$ is true, and use $P(n)$ to prove that $P(n+1)$ is true.
\end{itemize}
From the above two steps, we can conclude that $P(n)$ is true for all $n = 1, 2, 3, \ldots$.}

\note{Usually, the assertion $P(n+1)$ can be written in terms of the assertion $P(n)$ plus a new part.
Use what you know from $P(n)$ being true to make it easier to prove what you need to show.}

\guidedproof{Show that $3^n$ is odd for all $n = 1, 2, 3, \ldots$.
\blist{0.1in}
\item State $P(n)$:  $P(n)$ is that \blank{2in}
\item Basis step:  $P(1)$ is that \blank{1.5in}.  This is true because \blank{1.5in}.
\item State $P(n+1)$:  $P(n+1)$ is that \blank{2in}
\item Inductive step:  Let $n \geq 1$.  Assume that $P(n)$ is true.  Show that $P(n+1)$ is true.
You may use facts you have already proven about odd numbers.
\elist
\vspace{0.3in}
}

\example{List the first 7 positive odd integers in the table below, and write out the sum of the first $n$ positive odd numbers in the third row.
For example, when $n=3$, the sum is $1+3+5$.
\label{examplesum}
\centerline{
\begin{tabular}[c]{c|c|c|c|c|c|c|c|c|c|c}
  % after \\: \hline or \cline{col1-col2} \cline{col3-col4} ...
  $n$ & 1 & 2 & 3 & 4 & 5 & 6 & 7 & $\cdots$ & $n$ & $n+1$ \\
  \hline
  odd numbers & 1 & 3 & 5 & 7 & & & & $\cdots$ & & \\
  \hline
  sum & 1 & 4 & \hspace{0.2in} & \hspace{0.2in} & \hspace{0.2in} & \hspace{0.2in} & \hspace{0.2in} & $\cdots$ & \hspace{0.5in} & \hspace{0.5in} \\
\end{tabular}}
Write out a formula for the $n$th odd integer in terms of $n$.
Also write a formula for the $n+1$st odd integer in terms of $n$.
Make conjecture about the sum of the first $n$ odd integers, which will look like this: $1 + 3 + 5 + \cdots + (2n-1) =$ \blank{1in}.
}{0.0in}

\guidedproof{\label{oddsum}
Prove the conjecture in \ref{examplesum} using mathematical induction.
\blist{0.1in}
\item State $P(n)$:  $P(n)$ is that $1 + 3 + 5 + \cdots + (2n-1) = $ \blank{1in}
\item Basis step:  $P(1)$ is that \blank{1.5in}.  This is true because \blank{1.5in}.
\item Write out $P(n+1)$:  $P(n+1)$ is that \blank{3in}
\item Inductive step: Let $n \geq 1$.  Assume that $P(n)$ is true, and use that to show that $P(n+1)$ is true.
\elist
\vspace{1.5in}
By \blank{3in} we conclude that $1+3+5+\cdots+(2n-1)=$ \blank{1in} for all $n = 1, 2, 3, \ldots$.
}

\notation{The standard notation for the sum of $a_1,a_2,\ldots,a_n$ is $a_1+a_2+a_3+\cdots+a_n=\sum_{k=1}^{n}a_k$ and the notation for the product of $a_1,a_2,\ldots, a_n$ is $a_1\cdot a_2 \cdot a_3 \cdots \cdot a_n = \prod_{k=1}^n a_k$.}

\example{Rewrite the result in \ref{oddsum} using the standard notation for the sum.}{0.4in}

\show{Show that $\sum_{k=1}^n (4k-3) = n(2n-1)$ for all positive integers $n$.
\blist{0.1in}
\item State $P(n)$:  $P(n)$ is that:
\item Basis step:  $P(1)$ is that:
\item State $P(n+1)$:  $P(n+1)$ is that:
\item Inductive step: Let $n \geq 1$.  Assume that $P(n)$ is true, and use that to show that $P(n+1)$ is true.
\elist
}{0.8in}

\show{Use mathematical induction to show that $\sum_{k=1}^n 5^k = \frac{5}{4} (5^n-1)$ for all positive integers $n$.
\blist{0.1in}
\item State $P(n)$:  $P(n)$ is that:
\item Basis step:  $P(1)$ is that:
\item State $P(n+1)$:  $P(n+1)$ is that:
\item Inductive step: Let $n \geq 1$.  Assume that $P(n)$ is true, and use that to show that $P(n+1)$ is true.
\elist
}{0.8in}

\stop{Compare your proofs with the other people in your group before you move on.}

\note{The basis step need not use $n=1$, for example, it can use $n=-3, n= 0,$ or $n = 100$.}

\show{Use mathematical induction to show that $2n+1 < 2^n$ for all integers $n$ with $n \geq 4$.
\blist{0.1in}
\item State $P(n)$:  $P(n)$ is that:
\item Basis step:  $P(4)$ is that:
\item State $P(n+1)$:  $P(n+1)$ is that:
\item Inductive step: Let $n \geq 4$.  Assume that $P(n)$ is true, and use that to show that $P(n+1)$ is true.
\elist
}{1in}

\show{Use mathematical induction to show that $5^n > 2^n + 3^n$ for all integers $n$ with $n \geq 2$.  Use the same format as above.}{1.5in}

\show{Use induction to prove Bernoulli's inequality: For $x\in \mathbb{R}$, if $1+x >0$, then $(1+x)^n \geq 1+nx$ for all $n = 0, 1, 2, \ldots$.  Use the same format as above.}{1.5in}

\show{Use induction to prove that $\frac{1}{1\cdot 2} + \frac{1}{2\cdot 3} + \frac{1}{3\cdot 4} + \cdots+\frac{1}{n\cdot (n+1)} = \frac{n}{n+1}$ for all positive integers $n$. \label{sumeq}}{1.5in}

\note{In Problem \ref{sumeq}, we should be able to show that the statement is true without using mathematical induction.  How?}

\show{For each $n\in \mathbb{Z}^+$, let $P(n)$ denote the assertion ``$n^2+5n+1$ is an even integer.''
\blist{0.05in}
\item Prove that $P(n+1)$ is true whenever $P(n)$ is true.
\item For which $n$ is $P(n)$ actually true? 
\item What is moral of this exercise?
\elist
}{1in}

\show{Use induction to prove that $n^3-n$ is a multiple of 6 for all integers $n = 0, 1, 2, \ldots$.}{1.5in}

\show{Use induction to prove that $11^n-4^n$ is a multiple of 7 for all $n = 0, 1, 2, \ldots$.}{1.5in}

\show{Prove that $1^2-2^2+3^2-4^2+5^2+\cdots-(2n)^2+(2n+1)^2 = (n+1)(2n+1)$ for all $n = 0, 1, 2, \ldots$. \Hint{ It would be helpful to write down $P(0)$ and $P(1)$ first.}}{1.5in}

\pagebreak
\note{We also can use mathematical induction to show some propositions about sets.}

\show{Prove that if $A_1,A_2,\ldots,A_n$ and $B_1,B_2,\ldots,B_n$ are sets such that $A_j \subseteq B_j$ for $j=1,2,\ldots, n$, then $\bigcap_{j=1}^n A_j \subseteq \bigcap_{j=1}^n B_j$. \Hint{ In the initial step, show that if $A_1 \subseteq B_1$ and $A_2 \subseteq B_2$, then $A_1 \cap A_2 \subseteq B_1 \cap B_2$.}\label{set1}}{2in}

\show{Prove that if $A_1,A_2,\ldots,A_n$ and $B$ are sets, then $(A_1 \cup A_2 \cup \cdots \cup A_n)\cap B = (A_1 \cap B) \cup (A_2 \cap B) \cdots \cup (A_n \cap B)$ for any given positive integer $n$. \HintNN{ In the initial step, we have to show the distributive property of intersection over union of sets: $(A_1 \cup A_2) \cap B = (A_1 \cap B) \cup (A_2 \cap B)$. We have already shown that earlier in the semester.  You will also use this property in the inductive step. \label{set2}}}{2in}

\note{In \ref{set1} and \ref{set2}, instead of showing $P(1)$ is true, we show $P(2)$ is true in the initial step. But why? Explain!}

\vfill          % pad the rest of the page with white space

