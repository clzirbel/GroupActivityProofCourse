\yourname

\activitytitle{Mathematical Induction}{Proving that a claim is true for all $n=1, 2, 3, \ldots$.  }

\overview{One important task in mathematics is to find regular patterns and prove that they hold.
The main method we use for this is mathematical induction.
Co-authored by Ying-Ju Chen.}

\theorem{\bf Mathematical induction\label{MItheorem}}\\
{For each integer $n=1, 2, 3, \ldots$, let $P(n)$ denote a true/false statement involving $n$.
\begin{itemize}
\item[(i)] (The basis step) Prove that $P(1)$ is true.
\item[(ii)] (The inductive step) For each $n = 1, 2, 3, \ldots$, suppose that $P(n)$ is true, and use $P(n)$ to prove that $P(n+1)$ is true.
\end{itemize}
From the above two steps, we can conclude that $P(n)$ is true for all $n = 1, 2, 3, \ldots$.}

\note{Proving the inductive step is usually done as a ``rewrite'' proof, where you start with the left hand side of what you want to show and rewrite until you come to the desired right hand side.
Often, some quantity in the statement $P(n+1)$ can be written in terms of a similar quantity in the statement $P(n)$ plus a new part.
You will always use the fact that $P(n)$ is true.}

\question{A student working on an induction problem wrote $P(n+1) = P(n) + \frac{1}{n^2}$.
How can you tell that this must be wrong?}
{0.2in}

\guidedproof{\label{inductionguidedproof}
Use mathematical induction to show that $3^n$ is odd for all $n = 1, 2, 3, \ldots$.
\balist{0.0in}
\item State $P(n)$:  $P(n)$ is that \blank{2in}
\item Basis step:  $P(1)$ is that \blank{1.5in}.  This is true because \blank{1.5in}.
\item State $P(n+1)$:  $P(n+1)$ is that \blank{2in}
\item Inductive step:  Let $n \geq 1$.  suppose that $P(n)$ is true.  Show that $P(n+1)$ is true.
You may use facts you have already proven about odd numbers.

\noindent
Because $P(n)$ is true, \blank{1.5in}.
Now $3^{n+1}=$ \blank{1in}, which is odd because \blank{3.5in}.
Thus $P(n+1)$ is true.
Since $n \geq 1$ was arbitrary, by mathematical induction, $P(n)$ is true for all $n \geq 1$.
\elist
}

\exercise{Fill in the table using your powers of pattern recognition.

\vspace*{0.1in}
\label{examplesum}
\centerline{
\begin{tabular}[c]{c|c|c|c|c|c|c|c|c|c|c}
  $n$ & 1 & 2 & 3 & 4 & 5 & 6 & 7 & $\cdots$ & $n$ & $n+1$ \\
  \hline
  first $n$ odd integers & 1 & 3 & 5 & 7 & & & & $\cdots$ & & \\
  \hline
  sum of first $n$ odd integers & ~1~ & ~4~ & \hspace{0.2in} & \hspace{0.2in} & \hspace{0.2in} & \hspace{0.2in} & \hspace{0.2in} & $\cdots$ & \hspace{0.5in} & \hspace{0.5in} \\
\end{tabular}}

\vspace*{0.1in}
\noindent
Column $n$ of the table contains a conjecture about the sum of the first $n$ odd integers.
In the next problem, you will use mathematical induction to prove it.
}{0.0in}

\pagebreak
\guidedproof{\label{oddsum}
Prove the conjecture in \ref{examplesum} using mathematical induction.
\balist{0.1in}
\item State $P(n)$:  $P(n)$ is that the sum of the first $n$ odd integers equals \blank{1in}
\item Basis step:  $P(1)$ is that \blank{2.25in}.  This is true because \blank{0.75in}.
\item Write out $P(n+1)$:  $P(n+1)$ is that \blank{4in}
\item Inductive step: Let $n \geq 1$.  suppose that $P(n)$ is true, and use that to show that $P(n+1)$ is true.
\begin{eqnarray*}
\lefteqn{\mbox{the sum of the first $n+1$ odd integers}} \\
&=& \mbox{the sum of the first $n$ odd integers plus \blank{1in}} \\
&=& \blank{1.5in} + \blank{1.5in} \mbox{since $P(n)$ is true} \\
&=& \blank{1.5in}
\end{eqnarray*}
Thus $P(n+1)$ is true.
Since \blank{1in} was arbitrary, by \blank{2in} we conclude that the sum of the first $n$ odd integers equals \blank{1in} for all $n = 1, 2, 3, \ldots$.
\elist
}

\notation{Summation notation for $a_1+a_2+a_3+\cdots+a_n$ is $\displaystyle \sum_{k=1}^{n}a_k$.}

\example{Use summation notation to rewrite the result in \ref{oddsum}.}{0.2in}

\exercise{Fill in the blanks.

{\bf a.} $\displaystyle \sum_{k=1}^{n+1} k^2 = \sum_{k=1}^{n} k^2 + \blank{1in}$
\qqqq
{\bf b.} $\displaystyle \sum_{k=1}^{n+1} \frac{1}{k^3} = \sum_{k=1}^{n} \frac{1}{k^3} + \blank{1in}$
}{0in}

\show{Show that $\sum_{k=1}^n 4k-3 = n(2n-1)$ for all $n = 1, 2, 3, \ldots$.
\balist{0.0in}
\item State $P(n)$:  $P(n)$ is that
\item Basis step:  $P(1)$ is that  \hfill Check: \blank{1.5in}
\item State $P(n+1)$:  $P(n+1)$ is that

\vspace*{0.2in}
\noindent
Suggestion:  Use algebra to simplify the right hand side.

\item Inductive step: Let $n \geq 1$.  suppose that $P(n)$ is true, and use that to show that $P(n+1)$ is true.

\vspace*{-0.2in}
\begin{eqnarray*}
\sum_{k=1}^{n+1} 4k-3 &=& \blank{1.5in} + \blank{1.5in} \\
                                         &=& \blank{1.5in} + \blank{1.5in} \qq \mbox{since $P(n)$ is true} \\
                                         &=& \\
                                         &=& 
\end{eqnarray*}
Thus, \blank{1.5in}.  Since ... 
\elist
}{0.0in}

\stop{Compare your proofs with the other people in your group before you move on.}

\show{Use mathematical induction to show that $\sum_{k=1}^n 5^k = \frac{5}{4} (5^n-1)$ for all integers $n \geq 1$.
\balist{0.1in}
\item State $P(n)$:  
\item Basis step:  
\item State $P(n+1)$:  

\vspace*{0.2in}
\noindent
Suggestion:  Multiply out the right hand side.

\item Inductive step: Let $n \geq 1$.  suppose that $P(n)$ is true, and use that to show that $P(n+1)$ is true.
\elist
}{0.9in}

\note{The basis step need not use $n=1$, for example, it can use $n=-3, n= 0,$ or $n = 100$.}

\show{Use mathematical induction to show that $2n+1 < 2^n$ for all integers $n$ with $n \geq 4$.
\balist{0.10in}
\item State $P(n)$:
\item Basis step:  $P(4)$ is that:  \hfill Check: \blank{1.5in}
\item State $P(n+1)$:  
\item Inductive step: Let $n \geq 4$.  suppose that $P(n)$ is true, and use that to show that $P(n+1)$ is true.
\begin{eqnarray*}
2(n+1) + 1 = \blank{1.5in} &=& \blank{1.5in} \\
                                            &<& \blank{1.5in} \qq \mbox{since $P(n)$ is true}\\
                                            &<& \blank{1.5in} \\
                                            &=& 2^{n+1}.
\end{eqnarray*}
Thus, \blank{1.5in}.  Since ... 
\elist
}{0in}

\show{Use mathematical induction to show that $5^n > 2^n + 3^n$ for all integers $n$ with $n \geq 2$.  Use the same format as above.
\balist{0.0in}
\item
\item
\item
\item
\elist
}{1.5in}

\show{Use induction to prove Bernoulli's inequality: For all $x\in \R$, if $1+x >0$, then $(1+x)^n \geq 1+nx$ for all $n = 0, 1, 2, \ldots$.  Use the same format as above.

\noindent
Let $x$ be such that $1+x > 0$.

\vspace*{1.5in}
\noindent
Where did you use the assumption that $1+x > 0$?
}{0in}

\show{Use induction to prove that $\frac{1}{1\cdot 2} + \frac{1}{2\cdot 3} + \frac{1}{3\cdot 4} + \cdots+\frac{1}{n\cdot (n+1)} = \frac{n}{n+1}$ for all positive integers $n$. \label{sumeq}}{1.5in}

\note{It is possible to show that the statement in \ref{sumeq} is true without using mathematical induction, but using a different algebraic technique.  How?}

\show{For each $n\in \Z^+$, let $P(n)$ denote the statement ``$n^2+5n+1$ is an even integer.''
\balist{0.05in}
\item State $P(n+1)$
\item Suppose that $P(n)$ is true, and use that to prove that $P(n+1)$.
\vspace*{0.5in}
\item For which $n$ is $P(n)$ actually true? 
\item What is moral of this exercise?
\elist
}{0in}

\show{Use induction to prove that $n^3-n$ is a multiple of 6 for all integers $n = 0, 1, 2, \ldots$.
Do the induction step as a ``rewrite'' proof.}{1.5in}

\show{Use induction to prove that $11^n-4^n$ is a multiple of 7 for all $n = 0, 1, 2, \ldots$.
Do the induction step as a ``rewrite'' proof.
\Hint Use the equation $11^n-4^n = 7k$ once.}{1.5in}

\show{Prove that $1^2-2^2+3^2-4^2+5^2+\cdots-(2n)^2+(2n+1)^2 = (n+1)(2n+1)$ for all $n = 0, 1, 2, \ldots$. \Hint{ It would be helpful to write down $P(0)$ and $P(1)$ first.}}{1.5in}

\prove{Let $k > 0$ be an integer.
For each integer $n = 0, 1, 2, \ldots$, let $P(n)$ be the statement: ``There exist integers $q$ and $r$ with $0 \leq r < k$ such that $n = kq + r$.''
Use mathematical induction to show that $P(n)$ is true for all integers $n$.

\balist{0.5in}
\item Show that $P(0)$ is true.

\item Suppose that $P(n)$ is true and show that $P(n+1)$ is true.
It is helpful to do this with two cases.

\noindent
Case 1.  Suppose that $n = qk + r$ where $0 \leq r < k-1$.

\vspace*{0.5in}

\noindent
Case 2.  Suppose that $n = qk + r$ where $r = k-1$.

\item Suppose that $P(n)$ is true and show that $P(n-1)$ is true.
It is helpful to do this with two cases.
\elist

\vspace*{1.5in}
\noindent
Use steps b and c and the idea of mathematical to conclude the proof.

}{0in}

\show{
Use mathematical induction to prove that $\displaystyle \sum_{k=1}^{n} k = \frac{1}{2}n(n+1)$ for all integers $n = 1, 2, 3, \ldots$.
}{1.5in}

\show{
Use mathematical induction to prove that $\displaystyle \sum_{k=1}^{n} k^2 = \frac{1}{6}n(n+1)(2n+1)$ for all integers $n = 1, 2, 3, \ldots$.
}{1.5in}

\show{Let $r$ be a real number not equal to 1.
Use induction to prove that $\displaystyle \sum_{k=0}^{n} r^k = \frac{1-r^{n+1}}{1-r}$ for all integers $n = 0, 1, 2, \ldots$.
Note where you use the assumption on $r$.
}{1.5in}

\show{Use mathematical induction to show that $n! > 3^n$ for all $n = 7, 8, 9, \ldots$.}{1.5in}


\vfill          % pad the rest of the page with white space

