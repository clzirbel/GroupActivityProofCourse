% ---------------------------------- Packages

\usepackage{amsmath}    % allows AMS math things like the cases environment
\usepackage{amssymb}    % allows the use of AMS symbols like blackboard bold
\usepackage{amsthm}     % allows AMS definitions for theorems, proofs, etc.

% ----------------------------------- Packages for set theory activity

%\usepackage[utf8]{inputenc}
%\usepackage[english]{babel}
\usepackage{tikz}
\usetikzlibrary{shapes,backgrounds}

%\usepackage{geometry}
%\geometry{margin = 0.5in}


%\usepackage{hyperref}     % seems to cause troubles with \ref and \label, not sure why
\newcommand{\url}[1]{#1}  % use this when hyperref is not used

% ---------------------------------- Personalization

\newcommand{\coursenumber}{Math 3280}
\newcommand{\coursename}{Mathematical Foundations and Techniques}

% --------------------------------- Types of sections and their numbering

\newcommand{\yourname}{\hfill {\bf Your name: \underline{\hspace{2.5in}}}
%\vspace*{-0.1in}
}

\newcommand{\anonymous}{\hfill {\bf Anonymous!}}

\newcommand{\blank}[1]{\underline{\hspace{#1}}}

\newcommand{\activitytitle}[2]{\noindent {\bf \LARGE #1}\\\noindent #2\vspace*{0.1in}}

\newcommand{\overview}[1]{\noindent{\bf \Large Overview}\\\framebox{\parbox{7in}{#1}}}

\theoremstyle{definition}  % use roman font in theorems, definitions, etc.

\newtheorem{theorem}{\bf Theorem}                  % Allow numbered theorems
\newtheorem{proposition}[theorem]{\bf Proposition} % Number props with theorems
\newtheorem{remark}[theorem]{\bf Remark}           % Number remarks with theorems
\newtheorem{definitionX}[theorem]{{\bf Definition}}
\newtheorem{notationX}[theorem]{\bf Notation}
\newtheorem{exampleX}[theorem]{\bf Example}
\newtheorem{exerciseX}[theorem]{\bf Exercise}
\newtheorem{problemX}[theorem]{\bf Problem}
\newtheorem{showX}[theorem]{\bf Show}
\newtheorem{recallX}[theorem]{\bf Recall}
\newtheorem{noteX}[theorem]{\bf Note}
\newtheorem{guidedproofX}[theorem]{\bf Guided proof}
\newtheorem{proveX}[theorem]{\bf Prove}
\newtheorem{groupworkX}[theorem]{\bf Group work}
\newtheorem{questionX}[theorem]{\bf Question}
\newtheorem{challengeX}[theorem]{\bf Challenge}

% ------------------------------------------ Shortcuts for activity sections

\newcommand{\definition}[2]{\begin{definitionX}{\bf #1.} #2\end{definitionX}}
\newcommand{\notation}[1]{\begin{notationX}#1\end{notationX}}
\newcommand{\example}[2]{\begin{exampleX}#1\end{exampleX}\vspace*{#2}}
\newcommand{\exercise}[2]{\begin{exerciseX}#1\end{exerciseX}\vspace*{#2}}
\newcommand{\problem}[2]{\begin{problemX}#1\end{problemX}\vspace*{#2}}
\renewcommand{\show}[2]{\begin{showX}#1\end{showX}\vspace*{#2}}
\newcommand{\recall}[2]{\begin{recallX}#1\end{recallX}\vspace*{#2}}
\newcommand{\note}[1]{\begin{noteX}#1\end{noteX}}
\renewcommand{\stop}[1]{\noindent{\bf \Large \underline{Stop.}} #1}
\newcommand{\Hint}{{\bf Hint:~}}
\newcommand{\guidedproof}[1]{\begin{guidedproofX}#1\end{guidedproofX}}
\newcommand{\prove}[2]{\begin{proveX}#1\end{proveX}\vspace*{#2}}
\newcommand{\groupwork}[2]{\begin{groupworkX}#1\end{groupworkX}\vspace*{#2}}
\newcommand{\question}[2]{\begin{questionX}#1\end{questionX}\vspace*{#2}}
\newcommand{\challenge}[2]{\begin{challengeX}#1\end{challengeX}\vspace*{#2}}

% ----------------------------- Facilitate repeating sections with no number (NN)

\newcommand{\definitionNN}[2]{\noindent{\bf Definition #1.} #2}
\newcommand{\notationNN}[1]{\noindent{\bf Notation } #1}
\newcommand{\exampleNN}[2]{\noindent{\bf Example.} #1 \\ \vspace*{#2}} \newcommand{\showNN}[2]{\noindent{\bf Show.} #1 \\ \vspace*{#2}}
\newcommand{\recallNN}[2]{\noindent{\bf Recall.} #1 \\ \vspace*{#2}}
\newcommand{\noteNN}[1]{\begin{noteX}#1\end{noteX}}
\newcommand{\HintNN}{{\bf Hint:}}
\newcommand{\guidedproofNN}[1]{\begin{guidedproofX}#1\end{guidedproofX}}
\newcommand{\proveNN}[2]{\begin{proveX}#1\end{proveX}\vspace*{#2}}
\newcommand{\groupworkNN}[2]{\begin{groupworkX}#1\end{groupworkX}\vspace*{#2}}
\newcommand{\questionNN}[2]{\begin{questionX}#1\end{questionX}\vspace*{#2}}

% ---------------------------- Lists

\newcommand{\blist}[1]{\begin{list}{{\bf \arabic{enumi}.}}{\usecounter{enumi}\setlength{\itemsep}{#1}}} 
                                     % begin a numbered list.  The optional
                                     % argument is the spacing between items
\newcommand{\elist}{\end{list}}      % end the list

\newcommand{\balist}[1]{\begin{list}{{\bf \alph{enumii}.}}{\usecounter{enumii}\setlength{\itemsep}{#1}\setlength{\topsep}{0in}}} 
                                     % begin a numbered list.  The optional
                                     % argument is the spacing between items
\newcommand{\ealist}{\end{list}}      % end the list

\newcommand{\bilist}[1]{\begin{list}{{\bf \roman{enumii}.}}{\usecounter{enumii}\setlength{\itemsep}{#1}}} 
                                     % begin a numbered list.  The optional
                                     % argument is the spacing between items
\newcommand{\eilist}{\end{list}}      % end the list

% ----------------------------- Math shortcuts

\newcommand{\vect}[1]{\langle #1 \rangle}   % angle brackets for a vector
\newcommand{\tvec}[1]{\vect{#1_1, #1_2, #1_3}}
\newcommand{\qq}{\quad\quad}
\newcommand{\qqq}{\quad\quad\quad}
\newcommand{\qqqq}{\quad\quad\quad\quad}

\newcommand{\R}{\mathbb{R}}
\newcommand{\Z}{\mathbb{Z}}
\newcommand{\Q}{\mathbb{Q}}
\newcommand{\N}{\mathbb{N}}
\newcommand{\C}{\mathbb{C}}
\newcommand{\Rp}{\R^{+}}

\newcommand{\beqnarray}[1]{\vspace*{#1}\begin{eqnarray*}}
\newcommand{\eeqnarray}[1]{\end{eqnarray*}\vspace*{#1}}

\newcommand{\zero}{\boldsymbol{0}}
\newcommand{\one}{\boldsymbol{1}}

\newcommand{\circone}{\raisebox{.5pt}{\textcircled{\raisebox{-.9pt} {1}}}}
\newcommand{\circtwo}{\raisebox{.5pt}{\textcircled{\raisebox{-.9pt} {2}}}}

\newcommand{\DGreference}{\footnote{Reading, Writing, and Proving: A Closer Look at Mathematics, 2011, by Ulrich Daepp and Pamela Gorkin}}
