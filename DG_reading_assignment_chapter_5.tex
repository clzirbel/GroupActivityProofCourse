\activitytitle{Reading assignment, Chapter 5}{Due in the fourth week of classes.}

Read and understand Chapter 5 of the textbook by Daepp and Gorkin.
As with previous chapters,
{\small
\blist{0.0in}
\item Read somewhere quiet, minimizing distractions from phones and friends
\item Note the time that you start and stop reading, and add up the minutes
\item Read with a pencil in your hand and your notebook open in front of you
\item Write a sentence to summarize each paragraph, re--draw diagrams, work out examples and exercises on your own
\item Look up words you don't know, and write down ones you really don't know
\item Read slowly.  You are not reading a comic book or a newspaper.  It is not a goal of this class for you to learn how to read faster.  The goal is to learn how to get more out of the time you spend reading, and to learn to concentrate for longer periods of time.
\item At the end, tally up how much time you have spent on reading this chapter.
Write this number in your notebook and remember the number when you come to class.
\elist
}

This chapter walks you through a number of types of proofs and gives examples of each.  {\bf Rewrite these proofs in your notes, in your own words as much as possible, so that you make them yours.}  By the end of reading the chapter, you should {\bf know} the proof that the square root of 2 is irrational and you should know the other proofs as well.

It might help, in your notes, to make a list of proof techniques from the chapter and from previous chapters.
What chapter talked about proof by contrapositive?  Is that in Chapter 5?
What about truth tables?  You can prove things with those.  What kinds of things?

Read and understand Problem 1.  It is important.

Read the other problems, find the ones that are easy, and do them.
This may seem like a strange assignment, but I really mean it.  Think about each problem (if you can get through all of them), and make sure that if a problem is easy, that you recognize that and write out the solution.
Don't worry if a problem looks hard but turns out to be easy.
That happens all the time.
But hopefully you will spot a number of them that really are easy, and do them.
We will go over these problems in class next week.

\vfill          % pad the rest of the page with white space
