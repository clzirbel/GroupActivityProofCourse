\activitytitle{Reading assignment, Chapter 7}{Due in the seventh week of class.}

Read and understand Chapter 7 of the textbook by Daepp and Gorkin called ``Operations on sets.''

This is a short chapter, all about working with sets.
You can approach these problems in a number of ways.
Often it helps to draw a nice Venn diagram and get the right intuitive idea for what is being claimed, but don't stop there.
You can also just focus on letting $x \in A$ or whatever and working with that, without thinking about Venn diagrams.

Most of the chapter is devoted to one example, showing that, if $A,$ $B,$ and $C$ are sets, then $A \cup (B \cap C) = (A \cup B) \cap (A \cup C)$.
The book suggests working forward from one side, and backward from the other, just as people sometimes build a bridge by starting at each bank of a river and meeting in the middle.

It also suggests breaking into cases at some point.  For example, if $x \in A \cup (B \cap C)$, you can consider the case $x \in A$, which is great because then it's pretty clear that $x \in (A \cup B) \cap (A \cup C)$.  But you also need to consider the case $x \notin A$, so that $x \in B \cap C$.  But that's helpful, because then $x \in B$ and $x \in C$, and pretty soon it is clear that 
$x \in (A \cup B) \cap (A \cup C)$.

{\bf Do problem 7.1, all six parts.}  Take your time and use really good form so that the proof is crystal clear.  Notice that part (c) (statement 18 in the theorem) is an ``if and only if'' statement, so it has two parts.  It's going to look something like this:
\blist{0in}
\item Suppose that $A \subseteq B$.  We want to show that $(X \backslash B) \subseteq (X \backslash A).$  Let $x \in X \backslash B$.  Then $x \notin B$.  (More steps here.) Thus, $x \in X \backslash A$, and so $(X \backslash B) \subseteq (X \backslash A).$
\item Suppose that $(X \backslash B) \subseteq (X \backslash A).$  We want to show that $A \subseteq B$.  Let $x \in A$. (More steps here.)  Thus, $x \in B$.
\elist

{\bf Do problem 7.4.}

{\bf Do problem 7.6.}

I guess that these problems are a bit dull, but it really is helpful to be good at proving things about sets.  As the course goes on, I suspect that we will run into sets often, and these basic skills will pay off again and again.

\noindent As with the previous chapters,
\blist{0.0in}
\item Read somewhere quiet, minimizing distractions from phones and friends
\item Note the time that you start and stop reading, and add up the minutes
\item Read with a pencil in your hand and your notebook open in front of you
\item Write a sentence to summarize each paragraph, re-draw diagrams, work out examples and exercises on your own
\item Look up words you don't know, and write down ones you really don't know
\item Read slowly.  You are not reading a comic book or a newspaper.  It is not a goal of this class for you to learn how to read faster.  The goal is to learn how to get more out of the time you spend reading, and to learn to concentrate for longer periods of time.
\item At the end, tally up how much time you have spent on reading this chapter.
Write this number in your notebook and remember the number when you come to class.
\elist


\vfill          % pad the rest of the page with white space
