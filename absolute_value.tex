\yourname

\activitytitle{Absolute value and related functions}{A careful development of the properties of the absolute value function.}

\overview{The absolute value function is easy to understand for numbers like 9 and $-13$, but it's harder to show its properties because our intuition works so hard to see all variables as having positive values.
In this activity, we will not use the standard notation for the absolute value function and will have to keep our intuition at bay.
We will instead rely completely on the definition.
{\bf When you use a property of inequalities, cite it by number.}}

\definition{Absolute value}{The function $f : \R \to \R$ defined by
\[
    f(x) = \left\{ \begin{array}{cl} x, & \mbox{if $x \geq 0$} \\
                          -x, & \mbox{if $x < 0$} \end{array} \right.
\]
is called the {\em absolute value} function.}

\notation{In this activity, do not use the standard notation for absolute value, not even once.
Every time you work with the absolute value function, use and cite the definition.}

\show{\label{absofproduct}Show that $f(ab) = f(a) f(b)$ for all real numbers $a$ and $b$.
Follow the model.\\

Let $a$ and $b$ be real numbers.
There are four cases.
\blist{0.5in}
\item Suppose that $a \geq 0$ and $b \geq 0$.  Then $ab \geq 0$ so $f(ab) = ab$ and $f(a) = a$ and $f(b) = b$, so $f(ab) = ab = f(a)f(b)$.
\item Suppose that $a \geq 0$ and $b < 0$.
\item Suppose that $a < 0$ and $b \geq 0$.
\item Suppose that $a < 0$ and $b < 0$.
\elist
In each case, we see that \blank{2in}.
We made no further assumptions about \blank{1in}, thus \blank{3in}.}{0in}

\show{Following the model above, show that $f(f(a)) = f(a)$ for all real numbers $a$.}{1in}

\show{Show that $f(-a) = f(a)$ for all real numbers $a$.}{1in}

\show{Show that $f(a-b) = f(b-a)$ for all real numbers $a$ and $b$.}{1in}

\show{Show that $f(a) \geq 0$ for all real numbers $a$.}{1in}

\show{Follow the model in \ref{absofproduct} to show that $f(a+b) \leq f(a) + f(b)$ for all real numbers $a$ and $b$.
When $a$ and $b$ have different signs, consider two cases, $a+b \geq 0$ and $a+b < 0$.
You will probably want to show that if $b < 0$, then $b < -b$.
Make a good, solid argument using transitivity of $<$.}{3in}

\show{Show that for real numbers $a$ and $b$, $f(a) \leq b$ if and only if $-b \leq a \leq b$.
Remember that an ``if and only if'' proof has two directions.
In both directions, you will have to consider two cases, $a \geq 0$ and $a < 0$.
Note that the statement $-b \leq a \leq b$ is equivalent to ($-b \leq a$ and $a \leq b$).}{3in}

\show{Show that for all real numbers $a$ and $b$, $f(a) \geq b$ if and only if ($a \geq b$ or $a \leq -b$).}{2in}

\show{Show that for all real numbers $a$ and $b$, $f(a) \leq f(a-b) + f(b)$. {\bf Hint:} Look at $f((a-b)+b)$.}{1in}

\show{Show that for all real numbers $a$ and $b$, $f(a) - f(b) \leq f(a-b)$ and also $f(b)-f(a) \leq f(b-a)$.}{1in}

\show{Show that for all real numbers $a$ and $b$, $f(a-b) \geq f(f(a)-f(b))$.}{1in}

\definition{Minimum function}{The function $h : [0,\infty) \to \R$ defined by 
\[
    h(x) = \left\{ \begin{array}{cl} x, & \mbox{if $x \leq 1$} \\
                                     1, & \mbox{if $x > 1$} \end{array} \right.
\]
can be called the minimum function.}

\show{Show that for all real numbers $a$, $h(a) = 0$ if and only if $a = 0$.}{1.5in}

\show{Show that if $a \leq b$, then $h(a) \leq h(b)$.}{1.5in}

\show{Show that if $h(a) < h(b)$, then $a < b$.}{1.5in}

\show{Show that for all real numbers $a$ and $b$, $h(a+b) \leq h(a) + h(b)$.
{\bf Hint:}  Use a proof by cases.  But what are the cases?}{0in}

\vfill          % pad the rest of the page with white space
