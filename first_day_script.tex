\noindent {\bf What you might say on the first day of class}

This class is designed to help you develop key skills in math that will help you succeed in higher level, more abstract courses.
My goal is that by working hard in this course, you will become unstoppable in your future math courses.
\vskip 0.1in

There are two main components to the course:  learning how to read a math book on your own, and practicing the basic steps in mathematics, where we start with examples and non--examples, make a definition, consider more examples, make a conjecture, see if we can find a counterexample to the conjecture, make another conjecture, and if we can find a proof, call the result a theorem.
\vskip 0.1in

The simplest way to describe this course is that it is the ``proof course.''
A better way to describe it is that it focuses on the core ideas of mathematics, over and over again, in different settings.
The core idea of mathematics is to make definitions of mathematical objects, consider examples and non--examples of those definitions to make sure we understand them well, make conjectures about what we think might be true, and then try to prove those things.
It is all about understanding mathematical ideas and how they fit together.
Proofs are a part of that, but they are only a part.
This course will not have the same feel as algebra and calculus courses, which are heavier on calculations and don't have quite as many different ideas or different types of examples.
This course is much more about skills than content.
\vskip 0.1in

Most people learn by doing.
Me talking at the board is not the same as you learning.
Most of your time in the course will be spent working on activities as a group while I go from group to group, seeing how you are doing, answering questions, and making suggestions.
You will sometimes hand the activities in at the end of class so that I can read over them and make comments, then give them back to you at the beginning of the next class.
There is no need to rush through the activities.
Take your time, think about what you're doing.
There are often multiple correct ways to do a problem.
It's not question of ``what I want'' as the teacher, but what works.
Work with your group to make sure you all understand everything along the way.
You do not need to finish the activity; I always try to add extra material at the end so that no group runs out of things to do.
\vskip 0.1in

Outside of class, you will be reading the textbook, taking notes on what you read, and solving some exercises.
You will bring your notes to class and they will be read over and returned by the start of the next class to make sure that you are doing the reading and thinking.
Your first assignment is to read Chapter 1 of the textbook and turn in your notebook on Wednesday.
Each chapter will also have a short reading quiz.
I will not lecture over the material in the book; that is one of the keys to you learning how to read a book on your own.
\vskip 0.1in

There will be a final exam, but rather than mid-term exams, there will be quizzes over the material in activities, once you have had a chance to get good at it.
