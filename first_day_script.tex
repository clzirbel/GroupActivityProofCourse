\noindent {\bf Craig Zirbel's script for the first day of class}

Welcome to Math 3280, Mathematical Foundations and Techniques, better known as the proof class.
I teach this class whenever I can because I love teaching it.
When I tell other faculty that I'm teaching the proof course, they smile and say good for you and wish that they could be teaching it.
When I tell undergraduates and people from outside the department, they often recoil in horror, as if proofs are some Medieval torture or, more likely, as if proofs are something hard that you are expected to know how to do from some innate knowledge, but have never been taught.
\vskip 0.1in

This class is designed to put you at ease and teach you what to do.
It is designed to help you develop key skills in math that will help you succeed in higher level, more abstract courses.
My goal is that by working hard in this course, you will become unstoppable in your future math courses.
Most of the time in undergraduate math courses, the goal is for you to learn a bunch of new concepts, new definitions, to do exercises that are a few steps away from the definitions, and to understand some harder proofs that are important in the field.
Being able to read and understand definitions is very important there, and knowing the mechanics of writing proofs is sometimes all you really need to know.
This class works a lot on the mechanics of writing proofs.
\vskip 0.1in

There are two main components to the course:  learning how to read a math book on your own, and practicing the basic steps in mathematics, where we start with examples and non--examples, make a definition, consider more examples, make a conjecture, see if we can find a counterexample to the conjecture, make another conjecture, and if we can find a proof, call the result a theorem.
It is all about understanding mathematical ideas and how they fit together.
Proofs are a big part of that.
This course will not have the same feel as algebra and calculus courses, which are heavier on calculations and don't have quite as many different ideas or different types of examples.
This course is much more about skills than content.
\vskip 0.1in

Most people learn by doing.
Me talking at the board is not the same as you learning.
Most of your time in the course will be spent working on activities as a group while I go from group to group, seeing how you are doing, answering questions, and making suggestions.
You will sometimes hand the activities in at the end of class so that I can read over them and make comments, then give them back to you at the beginning of the next class.
There is no need to rush through the activities.
Take your time, think about what you're doing.
There are often multiple correct ways to do a problem.
It's not question of ``what I want'' as the teacher, but what works.
Work with your group to make sure you all understand everything along the way.
You do not need to finish the activity; I always try to add extra material at the end so that no group runs out of things to do.
\vskip 0.1in

Outside of class, you will be reading the textbook, taking notes on what you read, and solving some exercises.
You will bring your notes to class and they will be read over and returned by the start of the next class to make sure that you are doing the reading and thinking.
Your first assignment is to read Chapter 1 of the textbook and turn in your notebook next Tuesday.
The assignment is posted on Canvas already.
I will not lecture over the material in the book; that is one of the keys to you learning how to read a book on your own.
\vskip 0.1in

There will be a final exam, but rather than mid-term exams, there will be half a dozen quizzes over the material in activities, once you have had a chance to get good at it.
